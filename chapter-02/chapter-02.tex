% \documentclass[class=report, crop=false]{standalone}
\documentclass[../master.tex]{subfiles}
% % This is a file containing commonly used environments and such to remove clutter from other files

% Common packages
\usepackage[utf8]{inputenc}
\usepackage[leqno]{amsmath}
\usepackage{amsfonts, amssymb, amsthm, enumitem, bm}
\usepackage{mathrsfs, tikz-cd}
\usepackage{mathtools}
\usepackage{systeme}
% \usepackage{quiver}
% \usepackage{ntheorem, thmtools}



% Environments

% Problem Environment
\theoremstyle{definition}
\newtheorem{problem-internal}{Exercise}[section]
\newenvironment{problem}{
  \medskip
  \begin{problem-internal}
}{
\end{problem-internal}
}

% Solution Environment
\newenvironment{solution}{
  \begin{proof}[Solution]
    \vspace{-2px}
    \setlength{\parskip}{4px}
    \setlength{\parindent}{0px}
}{
\end{proof}
}

% Axiom Environment
\newtheoremstyle{emptyplain}
    {}          % default space above
    {}          % default space below
    {\itshape}  % body font
    {}          % no indent
    {\bfseries} % head font
    {.}         % punctuation after theorem head
    { }         % space after theorem head
    {#3}
\theoremstyle{emptyplain}
\newtheorem*{axiom}{}

% Proposition Environment
\newtheorem{proposition}{Proposition}[section]

% Norm definitions
\DeclarePairedDelimiter\abs{\lvert}{\rvert}%
\DeclarePairedDelimiter\norm{\lVert}{\rVert}%

% Swap the definition of \abs* and \norm*, so that \abs
% and \norm resizes the size of the brackets, and the 
% starred version does not.
\makeatletter
\let\oldabs\abs
\def\abs{\@ifstar{\oldabs}{\oldabs*}}
%
\let\oldnorm\norm
\def\norm{\@ifstar{\oldnorm}{\oldnorm*}}
\makeatother

% Operator names
\DeclareMathOperator{\im}{im}
\DeclareMathOperator{\coker}{coker}


\begin{document}
  \chapter{Groups, first encounter}

  \section{Definition of group}

  % Problem 1.1
  \begin{problem}
    Write a careful proof that every group is the group of isomorphisms of a groupoid.
    In particular, every group is the group of automorphisms of some object in some category.
  \end{problem}

  \begin{solution}
    Let \(G\) be a group.
    Consider a category \(\mathsf{C}_{G}\) with a single object \(\bullet\).
    Morphisms from \(\bullet\) to itself are the elements of \(G\).
    If \(g, h \in G\), then the morphism \(gh\) is their composition which is also in \(G\).
    The existence of inverse elements indicates that each morphism has an inverse, so they are all isomorphisms.
    Hence, each morphism is an isomorphism from \(\bullet\) to itself, thus making each morphism an automorphism.
    Therefore \(\mathsf{C}_{G}\) is a grouped with one object.
  \end{solution}

  % Problem 1.2
  \begin{problem}
     Consider the 'sets of numbers' listed in \S 1.1, and decide which are made
     into groups by conventional operations such as $+$ and $\cdot$. Even if the
     answer is negative (for example, $(\mathbb{R}, \cdot)$ is not a group), see
     if variations on the definition of these sets lead to groups (for example,
     $(\mathbb{R}^{*}, \cdot)$ \textit{is} a group; cf. \S 1.4).
  \end{problem}

  \begin{solution}
      $\mathbb{Z}$ is a group under normal addition, as are other sets with
      negatives and 0. Multiplicative groups require reciprocals and exclude
      zero since it has no multiplicative inverse. 
  \end{solution}

  % Problem 1.3
  \begin{problem}
      Prove that $(gh)^{-1} = h^{-1} g^{-1}$ for all elements $g, h$ of a group
      $G$.
  \end{problem}

  \begin{solution}
      First note that
      \[
          (gh) (gh)^{-1} = (gh) (h^{-1} g^{-1}) = g (h h^{-1}) g^{-1} = g g^{-1}
          = e
      \] 
      Furthermore, we have
      \[
          (gh)^{-1} (gh) = (h^{-1} g^{-1}) (gh) = h^{-1} g^{-1} g h = h^{-1} h =
          e
      \] 
      Thus, $(gh)^{-1} = h^{-1} g^{-1}$ is a two-sided inverse of $gh$.
  \end{solution}

  % Problem 1.4
  \begin{problem}
      Suppose that $g^2 = e$ for all elements $g$ of a group $G$; prove that $G$ 
      is commutative.
  \end{problem}

  \begin{solution}
      First note that $g^2 = e \Longrightarrow g = g^{-1}$. Then we have
      \[
          (gh)^{-1} = h^{-1} g^{-1} \Longrightarrow gh = hg
      \] 
      Thus, $G$ is commutative.
  \end{solution}

\end{document}
