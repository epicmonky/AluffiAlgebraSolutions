\documentclass[../../master.tex]{subfiles}

\begin{document}
\section{Naive Set Theory}

  % Problem 1.1
  \begin{problem}
    Locate a discussion of Russell's paradox, and understand it.
  \end{problem}

  \begin{solution}
    Consider the set of all sets which do not contain themselves.
    Does this set contain itself? If it is an element of itself, then clearly it
    contains itself. Thus it fails to satisfy its defining property and does not
    contain itself. If it does not contain itself, then it satisfies its defining
    property and does contain itself. The paradox demonstrates that not all
    properties can define a set.
  \end{solution}

  % Problem 1.2
  \begin{problem}
    Prove that if \(\sim\) is an equivalence relation on a set \(S\), then the
    corresponding family \(\mathscr{P}_{\sim}\) defined in \S 1.5 is indeed a
    partition of \(S\): that is, its elements are nonempty, disjoint, and their
    union is \(S\).
  \end{problem}

  \begin{solution}
    Let \(S\) be a set with the equivalence relation \(\sim\). Consider \\
    \(\mathscr{P}_{\sim} = \{[a]_{\sim} \mid a \in S \}\). Let \([a]_{\sim} \in
    \mathscr{P}_{\sim}\). Since \(\sim\) is reflexive, \(a \sim a\) so
    \([a]_{\sim}\) is nonempty.

    Now suppose \(a, b \in S\) and \(a \nsim b\). Suppose \(x \in [a]_{\sim}
    \cap [b]_{\sim}\). Then, since \(\sim\) is transitive, \(x \sim a\) and
    \(x \sim b\) so \(a \sim b\), a contradiction. Thus, each \([a]_{\sim}\) is
    disjoint.

    Finally, consider \(\bigcup_{[a]_{\sim} \in \mathscr{P}_{\sim}} [a]_{\sim}\). If \(a \in S\), then \(a \in
    [a]_{\sim}\). Thus, \(\bigcup [a]_{\sim} = S\).
  \end{solution}

  % Problem 1.3
  \begin{problem}
    Given a partion \(\mathscr{P}\) on a set \(S\), show how to define a
    relation \(\sim\) on \(S\) such that \(\mathscr{P}\) is the corresponding
    partition.
  \end{problem}

  \begin{solution}
    Let \(a \sim b\) if and only if \(\exists X \in \mathscr{P}\) such that \(a
    \in X\) and \(b \in X\) and let \(\mathscr{P}_{\sim}\) be the corresponding
    partition.

    Let \(X \in \mathscr{P}\). Certainly \(X\) is nonempty, so let \(a \in X\)
    and consider \([a]_{\sim} \in \mathscr{P}_{\sim}\). We must show that \(X =
    [a]_{\sim}\). Suppose \(a' \in X\) (it may be the case that \(a' = a\)).
    Since \(a, a' \in X\), we have \(a \sim a'\), so \(a' \in [a]_{\sim}\). Now
    suppose \(a' \in [a]_{\sim}\). Then \(a' \sim a\) so \(a' \in X\). Thus,
    \(X = [a]_{\sim} \in \mathscr{P}_{\sim}\), so \(\mathscr{P} \subseteq
    \mathscr{P}_{\sim}\).

    Now let \([a]_{\sim} \in \mathscr{P}_{\sim}\). We know that \([a]_{\sim}\) is nonempty, so choose \(a' \in [a]_{\sim}\).
    Then \(a' \sim a\) and there exists \(X \in \mathscr{P}\) such that \(a, a' \in X\). Hence, \([a]_{\sim} \subseteq X\).
    Furthermore, if \(a, a' \in X\) then \(a \sim a'\). Therefore, \(\mathscr{P}_{\sim} \subseteq \mathscr{P}\) and we have that \(\mathscr{P} = \mathscr{P}_{\sim}\).
  \end{solution}

  % Problem 1.4
  \begin{problem}
    How many different equivalence relations may be defined on the set \(\{1, 2, 3\}\)?
  \end{problem}

  \begin{solution}
    The number of equivalence relations is in bijection with the number of partitions. We can count these by hand:
    \begin{align*}
      \mathscr{P}_{0} &= \{ \{1, 2, 3\} \} \\
      \mathscr{P}_{1} &= \{ \{1\}, \{2\}, \{3\} \} \\
      \mathscr{P}_{2} &= \{ \{1, 2\}, \{3\} \} \\
      \mathscr{P}_{3} &= \{ \{1\}, \{2, 3\} \} \\
      \mathscr{P}_{4} &= \{ \{1, 3\}, \{2\} \}
    \end{align*}
    There are 5 equivalence relations defined on \(\{1, 2, 3\}\).
  \end{solution}

  % Problem 1.5
  \begin{problem}
    Give an example of a relation that is reflexive and symmetric but not transitive. What happens if you attempt to use this relation to define a partition on the set?
  \end{problem}

  \begin{solution}
    Consider the set of integers \(\mathbb{Z}\) and define \(a \sim b\) if and only if \(|a - b| \leq 1\).
    Certainly this is reflexive since \(a \sim a\) if and only if \(|a - a| = 0 \leq 1\), which holds for all integers.
    It is also symmetric because if \(a \sim b\) then \(|a - b| \leq 1\), but \(|a - b| = |b - a|\) so \(|b - a| \leq 1\), implying that \(b \sim a\).
    However, it is not transitive. For example, consider \(a = 0, b = 1, c = 2\). Then \(a \sim b\) and \(b \sim c\), but \(a \nsim c\).

    Attempting to define a partition using a relation which is not transitive means that partitions are not necessarily disjoint. For example, \([2]_{\sim} = \{1, 2, 3\}\), but \([3]_{\sim} = \{2, 3, 4\}\). Hence \(\mathscr{P}_{\sim}\) is not a partition of \(\mathbb{Z}\).
  \end{solution}

  % Problem 1.6
  \begin{problem}
    Define a relation \(\sim\) on the set \(\mathbb{R}\) of real numbers by setting \(a \sim b \Longleftrightarrow b - a \in \mathbb{Z}\).
    Prove that this is an equivalence relation, and find a `compelling' description for \(R / {\sim}\).
    Do the same for the relation \(\approx\) on the plane \(\mathbb{R} \times \mathbb{R}\) defined by declaring
    \((a_{1}, a_{2}) \approx (b_{1}, b_{2}) \Longleftrightarrow b_{1} - a_{1} \in \mathbb{Z}\) and \(b_{2} - a_{2} \in \mathbb{Z}\).
  \end{problem}

  \begin{solution}
    Let \(a, b, c \in \mathbb{R}\). Then \(a - a = 0 \in \mathbb{R}\) so \(a \sim a\) and \(\sim\) is reflexive.
    If \(a \sim b\) then \(b - a = n \in \mathbb{Z}\). Then \(a - b = -n \in \mathbb{Z}\) so \(b \sim a\) and \(\sim\) is symmetric.
    If \(a \sim b\) and \(b \sim c\) then \(b - a = m \in \mathbb{Z}\) and \(c - b = n \in \mathbb{Z}\). Then \(c - a = (c - b) + (b - a) = n + m \in \mathbb{Z}\), so \(a \sim c\) and \(\sim\) is transitive.
    Thus, \(\sim\) is an equivalence relation.

    \(\mathbb{R} / {\sim}\) is the set of equivalence classes under the given relation. It may be interpreted as the set of integers shifted by a real number \(\epsilon \in [0, 1)\).
    That is, for every set \(X \in \mathbb{R} / {\sim}\), there is a real number \(\epsilon \in [0, 1)\) such that every \(x \in X\) is of the form \(n + \epsilon\) for some \(n \in \mathbb{Z}\).

    We use a similar procedure to show that \(\approx\) is an equivalence relation. Let \((a_{1}, a_{2}) \in \mathbb{R} \times \mathbb{R}\).
    Then we have \(a_{1} - a_{1} = a_{2} - a_{2} = 0 \in \mathbb{Z}\). Thus, \((a_{1}, a_{2}) \approx (a_{1}, a_{2})\) and \(\approx\) is reflexive.
    Let \((b_{1}, b_{2}), (c_{1}, c_{2}) \in \mathbb{R} \times \mathbb{R}\). If we have \((a_{1}, a_{2}) \approx (b_{1}, b_{2})\), then \(b_{1} - a_{1} = m_{1} \in \mathbb{Z}\) and \(b_{2} - a_{2} = m_{2} \in \mathbb{Z}\).
    Hence \(a_{1} - b_{1} = -m_{1} \in \mathbb{Z}\) and \(a_{2} - b_{2} = -m_{2} \in \mathbb{Z}\) so \((b_{1}, b_{2}) \approx (a_{1}, a_{2})\) and \(\approx\) is symmetric.
    Finally, suppose \((a_{1}, a_{2}) \approx (b_{1}, b_{2})\) and \((b_{1}, b_{2}) \approx (c_{1}, c_{2})\).
    Then \(b_{1} - a_{1} = m_{1} \in \mathbb{Z}\), \(b_{2} - a_{2} = m_{2} \in \mathbb{Z}\), \(c_{1} - b_{1} = n_{1} \in \mathbb{Z}\), and \(c_{2} - b_{2} = n_{2} \in \mathbb{Z}\).
    Therefore, \(c_{1} - a_{1} = (c_{1} - b_{1}) + (b_{1} - a_{1}) = n_{1} + m_{1} \in \mathbb{Z}\) and \(c_{2} - a_{2} = (c_{2} - b_{2}) + (b_{2} - a_{2}) = n_{2} + m_{2} \in \mathbb{Z}\).
    Thus, \((a_{1}, a_{2}) \approx (c_{1}, c_{2})\) and \(\approx\) is transitive. Then \(\approx\) is an equivalence relation over \(\mathbb{R} \times \mathbb{R}\).

    \(\mathbb{R} \times \mathbb{R} / {\approx}\) is the set of equivalence classes under the given relation.
    Every element is the 2-dimensional integer lattice shifted by a pair of real numbers \((\epsilon_{1}, \epsilon_{2}) \in [0, 1) \times [0, 1)\).
  \end{solution}

\end{document}
