\documentclass[../../master.tex]{subfiles}

\begin{document}
\section{Functions between sets}

  % Problem 2.1
  \begin{problem}
    How many different bijections are there between a set \(S\) with \(n\) elements and itself?
  \end{problem}

  \begin{solution}
    A function \(f: S \to S\) is a subset \(\Gamma_{f} \subseteq S \times S\).
    Since \(f\) is bijective, then for all \(y \in S\), there exists a unique \(x \in S\) such that \((x, y) \in \Gamma_{f}\).
    Certainly \(|\Gamma_{f}| = n\). Since each \(x\) is unique, every element \(x \in S\) must be present in the first component of exactly one element in \(\Gamma_{f}\).
    Similarly, each element \(y \in S\) must be present in the second component of exactly one element in \(\Gamma_{f}\).
    Then each bijection is merely a permutation of \(S\), and there are \(n!\) permutations.
    Thus, there are \(n!\) bijections from \(S\) to itself.
  \end{solution}

  % Problem 2.2
  \begin{problem}
    Prove statement (2) in Proposition 2.1. You may assume that given a family of disjoint subsets of a set, there is a way to choose one element in each member of the family.
  \end{problem}

  \begin{proposition}[Proposition 2.1]
    Assume \(A \neq \emptyset \), and let \(f: A \to B\) be a function.
    Then (1) \(f\) has a left-inverse if and only if \(f \) is injective; and (2) \(f\) has a right-inverse if and only if \(f\) is surjective.
  \end{proposition}

  \begin{solution}
    Assume \(A \neq \emptyset\) and let \(f: A \to B\) be a function.

    \(( \Longrightarrow)\) Suppose there exists a function \(g\) that is a right-inverse of \(f\). Then \(f \circ g = \text{id}_{B}\).
    Let \(b \in B\). Then \(g(b) \in A\) and \(f(g(b)) = b\). Thus for all \(b \in B\), there exists \(a = g(b)\) such that \(f(a) = b\). Hence, \(f\) is surjective.

    \(( \Longleftarrow)\) Suppose that \(f\) is surjective. We want a function \(g: B \to A\) such that \(f(g(b)) = b\) for all \(b \in B\).
    Since \(f\) is surjective, for all \(b \in B\), there exists an \(a \in A\) such that \(f(a) = b\).
    Construct a set \(\Gamma = \{(b, a) \mid f(a) = b\} \subseteq B \times A\).
    Note that \(\Gamma\) is not necessarily unique since there may be several \(a\) such that \(f(a) = b\).
    However, its existence is guaranteed since \(f\) is surjective.
    Then this set may be used to define \(g\) where \(g(b) = a\) if and only if \((a, b) \in \Gamma\).
    Now let \(b \in B\). Then there exists an \(a \in A\) such that \(f(a) = b\). Therefore, \((a, b) \in \Gamma\) so \(g(b) = a\).
    We get that \(f(g(b)) = f(a) = b\) so \(g\) is a right-inverse of \(f\).
  \end{solution}

  % Problem 2.3
  \begin{problem}
    Prove that the inverse of a bijection is a bijection and that the composition of two bijections is bijection.
  \end{problem}

  \begin{solution}
    Let \(f: A \to B\) be a bijection. Consider \(f^{-1}: B \to A\). We have that \(f^{-1} \circ f = \text{id}_{A}\) and \(f \circ f^{-1} = \text{id}_{B}\).
    Then \(f\) is the left- and right-inverse of \(f^{-1}\), so \(f^{-1}\) is also a bijection.

    Let \(f: A \to B\) and \(g: B \to C\) be bijections and consider \(g \circ f\). Suppose \(a, a' \in A\) such that \((g \circ f)(a) = (g \circ f)(a')\).
    Since \(g\) is bijective, and in particular it is injective, we have \((g \circ f)(a) = (g \circ f)(a') \Longrightarrow f(a) = f(a')\).
    Similarly, \(f\) is injective so \(f(a) = f(a') \Longrightarrow a = a'\). Thus, \(g \circ f\) is injective.
    Now let \(c \in C\). Since \(g\) is surjective, there exists a \(b \in B\) such that \(g(b) = c\).
    Similarly, since \(f\) is surjective, there exists an \(a \in A\) such that \(f(a) = b\).
    Then \((g \circ f)(a) = g(b) = c\) so \(g \circ f\) is surjective. Hence, \(g \circ f\) is bijective.
  \end{solution}

  % Problem 2.4
  \begin{problem}
    Prove that `isomorphism' is an equivalence relation (on any set of sets).
  \end{problem}

  \begin{solution}
    Let \(A\) be a set. Then \(\text{id}_{A}\) is a bijection so \(A \cong A\).
    Let \(B\) be another set such that \(A \cong B\). That is, there exists a bijection \(f: A \to B\).
    Since \(f\) is bijective, it has an inverse \(f^{-1}: B \to A\), so \(B \cong A\).
    If \(C\) is another set such that \(B \cong C\), then there exists a bijection \(g: B \to C\).
    The composition of bijections is a bijection so \(g \circ f: A \to C\) is bijective.
    Hence \(A \cong C\) and \(\cong\) is an equivalence relation.
  \end{solution}

  % Problem 2.5
  \begin{problem}
    Formulate a notion of \textit{epimorphism}, in the style of the notion of \textit{monomorphism} seen in \S 2.6,
    and prove a result analogous to Proposition 2.3, for epimorphisms and surjections.
  \end{problem}

  \begin{proposition}[Proposition 2.3]
    A function is injective if and only if it is a monomorphism.
  \end{proposition}

  \begin{solution}
    A function \(f: A \to B\) is an epimorphism if for all sets \(Z\) and all functions \(\beta, \beta': B \to Z\) we have \(\beta \circ f = \beta' \circ f \Longrightarrow \beta = \beta'\).
    Now we show that a function is surjective if and only if it is an epimorphism.

    \((\Longrightarrow)\) Suppose that \(f: A \to B \) is surjective. Then \(f\) has a right-inverse \(g: B \to A\).
    Let \(\beta, \beta'\) be functions from \(B\) to another set \(Z\) such that \(\beta \circ f = \beta' \circ f\).
    Compose on the right by \(g\) and use associativity of composition:
    \begin{equation*}
      \beta \circ (f \circ g) = (\beta \circ f) \circ g = (\beta' \circ f) \circ g = \beta' \circ (f \circ g)
    \end{equation*}
    Since \(g\) is a right-inverse of \(f\), we have
    \begin{equation*}
      \beta \circ \text{id}_{B} = \beta' \circ \text{id}_{B}
    \end{equation*}
    and thus \(\beta = \beta'\) and \(f\) is an epimorphism.

    \((\Longleftarrow)\) Now suppose that \(f: A \to B\) is an epimorphism.
    Let \(Z = \{0, 1\}\) and consider the morphisms \(\beta, \beta': B \to Z\) where \(\beta(b) = 0\) for all \(b \in B\) and \(\beta'(b) = 0\) if \(b \in \text{im}(f)\) or \(\beta'(b) = 1\) otherwise.
    By construction, \(\beta \circ f = \beta' \circ f\). This implies that \(\beta = \beta'\),
    which is only the case if every element \(b \in B\) is sent to the same element of \(Z\).
    \(\beta\) sends every element of \(B\) to 0, and \(\beta'\) sends every element of \(\text{im}(f)\) to 0, so \(\text{im}(f) = B\) and \(f\) is surjective.
  \end{solution}

  % Problem 2.6
  \begin{problem}
    With notation as in Example 2.4, explain how any function \(f: A \to B\) determines a section of \(\pi_{A}\).
  \end{problem}

  \begin{solution}
    We know \(f\) corresponds to a subset \(\Gamma_{f} = \{(a, b) \mid f(a) = b\} \subseteq A \times B\).
    The projection \(\pi_{A}: A \times B \to A\) is defined such that \(\pi_{A} (a, b) = a\).
    Let \(g: A \to A \times B\) be a function such that \(g(a) = (a, f(a)) \in \Gamma_{f}\).
    Since \((\pi_{A} \circ g) (a) = \pi_{A}(a, f(a)) = a\) for all \(a \in A\), \(g\) is a section of \(\pi_{A}\) which is determined by \(f\).
  \end{solution}

  % Problem 2.7
  \begin{problem}
    Let \(f: A \to B\) be any function. Prove that the graph \(\Gamma_{f}\) of \(f\) is isomorphic to \(A\).
  \end{problem}

  \begin{solution}
    Recall that \(\Gamma_{f} = \{(a, b) \mid b = f(a)\} \subseteq A \times B\).
    Let \(g: A \to \Gamma_{f}\) be defined as \(g(a) = (a, f(a))\).
    For all \((a, b) \in \Gamma_{f}\), we have \(g(a) = (a, f(a)) = (a, b)\) so \(g\) is surjective.
    If \(g(a) = g(a')\), then \((a, f(a)) = (a', f(a'))\).
    That is, \(a = a'\) so \(g\) is injective, hence it is a bijection.
    Therefore, \(\Gamma_{f} \cong A\).
  \end{solution}

  % Problem 2.8
  \begin{problem}
    Describe as explicitly as you can all terms in the canonical decomposition of the function \(\mathbb{R} \to \mathbb{C}\) defined by \(r \mapsto e^{2\pi ir}\).
    (This exercise matches one assigned previously. Which one?)
  \end{problem}

  \begin{solution}
    Let \(f: \mathbb{R} \to \mathbb{C}\) be the function defined above.
    The first part of the decomposition is defined by letting \(\sim\) be an equivalence relation on \(\mathbb{R}\) such that \(a \sim b \Longleftrightarrow f(a) = f(b)\).
    That is, \([a]_{\sim}\) is the set of elements in \(\mathbb{R}\) that are mapped to the same element as \(a\) in \(\mathbb{C}\).
    Then we have a projection \(\mathbb{R} \twoheadrightarrow \mathbb{R} / {\sim}\) which sends each element \(a \in \mathbb{R}\) to its equivalence class \([a]_{\sim}\).
    Note that \(f(x) = f(x+1)\). That is, the function is periodic about the integers so real numbers which differ by an integer amount belong to the same equivalence class.
    Then \(\mathbb{R} / {\sim} = \{ \{r + k \mid k \in \mathbb{Z} \} \mid r \in [0, 1)\) which is identical to the quotient set in Exercise 1.1.6.

    The function \(\tilde{f}: \mathbb{R} \to \text{im}(f)\) maps each equivalence class to the complex number that \(f\) maps the representative to.
    Certainly if \(\tilde{f}([a]_{\sim}) = \tilde{f}([a']_{\sim})\) then \(f(a) = f(a')\) and \(a \sim a'\) by definition.
    Thus \([a]_{\sim} = [a']_{\sim}\) so \(\tilde{f}\) is injective.
    Similarly, let \(b \in \text{im}(f)\). Then there is an element \(a \in \mathbb{R}\) such that \(f(a) = b\).
    Then \(\tilde{f}([a]_{\sim}) = f(a) = b\) so \(\tilde{f}\) is surjective and hence a bijection.
    Finally, we have the inclusion \(\text{im}(f) \hookrightarrow \mathbb{C}\) which embeds the image of \(f\) into its codomain.
  \end{solution}

  % Problem 2.9
  \begin{problem}
    Show that if \(A' \cong A''\) and \(B' \cong B''\), and further \(A' \cap B' = \emptyset\) and \(A'' \cap B'' = \emptyset\), then \(A' \cup B' \cong A'' \cup B''\).
    Conclude that the operation \(A \coprod B\) is well-defined \textit{up to isomorphism}.
  \end{problem}

  \begin{solution}
    There exist bijections \(f: A' \to A''\) and \(g: B' \to B''\).
    Then we can define \(h: A' \cup B' \to A'' \cup B''\) where
    \begin{equation*}
      h(x) =
      \begin{cases}
        f(x) \text{ if } x \in A' \\
        g(x) \text{ if } x \in B'
      \end{cases}
    \end{equation*}
    Let \(y \in A'' \cup B''\). Since \(A'' \cap B'' = \emptyset\), we have either \(y \in A''\) or \(y \in B''\).
    WLOG, suppose that \(y \in A''\).
    Note that since \(f\) is surjective, there exists \(x \in A'\) such that \(f(x) = y\).
    Then \(h(x) = f(x) = y\) so \(h\) is surjective.
    Suppose \(x \neq x'\) for \(x, x' \in A' \cup B'\). If \(x, x' \in A'\) then since \(f\) is injective and \(h(x) = f(x)\) for all \(x \in A'\), we have \(h(x) \neq h(x')\).
    A similar reasoning shows that if \(x, x' \in B'\), then \(h(x) \neq h(x')\).
    WLOG, suppose that \(x \in A'\) and \(x' \in B'\). Then \(h(x) = f(x) \neq g(x') = h(x')\) since \(A'' \cap B'' = \emptyset\). Thus \(h\) is surjective and hence a bijection, showing that \(A' \cup B' \cong A'' \cup B''\).

    The constructions of \(A', A'', B', B''\) are equivalent to creating ``copies" of sets \(A\) and \(B\) to use in the disjoint union.
    Thus, the disjoint union \(A \coprod B\) is well-defined up to isomorphism.
  \end{solution}

  % Problem 2.10
  \begin{problem}
    Show that if \(A\) and \(B\) are finite sets, then \(|B^{A}| = |B|^{|A|}|\).
  \end{problem}

  \begin{solution}
    Recall that \(|B^{A}|\) is the number of functions from \(A\) to \(B\).
    Each functions assigns a single element of \(A\) to a single element of \(B\).
    There are \(|B|\) choices for each of the \(|A|\) elements.
    This is equivalent to \(|B|^{|A|}\) total choices.
    Thus, \(|B^{A}| = |B|^{|A|}\).
  \end{solution}

  % Problem 2.11
  \begin{problem}
    In view of Exercise 2.10, it is not unreasonable to use \(2^{A}\) to denote the set of functions from an arbitrary set \(A\) to a set with 2 elements (say \(\{0, 1\}\)).
    Prove that there is a bijection between \(2^{A}\) and the \textit{power set} of \(A\).
  \end{problem}

  \begin{solution}
    Consider \(f: \mathscr{P} (A) \to 2^{A}\) defined as
    \begin{equation*}
      f(X) = \{(a, 1) \text{ if } a \in X \text{, and } (a, 0) \text{ otherwise} \}
    \end{equation*}
    Let \(g \in 2^{A}\). Then \(g\) is a function from \(A\) to \(\{0, 1\}\).
    Let \(A_{1} = \{a \in A \mid g(a) = 1\). Then \(A_{1} \in \mathscr{P}(A)\) and \(f(A_{1}) = g\), so \(f\) is surjective.
    Now suppose that \(X, Y \subseteq A\) such that \(f(X) = f(Y)\).
    That is, for all \(a \in A\), \(a \in X \Longleftrightarrow (a, 1) \in f(X) \Longleftrightarrow (a, 1) \in f(Y) \Longleftrightarrow a \in Y\).
    Thus, \(X = Y\) so \(f\) is injective and a bijection.
    Therefore, \(2^{A} \cong \mathscr{P}(A)\).
  \end{solution}
\end{document}
