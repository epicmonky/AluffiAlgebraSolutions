\documentclass[../../master.tex]{subfiles}

\begin{document}
\section{Categories}

  % Problem 3.1
  \begin{problem}
    Let \(\mathsf{C}\) be a category. Consider a structure \(\mathsf{C}^{op}\) with
    \begin{itemize}
      \item \(\text{Obj}(\mathsf{C}^{op}) := \text{Obj}(\mathsf{C})\);
      \item for \(A, B\) objects of \(\mathsf{C}^{op}\) (hence objects of \(\mathsf{C}\)),
      \(\text{Hom}_{\mathsf{C}^{op}} (A, B) := \text{Hom}_{\mathsf{C}}(B, A)\).
    \end{itemize}
    Show how to make this into a category (that is, define composition of moprhisms in \(\mathsf{C}^{op}\) and verify the properties listed in \S 3.1).

    Intuitively, the `opposite' category \(\mathsf{C}^{op}\) is simply obtained by `reversing all the arrows' in \(\mathsf{C}\).
  \end{problem}

  \begin{solution}
    For objects \(A, B, C \in \text{Obj}(\mathsf{C}^{op})\), the set of morphisms from \(A\) to \(B\), \(\text{Hom}_{\mathsf{C}^{op}}(A, B)\), is defined as \(\text{Hom}_{\mathsf{C}}(B, A)\).
    For morphisms \(f \in \text{Hom}_{\mathsf{C}^{op}}(A, B)\) and \(g \in \text{Hom}_{\mathsf{C}^{op}}(B, C)\), define composition as follows:
    \begin{equation*}
      \circ_{\mathsf{C}^{op}}: \text{Hom}_{\mathsf{C}^{op}}(A, B) \times \text{Hom}_{\mathsf{C}^{op}}(B, C) \to \text{Hom}_{\mathsf{C}^{op}}(A, C)
    \end{equation*}
    such that
    \begin{equation*}
      \circ_{\mathsf{C}^{op}}(g, f) = \circ_{\mathsf{C}}(f, g)
    \end{equation*}
    Then if \(f \in \text{Hom}_{\mathsf{C}^{op}}(A, B), g \in \text{Hom}_{\mathsf{C}^{op}}(B, C), h \in \text{Hom}_{\mathsf{C}^{op}}(C, D)\), then
    \begin{equation*}
      (h \circ_{\mathsf{C}^{op}} g) \circ_{\mathsf{C}^{op}} f = f \circ_{\mathsf{C}} (g \circ_{\mathsf{C}} h) = (f \circ_{\mathsf{C}} g) \circ_{\mathsf{C}} h = h \circ_{\mathsf{C}^{op}} (g \circ_{\mathsf{C}^{op}} f)
    \end{equation*}
    so composition is associative.
    Furthermore, define the identity morphism \(1_{A_{\mathsf{C}^{op}}} = 1_{A_{\mathsf{C}}}\).
    Then for all \(f \in \text{Hom}_{\mathsf{C}^{op}}(A, B)\) we have
    \begin{gather*}
      f \circ_{\mathsf{C}^{op}} 1_{A_{\mathsf{C}^{op}}} = 1_{A_{\mathsf{C}}} \circ_{\mathsf{C}} f = f \\
      1_{B_{\mathsf{C}^{op}}} \circ_{\mathsf{C}^{op}} f = f \circ_{\mathsf{C}} 1_{B_{\mathsf{C}}} = f
    \end{gather*}
    so identities preserve morphisms.
    Finally, let \(A, B, C, D \in \text{Obj}(\mathsf{C}^{op})\) where \(A \neq C\) and \(B \neq D\).
    Consider the sets \(\text{Hom}_{\mathsf{C}^{op}} (A, B)\) and \(\text{Hom}_{\mathsf{C}^{op}} (C, D)\).
    These are equal to the sets \(\text{Hom}_{\mathsf{C}} (B, A)\) and \(\text{Hom}_{\mathsf{C}} (D, C)\) respectively, which are disjoint since \(\mathsf{C}\) is a category.
    Thus, \(\mathsf{C}^{op}\) forms a category.
  \end{solution}

  % Problem 3.2
  \begin{problem}
    If \(A\) is a finite set, how large is \(\text{End}_{\mathsf{Set}}(A)\)?
  \end{problem}

  \begin{solution}
    Recall that \(\text{End}_{\mathsf{Set}}(A)\) is the set of functions from \(A\) to \(A\).
    By Exercise 2.10, we have \(|B^{A}| = |B|^{|A|}\).
    Thus, \(|\text{End}_{\mathsf{Set}}(A)| = |A|^{|A|}\).
  \end{solution}

  % Problem 3.3
  \begin{problem}
    Formulate precisely what it means to say that \(1_{a}\) is an identity with respect to composition in Example 3.3, and prove this assertion.
  \end{problem}

  \begin{solution}
    Let \(S\) be a set and \(\sim\) be a reflexive and transitive relation on \(S\).
    Consider a category \(\mathsf{C}\) where
    \begin{itemize}
      \item \(\text{Obj}(\mathsf{C})\) are the elements in \(S\)
      \item If \(a, b\) are objects, then let \(\text{Hom}(a, b) = (a, b) \in S \times S\) if \(a \sim b\) and let \(\text{Hom}(a, b) = \emptyset\) otherwise.
    \end{itemize}
    This forms a category and composition is defined as follows. Let \(a, b, c\) be objects and \(f \in \text{Hom}(a, b)\), \(g \in \text{Hom}(b, c)\).
    Then \(g \circ f = (a, c) \in \text{Hom}(a, c)\) by the transitivity of \(\sim\).

    Now we verify that the identity preserves morphisms in this category.
    Let \(a, b \in S\) and \(f \in \text{Hom}(a, b)\).
    A morphism \(1_{a} = (a, a) \in \text{End}(a)\) is an identity with respect to composition if
    \begin{equation*}
      f \circ 1_{a} = f
    \end{equation*}
    Indeed, we have \(f = (a, b)\) and \(1_{a} = (a, a)\).
    Then by definition we have
    \begin{equation*}
      f \circ 1_{a} = (a, b) (a, a) = (a, b) = f
    \end{equation*}
    Thus \(1_{a}\) is an identity with respect to composition as required.
  \end{solution}

  % Problem 3.4
  \begin{problem}
    Can we define a category in the style of Example 3.3 using the relation \(<\) on the set \(\mathbb{Z}\).
  \end{problem}

  \begin{solution}
    No, since the relation \(<\) is not reflexive. That is, \(a < a\) does not hold for any \(a \in \mathbb{Z}\).
    There is no reasonable way to define an identity morphism.
  \end{solution}

  % Problem 3.5
  \begin{problem}
    Explain in what sense Example 3.4 is an instance of the categories considered in Example 3.3.
  \end{problem}

  \begin{solution}
    Let \(S\) be a set and consider the category \(\hat{S}\) where
    \begin{itemize}
      \item \(\text{Obj}(\hat{S}) = \mathscr{P}(S)\)
      \item For \(A, B \in \text{Obj}(\hat{S})\), let \(\text{Hom}_{\hat{S}}(A, B)\) be the pair \((A, B)\) if \(A \subseteq B\),
      and let \(\text{Hom}_{\hat{S}}(A, B) = \emptyset\) otherwise.
      \end{itemize}
      Composition is obtained by using the transitivity of inclusion.

      This is equivalent to the category in Example 3.3 by considering the relation \(\sim\) defined on \(\mathscr{P}(S)\) where \(A \sim B\) if and only if \(A \subseteq B\).
      Indeed, this relation is both reflexive and transitive so we may construct the category considered in Example 3.3, and the two are equivalent.
  \end{solution}

  % Problem 3.6
  \begin{problem}
    (Assuming some familiarity with linear algebra.)
    Define a category \(\mathsf{V}\) by taking \(\text{Obj}(\mathsf{V}) = \mathbb{N}\)
    and letting \(\text{Hom}_{\mathsf{V}}(n, m) =\) the set of \(m \times n\) matrices with real entries, for all \(n, m \in \mathbb{N}\).
    (We will leave the reader the task of making sense of a matrix with 0 rows or columns.)
    Use products of matrices to define composition. Does this category `feel' familiar?
  \end{problem}

  \begin{solution}
    First of all, the identity morphism for the object \(n\) is the set of \(n \times n\) matrices.
    Let \(l, m, n \in \mathbb{N}\) and
    \begin{equation*}
      f \in \text{Hom}(l, m), \quad g \in \text{Hom}(m, n)
    \end{equation*}
    Then \(fg\) is an \(l \times n\) matrix and is in \(\text{Hom}(l, n)\).
    Furthermore, matrix multiplication is associative.

    This category is another instance of Example 3.3 where the set is \(\mathbb{N}\) and the relation \(\sim\) is defined as follows:
    \(m \sim n\) if and only if \(\text{Hom}(m, n)\) is nonempty.
    Certainly this relation is both reflexive and transitive so it is an instance of Example 3.3.
    \end{solution}

    % Problem 3.7
    \begin{problem}
      Define carefully objects and morphisms in Example 3.7, and draw the diagram corresponding to composition.
    \end{problem}

    \begin{solution}
      Given a category \(\mathsf{C}\) and an object \(A \in \text{Obj}(\mathsf{C})\), consider the category \(\mathsf{C}^{A}\) where
      \begin{itemize}
        \item \(\text{Obj}(\mathsf{C}^{A}) =\) all morphisms from \(A\) to any object of \(\mathsf{C}\);
        \item Let \(f_{1}, f_{2}\) be objects of \(\mathsf{C}^{A}\), or two arrows
        \[
        \begin{tikzcd}
	        {A} && {A} \\
	        \\
	        {Z_{1}} && {Z_{2}}
	        \arrow["{f_{1}}"', from=1-1, to=3-1]
	        \arrow["{f_{2}}", from=1-3, to=3-3]
        \end{tikzcd}
        \]
        in \(\mathsf{C}\). Morphisms \(f_{1} \to f_{2}\) are \textit{commutative diagrams}
        \[
        \begin{tikzcd}
	        && {A} \\
          \\
          {Z_{1}} &&&& {Z_{2}}
          \arrow["{f_{1}}"', from=1-3, to=3-1]
	        \arrow["{f_{2}}", from=1-3, to=3-5]
          \arrow["{\sigma}"', from=3-1, to=3-5]
        \end{tikzcd}
        \]
        in the category \(\mathsf{C}\).
      \end{itemize}
      That is, morphisms \(\sigma \in \text{Hom}_{\mathsf{C}^{A}}(f_{1}, f_{2})\)
      are precisely the morphisms \(\sigma: Z_{1} \to Z_{2}\) in \(\mathsf{C}\) such that \(f_{2} = \sigma \circ f_{1}\).

      If \(\sigma \in \text{Hom}(f, g)\) and \(\tau \in \text{Hom}(g, h)\), then \(\tau \circ \sigma \in \text{Hom}(f, h)\) is the morphism in \(\mathsf{C}\) making the following diagram commute:
      \[
      \begin{tikzcd}
	       && {A} \\
	        \\
	        {Z_{1}} && {Z_{2}} && {Z_{3}}
	        \arrow["{f}"', from=1-3, to=3-1]
          \arrow["{g}", from=1-3, to=3-3]
          \arrow["{\sigma}"', from=3-1, to=3-3]
	        \arrow["{h}", from=1-3, to=3-5]
	        \arrow["{\tau}"', from=3-3, to=3-5]
          \arrow["{\tau \circ \sigma}"', from=3-1, to=3-5, bend right=30]
      \end{tikzcd}
      \]
    \end{solution}

    % Problem 3.8
    \begin{problem}
      A \textit{subcategory} \(\mathsf{C'}\) of a category \(\mathsf{C}\) consists of a collection of objects of \(\mathsf{C}\),
      with morphisms \(\text{Hom}_{\mathsf{C'}}(A, B) \subseteq \text{Hom}_{\mathsf{C}}(A, B)\) for all objects \(A, B\) in \(\text{Obj}(\mathsf{C'})\),
      such that identities and compositions in \(\mathsf{C}\) make \(\mathsf{C'}\) into a category.
      A subcategory \(\mathsf{C'}\) is \textit{full} if \(\text{Hom}_{\mathsf{C'}}(A, B) = \text{Hom}_{\mathsf{C}}(A, B)\) for all \(A, B\) in \(\text{Obj}(\mathsf{C'})\).
      Construct a category of \textit{infinite sets} and explain how it may be viewed as a full subcategory of \(\mathsf{Set}\).
    \end{problem}

    \begin{solution}
      Let \(\mathsf{Set}^{\infty}\) be a category whose objects are infinite sets and whose morphisms are set functions between them.
      That is, for infinite sets \(A, B\) we let \(\text{Hom}_{\mathsf{Set}^{\infty}}(A, B)\) be the set of set functions from \(A\) to \(B\).
      Certainly this is equivalent to \(\text{Hom}_{\mathsf{Set}}(A, B)\) so the subcategory is full.
    \end{solution}

    % Problem 3.9
    \begin{problem}
      An alternative to the notion of \textit{multiset} introduced in \S 2.2 is obtained by considering sets endowed with equivalence relations;
      equivalent elements are taken to be multiple instances of elements `of the same kind'.
      Define a notion of morphism between such enhanced sets, obtaining a category \(\mathsf{MSet}\) containing (a `copy' of) \(\mathsf{Set}\) as a full subcategory.
      (There may be more than one reasonable way to do this! This is intentionally an open-ended exercise.)
      Which objects in \(\mathsf{MSet}\) determine ordinary multisets as defined in \S 2.2 and how?
      Spell out what a morphism of multisets would be from this point of view.
      (There are several natural notions of morphisms of multisets.
      Try to define morphisms in \(\mathsf{MSet}\) so that the notion you obtain for ordinary multisets captures your intuitive understanding of these objects.)
    \end{problem}

    \begin{solution}
      Consider the category \(\mathsf{MSet}\) where
      \begin{itemize}
        \item \(\text{Obj}(\mathsf{MSet}) =\) sets endowed with equivalence relations;
        \item If \(A, B \in \text{Obj}(\mathsf{MSet})\) then \(\text{Hom}_{\mathsf{MSet}}(A, B)\) is the collection of functions from \(A\) to \(B\) which preserve equivalence classes.
        That is, if \(\sim\) is an equivalence relation on \(A\) and \(\approx\) is an equivalence relation on \(B\) then
        for \(a, b \in A\) and \(f \in \text{Hom}_{\mathsf{MSet}}(A, B)\) we have \(a \sim b \Longrightarrow f(a) \approx f(b)\).
      \end{itemize}
      Composition is naturally defined as it is \(\mathsf{Set}\).
      For objects \(A, B, C\), let
      \(f \in \text{Hom}_{\mathsf{MSet}} (A, B)\) and \(g \in \text{Hom}_{\mathsf{MSet}} (B, C)\).
      If \(a, b \in A\) and \(a \sim_{A} b\) then, since \(f\) is a morphism, \(f(a) \sim_{B} f(b)\).
      Furthermore, \(g\) is a morphism so \(g(f(a)) \sim_{C} g(f(b))\) so \(g \circ f \in \text{Hom}_{\mathsf{MSet}}(A, C)\).
      The identity morphism has a natural definition where \(1_{S}: S \to S\) is the identity function \(\mathsf{Set}\).
      It obviously preserves equivalence classes.
      Associativity is similarly inherited from \(\mathsf{Set}\).

      In \S 2.2, multisets are defined as a set \(A\) along with a function \(m: A \to \mathbb{N}^{*}\) which takes each element of \(A\) to the number denoting its multiplicity.
      We define the equivalence relation \(\sim\) on \(A\) which partitions \(A\) into its distinct elements, or those elements which are not equal.
      In other words, \(m(a) \neq m(b) \Longrightarrow a \nsim b\).
      Morphisms between these objects as defined above can intuitively be expressed as
      the functions which allow elements to be renamed and naturally mapped to other multisets which preserve multiplicity.
    \end{solution}

    % Problem 3.10
    \begin{problem}
      Since the objects of a category \(\mathsf{C}\) are not (necessarily) sets, it is not clear how to make sense of a notion of `subobject' in general.
      In some situations it \textit{does} make sense to talk about subobjects,
      and the subobjects of any given object \(A\) in \(\mathsf{C}\) are in one-to-one correspondence with the morphisms \(A \to \Omega\)
      for a fixed special object \(\Omega\) of \(\mathsf{C}\), called a \textit{subobject classifier}.
      Show that \(\mathsf{Set}\) has a subobject classifier.
    \end{problem}

    \begin{solution}
      Consider the set \(\Omega = \{0, 1\}\).
    Let \(A\) be any set.
    The subsets \(X \subseteq A\) induce morphisms \(f: A \to \Omega\) where
      \begin{equation*}
        f(x) =
        \begin{cases}
          1 \text{ if } x \in X \\
          0 \text{ if } x \notin X
        \end{cases}
      \end{equation*}
      Certainly these morphisms are in bijection with subsets of \(A\).
      Thus \(\{0, 1\}\) is a subobject classifier of \(\mathsf{Set}\), though any set with 2 elements works.
    \end{solution}

    % Problem 3.11
    \begin{problem}
      Draw the relevant diagrams and define composition and identities for the category \(\mathsf{C}^{A, B}\) mentioned in Example 3.9.
      Do the same for the category \(\mathsf{C}^{\alpha, \beta}\) mentioned in Example 3.10.
    \end{problem}

    \begin{solution}
      Consider the category \(\mathsf{C}^{A, B}\) where
      \begin{itemize}
        \item \(\text{Obj}(\mathsf{C}^{A, B}) =\) diagrams
        \[
        \begin{tikzcd}
	        {A} \\
	        && {Z} \\
	        {B}
	        \arrow["{f}", from=1-1, to=2-3]
	        \arrow["{g}"', from=3-1, to=2-3]
        \end{tikzcd}
        \]
        in \(\mathsf{C}\)
        \item Morphisms between objects \((Z_{1}, f_{1}, g_{1})\) and \((Z_{2}, f_{2}, g_{2})\) are commutative diagrams
        \[
        \begin{tikzcd}
	        {A} \\
	        && {Z_{1}} && {Z_{2}} \\
	        {B}
	        \arrow["{f_{1}}", from=1-1, to=2-3]
	        \arrow["{g_{1}}"', from=3-1, to=2-3]
	        \arrow["{f_{2}}", from=1-1, to=2-5, bend left=15]
	        \arrow["{g_{2}}"', from=3-1, to=2-5, bend right=15]
	        \arrow["{\sigma}", from=2-3, to=2-5]
        \end{tikzcd}
        \]
        That is, we have a morphism \(\sigma \in \text{Hom}_{\mathsf{C}}(Z_{1}, Z_{2})\) such that
        \(f_{2} = \sigma \circ f_{1}\) and \(g_{2} = \sigma \circ g_{1}\).
      \end{itemize}
      Composition has a natural definition.
      Given a third object \((Z_{3}, f_{3}, g_{3})\) with a morphism \(\tau: Z_{2} \to Z_{3}\) we define \(\tau \circ \sigma: Z_{1} \to Z_{3}\)
      such that \(f_{3} = \tau \circ \sigma (f_{1})\) and \(g_{3} = \tau \circ \sigma (g_{1})\).
      Given an object \((Z, f, g)\), the identity morphism \(1_{Z} \in \text{End}_{\mathsf{C}}(Z)\) serves as an identity in \(\mathsf{C}^{A, B}\) as well.
      Specifically, we have \(f = 1_{Z} \circ f\) and \(g = 1_{Z} \circ g\).

      Now consider the category \(\mathsf{C}^{\alpha, \beta}\) where \(\alpha: C \to A\) and \(\beta: C \to B\). Then we have
      \begin{itemize}
        \item \(\text{Obj}(\mathsf{C}^{\alpha, \beta}) = \) commutative diagrams
        \[
        \begin{tikzcd}
	        && {A} \\
	        {C} &&&& {Z} \\
	        && {B}
	        \arrow["{\alpha}", from=2-1, to=1-3]
	        \arrow["{\beta}"', from=2-1, to=3-3]
	        \arrow["{f}", from=1-3, to=2-5]
	        \arrow["{g}"', from=3-3, to=2-5]
        \end{tikzcd}
        \]
        where \(Z\) is an object in \(\mathsf{C}\)
        \item Morphisms between objects \((Z_{1}, f_{1}, g_{1})\) and \((Z_{2}, f_{2}, g_{2})\) are commutative diagrams
        \[
        \begin{tikzcd}
	        && {A} \\
	        {C} &&&& {Z_{1}} && {Z_{2}} \\
	        && {B}
	        \arrow["{\alpha}", from=2-1, to=1-3]
	        \arrow["{\beta}"', from=2-1, to=3-3]
	        \arrow["{f_{1}}", from=1-3, to=2-5]
	        \arrow["{g_{1}}"', from=3-3, to=2-5]
	        \arrow["{f_{2}}", from=1-3, to=2-7, bend left=15]
	        \arrow["{g_{2}}"', from=3-3, to=2-7, bend right=15]
	        \arrow["{\sigma}", from=2-5, to=2-7]
        \end{tikzcd}
        \]
        That is, we have a morphism \(\sigma \in \text{Hom}_{\mathsf{C}}(Z_{1}, Z_{2})\) such that the diagram commutes.
      \end{itemize}
      Composition again has a natural definition. Given a third object \((Z_{3}, f_{3}, g_{3})\) and a morphism \(\tau: Z_{2} \to Z_{3}\),
      we can define a morphism \(\tau \circ \sigma: Z_{1} \to Z_{3}\) such that the corresponding diagram commutes.
      Finally, given an object \((Z, f, g)\) we inherit the identity morphism \(1_{Z}\) from \(\mathsf{C}\).
      Certainly the corresponding diagram commutes.
    \end{solution}
\end{document}
