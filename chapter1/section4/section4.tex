\documentclass[../../master.tex]{subfiles}

\begin{document}
\section{Morphisms}

    % Problem 4.1
    \begin{problem}
      Composition is defined for \textit{two} morphisms.
      If more than two morphisms are given, e.g.,
      \[
      \begin{tikzcd}
        {A} & {B} & {C} & {D} & {E}
        \arrow["{f}", from=1-1, to=1-2]
        \arrow["{g}", from=1-2, to=1-3]
        \arrow["{h}", from=1-3, to=1-4]
        \arrow["{i}", from=1-4, to=1-5]
      \end{tikzcd}
      \]
      then one may compose them in several ways, for example:
      \begin{equation*}
        (ih)(gf), \quad (i(hg))f, \quad i((hg)f), \quad \text{etc.}
      \end{equation*}
      so that at every step one is only composing two morphisms.
      Prove that the result of any such nested composition is independent of the placement of the parentheses.
      (Hint: Use induction on \(n\) to show that any such choice for \(f_{n} f_{n-1} \cdots f_{1}\) equals
      \begin{equation*}
        ((\cdots ((f_{n} f_{n-1}) f_{n-2}) \cdots) f_{1}).
      \end{equation*}
      Carefully working out the case \(n=5\) is helpful.)
    \end{problem}

    \begin{solution}
      For \(n = 3\), we have \((fg)h = f(gh)\) by the associativity of composition in a category.
      Suppose \(n \geq 4\) and that for \(n - 1\) morphisms we have shown that composition is independent of the placement of the parentheses.
      Let \(f_{1}, \ldots, f_{n}\) be morphisms in a category:
      \[
      \begin{tikzcd}
        {Z_{1}} & {Z_{2}} & {\cdots} & {Z_{n}} & {Z_{n+1}}
        \arrow["{f_{1}}", from=1-1, to=1-2]
        \arrow["{f_{2}}", from=1-2, to=1-3]
        \arrow["{f_{n-1}}", from=1-3, to=1-4]
        \arrow["{f_{n}}", from=1-4, to=1-5]
      \end{tikzcd}
      \]
      Suppose that a parenthesization of \(f_{n}, f_{n-1}, \ldots, f_{1}\) is \(f\) and that
      \(f = hg\) where \(h\) is some parenthesization of \(f_{n}, f_{n-1}, \ldots, f_{i+1}\),
      and \(g\) is some parenthesization of \(f_{i}, f_{i-1}, \ldots, f_{1}\), where \(1 \leq i \leq n\).
      Applying the inductive to \(h\) and \(g\), we see that
      \begin{gather*}
        h = ((\cdots ((f_{n}f_{n-1}) f_{n-2}) \cdots) f_{i+1}) \\
        g = (f_{i} (f_{i-1} (\cdots (f_{2} f_{1}) \cdots)) = f_{i} g'
      \end{gather*}
      hence \(f = hg = h (f_{i} g') = (h f_{i}) g'\).
      Effectively, we remove morphisms \(f_{i}\) from the left side of \(g'\) and attach them to the right side of \(h\) to obtain the form
      \begin{equation*}
        f = ((\cdots ((f_{n} f_{n-1}) f_{n-2}) \cdots) f_{1})
      \end{equation*}
    \end{solution}

    % Problem 4.2
    \begin{problem}
      In Example 3.3 we have seen how to construct a category from a set endowed with a relation,
      provided this latter is reflexive and transitive.
      For what types of relations is the corresponding category a groupoid (cf. Example 4.6)?
    \end{problem}

    \begin{solution}
      Recall that a groupoid is a category in which every morphism is an isomorphism and hence has a two-sided inverse.
      The corresponding category is a groupoid when the relation is also symmetric and hence an equivalence relation.
      Indeed, if \((x, y) \in \text{Hom}(x, y)\) then \(x \sim y\).
      If \(\sim\) is reflexive then this implies that \(y \sim x\) so \((y, x) \in \text{Hom}(y, x)\).
      Then \((x, y) (y, x) = (x, x)\) and \((y, x) (x, y) = (y, y)\), both of which are the identity morphisms of their respective objects.
      Thus, \((x, y)\) is an isomorphism and the category is a groupoid.
    \end{solution}

    % Problem 4.3
    \begin{problem}
      Let \(A, B\) be objects of a category \(\mathsf{C}\), and let \(f \in \text{Hom}_{\mathsf{C}}(A, B)\) be a morphism.
      \begin{itemize}
        \item Prove that if \(f\) has a right-inverse, then \(f\) is an epimorphism.
        \item Show that the converse does not hold, by giving an explicity example of a category and an epimorphism without a right-inverse.
      \end{itemize}
    \end{problem}

    \begin{solution}
      Suppose \(f\) has a right-inverse. That is, there exists a morphism \(g \in \text{Hom}_{\mathsf{C}}(B, A)\) such that
      \(f \circ g = 1_{B}\).
      Then if we consider two morphisms \(\beta, \beta' \in \text{Hom}_{\mathsf{C}}(B, Z)\) such that \(\beta \circ f = \beta' \circ f\) we have
      \begin{align*}
        (&\beta \circ f) \circ g = (\beta' \circ f) \circ g \\
        \Longrightarrow &\beta \circ (f \circ g) = \beta' \circ (f \circ g) \\
        \Longrightarrow &\beta \circ 1_{B} = \beta' \circ 1_{B} \\
        \Longrightarrow &\beta = \beta'
      \end{align*}
      Thus, \(f\) is an epimorphism.

      However, consider the category \(\mathsf{C}\) where
      \begin{itemize}
        \item \(\text{Obj}(\mathsf{C}) = \mathbb{Z}\)
        \item For objects \(a, b \in \mathbb{Z}\) we have
        \(\text{Hom}_{\mathsf{C}}(a, b) = \{(a, b)\}\) if \(a \leq b\) and \(\emptyset\) otherwise.
      \end{itemize}
      The reflexivity and transitivity of \(\leq\) makes this a category.
      Given morphisms \(f \in \text{Hom}_{\mathsf{C}}(a, b)\) and \(g \in \text{Hom}_{\mathsf{C}}(b, c)\) we define composition as
      \(g \circ f = (b, c) \circ (a, b) = (a, c) \in \text{Hom}_{\mathsf{C}}(a, c)\).
      Consider two objects \(a, b \in \mathbb{Z}\) such that \(a < b\) and let \(f: a \to b = (a, b)\) be the morphism from \(a\) to \(b\).
      Consider two morphisms \(\beta, \beta' \in \text{Hom}_{\mathsf{C}}(b, c)\) such that \(\beta \circ f = \beta' \circ f\).
      Then we have \(\beta = \beta'\) since each Hom set has at most one morphism.
      Thus \(f\) is an epimorphism. However, it does not have a right-inverse.
      Indeed, suppose \(\text{Hom}_{\mathsf{C}}(b, a)\) is nonempty.
      Then it can only contain \((b, a)\) which would imply that \(b \leq a\), a contradiction since we assumed \(a < b\).
      Thus, we have a category where epimorphisms do not necessarily have right-inverses.
    \end{solution}

    % Problem 4.4
    \begin{problem}
      Prove that the composition of two monomorphims is a monomorphism.
      Deduce that one can define a subcategory \(\mathsf{C}_{\text{mono}}\) of a category \(\mathsf{C}\)
      by taking the same objects as in \(\mathsf{C}\) and defining \(\text{Hom}_{\mathsf{C}_{\text{mono}}}(A, B)\)
      to be subset of \(\text{Hom}_{\mathsf{C}}(A, B)\) consisting of monomorphisms, for all objects \(A, B\).
      (Cf. Exercise 3.8; of course, in general \(\mathsf{C}_{\text{mono}}\) is not full in \(\mathsf{C}\).)
      Do the same for epimorphisms.
      Can you define a subcategory \(\mathsf{C}_{\text{nonmono}}\) of \(\mathsf{C}\) by restricting to morphisms that are \textit{not} monomorphisms?
    \end{problem}

    \begin{solution}
      Suppose that \(f \in \text{Hom}_{\mathsf{C}}(A, B)\) and \(g \in \text{Hom}_{\mathsf{C}}(B, C)\) are two monomorphisms.
      Let \(\alpha, \alpha' \in \text{Hom}_{\mathsf{C}}(Z, A)\) be two morphisms such that \((g \circ f) \circ \alpha = (g \circ f) \circ \alpha'\).
      Then we have
      \begin{align*}
        (&g \circ f) \circ \alpha = (g \circ f) \circ \alpha' \\
        \Longrightarrow &g \circ (f \circ \alpha) = g \circ (f \circ \alpha') && \text{by the associativity of composition} \\
        \Longrightarrow &f \circ \alpha = f \circ \alpha' && \text{since \(g\) is a monomorphism} \\
        \Longrightarrow &\alpha = \alpha' && \text{since \(f\) is a monomorphism}
      \end{align*}
      Hence, \(g \circ f\) is a monomorphism.
      Therefore, the subcategory \(\mathsf{C}_{\text{mono}}\) is closed with respect to composition.

      We use a similar proof to show that the composition of two epimorphisms is an epimorphism.
      Suppose that \(f \in \text{Hom}_{\mathsf{C}}(A, B)\) and \(g \in \text{Hom}_{\mathsf{C}}(B, C)\) are epimorphisms.
      Let \(\beta, \beta' \in \text{Hom}_{\mathsf{C}}(C, Z)\) be two morphisms such that \(\beta \circ (g \circ f) = \beta' \circ (g \circ f)\).
      Then we have
      \begin{align*}
        &(\beta \circ g) \circ f = (\beta' \circ g) \circ f && \text{by the associativity of composition} \\
        \Longrightarrow &\beta \circ g = \beta' \circ g && \text{since \(f\) is an epimorphism} \\
        \Longrightarrow &\beta = \beta' && \text{since \(g\) is an epimorphism}
      \end{align*}
      Thus, \(g \circ f\) is an epimorphism so we can define a similar subcategory \(\mathsf{C}_{\text{epi}}\) which is closed with respect to compositon.

      We can also define a category \(\mathsf{C}_{\text{nonmono}}\) whose morphisms are restricted to those of \(\mathsf{C}\) which are not monomorphisms.
      Indeed, suppose \(f \in \text{Hom}_{\mathsf{C}}(A, B)\) is not a monomorphism.
      That is, there exist morphisms \(\alpha, \alpha' \in \text{Hom}_{\mathsf{C}}(Z, A)\) such that \(f \circ \alpha = f \circ \alpha'\) but \(\alpha \neq \alpha'\).
      Let \(g \in \text{Hom}_{\mathsf{C}}(B, C)\) be a non-monomorphism.
      Then we have \((g \circ f) \circ \alpha = (g \circ f) \circ \alpha'\) but \(\alpha \neq \alpha'\).
      Thus, \((g \circ f)\) is not a monomorphism so the category \(\mathsf{C}_{\text{nonmono}}\) is closed under composition.
      Interestingly, this only relies on the fact that \(f\) is not a monomorphism.
    \end{solution}

    % Problem 4.5
    \begin{problem}
      Give a concrete description of monomorphims and epimorphisms in the category \(\mathsf{MSet}\) you constructed in Exercise 3.9.
      (Your answer will depend on the notion of morphism you defined in that exercise!)
    \end{problem}

    \begin{solution}
      Recall that we defined multisets to be sets equipped with equivalence relations.
      A morphism between two multisets is a set function which preserves the equivalence relation.
      The notions of monomorphism and epimorphism are naturally inherited from \(\mathsf{Set}\).
      \begin{itemize}
        \item A morphism \(f \in \text{Hom}_{\mathsf{MSet}}(A, B)\) is a monomorphism if for all \(a_{1}, a_{2} \in A\) we have
        \(f(a_{1}) \sim_{B} f(a_{2}) \Longrightarrow a_{1} \sim_{A} a_{2}\). We call these morphisms \textit{injective}.
        \item A morphism \(f \in \text{Hom}_{\mathsf{MSet}}(A, B)\) is an epimorphism if for all \(b \in B\) there exists an \(a \in A\) such that \(f(a) = b\). We call these morphisms \textit{surjective}.
      \end{itemize}
      We will prove that these definitions satisfy the category theoretical definitions of monomorphisms and epimorphisms.
      We start by proving an analogue of Proposition 2.1 in \(\mathsf{MSet}\).
      \begin{proposition}[Lemma]
        Assume \(A \neq \emptyset\) and let \(f: A \to B\) be a morphism of multisets. Then
        \begin{enumerate}
          \item \(f\) has a left-inverse if and only if it is injective.
          \item \(f\) has a right-inverse if and only if it is surjective.
        \end{enumerate}
      \end{proposition}
      \begin{proof}
        First we prove (1). If \(f\) has a left-inverse, then there exists a morphism \(g \in \text{Hom}_{\mathsf{MSet}}(B, A)\) such that \(g \circ f = 1_{A}\).
        Let \(a_{1} \nsim_{A} a_{2}\) be elements in \(A\) not eqivalent under the relation.
        Then
        \begin{equation*}
          g \circ f(a_{1}) = 1_{A}(a_{1}) = a_{1} \nsim_{A} a_{2} = 1_{A}(a_{2}) = g \circ f(a_{2})
        \end{equation*}
        That is, \(a_{1} \nsim_{A} a_{2} \Longrightarrow f(a_{1}) \nsim_{B} f(a_{2})\) which is the contrapositive of the definition for an injective morphism.
        Thus, if \(f\) has a left-inverse it must be injective.

        Now suppose \(f: A \to B\) is injective. We will construct a left-inverse \(g: B \to A\).
        Choose one fixed element \(s \in A\).
        Now set
        \begin{equation*}
          g(b) =
          \begin{cases}
            a \text{ if } b = f(a) \text{ for some } a \in A, \\
            s \text{ if } b \notin \text{im} f
          \end{cases}
        \end{equation*}
        This definition guarantees that every \(b\) that is in the image of \(f\) maps to a unique element since \(f\) is injective.
        We can verify that \(g\) is a left-inverse of \(f\).
        If \(a \in A\), then \(g \circ f(a) = a = 1_{A}(a)\).

        A highly similar proof follows for (2). If \(f: A \to B\) has a right-inverse, then there exists a morphism \(g: B \to A\) such that \(f \circ g = 1_{B}\).
        Let \(b \in B\). Then \(g(b) \in A\) and \(f \circ g(b) = b\) for all such \(b\).
        Thus \(f\) is surjective.

        For the reverse direction, suppose that \(f: A \to B\) is surjective. We will construct a right-inverse \(g: B \to A\).
        Let \(S = \{(a, b) \mid f(a) = b\}\).
        Certainly \(S\) contains elements for each \(b \in B\) since \(f\) is surjective.
        Then define \(g: B \to A\), \(g(b) = a\) where \(a\) is the least element such that \((a, b) \in S\).
        This definition guarantees that every element of \(b\) is mapped to only one element since there may be several \(a\) which are mapped to \(b\).
        We can verify that \(g\) is a right-inverse of \(f\).
        Let \(b \in B\). Then \(f \circ g(b) = b = 1_{B}(b)\).
      \end{proof}
      With this lemma, we show that our definition of injective and surjective morphisms is precisely equivalent to monomorphisms and epimorphisms in the category \(\mathsf{MSet}\).

      First suppose that \(f: A \to B\) is injective. Then it has a left-inverse \(g: B \to A\).
      Let \(\alpha, \alpha' \in \text{Hom}_{\mathsf{MSet}}(Z, A)\) be morphisms such that \(f \circ \alpha = f \circ \alpha'\).
      Then we find
      \begin{align*}
        (&g \circ f) \circ \alpha = (g \circ f) \circ \alpha' && \text{by associativity of composition} \\
        \Longrightarrow & 1_{A} \circ \alpha = 1_{A} \circ \alpha' && \text{since \(g\) is a left-inverse of \(f\)} \\
        \Longrightarrow & \alpha = \alpha'
      \end{align*}
      Thus, \(f\) is a monomorphism in the category theoretical sense.

      Now suppose that \(f: A \to B\) is a monomorphism. We will show it is injective.
      Consider the set \(Z = \{p\}\) and let \(\alpha, \alpha' \in \text{Hom}_{\mathsf{MSet}}(Z, A)\) be morphisms such that \(f \circ \alpha = f \circ \alpha'\).
      Since \(f\) is a monomorphism, this forces \(\alpha = \alpha'\).
      In turn, this means \(\alpha(p) \sim_{A} \alpha'(p)\).
      Letting \(a_{1} = \alpha(p)\) and \(a_{2} = \alpha'(p)\), we have
      \begin{equation*}
        f(a_{1}) \sim_{A} f(a_{2}) \Longrightarrow a_{1} \sim_{A} a_{2}
      \end{equation*}
      Thus, \(f\) is injective.
      A nearly identical proof follows for epimorphisms and surjective morphisms.
    \end{solution}
\end{document}
