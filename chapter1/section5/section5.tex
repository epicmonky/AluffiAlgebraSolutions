\documentclass[../../master.tex]{subfiles}

\begin{document}
\section{Universal Properties}

    % Problem 5.1
    \begin{problem}
      Prove that a final object in a category \(\mathsf{C}\) is initial in the opposite category \(\mathsf{C}^{op}\).
    \end{problem}

    \begin{solution}
      Let \(A\) be a final object in \(\mathsf{C}\).
      That is, for every object \(Z\) of \(\mathsf{C}\), there exists exactly one morphism \(f \in \text{Hom}_{\mathsf{C}}(Z, A)\).
      Recall that the opposite category \(\mathsf{C}^{op}\) is formed by `reversing' all arrows.
      More formally, we set \(\text{Hom}_{\mathsf{C}^{op}}(Z, B) = \text{Hom}_{\mathsf{C}}(B, Z)\).
      In particular, for every object \(Z\) of \(\mathsf{C}^{op}\), there exists exactly one morphism \(f \in \text{Hom}_{\mathsf{C}^{op}}(A, Z)\).
      Thus, \(A\) is initial in \(\mathsf{C}^{op}\).
    \end{solution}

    % Problem 5.2
    \begin{problem}
      Prove that \(\emptyset\) is the \textit{unique} initial object in \(\mathsf{Set}\).
    \end{problem}

    \begin{solution}
      Note that the empty set \(\emptyset\) is initial in \(\mathsf{Set}\) with the only morphism to other sets being the empty mapping.
      Now let \(I\) be any other initial object in \(\mathsf{Set}\).
      Then \(I \cong \emptyset\).
      Recall that isomorphic sets are those which have the same order (so that a bijection exists between them).
      Thus, \(|I| = |\emptyset| = 0\) and \(I\) is necessarily the empty set \(\emptyset\) since it is the only set with no elements.
    \end{solution}

    % Problem 5.3
    \begin{problem}
      Prove that final objects are unique up to isomorphism.
    \end{problem}

    \begin{solution}
      First note that if \(F\) is a final object in a category \(\mathsf{C}\), then there is a unique morphism \(F \to F\), namely the identity \(1_{F}\).
      Now assume \(F_{1}\) and \(F_{2}\) are both final in \(\mathsf{C}\).
      Since \(F_{2}\) is final, there is a unique morphism \(f: F_{1} \to F_{2}\).
      We will show that \(f\) is an isomorphism.
      Since \(F_{1}\) is final, there is a unique morphism \(g: F_{2} \to F_{1}\).
      Consider the composition \(g \circ f: F_{1} \to F_{1}\).
      As noted earlier, this is necessarily the identity morphism \(1_{F_{1}}\).
      Similarly, \(f \circ g: F_{2} \to F_{2}\) is necessarily the identity morphism \(1_{F_{2}}\).
      Thus, \(f\) is an isomorphism and \(F_{1} \cong F_{2}\).
    \end{solution}

    % Problem 5.4
    \begin{problem}
      What are initial and final objects in the category of `pointed sets'?
      Are they unique?
    \end{problem}

    \begin{solution}
      Recall that the category of pointed sets \(\mathsf{Set}^{*}\) is defined as follows:
      \begin{itemize}
        \item \(\text{Obj}(\mathsf{Set}^{*}) =\) morphisms \(f: \{*\} \to S\) in \(\mathsf{Set}\) where \(S\) is any set.
        Note that objects may be denoted as pairs \((S, s)\) where \(S\) is the set the morphism maps to and \(s\) is the element that \(f\) sends \(*\) to.
        \item Given two objects \((S, s)\) and \((T, t)\), a morphism \(f: (S, s) \to (T, t)\) corresponds to a set-function \(\sigma: S \to T\) such that \(\sigma(s) = t\).
      \end{itemize}
      Then the pointed singleton sets \((\{s\}, s)\) are the initial and final objects of \(\mathsf{Set}^{*}\).
      Indeed, let \((T, t)\) be any object in \(\mathsf{Set}^{*}\).
      Then there is only one morphism \(\sigma: S \to T\) such that \(\sigma(s) = t\).
      Similarly, there is only one morphism \(\sigma': T \to S\) such that \(\sigma(t) = s\).
      Thus, pointed singleton sets are both initial and final.
      They are also clearly not unique as both \((\{a\}, a)\) and \((\{b\}, b)\) where \(a \neq b\) are distinct pointed singleton sets.
    \end{solution}

    % Problem 5.5
    \begin{problem}
      What are the final objects in the category considered in \S 5.3?
    \end{problem}

    \begin{solution}
      The category considered in \S 5.3 is defined as follows:
      Let \(\sim\) be an equivalence relation defined on a set \(A\).
      Consider the category \(\mathsf{C}_{A}\) where
      \begin{itemize}
        \item \(\text{Obj}(\mathsf{C}_{A}) =\) morphisms \(\varphi: A \to Z\) where \(Z\) is an arbitrary set such that \(a \sim a' \Longrightarrow \varphi(a) = \varphi(a')\).
        Objects are frequently denoted \((\varphi, Z)\).
        \item Morphisms \((\varphi_{1}, Z_{1}) \to (\varphi_{2}, Z_{2})\) are commutative diagrams
        \[
        \begin{tikzcd}
          {Z_{1}} && {Z_{2}} \\
          & {A}
          \arrow["{\varphi_{1}}", from=2-2, to=1-1]
          \arrow["{\varphi_{2}}"', from=2-2, to=1-3]
          \arrow["{\sigma}", from=1-1, to=1-3]
        \end{tikzcd}
        \]
      \end{itemize}
      Then the objects \((\varphi^{*}, \{*\})\) are final in this category, where \(\varphi^{*}\) is the morphism mapping every element of \(A\) to \(*\).
      To verify, let \((\varphi, Z)\) be an object.
      Then there exists a unique morphism \(\sigma: Z \to \{*\}\), namely the one mapping every element of \(Z\) to \(*\).
      Certainly this morphism makes the diagram commute, and since it exists for all objects, \(\varphi^{*}, \{*\}\) is final.
    \end{solution}

    % Problem 5.6
    \begin{problem}
      Consider the category corresponding to endowing (as in Example 3.3) the set \(\mathbb{Z}^{+}\) of positive integers with the \textit{divisibility} relation.
      Thus there is exactly one morphism \(d \to m\) in this category if and only if \(d\) divides \(m\) without remainder;
      there is no morphism between \(d\) and \(m\) otherwise.
      Show that this category has products and coproducts.
      What are their conventional names?
    \end{problem}

    \begin{solution}
      Given two positive integers \(m, n\), their categorical product \(m \times n\) is the positive integer such that, given any positive integer \(z\), the diagram
      \[
      \begin{tikzcd}
        &&&& {m} \\
        {z} && {m \times n} \\
        &&&& {n}
        \arrow["{\sigma}"', from=2-1, to=2-3]
        \arrow["{\pi_{m}}"', from=2-3, to=1-5]
        \arrow["{\pi_{n}}"', from=2-3, to=3-5]
        \arrow["{f_{m}}", from=2-1, to=1-5, bend left=15]
        \arrow["{f_{n}}", from=2-1, to=3-5, bend right=15]
      \end{tikzcd}
      \]
      commutes.

      Note that the existence of projections \(\pi_{m}, \pi_{n}\) implies \(m \times n\) divides \(m\) and \(m \times n\) divides \(n\).
      Thus, we have \(m \times n\) divides \(\text{gcd}(m, n)\).
      Furthermore, consider \(z = \text{gcd}(m, n)\).
      Certainly there exist morphisms \(f_{m}: z \to m\) and \(f_{n}: z \to n\).
      Then by the definition of categorical products, there exists a unique morphism \(\sigma: z \to m \times n\).
      That is, we have \(\text{gcd}(m, n)\) divides \(m \times n\).
      Combined with the earlier observation, we find \(m \times n = \text{gcd}(m, n)\).

      Now let us consider the categorical coproduct \(m \coprod n\).
      This is a positive integer such that, given any positive integer \(z\), the diagram
      \[
      \begin{tikzcd}
        {m} \\
        && {m \coprod n} && {z} \\
        {n}
        \arrow["{i_{m}}", from=1-1, to=2-3]
        \arrow["{i_{n}}"', from=3-1, to=2-3]
        \arrow["{\sigma}", from=2-3, to=2-5]
        \arrow["{f_{m}}", from=1-1, to=2-5, bend left=15]
        \arrow["{f_{n}}"', from=3-1, to=2-5, bend right=15]
      \end{tikzcd}
      \]
      commutes.

      The existence of the inclusion morphisms imply that both \(m\) and \(n\) divide \(m \coprod n\),
      so \(\text{lcm}(m, n)\) divides \(m \coprod n\).
      Furthermore, take \(z\) to be \(\text{lcm}(m, n)\).
      Then there certainly exist morphisms \(f_{m}: m \to z\) and \(f_{n}: n \to z\).
      By the definition of the categorical coproduct, there exists a unique morphism \(\sigma: m \coprod n \to z\), so \(m \coprod n\) divides \(\text{lcm}(m, n)\).
      Thus, we have \(m \coprod n = \text{lcm}(m, n)\).
    \end{solution}

    % Problem 5.7
    \begin{problem}
      Redo Exercise 2.9, this time using Proposition 5.4.
    \end{problem}

    \begin{solution}
      Exercise 2.9 asks that we show if \(A' \cong A''\) and \(B' \cong B''\),
      and further \(A' \cap B' = A'' \cap B'' = \emptyset\),
      then \(A' \cup B' \cong A'' \cup B''\).
      We can conclude that \(A \coprod B\) is well-defined up to isomorphism.

      First consider \(i_{A'}: A' \to A' \cup B', i_{A'}(a) = a\) for all \(a \in A'\).
      Define a similar function \(i_{B'}\).
      If \(Z\) is a set with morphisms \(f_{A'}: A' \to Z\) and \(f_{B'}: B' \to Z\),
      we have a unique morphism \(\sigma: A' \coprod B' = A' \cup B' \to Z\) where
      \begin{equation*}
        \sigma(x) =
        \begin{cases}
          f_{A'}(x) \text{ if } x \in A' \\
          f_{B'}(x) \text{ if } x \in B'
        \end{cases}
      \end{equation*}
      This shows that the disjoint union is a coproduct.

      We define entirely analagous morphisms for \(A''\) and \(B''\).
      Then we have a second coproduct \(A'' \coprod B'' = A'' \cup B''\).

      Proposition 5.4 states that in any category \(\mathsf{C}\), two initial objects \(I_{1}\) and \(I_{2}\) are isomorphic.
      Note that the coproducts \(A' \coprod B'\) and \(A'' \coprod B''\) we have defined are initial in the category \(\mathsf{Set}_{A, B}\).
      Thus, they are isomorphic.
    \end{solution}

    % Problem 5.8
    \begin{problem}
      Show that in every category \(\mathsf{C}\) the products \(A \times B\) and \(B \times A\) are isomorphic, if they exist.
      (Hint: Observe that they both satisfy the universal property for the product of \(A\) and \(B\);
      then use Proposition 5.4.)
    \end{problem}

    \begin{solution}
      Let \(A \times B\) and \(B \times A\) be products in a category \(\mathsf{C}\).
      Certainly \(A \times B\) satisfies the universal property for products.
      That is, given an object \(Z\) and morphisms \(f_{A}: Z \to A\) and \(f_{B}: Z \to B\), we can construct a unique morphism
      \(\sigma: Z \to A \times B\).

      Now consider the morphism \(\tau: A \times B \to B \times A, \tau(a, b) = (b, a)\).
      Certainly this morphism is an isomorphism since it has an inverse \(\tau^{-1}(b, a) = (a, b)\).
      Then for any object \(Z\) and morphisms \(f_{A}, f_{B}\) as defined above, we consider the morphism
      \(\varphi: Z \to B \times A, \varphi = \tau \circ \sigma\).
      It is unique since it is determined by the product \(A \times B\).
      Therefore, \(B \times A\) also satisfies the universal property for the product of \(A\) and \(B\).
      By Proposition 5.4, the two objects are isomorphic.
      Admittedly, we already observed that an isomorphism exists between the two objects.
    \end{solution}

    % Problem 5.9
    \begin{problem}
      Let \(\mathsf{C}\) be a category with products.
      Find a reasonable candidate for the universal property that the product \(A \times B \times C\) of \textit{three} objects of \(\mathsf{C}\) ought to satisfy,
      and prove that both \((A \times B) \times C\) and \(A \times (B \times C)\) satisfy this universal property.
      Deduce that \((A \times B) \times C\) and \(A \times (B \times C)\) are necessarily isomorphic.
    \end{problem}

    \begin{solution}
      Given three objects \(A, B, C\) of a category \(\mathsf{C}\),
      we can consider the product \(A \times B \times C\) with three natural projections \(\pi_{A}, \pi_{B}, \pi_{C}\).
      The reasonable definition of the universal property is as follows:
      For every object \(Z\) and morphisms \(f_{A}: Z \to A\),  \(f_{B}: Z \to B\), and \(f_{C}: Z \to C\),
      there exists a unique morphism \(\sigma: Z \to A \times B \times C\) such that the diagram
      \[
      \begin{tikzcd}
        &&&& {A} \\
        {Z} && {A \times B \times C} \\
        &&&& {B} \\
        &&&&& {C}
        \arrow["{\sigma}", from=2-1, to=2-3]
        \arrow["{\pi_{A}}", from=2-3, to=1-5]
        \arrow["{\pi_{B}}", from=2-3, to=3-5]
        \arrow["{\pi_{C}}", from=2-3, to=4-6, bend left=30]
	      \arrow["{f_{A}}", from=2-1, to=1-5, bend left=18]
	      \arrow["{f_{B}}"', from=2-1, to=3-5, bend right=6]
	      \arrow["{f_{A}}"', from=2-1, to=4-6, bend right=18]
      \end{tikzcd}
      \]
      commutes.

      First we will show that \((A \times B) \times C\) satisfies this universal property.
      For every object \(Z\), we have a unique morphism \(\tau: Z \to A \times B, \tau(z) = (f_{A}(z), f_{B}(z))\).
      Now we define \(\sigma: Z \to (A \times B) \times C\),
      \begin{equation*}
        \sigma(z) = (\tau(z), f_{C}(z)) = ((f_{A}(z), f_{B}(z)), f_{C}(z))
      \end{equation*}
      We define a natural projection \(\pi_{A}':(A \times B) \times C \to A, \pi_{A}' = \pi_{A} \circ \pi_{A \times B}\)
      along with an analogous projection \(\pi_{B}'\) and the typical \(\pi_{C}\).
      These morphisms make the diagram commute because for all \(z \in Z\) we have
      \begin{equation*}
        \pi_{A}' \circ \sigma(z) = \pi_{A} \circ \pi_{A \times B} ((f_{A}(z), f_{B}(z)), f_{C}(z)) = \pi_{A} (f_{A}(z), f_{B}(z)) = f_{A}(z)
      \end{equation*}
      and similarly for \(f_{B}\) and \(f_{C}\).
      Thus, \((A \times B) \times C\) satisfies the universal property for the product \(A \times B \times C\).

      An entirely analogous construction shows that \(A \times (B \times C)\) also satisfies this universal property.
      By Proposition 5.4, we must have \((A \times B) \times C \cong A \times (B \times C)\).
    \end{solution}

    % Problem 5.10
    \begin{problem}
      Push the envelope a little further still, and define products and coproducts for \textit{families} (i.e., indexed sets) of objects of a category.
      Do these exist in \(\mathsf{Set}\)?
      It is common to denote the product \(\underbrace{A \times \cdots \times A}_{n \text{ times}}\) by \(A^{n}\).
    \end{problem}

    \begin{solution}
      Given a family of objects \(\{A_{i}\}_{i \in I}\) for some set \(I\) in a category \(\mathsf{C}\),
      the product \(\Pi_{i \in I} A_{i}\) with natural projections \(\{\pi_{A_{i}}\}_{i \in I}\) should satisfy the universal property that
      for all objects \(Z\) and morphisms \(\{f_{A_{i}}\}_{i \in I}, f_{A_{i}}: Z \to A_{i}\), there exists a unique morphism
      \(\sigma: Z \to \Pi_{i \in I} A_{i}\) such that \(\pi_{A_{i}} \circ \sigma = f_{A_{i}}\) for all \(i \in I\).

      Similarly, the coproduct \(\coprod_{i \in I} A_{i}\) with natural inclusions \(\{i_{A_{i}}\}_{i \in I}\) should satisfy the following universal property:
      for all objects \(Z\) and morphisms \(\{f_{A_{i}}\}_{i \in I}, f_{A_{i}}: A_{i} \to Z\), there exists a unique morphism
      \(\sigma: \coprod_{i \in I} A_{i} \to Z\) such that \(\sigma \circ i_{A_{i}} = f_{A_{i}}\) for all \(i \in I\).

      The product for finite families of sets exists.
      However, we require the Axiom of Choice to ensure that the infinite product of nonempty sets is nonempty.
      The coproduct should exist for any family of sets since the family is indexed so we can just take the coproduct to be \(\bigcup \{i\} \times \{A_{i}\}\) but I'm not positive.
    \end{solution}

    % Problem 5.11
    \begin{problem}
      Let \(A\), resp. \(B\), be a set endowed with an equivalence relation \(\sim_{A}\), resp. \(\sim_{B}\).
      Define a relation \(\sim\) on \(A \times B\) by setting
      \begin{equation*}
        (a_{1}, b_{1}) \sim (a_{2}, b_{2}) \Longleftrightarrow a_{1} \sim_{A} a_{2} \text{ and } b_{1} \sim_{B} b_{2}.
      \end{equation*}
      (This is immediately seen to be an equivalence relation.)
      \begin{itemize}
        \item Use the universal property for quotients (\S 5.3) to establish that there are functions
        \((A \times B) / {\sim} \to A / {\sim_{A}}\), \((A \times B) / {\sim} \to B / {\sim_{B}}\).
        \item Prove that \((A \times B) / {\sim}\), with these two functions, satisfies the universal property for the product of
        \(A / {\sim_{A}}\) and \(B / {\sim_{B}}\).
        \item Conclude (without further work) that \((A \times B) / {\sim} \; \cong (A / {\sim_{A}}) \times (B / {\sim_{B}})\).
      \end{itemize}
    \end{problem}

    \begin{solution}
      Let \(\pi_{A}: A \times B \to A\) and \(\pi_{B}: A \times B \to B\) be the canonical projections for \(A\) and \(B\).
      Let \(\pi_{\sim}^{Z}: Z \to Z / {\sim}\) be the canonical quotient mapping for all objects \(Z\) and equivalence relations \(\sim\).
      Consider the morphism \(\varphi_{A}: A \times B \to A / {\sim_{A}},\)
      \begin{equation*}
        \varphi_{A} = \pi_{\sim_{A}}^{Z} \circ \pi_{A}
      \end{equation*}
      We then use the universal property of quotients to see that there exists a unique morphism
      \(\overline{\varphi}_{A}: (A \times B) / {\sim} \to A / {\sim_{A}}\).
      By analogous means, there exists a unique morphism \(\overline{\varphi}_{B}: (A \times B) / {\sim} \to B / {\sim_{B}}\).

      Now we will show that these morphisms act as natural projections from the product of \(A / {\sim_{A}}\) and \(B / {\sim_{B}}\).
      Let \(Z\) be a set with morphisms \(f_{A}: Z \to A / {\sim_{A}}\) and \(f_{B}: Z \to B / {\sim_{B}}\).
      Then there exists a unique morphism \(\sigma: Z \to (A \times B) / {\sim}\) such that the diagram
      \[
      \begin{tikzcd}
        &&&& {A / {\sim_{A}}} \\
        {Z} && {(A \times B) / {\sim}} \\
        &&&& {B / {\sim_{B}}}
        \arrow["{\sigma}", from=2-1, to=2-3]
        \arrow["{\overline{\varphi}_{A}}"', from=2-3, to=1-5]
        \arrow["{\overline{\varphi}_{B}}", from=2-3, to=3-5]
        \arrow["{f_{A}}", from=2-1, to=1-5, bend left=15]
        \arrow["{f_{B}}"', from=2-1, to=3-5, bend right=15]
      \end{tikzcd}
      \]
      commutes.
      Define a function \(\tau: Z \to A / {\sim_{A}} \times B / {\sim_{B}}, \tau(z) = (f_{A}(z), f_{B}(z))\).
      Note that by the universal property of the quotient there exists a unique function \(\overline{1}_{A}: A / {\sim_{A}} \to A\), \(\overline{1}_{A}([a]_{\sim_{A}}) = a\).
      We define a similar function \(\overline{1}_{B}\).
      Then we construct a morphism \(\overline{1}_{A \times B}: A / {\sim_{A}} \times B / {\sim_{B}} \to A \times B\),
      \begin{equation*}
        \overline{1}_{A \times B}([a]_{\sim_{A}}, [b]_{\sim_{B}}) = (\overline{1}_{A}([a]_{\sim_{A}}), \overline{1}_{B}([b]_{\sim_{B}}))
      \end{equation*}
      We now finally define \(\sigma = \pi_{\sim}^{A \times B} \circ \overline{1}_{A \times B} \circ \tau\).
      Then we have
      \begin{align*}
        \overline{\varphi}_{A} \circ \sigma(z) &= \overline{\varphi}_{A} \circ \pi_{\sim}^{A \times B} (\overline{1}_{A \times B}(f_{A}(z), f_{B}(z))) \\
        &= \overline{\varphi}_{A} (f_{A}(z), f_{B}(z)) \\
        &= f_{A}(z)
      \end{align*}
      Similarly, \(\overline{\varphi}_{B} \circ \sigma(z) = f_{B}(z)\).
      Thus, \((A \times B) / {\sim}\) satisfies the universal property for the product of \(A / {\sim_{A}}\) and \(B / {\sim_{B}}\).
      Therefore, \((A \times B) / {\sim} \cong A / {\sim_{A}} \times B / {\sim_{B}}\).
    \end{solution}

    % Problem 5.12
    \begin{problem}
      Define the notions of \textit{fibered products} and \textit{fibered coproducts},
      as terminal objects of the categories \(\mathsf{C}_{\alpha, \beta}\), \(\mathsf{C}^{\alpha, \beta}\) considered in Example 3.10
      (cf. also Exercise 3.11),
      by stating carefullly the corresponding universal properties.
      As it happens, \(\mathsf{Set}\) has both fibered products and coproducts.
      Define these objects `concretely', in terms of naive set theory.
    \end{problem}

    \begin{solution}
      Recall that given two morphisms \(\alpha: A \to C\) and \(\beta: B \to C\),
      the category \(\mathsf{C}_{\alpha, \beta}\) is defined as follows:
      \begin{itemize}
        \item \(\text{Obj}(\mathsf{C}_{\alpha, \beta}) = \) commutative diagrams
        \[
        \begin{tikzcd}
	        && {A} \\
	        {Z} &&&& {C} \\
	        && {B}
	        \arrow["{f}", from=2-1, to=1-3]
	        \arrow["{g}"', from=2-1, to=3-3]
	        \arrow["{\alpha}", from=1-3, to=2-5]
	        \arrow["{\beta}"', from=3-3, to=2-5]
        \end{tikzcd}
        \]
        where \(Z\) is an object in \(\mathsf{C}\)
        \item Morphisms between objects \((Z_{1}, f_{1}, g_{1})\) and \((Z_{2}, f_{2}, g_{2})\) are commutative diagrams
        \[
        \begin{tikzcd}
	        &&&& {A} \\
	        {Z_{1}} && {Z_{2}} &&&& {C} \\
	        &&&& {B}
	        \arrow["{\alpha}", from=1-5, to=2-7]
	        \arrow["{\beta}"', from=3-5, to=2-7]
	        \arrow["{f_{1}}", from=2-1, to=1-5, bend left=15]
	        \arrow["{g_{1}}"', from=2-1, to=3-5, bend right=15]
	        \arrow["{f_{2}}", from=2-3, to=1-5]
	        \arrow["{g_{2}}"', from=2-3, to=3-5]
	        \arrow["{\sigma}", from=2-1, to=2-3]
        \end{tikzcd}
        \]
        That is, we have a morphism \(\sigma \in \text{Hom}_{\mathsf{C}}(Z_{1}, Z_{2})\) such that the diagram commutes.
      \end{itemize}
      The fibered product \(A \times_{C} B\) is a final object in this category.
      In other words, for every object \(Z\) with morphisms \(f: Z \to A\) and  \(g: Z \to B\) where \(\alpha \circ f = \beta \circ g\),
      there exists a unique morphism \(\sigma: Z \to A \times_{C} B\) such that the diagram
      \[
      \begin{tikzcd}
        &&&& {A} \\
        {Z} && {A \times_{C} B} &&&& {C} \\
        &&&& {B}
        \arrow["{\alpha}", from=1-5, to=2-7]
        \arrow["{\beta}"', from=3-5, to=2-7]
        \arrow["{f}", from=2-1, to=1-5, bend left=15]
        \arrow["{g}"', from=2-1, to=3-5, bend right=15]
        \arrow["{\pi_{A}}", from=2-3, to=1-5]
        \arrow["{\pi_{B}}"', from=2-3, to=3-5]
        \arrow["{\sigma}", from=2-1, to=2-3]
      \end{tikzcd}
      \]
      commutes.

      We claim that the fibered product in \(\mathsf{Set}\) is defined as
      \begin{equation*}
        A \times_{C} B = \{(a, b) \mid \alpha(a) = \beta(b)\}
      \end{equation*}
      with the natural projections \(\pi_{A}\) and \(\pi_{B}\).
      Let \(Z\) be an arbitrary object with appropriate morphisms \(f\) and \(g\).
      Define \(\sigma: Z \to A \times_{C} B\) as \(\sigma(z) = (f_{A}(z), f_{B}(z))\).
      Then we have \(\pi_{A} \circ \sigma = f\) and \(\pi_{B} \circ \sigma = g\).
      Combined with the condition that \(\alpha \circ f = \beta \circ g\),
      it becomes clear that these definitions make the diagram commute.

      We define the fibered coproduct analogously.
      Recall that given morphisms \(\alpha: C \to A\) and \(\beta: C \to B\),
      the category \(\mathsf{C}^{\alpha, \beta}\) is defined as:
      \begin{itemize}
        \item \(\text{Obj}(\mathsf{C}^{\alpha, \beta}) = \) commutative diagrams
        \[
        \begin{tikzcd}
	        && {A} \\
	        {C} &&&& {Z} \\
	        && {B}
	        \arrow["{\alpha}", from=2-1, to=1-3]
	        \arrow["{\beta}"', from=2-1, to=3-3]
	        \arrow["{f}", from=1-3, to=2-5]
	        \arrow["{g}"', from=3-3, to=2-5]
        \end{tikzcd}
        \]
        where \(Z\) is an object in \(\mathsf{C}\)
        \item Morphisms between objects \((Z_{1}, f_{1}, g_{1})\) and \((Z_{2}, f_{2}, g_{2})\) are commutative diagrams
        \[
        \begin{tikzcd}
	        && {A} \\
	        {C} &&&& {Z_{1}} && {Z_{2}} \\
	        && {B}
	        \arrow["{\alpha}", from=2-1, to=1-3]
	        \arrow["{\beta}"', from=2-1, to=3-3]
	        \arrow["{f_{1}}", from=1-3, to=2-5]
	        \arrow["{g_{1}}"', from=3-3, to=2-5]
	        \arrow["{f_{2}}", from=1-3, to=2-7, bend left=15]
	        \arrow["{g_{2}}"', from=3-3, to=2-7, bend right=15]
	        \arrow["{\sigma}", from=2-5, to=2-7]
        \end{tikzcd}
        \]
        That is, we have a morphism \(\sigma \in \text{Hom}_{\mathsf{C}}(Z_{1}, Z_{2})\) such that the diagram commutes.
      \end{itemize}

      The fibered coproduct \(A \coprod_{C} B\) is initial in this category.
      Thus, for every object \(Z\) with morphisms \(f: A \to Z\) and \(g: B \to Z\) where \(f \circ \alpha = g \circ \beta\), the diagram
      \[
      \begin{tikzcd}
        && {A} \\
        {C} &&&& {A \coprod_{C} B} && {Z} \\
        && {B}
        \arrow["{\alpha}", from=2-1, to=1-3]
        \arrow["{\beta}"', from=2-1, to=3-3]
        \arrow["{i_{A}}", from=1-3, to=2-5]
        \arrow["{i_{B}}"', from=3-3, to=2-5]
        \arrow["{f}", from=1-3, to=2-7, bend left=15]
        \arrow["{g}"', from=3-3, to=2-7, bend right=15]
        \arrow["{\sigma}", from=2-5, to=2-7]
      \end{tikzcd}
      \]
      commutes.

      To construct the fibered coproduct \(A \coprod_{C} B\) in \(\mathsf{Set}\),
      first consider the disjoint union \((\{0\} \times A) \cup (\{1\} \times B)\).
      We define an equivalence relation \(\sim\) on this set, setting
      \begin{align*}
        &(0, a) \sim (0, a') \Longleftrightarrow a = a', \\
        &(1, b) \sim (1, b') \Longleftrightarrow b = b', \\
        &(0, a) \sim (1, b) \Longleftrightarrow \exists c \in C: \alpha(c) = a \text{ and } \beta(c) = b
      \end{align*}
      Interestingly, note that equivalence classes have at most 2 elements.

      We claim that \(A \coprod_{C} B / {\sim}\) is a fibered coproduct in \(\mathsf{Set}\) with the maps \(i_{A}(a) = [(0, a)]_{\sim}\) and \(i_{B}(b) = [(1, b)]_{\sim}\).
      Let \(Z\) be a set with functions \(f: A \to Z\) and \(g: B \to Z\) such that \(f \circ \alpha = g \circ \beta\).
      By the universal property of the coproduct, there is a unique morphism \(\sigma': A \coprod B \to Z\).
      Now we use the universal property of the quotient to construct a unique function \(\sigma: A \coprod B / {\sim} \to Z\).
      We can verify that
      \begin{equation*}
        \sigma \circ i_{A}(a) = \sigma([(0, a)]_{\sim}) = \sigma'(0, a) = f(a)
      \end{equation*}
      Similarly, we have \(\sigma \circ i_{B}(b) = g(b)\).
      Combined with the condition that \(f \circ \alpha = g \circ \beta\), it becomes clear that the diagram commutes.
    \end{solution}
\end{document}
