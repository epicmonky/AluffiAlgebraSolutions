\documentclass[../../master.tex]{subfiles}

\begin{document}
\section{UFDs, PIDs, Euclidean domains}
    
    % Problem 2.1
    \begin{problem}
        Prove Lemma 2.1.
        \begin{proposition}[Lemma 2.1]
            Let $R$ be a UFD, and let $a, b, c$ be nonzero elements of $R$. Then
            \begin{itemize}
                \item $(a) \subseteq (b) \Longleftrightarrow$ the multiset of
                    irreducible factors of $b$ is contained in the multiset of
                    irreducible factors of a;
                \item $a$ and $b$ are associates (that is, $(a) = (b)$ )
                    $\Longleftrightarrow$ the two multisets coincide;
                \item the irreducible factors of a product $bc$ are the
                    collection of all irreducible factors of $b$ and $c$.
            \end{itemize}
        \end{proposition}
    \end{problem}
    
    \begin{solution}
        Let $M_{a}$ denote the multiset containing the irreducible factors of
        $a$.
        \begin{itemize}
            \item $(a) \subseteq (b) \Longleftrightarrow a = bc
                \Longleftrightarrow a = (q_1^{\alpha_1} \cdots q_r^{\alpha_r}) c
                \Longleftrightarrow M_b \subseteq M_a$.

            \item $(a) = (b) \Longleftrightarrow (a) \subseteq (b)$ and $(b)
                \subseteq (a) \Longleftrightarrow M_a \subseteq M_b$ and $M_b
                \subseteq M_a$. That is, the multisets coincide.

            \item It is clear from point 1 that the irreducible factors of $b$
              and $c$ are contained in the irreducible factors of $bc$. Now
              suppose $q$ is an irreducible factor of $bc$. If $q$ is a
              factor of $b$ then we are done so suppose not. Then we may
              factor $bc = b q r$ where $r$ is some collection of units and
              irreducible factors. Since $R$ is a UFD and in particular an
              integral domain, we cancel $b$ on both sides and obtain $c =
              qr$. That is, $q$ is a factor of $c$. Thus, the irreducible
              factors of $bc$ are the collection of irreducible factors of $b$
              and $c$.
        \end{itemize}
    \end{solution}

    % Problem 2.2
    \begin{problem}
        Let $R$ be a UFD, and let $a, b, c$ be elements of $R$ such that $a \mid
        bc$ and $\gcd(a, b) = 1$. Prove that $a$ divides $c$.
    \end{problem}

    \begin{solution}
        Since $a \mid bc$, there exists $r \in R$ such that $ar = bc$. By
        uniqueness, both sides of this equation share the same multiset of
        irreducible factors. Since $\gcd(a, b) = 1$, $a$ and $b$ share no
        irreducible factors. Thus, the irreducible factors of $a$ are contained
        in those of $c$ and we have $a \mid c$.
    \end{solution}

    % Problem 2.3
    \begin{problem}
        Let $n$ be a positive integer. Prove that there is a one-to-one
        correspondence preserving multiplicities between the irreducible factors
        of $n$ (as an integer) and the composition factors of $\mathbb{Z} /
        n\mathbb{Z}$ (as a group). (In fact, the Jordan-H\"{o}lder theorem may
        be used to prove that $\mathbb{Z}$ is a UFD.)
    \end{problem}

    \begin{solution}
        Let $d$ be the largest proper divisor of $n$ and let $G_1 = \mathbb{Z} /
        d \mathbb{Z}$. Then $G / G_1$ is simple of cyclic, hence it has prime
        order. Repeating this process (a finite number of times since $n$ is
        finite), we obtain a composition series of $G$,
        \[
        G = G_0 \trianglerighteq G_1 \trianglerighteq \cdots \trianglerighteq
        G_m = 1,
        \] 
        where $G_i / G_{i+1}$ has prime order. Then 
        \[
            n = |G| = |G / G_1| |G_1 / G_2| \cdots |G_{m-1} / G_{m-2}| = p_1 p_2
            \cdots p_{m-1}.
        \] 
        Thus, this process produces a composition series whose factors are in
        bijection with the prime (and irreducible, since we are in $\mathbb{Z}$
        ) factors of $n$.
    \end{solution}

    % Problem 2.4
    \begin{problem}
        Consider the elements $x, y$ in $\mathbb{Z}[x, y]$. Prove that 1 is a
        gcd of $x$ and $y$, and yet 1 is \textit{not} a linear combination of
        $x$ and $y$. (Cf. Exercise II.2.13.)
    \end{problem}

    \begin{solution}
        Certainly $(x, y) \subseteq (1) = R$. Now consider $d$ such
        that $(x, y) \subseteq (d)$. Then $d \mid x$ and $d \mid y$. However,
        both $x$ and $y$ are irreducible and $(x) \subsetneq (d)$ so the two are
        not associate. Thus, $d$ is a unit in $\mathbb{Z}[x, y]$ such as $1$.
        However, 1 cannot be written as a linear combination of $x$ and $y$ by
        comparing degrees.
    \end{solution}

    % Problem 2.5
    \begin{problem}
        Let $R$ be the subring of $\mathbb{Z}[t]$ consisting of polynomials with
        no term of degree 1: $a_0 + a_2 t^2 + \cdots + a_d t^{d}$.
        \begin{itemize}
            \item Prove that $R$ is indeed a subring of $\mathbb{Z}[t]$, and
                conclude that $R$ is an integral domain.
            \item List all common divisors of $t^{5}$ and $t^{6}$ in $R$.
            \item Prove that $t^{5}$ and $t^{6}$ have no gcd in $R$.
        \end{itemize}
    \end{problem}

    \begin{solution}
        Certainly if $f, g \in R$, then $f - g \in R$ since adding polynomials
        cannot introduce terms of a new degree. We also have
        \[
            fg = (a_0 + a_2t^2 + \cdots)(b_0 + b_2t^2 + \cdots) = a_0b_0 +
            (a_0b_2 + a_2b_0) t^2 + \cdots \in R
        \] 
        Thus, $R$ is a subring of $\mathbb{Z}[t]$. A subring of an integral
        domain is also an integral domain (or else non-zero elements $x, y$ such
        that $xy = 0$ would also be in the ring). Thus, $R$ is an integral
        domain.

        The common divisors of $t^{5}$ and $t^{6}$ in $R$ are $1, \, t^2,$ and
        $t^3$. However, note that $t^{6} = t^{5} \cdot t$ and $t \notin R$.
        Suppose $d = \gcd(t^{5}, t^{6})$. Then $t^{6} \in (d)$. That is, there
        is an element $a$ such that $t^{6} = t^{5} \cdot t = ad$. We may cancel
        since $R$ is an integral domain to find that $t = bd$ and thus $t \in
        (d)$, a contradiction. Therefore, $t^{5}$ and $t^{6}$ have no greatest
        common divisor.
    \end{solution}

    % Problem 2.6
    \begin{problem}
        Let $R$ be a domain with the property that the intersection of any
        family of principal ideals in $R$ is necessarily a principal ideal.
        \begin{itemize}
            \item Show that greatest common divisors exist in $R$.
            \item Show that UFDs satisfy this property.
        \end{itemize}
    \end{problem}

    \begin{solution}
        Since the intersection is associative, we may consider only two elements
        $a, b \in R$. Consider their intersection $(a) \cap (b) = (m)$. Then we
        have $ab = dm$ for some $d \in R$. We claim that $d = \gcd(a, b)$.
        Indeed, we have $(m) \subseteq (a)$ so $m = a \cdot r$ for some $r$.
        Then $ab = dm = d a r \Longrightarrow b = dr \Longrightarrow d \mid b$.
        Similarly, $d \mid a$ so it is a common divisor of both. Now let $c \mid
        a$ and $c \mid b$. That is, $a = c r_1$ and $b = c r_2$. Then $c \mid
        ab$, or $ab = cx$ for some $x$. Rewriting, we have $c r_1 b = cx
        \Longrightarrow (x) \subseteq (b)$. Similarly, $(x) \subseteq (a)$. Then
        $(x) \subseteq (a) \cap (b) = (m)$ so $x = m s$ for some $s$. Finally,
        we have $dm = ab = cx = c (ms) \Longrightarrow d = cs \Longrightarrow c
        \mid d$. Thus, $d$ is indeed a $\gcd$ for $a$ and $b$.

        Let $R$ be a UFD and consider a family of principal ideals $\{ (a_i)
        \}$. Let $I \ \bigcap_i (a_i)$ and pick any $r_0 \in I$. If $(r_0) = I$,
        we are done so suppose not. Then pick $s \in I - (r_0)$. We may then set
        $r_1 = \gcd(r_0, s)$. The ideal $(r_1)$ is the smallest principal ideal
        containing $(r_0, s)$, which is a subset of each $(a_i)$ since both
        generators are chosen from the intersection of these ideals. Thus $(r_1)
        \subseteq I$ and we have the chain 
        \[
            (r_0) \subsetneq (r_0, s) \subseteq (r_1) \subseteq I.
        \] 
        This process can be repeated as long as $(r_n) \subsetneq I$. Thus, we
        form an ascending chain of principal ideals and since $R$ is a UFD, it
        must stabilize. This occurs when $(r_n) = I$.
    \end{solution}

    % Problem 2.7
    \begin{problem}
        Let $R$ be a Noetherian domain, and assume that for all nonzero $a, b$
        in $R$, the greatest common divisors of $a$ and $b$ are linear
        combinations of $a$ and $b$. Prove that $R$ is a PID.
    \end{problem}

    \begin{solution}
        Suppose that $R$ is not a PID and let $I$ be a non-principal ideal.
        Choose $0 \neq a_0 \in I$. Then $(a_0) \subsetneq I$ so we may choose
        $b_0 \in I - (a_0)$. We may consider $a_1 = \gcd(a_0, b_0)$. Then we
        find
        \[
            (a_0) \subsetneq (a_0, b_0) = (a_1) \subsetneq I
        \] 
        Repeating this indefinitely yields an ascending chain of ideals which
        does not stabilize, a contradiction to the assumption that $R$ is
        Noetherian. Thus, $R$ must be a PID.
    \end{solution}

    % Problem 2.8
    \begin{problem}
        Let $R$ be a UFD, and let $I \neq (0)$ be an ideal of $R$. Prove that
        every descending chain of principal ideals containing $I$ must
        stabilize.
    \end{problem}

    \begin{solution}
        Consider a descending chain of principal ideals
        containing $I$ 
        \[
            (a_1) \supsetneq (a_2) \supsetneq \cdots
        \] 
        There is a corresponding ascending chain of multisets of irreducible
        factors. Let $0 \neq b \in I$. Then $(b) \subseteq (a_i)$ for all
        $(a_i)$ in the ascending chain. Letting $M_b$ denote the multiset of
        irreducible factors of $b$, we have that each multiset in the
        corresponding ascending chain is contained in $M_b$. If the chain does
        not stabilize, then eventually the multiset of irreducible factors for
        say $a_n$ will have greater size than $M_b$,  a contradiction. Therefore the
        descending chain of principal ideals must stabilize.
    \end{solution}

    % Problem 2.9
    \begin{problem}
        The \textit{height} of a prime ideal $P$ in a ring $R$ is (if finite)
        the maximum length $h$ of a chain of prime ideals $P_0 \subsetneq P_1
        \subsetneq \cdots \subsetneq P_h = P$ in $R$. (Thus, the Krull dimension
        of $R$, if finite, is the maximum height of a prime ideal in $R$.) Prove
        that if $R$ is a UFD, then every prime ideal of height 1 in $R$ is
        principal.
    \end{problem}

    \begin{solution}
        First note that $(0)$ is prime in $R$ since $R$ is an integral domain.
        Thus, the chain of ideals looks like
        \[
            (0) \subsetneq P.
        \] 
        Since $P$ is non-empty, there is some non-zero element $a \in P$.
        Consider the factorization of $a$ into irreducibles. Since $P$ is prime,
        one of these elements belongs to $P$, say $p$. Since $R$ is a UFD,
        irreducible elements are prime so $(p)$ is a prime ideal. But then we
        have
        \[
            (0) \subsetneq (p) \subseteq P.
        \] 
        Since $P$ has height one, it must be the case that $(p) = P$, so $P$ is
        principal.
    \end{solution}

    % Problem 2.10
    \begin{problem}
        It is a consequence of a theorem known as \textit{Krull's
        Hauptidealsatz} that every nonzero, nonunit element in a Noetherian domain is
        contained in a prime ideal of height 1. Assuming this, prove a converse
        to Exercise 2.9, and conclude that a Noetherian domain $R$ is a UFD if
        and only if every prime ideal of height 1 in $R$ is principal.
    \end{problem}

    \begin{solution}
        Suppose $R$ is a Noetherian domain such that every prime ideal of height
        1 is principal. Since $R$ is Noetherian, the a.c.c. holds for all
        ideals, and principal ideals in particular. Therefore, we only need to
        show that irreducible elements are prime. Let $q$ be an irreducible
        element of $R$. By Krull's Hauptidealsatz, $q$ is contained in some
        prime ideal of height 1, say $(p)$. Then we have $q = pa$ for some unit
        $a$. Thus, $(p) = (q)$ and $(q)$ is prime, implying that $q$ is a prime
        element. Since every irreducible element is prime, $R$ is a UFD.
    \end{solution}

    % Problem 2.11
    \begin{problem}
        Let $R$ be a PID, and let $I$ be a nonzero ideal of $R$. Show that $R /
        I$ is an artinian ring (cf. Exercise 1.10), by proving explicitly that
        the d.c.c. holds in $R / I$.
    \end{problem}

    \begin{solution}
        Since $R$ is a PID, let $I = (a)$. Consider a descending chain of ideals
        in $R / I$
        \[
        \frac{I_0}{I} \supsetneq \frac{I_1}{I} \supsetneq \frac{I_2}{I}
        \supsetneq \cdots
        \] 
        This corresponds to a descending chain of ideals containing $I$ in $R$.
        Since $R$ is a PID, it is also a UFD and by Exercise 2.8, a descending
        chain of principal ideals containing a non-zero ideal must stabilize.
        Thus, this descending chain in $R$ stabilizes and so does the one in $R
        / I$.
    \end{solution}

    % Problem 2.12
    \begin{problem}
        Prove that if $R[x]$ is a PID, then $R$ is a field.
    \end{problem}

    \begin{solution}
        Consider the ideal $(x)$. By Exercise 2.11, the quotient $R[x] / (x)$ is
        artinian. Furthermore, $R$ is an integral domain (since $R[x]$ is) and
        by Exercise 1.10, an artinian integral domain is a field.
    \end{solution}

    % Problem 2.13
    \begin{problem}
        For $a, b, c$ positive integers with $c > 1$, prove that $c^{a}-1$
        divides $c^{b} - 1$ if and only if $a \mid b$. Prove that $x^{a} - 1$
        divides $x^{b} - 1$ in $\mathbb{Z}[x]$ if and only if $a \mid b$. (Hint:
        For the interesting implications, write $b = ad + r$ with $0 \leq r <
        a$, and take `size' into account.)
    \end{problem}

    \begin{solution}
        Since $\mathbb{Z}$ is a Euclidean domain, we may write $b = ad + r$ with
        $0 \leq r < a$. Then we have 
        \[
            x^{b} - 1 = x^{b} - x^{r} + x^{r} - 1 = x^{r} \left( x^{ad} -
                1\right) + x^{r} - 1
        \] 
        Furthermore, note that
        \[
            x^{ad} - 1 = (x^{a} -1) \left(x^{a(d-1)} + x^{a(d-2)} + \cdots +
                1\right)
        \] 
        Then $x^{a} - 1$ divides the right side of the first equation if and
        only if $r = 0$, if and only if $a$ divides $b$. The first statement is
        a direct implication by setting $x = c$.
    \end{solution}

    % Problem 2.14
    \begin{problem}
        Prove that if $k$ is a field, then $k[[x]]$ is a Euclidean domain.
    \end{problem}

    \begin{solution}
        Define a valuation on $k[[x]] \setminus \{0\}$, setting $v(f)$ to be the
        degree of the smallest term of $f$ with non-zero coefficient. Indeed,
        given power series $f, g$, we write
        \[
        f = qg + r.
        \] 
        This is possible since $k$ is a field. If $v(g) > v(f)$ then let $q =
        0$ and set $r = f$ so that $v(r) < v(g)$. If $v(g) = v(f)$, then define
        $q$ such that the first non-zero term of $qg$ equals that of $f$. Then
        define $r$ such that the remaining terms are equivalent and we have 
        $v(r) < v(g)$. Similarly, if $v(g) < v(f)$, define $q$ such that the
        first $v(f) - v(g)$ terms of $qg$ are equal to those of $f$ (possible
        since $k$ is a field). Then $v(r) < v(g)$. Thus, this is indeed a
        Euclidean valuation.
    \end{solution}

    % Problem 2.15
    \begin{problem}
        Prove that if $R$ is a Euclidean domain, then $R$ admits a Euclidean
        valuation $\bar{v}$ such that $\bar{v}(ab) \geq \bar{v}(b)$ for all
        nonzero $a, b \in R$. (Hint: Since $R$ is a Euclidean domain, it admits
        a valuation $v$ as in Definition 2.7. For $a \neq 0$, let $\bar{v}(a)$
        be the minimum of all $v(ab)$ as $b \in R, b \neq 0$. To see that $R$ is
        a Euclidean domain with respect to $\bar{v}$ as well, let $a, b$ be
        nonzero in $R$, with $b \nmid a$; choose $q, r$ so that $a = bq + r$,
        with $v(r)$ minimal; assume that $\bar{v}(r) \geq \bar{v}(b)$, and get a
        contradiction.)
    \end{problem}

    \begin{solution}
        Define $\bar{v}$ as above; that is, set $\bar{v}(a) = \min \{v(ab) \mid
        b \in R, b \neq 0 \}$. Clearly, $\bar{v}$ satisfies the property that
        $\bar{v}(ab) \geq \bar{v}(b)$. Let $a, b \in R$ be non-zero and $b \nmid
        a$.  Write $a = bq + r$ with minimal $v(r)$. Suppose that $\bar{v}(r)
        \geq \bar{v}(b)$. That is, there exists $c \in R$ such that for all $x
        \in R$, $v(rx) \geq v(bc)$. In particular, for $x = c$, we have $v(rc)
        \geq v(bc)$. However, multiplying the initial equation by $c$ yields $ac
        = bcq + rc$ where $v(rc) < v(bc)$, a contradiction. Thus, $\bar{v}$ is a
        Euclidean valuation.
    \end{solution}

    % Problem 2.16
    \begin{problem}
        Let $R$ be a Euclidean domain with Euclidean valuation $v$; assume that
        $v(ab) \geq v(b)$ for all nonzero $a, b \in R$ (cf. Exercise 2.15).
        Prove that associate elements have the same valuation and that units
        have minimum valuation.
    \end{problem}

    \begin{solution}
        Let $a$ and $b$ be associates. That is, we can write $a = ub$ for some
        unit $u$. Then we have $v(a) = v(ub) \geq v(b)$. Furthermore, we have $b
        = u^{-1}a$ so $v(b) = v(u^{-1}a) \geq v(a)$. Thus, $v(a) = v(b)$.

        Now consider a unit $u$. For all $r \in R$, we have $r = ru^{-1}u$.
        This implies that $v(u) \leq v(r)$ so units have minimum valuation.
    \end{solution}

    % Problem 2.17
    \begin{problem}
        Let $R$ be a Euclidean domain that is not a field. Prove that there
        exists a nonzero, nonunit element $c$ in $R$ such that $\forall a \in R,
        \exists q, r \in R$ with $a = qc + r$ and either $r = 0$ or $r$ a unit.
    \end{problem}

    \begin{solution}
        The existence of a nonzero, nonunit element $c$ is guaranteed since $R$
        is not a field. Choose such a $c$ with minimal valuation. Let $a \in R$
        and choose $q, r$ such that $a = qc + r$. If $r = 0$ then we are done so
        suppose not. We have $v(r) < v(c)$. If $r$ is not a unit, then a
        contradiction arises as we chose $c$ to have minimal valuation. Thus
        $r$ must be a unit.
    \end{solution}

    % Problem 2.18
    \begin{problem}
        For an integer $d$, denote by $\mathbb{Q}(\sqrt{d})$ the smallest
        subfield of $\mathbb{C}$ containing $\mathbb{Q}$ and $\sqrt{d}$, with
        norm $N$ defined as in Exercise III.4.10. See Exercise 1.17 for the case
        $d = -5$; in this problem, you will take $d = -19$.

        Let $\delta = (1 + i \sqrt{19}) / 2$, and consider the following subring
        of $\mathbb{Q}(\sqrt{-19})$ :
        \[
            \mathbb{Z}[\delta] := \left\{ a + b \frac{1 + i \sqrt{19}}{2} \mid a,
            b \in \mathbb{Z} \right\}.
        \] 
        \begin{itemize}
            \item Prove that the smallest values of $N(z)$ for $z = a + b \delta
                \in \mathbb{Z}[\delta]$ are 0, 1, 4, 5. Prove that $N(a + b
                \delta) \geq 5$ if $b \neq 0$.
            \item Prove that the units in $\mathbb{Z}[\delta]$ are $\pm 1$.
            \item If $c \in \mathbb{Z}[\delta]$ satisfies the condition
                specified in Exercise 2.17, prove that $c$ must divide 2 or 3 in
                $\mathbb{Z}[\delta]$, and conclude that $c = \pm 2$ or $c = \pm
                3$.
            \item Now show that $\nexists q \in \mathbb{Z}[\delta]$ such that
                $\delta = qc + r$ with $c = \pm 2, \pm 3$ and $r = 0, \pm 1$.
        \end{itemize}
        Conclude that $\mathbb{Z}[(1 + \sqrt{-19}) / 2]$ is not a Euclidean
        domain.
    \end{problem}

    \begin{solution}
        Certainly $N(z)$ takes on those values for values $(0, 0)$, $(\pm1, 0)$,
        $(\pm2, 0)$, and $(0, \pm1)$. To prove these are minimal, let $|a| > 2$.
        Then
        \[
             N(a + b \delta) \geq N(a) = a^2 > 4 = N(\pm2).
        \]    
        Furthermore, if $b \neq 0$ then 
        \[
            N(a + b \delta) \geq N(b \delta) = \frac{b^2}{4} + 19 \cdot
            \frac{b^2}{4} = 5 b^2 \geq 5
        \] 

        Clearly two units in $\mathbb{Z}[\delta]$ are $\pm 1$. Now let $u$ be a
        unit. Then $N(u) = 1$. By Point 1, we have $u = \pm 1$.

        If $c \in \mathbb{Z}[\delta]$ satisfies the condition from the previous
        problem then we have $2 = q_1c + r_1$ and $3 = q_2c + r_2$. If $r_1 =
        0$ then $c \mid 2$. If $r_1 \neq 0$ then $r_1 = \pm 1$. If $r_1 = 1$
        then $2 = q_1c + 1 \Longrightarrow q_1c = 1$, contradicting that $c$ is
        not a unit. If $r_1 = -1$, then we have
        \[
        q_2c + r_2 = 3 = 2 + 1 = q_1c - 1 + 1 = q_1c
        \] 
        so $c \mid 3$. Given the condition and point 1, it must be the case that
        $c = \pm 2$ or $c = \pm 3$.

        Now suppose there exists $q = a + b\delta \in \mathbb{Z}[\delta]$ such
        that $\delta = qc + r$ with $c = \pm 2, \pm 3$ and $r = 0, \pm 1$. If $r
        = 0$, then we have $N(q) N(c) = N(qc) = N(\delta) = 5$. Since 5 is
        prime and $N(c) = 4$ or 9 respectively, $q$ cannot exist. Similarly, if
        $r = 1$, then we have $N(q) N(c) = N(qc) = N(\delta - 1) = 5$ and the
        same contradiction arises. If $r = -1$, then $N(qc) = 7$, another
        contradiction. Thus, there can be no such $q$ and $\mathbb{Z}[(1 +
        \sqrt{-19}) / 2]$ is not a Euclidean domain.
    \end{solution}

    % Problem 2.19
    \begin{problem}
        A \textit{discrete valuation} on a field $k$ is a surjective
        homomorphism of abelian groups $v : (k^{*}, \cdot) \to (\mathbb{Z}, +)$
        such that $v(a + b) \geq \min(v(a), v(b))$ for all $a, b \in k^{*}$ such
        that $a + b \in k^{*}$.
        \begin{itemize}
            \item Prove that the set $R := \{a \in k^{*} \mid v(a) \geq 0 \}
                \cup \{0\}$ is a subring of $k$.
            \item Prove that $R$ is a Euclidean domain.
        \end{itemize}
        Rings arising in this fashion are called \textit{discrete valuation
        rings}, abbreviated DVR. They arise naturally in number theory and
        algebraic geometry. Note that the Krull dimension of a DVR is 1 (Example
        III.4.14); in algebraic geometry, DVRs correspond to particularly nice
        points on a `curve'.
        \begin{itemize}
            \item Prove that the ring of rational numbers $a / b$ with $b$ 
                \textit{not} divisible by a fixed prime integer $p$ is a DVR.
        \end{itemize}
    \end{problem}

    \begin{solution}
        To show that $R$ is a subring, first note that it is a subgroup under
        addition. Indeed, for nonzero $a, b \in R$ we have
        \[
            v(a - b) \geq \min(v(a), v(-b)).
        \] 
        Note that $v(-b) = v(-1 \cdot b) = v(-1) + v(b)$ where $-1$ is the
        additive inverse of 1. Furthermore, 
        \[
            v(-1) + v(-1) = v(-1 \cdot -1) = v(1) = 0
        \]
        implies that $v(-1) = 0$. Thus, we have $v(-b) = v(b)$ so $v(a
        - b) \geq \min(v(a), v(-b)) \geq 0$, meaning $a - b \in R$.

        To show that $R$ is closed under multiplication, see that $v(ab) = v(a)
        + v(b)$. Since both $v(a)$ and $v(b)$ are non-negative, so is their sum.
        Therefore, $ab \in R$ and $R$ is a ring.

        To prove that $R$ is a Euclidean domain, we must show that $v$ is a
        Euclidean valuation which we do by cases. Let $a, b \in R$ be nonzero.
        If $v(a) \geq v(b)$, then we have $v(a / b) = v(a) - v(b) \geq 0$ so $a
        / b \in R$. Therefore we can write $a = (a / b)  b + 0$. If $v(a) < v(b)$,
        then we have $a = 0b + a$. Thus, in any case we can choose $q, r \in R$
        such that $a = qb + r$ with either $r = 0$ or $v(r) < v(b)$.

        Consider the ring $R$ of rational numbers $a / b$ with $b$ not divisible
        by a fixed prime integer $p$. We should define a discrete valuation,
        that is a group homomorphism to $\mathbb{Z}$, on the field $\mathbb{Q}$ so that
        the resulting ring arises in the manner defined above. Given a rational
        number $a / b$ such that a fixed prime $p \nmid b$, we can use the
        unique factorization of $\mathbb{Z}$ to write 
        \[
        \frac{a}{b} = \frac{p^{k}z}{b}
        \] 
        for integers $k, z$ such that $p \nmid z$. Then define $v(a / b) = k$.
        To verify that $v$ is a discrete valuation, we first show that it is a
        homomorphism of groups.
        Indeed, if $x, y \in \mathbb{Q}^{*}$, then 
        \[
            v(xy) = v\left(\frac{a_1a_2}{b_1 b_2}\right) = 
            v \left(\frac{p^{k_1}z_1 p^{k_2}z_2}{b_1 b_2}\right) =
            v \left(\frac{p^{k_1 + k_2}z_1 z_2}{b_1 b_2}\right) = k_1 + k_2 =
            v(x) + v(y)
        \] 
        Thus, $v$ is a group homomorphism. Furthermore, we find that
        \[
            v(x + y) = v\left(\frac{a_1 b_2 + a_2 b_1}{b_1 b_2}\right) =
            v\left( \frac{p^{k_1}z_1 b_2 + p^{k_2}z_2 b_1}{b_1 b_2} \right)
        \] 
        WLOG, we may assume $k_1 \leq k_2$. Then
        \[
            v\left(\frac{p^{k_1}z_1 b_2 + p^{k_2}z_2 b_1}{b_1 b_2}\right) = 
            v\left(p^{k_1} \frac{z_1 b_2 + p^{k_2 - k_1}z_2 b_1}{b_1 b_2}\right)
            = k_1 \geq \min(v(x), v(y))
        \] 
        Therefore, $v$ is a discrete valuation and the resulting ring is in fact
        the one defined above. I did not formulate this valuation myself and I
        don't see how it's at all a natural definition but it works out.
    \end{solution}

    % Problem 2.20
    \begin{problem}
        As seen in Exercise 2.19, DVRs are Euclidean domains. In particular,
        they must be PIDs. Check this directly, as follows. Let $R$ be a DVR,
        and let $t \in R$ be an element such that $v(t) = 1$. Prove that if $I
        \subseteq R$ is any nonzero ideal, then $I = (t^{k})$ for some $k \geq
        1$. (The element $t$ is called a `local parameter' of $R$.)
    \end{problem}

    \begin{solution}
        Let $a \in I$ be a nonzero element with minimal valuation $v(a) = n$.
        Then for all nonzero $b \in I$, we have
        \[
            v(b / a) = v(b) - v(a) \geq 0 \Longrightarrow b / a \in R
            \Longrightarrow b \in (a).
        \] 
        Although this is sufficient, we can go on to show that if $v(a) = v(b)$
        then $(a) = (b)$. Indeed, we find
        \[
            v(a / b) = v(b / a) = 0 \Longrightarrow b \mid a \text{ and } a \mid
            b \Longrightarrow (a) = (b)
        \] 
        For a local parameter $t$, we have $v(t^{k}) = k$ so for an element $a
        \in I$ with minimal valuation $n$, we have $I = (t^{n})$.
    \end{solution}

    % Problem 2.21
    \begin{problem}
        Prove that an integral domain is a PID if and only if it admits a
        Dedekind-Hasse valuation. (Hint: For the $\Longleftarrow$ implication,
        adapt the argument in Proposition 2.8; for $\Longrightarrow$, let
        $v(a)$ be the size of the multiset of irreducible factors of $a$.)
    \end{problem}

    \begin{solution}
        First suppose that $R$ is an integral domain admitting a Dedekind-Hasse
        valuation. Let $I$ be an ideal of $R$. If $I$ is zero then it is clearly
        principal so suppose not. Then choose $0 \neq b \in I$ to have minimal
        valuation. For all $a \in I$, we either have $(a, b) \in (b)$ or there
        exists $q, r, s \in R$ such that $as = bq + r$ with $v(r) < v(b)$. In
        the first case, $a \in (b)$. In the latter case, $r = as - bq \in I$. By
        choice of $b$, we cannot have $v(r) < v(b)$. Thus, $r = 0$ and $a \in
        (b)$. Therefore, $I = (b)$ so $R$ is a PID.

        Now suppose that $R$ is a PID. We must show that it admits a
        Dedekind-Hasse valuation. Define $v: R \to \mathbb{Z}^{\geq 0}$ to send
        $v(a)$ to the size of the multiset of irreducible factors of $a$ (recall
        that a PID is a UFD). To verify that this is a Dedekind-Hasse valuation,
        let $a, b \in R$. We have $(a, b) = (d)$ for some
        $d \in R$. In particular, $d \mid b$ so $v(d) \leq v(b)$. If $v(d) =
        v(b)$, then $(b) = (d)$ by considering the size of multisets of irreducible
        factors so we have $(a, b) = (b)$ and $b \mid a$. If $v(d) < v(b)$, we can
        write
        \[
        -d = as + bq \Longrightarrow as = bq + d
        \] 
        for $q, s \in R$. Thus, $v$ is indeed a Dedekind-Hasse valuation.
    \end{solution}

    % Problem 2.22
    \begin{problem}
        Suppose $R \subseteq S$ is an inclusion of integral domains, and assume
        that $R$ is a PID. Let $a, b \in R$ and let $d \in R$ be a gcd for $a$
        and $b$ in $R$. Prove that $d$ is also a gcd for $a$ and $b$ in $S$.
    \end{problem}

    \begin{solution}
        Since $R$ is a PID, we have $(a, b) = (d)$. That is, there exist  $x, y
        \in R$ such that $ax + by = d$. Now let $c \in S$ such that $c \mid a$
        and $c \mid b$. Then $c \mid ax + by = d$. Thus, $d $ is a gcd for $a$
        and $b$ in $S$ as well.
    \end{solution}

    % Problem 2.23
    \begin{problem}
        Compute $d = \gcd(5504227617645696, 2922476045110123)$. Further, find
        $a, b$ such that $d = 5504227617645696a + 2922476045110123b$.
    \end{problem}

    \begin{solution}
        A brief application of the extended Euclidean algorithm shows that $d =
        234982394879$. Furthermore, we have $a = 1055$ and $b = -1987$.
    \end{solution}

    % Problem 2.24
    \begin{problem}
        Prove that there are infinitely many prime integers. (Hint: Assume by
        contradiction that $p_1, \ldots, p_N$ is a complete list of all positive
        prime integers. What can you say about $p_1 \cdot \cdots \cdot p_N + 1$?
        This argument was already known to Euclid, more than 2,000 years ago.)
    \end{problem}

    \begin{solution}
        Let $P = p_1 \cdot \cdots \cdot p_N + 1$. By assumption, $P$ is not
        prime so it is divisible by some prime in our list, say $p_i$. But then
        we have $p \mid P - p_1 \cdot \cdots \cdot p_N = 1$, a contradiction.
        Therefore the list of primes is not complete.
    \end{solution}

    % Problem 2.25
    \begin{problem}
        Variation on the theme of Euclid from Exercise 2.24: Let $f(x) \in
        \mathbb{Z}[x]$ be a nonconstant polynomial such that $f(0) = 1$. Prove
        that infinitely many primes divide the numbers $f(n)$, as $n$ ranges in
        $\mathbb{Z}$. (If $p_1, \ldots, p_n$ were a complete list of primes
        dividing the numbers $f(n)$, what could you say about $f(p_1 \cdots
        p_Nx)$?)

        Once you are happy with this, show that the hypothesis $f(0) = 1$ is
        unnecessary. (If $f(0) = a \neq 0$, consider $f(p_1 \cdots p_N ax)$.
        Finally note that there is nothing special about 0.)
    \end{problem}

    \begin{solution}
        First note that the requirement $f(0) = 1$ implies that the constant
        term of the polynomial is 1. Suppose there were a complete list of
        primes dividing the values of $f(n)$. Let $P = p_1 \cdots p_N$ and
        consider $f(Px)$. We find
        \[
            f(Px) = 1 + a_1(Px) + a_2 (Px)^2 + \cdots + a_n (Px)^{n}
        \] 
        In particular, for $x = 1$, we have $p_i$ divides the left side. But
        $p_i$ also divides $P$ and so it divides the difference
        \[
            p_i \mid f(Px) - (a_1 Px + a_2 (Px)^2 + \cdots + a_n (Px)^{n}) = 1,
        \] 
        a contradiction.

        An entirely analogous proof works for $f(0) = a \neq 0$ and considering
        the product $f(Pax)$. The case $f(0)
        = 0$ is trivial since all primes $p$ divide $f(p)$.
    \end{solution}
\end{document}
