\documentclass[../../master.tex]{subfiles}

\begin{document}
\section{Intermezzo: Zorn's lemma}

    % Problem 3.1
    \begin{problem}
        Prove that every well-ordering is total.
    \end{problem}

    \begin{solution}
        Recall that a well-ordering on $Z$ is an order relation such that every
        nonempty subset of $Z$ has a least element. For any two elements $a, b
        \in Z$, consider the subset $\{a, b\} \subseteq Z$. Since this subset
        has a least element, it must be the case that either $a \preceq b$ or $b
        \preceq a$. As this holds for any pair of elements in $Z$, it follows
        that $\preceq$ is total on $Z$.
    \end{solution}

    % Problem 3.2
    \begin{problem}
        Prove that a totally ordered set $(Z, \preceq)$ is a woset if and only
        if every descending chain
        \[
        z_1 \succeq z_2 \succeq z_3 \succeq \cdots
        \] 
        in $Z$ stabilizes.
    \end{problem}

    \begin{solution}
        Suppose every such descending chain stabilizes. Let $S \subseteq Z$ be
        a nonempty subset. Since $Z$ is totally ordered, the elements of $S$
        form a descending chain as described above. Then there is some element
        $a$ such that for all $b \in Z$, $a \preceq b$. That is, $a$ is a least
        element in $S$. Then $Z$ is well-ordered.

        Now suppose $Z$ is a woset. Assume there is a descending chain which
        does not stabilize. Then the set formed by these elements does not have
        a minimum element, a contradiction. Therefore, every descending chain in
        $Z$ stabilizes.
    \end{solution}

    % Problem 3.3
    \begin{problem}
        Prove that the axiom of choice is equivalent to the statement that a
        set-function is surjective if and only if it has a right-inverse (cf.
        Exercise I.2.2).
    \end{problem}

    \begin{solution}
        The proof of the statement about surjective set-functions assumes the
        axiom of choice, showing that it is sufficient. To see that it is
        necessary, assume that every surjective set-function has a
        right-inverse. Let $A$ be a set of disjoint nonempty sets and $B =
        \bigcup A$. Then for each $b \in B$, there exists exactly one set $X \in
        A$ such that $b \in X$. Thus, we have a surjective function $f: B \to
        A$. Then it has a right-inverse $g$. Define $C:= \{g(X) \mid X \in A\}$.
        Then $C$ is a choice set.
    \end{solution}

    % Problem 3.4
    \begin{problem}
        Construct explicitly a well-ordering on $\mathbb{Z}$. Explain why you
        know that $\mathbb{Q}$ can be well-ordered, even without performing an
        explicit construction.
    \end{problem}

    \begin{solution}
        The well-ordering on $\mathbb{N}$, namely $\leq$, does not work because
        of the negative numbers so we work around this by imposing conditions.
        Let $a, b \in \mathbb{Z}$ and set $a \preceq b$ if and only if one of
        the following holds:
        \begin{itemize}
            \item $|a| < |b|$.
            \item $|a| = |b|$ and $a \leq b$.
        \end{itemize}
        This well ordering yields the following visualization: $0, -1, 1, -2, 2,
        \ldots$ Assuming the Well-ordering Theorem, every set admits a
        well-ordering, including $\mathbb{Q}$. Without directly invoking the
        theorem, we also know that $\mathbb{Q}$ is a countable set and thus is
        in bijection with $\mathbb{N}$, which has a well-ordering. 
    \end{solution}

    % Problem 3.5
    \begin{problem}
        Prove that the (ordinary) principle of induction is equivalent to the
        statement that $\leq$ is a well-ordering on $\mathbb{Z}^{>0}$. (To prove
        by induction that $(\mathbb{Z}^{>0}, \leq)$ is well-ordered, assume it
        is known that 1 is the least element of $\mathbb{Z}^{>0}$ and that
        $\forall n \in \mathbb{Z}^{>0}$ there are no integers between $n$ and $n
        + 1$.)
    \end{problem}

    \begin{solution}
        In Claim 3.2, it was shown that the principle of induction holds for any
        well-ordered set. That is, $\leq$ being a well-ordering on
        $\mathbb{Z}^{> 0}$ implies that the principle of induction holds. To
        show the converse, we can assume that 1 is the least element of
        $\mathbb{Z}^{> 0}$ and that there are no integers between $n$ and $n +
        1$ for all $n \in \mathbb{Z}$. Suppose that there exist a non-empty
        subset $S$ of $\mathbb{Z}^{> 0}$ such that $S$ has no minimum element.
        Then $1 \notin S$ or else it would be a minimal element. Similarly, $2
        \notin S$ because there are no integers between $1$ and $2$, which would
        make $1$ a minimal element. If none of $1, 2, \ldots, n$ are in $S$,
        then $n + 1 \notin S$ or it would be minimal. Thus, the principle of
        induction implies that $S$ is empty, a contradiction. Therefore, $S$
        must have a minimal element so $\leq$ is a well-ordering on
        $\mathbb{Z}^{>0}$.
    \end{solution}

    % Problem 3.6
    \begin{problem}
        In this exercise assume the truth of Zorn's lemma and the conventional
        set-theoretic constructions; you will be proving the well-ordering
        theorem.

        Let $Z$ be a nonempty set, and let $\mathscr{Z}$ be the set of pairs
        $(S, \leq)$ consisting of a subset $S$ of $Z$ and of a
        \textit{well-ordering} $\leq$ on $S$. Note that $\mathscr{Z}$ is not
        empty (singletons can be well-ordered). Define a relation $\preceq$ on
        $\mathscr{Z}$ by prescribing
        \[
            (S, \leq) \preceq (T, \leq')
        \] 
        if and only if $S \subseteq T, \leq$ is the restriction of $\leq'$ to
        $S$, and every element of $S$ precedes every element of $T \setminus S$
        w.r.t.  $\leq'$.
        \begin{itemize}
            \item Prove that $\preceq$ is an order relation in $\mathscr{Z}$.
            \item Prove that every chain in $\mathscr{Z}$ has an upper bound in
                $\mathscr{Z}$.
            \item Use Zorn's lemma to obtain a maximal element $(M, \leq)$ in
                $\mathscr{Z}$. Prove that $M = Z$.
        \end{itemize}
        Thus every set admits a well-ordering, as stated in Theorem 3.3.
    \end{problem}

    \begin{solution}
        Recall that an order relation is reflexive, transitive, and
        antisymmetric. Given a pair $(S, \leq)$, certainly we have $S \subseteq
        S$ and every element of $S$ precedes every element of $S \setminus S =
        \emptyset$ with respect to $\leq$. Therefore, $\preceq$ is reflexive.
        Let $(T, \leq'), (R, \leq'') \in \mathscr{Z}$ such that $(S, \leq)
        \preceq (T, \leq')$ and $(T, \leq') \preceq (R, \leq'')$. Then $S
        \subseteq R$ (by transitivity of subsets) and $\leq$ is the restriction
        of $\leq'$ to $S$, which is the restriction of $\leq''$ to $S$.
        Furthermore, $S \subseteq T$ and every element of $T$ precedes every
        element of $R \setminus T$ w.r.t. $\leq''$. In particular, every element
        of $S$ precedes the elements of $R \setminus T$ w.r.t. $\leq''$. Thus,
        we have $(S, \leq) \preceq (R, \leq'')$, proving transitivity. Finally,
        suppose we have $(S, \leq) \preceq (T, \leq')$ and $(T, \leq') \preceq
        (S, \leq)$. Then $S \subseteq T$ and $T \subseteq S$ so $S = T$. To show
        the two order relations are equivalent, let $a, b \in S$ such that $a
        \leq b$. Since $\leq$ is the restriction of $\leq'$, we have $a \leq'
        b$. Similarly, we find $a \leq' b \Longrightarrow a \leq b$. Thus, the
        two order relations are equivalent and we find $(S, \leq) = (T, \leq')$,
        proving antisymmetry and showing that $\preceq$ is in fact an order
        relation on $\mathscr{Z}$.

        Now consider a chain $\mathscr{C}$ of subsets.
        We must show it has an upper bound in $\mathscr{Z}$. Consider the set
        \[
            U := \bigcup_{S \in \mathscr{C}} S.
        \]
        Certainly each $S \subseteq U$. Furthermore, there is a natural order
        relation on $U$ since for all $a, b \in U$, there exists some $S \in
        \mathscr{C}$ containing both $a$ and $b$. Then the order relation on
        $S$ has $a \leq b$ which also holds in $U$. Thus, $U$ is well-ordered
        and is an upper bound for $\mathscr{C}$.

        Since every chain has an upper bound, Zorn's lemma states that there is
        a maximal element $(M, \leq)$ in $\mathscr{Z}$. Clearly $M \subseteq Z$.
        To show that $M = Z$, suppose otherwise. That is, suppose there is some
        element $x_0 \in Z \setminus M$. Then consider the set $M \cup \{x_0\}$ with
        the order relation $\leq'$ such that for all $x \in M$, $x \leq' x_{0}$.
        Then $(M, \leq) \preceq (M \cup \{x_0\}, \leq')$, contradicting the maximality
        of $M$. Thus, $M = Z$ so $Z$ has a well-ordering.
    \end{solution}

    % Problem 3.7
    \begin{problem}
        In this exercise assume the truth of the axiom of choice and the
        conventional set-theoretic constructions; you will be proving the
        well-ordering theorem.

        Let $Z$ be a nonempty set. Use the axiom of choice to choose an element
        $\gamma(S) \notin S$ for each proper subset $S \subsetneq Z$. Call a
        pair $(S, \leq)$ a $\gamma$-\textit{woset} if $S \subseteq Z$, $\leq$ is
        a well-ordering on $S$, and for every $a \in S$, $a = \gamma(\{b \in S,
        b < a\})$.
        \begin{itemize}
            \item Show how to begin constructing a $\gamma$-woset, and show that
                all $\gamma$-wosets must begin in the same way.
        \end{itemize}
        Define an ordering on $\gamma$-wosets by prescribing that $(U, \leq'')
        \preceq (T, \leq')$ if and only if $U \subseteq T$ and $\leq''$ is the
        restriction of $\leq'$.
        \begin{itemize}
            \item Prove that if $(U, \leq'') \prec (T, \leq')$, then $\gamma(U)
                \in T$.
            \item For two $\gamma$-wosets $(S, \leq)$ and $(T, \leq')$, prove
                that there is a maximal $\gamma$-woset $(U, \leq'')$ preceding
                both w.r.t. $\preceq$. (Note: There is no need to use Zorn's
                lemma!)
            \item Prove that the maximal $\gamma$-woset found in the previous
                point in fact equals $(S, \leq)$ or $(T, \leq')$. Thus,
                $\preceq$ is a total ordering.
            \item Prove that there is a maximal $\gamma$-woset $(M, \leq)$
                w.r.t. $\preceq$. (Again, Zorn's lemma need not and should not
                be invoked.)
            \item Prove that $M = Z$.
        \end{itemize}
        Thus every set admits a well-ordering, as stated in Theorem 3.3.
    \end{problem}

    \begin{solution}
        Given $\gamma(S)$, one can begin constructing a $\gamma$-woset $(S,
        \leq)$ by including $\gamma(\emptyset)$. In some sense, $a =
        \gamma(\emptyset)$ is minimal in $S$ since no elements precede it.
        Furthermore, since every $\gamma$-woset is well-ordered, they all have a
        minimal element. That is, they all contain $\gamma(\emptyset)$. One can
        continue the construction of the $\gamma$-woset by letting the next
        element be $\gamma$ of the elements currently in the set. The
        well-ordering on the set follows naturally.

        Now suppose we have $(U, \leq'') \prec (T, \leq')$. By the definition of
        $\prec$, we have $U \subset T$. Since $T$ is well-ordered, there is some
        minimum element $a$ such that for all $b \in U$, $b <' a$. Then $a =
        \gamma(\{b \in S, b <' a\}) = \gamma(U)$.

        Given two $\gamma$-wosets $(S, \leq)$ and $(T, \leq')$, consider the set
        $R = S \cap T$ with the obvious well ordering. Indeed, since $R
        \subseteq S$ and $R \subseteq T$, $R$ precedes both w.r.t. $\preceq$.
        Furthermore, if there were any more elements then it would not satisfy
        the defining property of being a subset of both $S$ and $T$ so it is
        maximal.

        If $R = S$, then there is nothing to prove so suppose otherwise. Then $R
        \prec S$ so $\gamma(R) = a \in S$ for some $s$. If $R \prec T$ then
        $\gamma(R) = b \in T$ for some $b$. But then $a = b \in S \cap T = R$, a
        contradiction (since $\gamma(R) \notin R$ ). Thus, $R = S$ or $R = T$
        and $\preceq$ is a total ordering.

        Since $\preceq$ is a total ordering, we can construct a chain of
        $\gamma$-wosets. Let $M$ be the union of these $\gamma$-wosets with the
        ordering inherited from the wosets. Certainly each $\gamma$-woset $S
        \subseteq M$ so $M$ is maximal.

        Finally, we know $M \subseteq Z$. Suppose $Z \subsetneq M$. Then there
        exists some element $x \in Z \setminus M$. Consider $M \cup \{x\}$.
        Since $\gamma(\{x\})$ is defined, this set is a $\gamma$-woset properly
        containing $M$, contradicting the maximality of $M$. Thus, $M = Z$ so
        there is a well-ordering on $Z$.
    \end{solution}

    % Problem 3.8
    \begin{problem}
        Prove that every nontrivial finitely generated group has a maximal
        proper subgroup. Prove that $(\mathbb{Q}, +)$ has no maximal proper
        subgroup.
    \end{problem}

    \begin{solution}
        Let $\mathscr{S}$ be the set of all proper subgroups of a finitely
        generated group $G$. Then $\mathscr{S}$ is partially ordered by
        inclusion so let $\mathscr{C}$ be a chain in this poset. Let $H$ be the
        union of all subgroups in this chain. Since the chain is nonempty, there
        is one subgroup $K_0$ containing the identity, so $H$ contains the
        identity. Furthermore, suppose $x, y \in H$. Then there are subgroups
        $K_1, K_2$ with $x \in K_1$, $y \in K_2$. Suppose WLOG that $K_1
        \subseteq K_2$. Then both $x, y \in K_2$ and since $K_2$ is a subgroup,
        $xy^{-1} \in K_2 \subseteq H$. Thus $H$ is a subgroup.

        To show $H$ is a proper subgroup, suppose otherwise. In particular, $H$
        contains the generators $g_1, g_2, \ldots, g_n$ of $G$. Then there is
        some subgroup $K_n$ containing all such generators, implying that $K_n =
        G$, a contradiction. Thus, $H$ must be proper.

        Since every chain in $\mathscr{S}$ has an upper bound in $\mathscr{S}$,
        Zorn's lemma applies and $\mathscr{S}$ has a maximal element. That is,
        $G$ has a maximal proper subgroup.

        Suppose that $(\mathbb{Q}, +)$ has a maximal proper subgroup $H$.
        Then the quotient $\mathbb{Q} / H$ is simple and abelian, so it must be
        cyclic with prime order. Say $\mathbb{Q} / H \cong \mathbb{Z} /
        p\mathbb{Z}$. Choose $x \in \mathbb{Q} \setminus H$. Then $H = p
        (\frac{x}{p} + H) = x + N$, implying that $x \in N$, a contradiction.
        Thus, $\mathbb{Q}$ has no maximal proper subgroup.
    \end{solution}

    % Problem 3.9
    \begin{problem}
        Consider the rng (= ring without 1; cf. \S III.1.1) consisting of the
        abelian group $(\mathbb{Q}, +)$ endowed with the trivial multiplication
        $qr = 0$ for all $q, r \in \mathbb{Q}$. Prove that this rng has no
        maximal ideals.
    \end{problem}

    \begin{solution}
        Suppose the ring $R$ has a maximal ideal $M$. Then $M$ is also a maximal
        subgroup of $\mathbb{Q}$ (a larger subgroup would also act as an ideal).
        As shown above, $\mathbb{Q}$ does not contain maximal subgroups so
        neither can $M$ be a maximal ideal.
    \end{solution}

    % Problem 3.10
    \begin{problem}
        As shown in Exercise III.4.17, every maximal ideal in the ring of
        continuous real-valued functions on a \textit{compact} topological space
        $K$ consists of the functions vanishing of a point of $K$.

        Prove that there are maximal ideals in the ring of continuous real-value
        functions on the \textit{real line} that do not correspond to points of
        the real line in the same fashion. (Hint: Produce a proper ideal that is
        not contained in any maximal ideal corresponding to a point, and apply
        Proposition 3.5.)
    \end{problem}

    \begin{solution}
        I still don't know topology but I imagine the solution uses something
        about the fact that the real line is not compact (whatever that means).
    \end{solution}

    % Problem 3.11
    \begin{problem}
        Prove that a UFD $R$ is a PID if and only if every nonzero prime ideal
        in $R$ is maximal. (Hint: One direction is Proposition III.4.13. For the
        other, assume that every nonzero prime ideal in a UFD $R$ is maximal,
        and prove that every maximal ideal in $R$ is principal; then use
        Proposition 3.5 to relate arbitrary ideals to maximal ideals, and prove
        that every ideal of $R$ is principal.)
    \end{problem}

    \begin{solution}
        First suppose that $R$ is a PID and let $I = (a)$ be a nonzero prime
        ideal. Assume $I \subseteq J$ for an ideal $J = (b)$ of $R$. Since $a
        \in (b)$, we have $a = bc$ for some $c \in R$. But since $a$ is prime,
        we have $b \in (a)$ or $c \in (a)$. In the first case, there is nothing
        more to prove. In the second, we have $c = da$. Then
        \[
            a = bda \Longrightarrow bd = 1 \Longrightarrow (b) = (1) = R.
        \] 
        Thus, $I$ is maximal.

        Now let $R$ be a UFD such that every prime ideal is maximal. Let $I$ be
        a maximal ideal. Then $I$ is also a prime ideal of height 1. By Exercise
        2.9, $I$ is principal. Thus, every maximal ideal is principal. Now let
        $I_0$ be an arbitrary ideal. It is contained in some maximal ideal
        $\mathfrak{m}_0 = (a_0)$. In particular, every element admits a factor of
        $a$, which is irreducible (by Exercise 1.12). Then we may write $I = a_0
        J_0$ for an ideal $J_0$. If $J_0 = R$ then $I = (a_0)$ and we are done.
        Otherwise, $J_0$ is properly contained in a maximal ideal
        $\mathfrak{m}_1 = (a_1)$ so we may write $J_0 = a_1 J_1$. We may repeat
        this and it will terminate since the elements of $I$ only have
        finitely many irreducible factors. When it terminates, we find that $J_t
        = R$ so $I = (a_0 a_1 \cdots a_t)$.
    \end{solution}

    % Problem 3.12
    \begin{problem}
        Let $R$ be a commutative ring, and let $I \subseteq R$ be a proper
        ideal. Prove that the set of prime ideals containing $I$ has minimal
        elements. (These are the \textit{minimal primes} of $I.$)
    \end{problem}

    \begin{solution}
        Consider the set $\mathscr{I}$ of prime ideals of $R$ which contain $I$.
        The set is ordered by inclusion so consider a chain $\mathscr{C}$ and
        let $\mathfrak{B}$ be the intersection of the prime ideals in
        $\mathscr{C}$. Certainly $I \subseteq \mathfrak{B}$. Now we must check
        that $\mathfrak{B}$ is in fact prime. Suppose $ab \in \mathfrak{B}$ but
        neither $a$ nor $b$ is. Then there exist two prime ideals
        $\mathfrak{p}, \mathfrak{p}'$ such that $a \notin \mathfrak{p}, b \notin
        \mathfrak{p}'$ and WLOG $\mathfrak{p} \subseteq \mathfrak{p}'$. Then $a,
        b \notin \mathfrak{p}$ but $ab \in \mathfrak{p}$, contradicting that
        $\mathfrak{p}$ is prime. Thus, $\mathfrak{B}$ is prime. Since every
        chain in $\mathscr{I}$ has a lower bound, $\mathscr{I}$ has a minimal
        element.
    \end{solution}

    % Problem 3.13
    \begin{problem}
        Let $R$ be a commutative ring, and let $N$ be its nilradical (Exercise
        III.3.12). Let $r \notin N$.
        \begin{itemize}
            \item Consider the family $\mathscr{F}$ of ideals of $R$ that do not
                contain any power $r^{k}$ of $r$ for $k > 0$. Prove that
                $\mathscr{F}$ has maximal elements.
            \item Let $I$ be a maximal element of $\mathscr{F}$. Prove that $I$
                is prime.
            \item Conclude $r \notin N \Longrightarrow r$ is not in the
                intersection of all prime ideals of $R$.
        \end{itemize}
        Together with Exercise III.4.18, this shows that the nilradical of a
        commutative ring $R$ equals the intersection of all prime ideals of $R$.
    \end{problem}

    \begin{solution}
        Recall that the nilradical of a ring is the set of nilpotent elements
        (elements $a$ such that $a^{n}=0$ for some $n$). The nilradical is an
        ideal of $R$. 
        
        The family $\mathscr{F}$ of ideals not containing any power of $r^{k}$
        is ordered by inclusion. Each chain in this family has a maximal
        element, namely the union of all of the ideals in the chain. Therefore,
        by Zorn's lemma $\mathscr{F}$ has maximal elements.

        Let $I$ be a maximal element of $\mathscr{F}$ and suppose $ab \in I$ but
        $a, b \notin I$. Then the ideals $I + (a)$ and $I + (b)$ both properly
        contain $I$. By the maximality of $I$, we have $r^{m} \in I + (a)$ and
        $r^{n} \in I + (b)$. But then we find
        \[
            r^{m + n} = (s_1 + ax) (s_2 + by) = s_1s_2 + s_1 \cdot by + ax \cdot
            s_2 + ax \cdot by \in I
        \] 
        for $s_1, s_2 \in I$, a contradiction. Thus one of $a, b \in I$ so $I$
        is prime.

        Suppose $r$ is not in the nilradical of $R$. Then there is some prime
        ideal not containing any power of $r$, so $r$ is not in the
        intersection of all prime ideals. In particular, $\bigcap \mathfrak{p}
        \subseteq N$.
    \end{solution}

    % Problem 3.14
    \begin{problem}
        The \textit{Jacobson radical} of a commutative ring $R$ is the
        intersection of the maximal ideals in $R$. (Thus, the Jacobson radical
        contains the nilradical.) Prove that $r$ is in the Jacobson radical if
        and only if $1 + rs$ is invertible for every $s \in R$.
    \end{problem}

    \begin{solution}
        If $r$ is in the Jacobson radical, then it is in every maximal ideal.
        Suppose there exists some $s \in R$ such that $1 + rs$ is not
        invertible. Then $(1 + rs)$ is a proper ideal and hence is contained in
        a maximal ideal $\mathfrak{m}$. But $r \in \mathfrak{m}$ so $1 = rs - r
        \cdot s \in \mathfrak{m}$, a contradiction. Thus $1 + rs$ is invertible
        for all $s \in R$.

        Now suppose that $1 + rs$ is invertible for all $s \in R$ and let
        $\mathfrak{m}$ be a maximal ideal. If $r \notin \mathfrak{m}$ then
        $\mathfrak{m} + (r) = R$ so there exists $y \in \mathfrak{m}$ and $s \in
        (r)$ such that $rs + y = 1$. But then $y = 1 - rs$ is invertible so $1 =
        y y^{-1} \in \mathfrak{m}$, a contradiction. Thus, $r \in \mathfrak{m}$.
    \end{solution}

    % Problem 3.15
    \begin{problem}
        Recall that a (commutative) ring $R$ is Noetherian if every ideal of
        $R$ is finitely generated. Assume the seemingly weaker condition that
        every \textit{prime} ideal of $R$ is finitely generated. Let
        $\mathscr{F}$ be the family of ideals that are not finitely generated in
        $R$. You will prove $\mathscr{F} = \emptyset$.
        \begin{itemize}
            \item If $\mathscr{F} \neq 0$, prove that it has a maximal element
                $I$.
            \item Prove that $R / I$ is Noetherian.
            \item Prove that there are ideals $J_1, J_2$ properly containing $I$, such
                that $J_1 J_2 \subseteq I$.
            \item Give a structure of $R / I$ module to $I / J_1 J_2$ and $J_1 /
                J_1 J_2$.
            \item Prove that $I / J_1 J_2$ is a finitely generated $R /
                I$-module.
            \item Prove that $I$ is finitely generated, thereby reaching a
                contradiction.
        \end{itemize}
        Thus, a ring is Noetherian if and only if its \textit{prime} ideals are
        finitely generated.
    \end{problem}

    \begin{solution}
        If $\mathscr{F}$ is nonempty, it is partially ordered by inclusion. For
        each chain $\mathscr{C}$ in $\mathscr{F}$, the ideal defined as the
        union of ideals in the chain is an upper bound for $\mathscr{C}$.
        Indeed, if it were finitely generated then the generating set would be
        contained in one of the ideals, contradicting the assumption that ideals
        in $\mathscr{F}$ are not finitely generated. By Zorn's lemma,
        $\mathscr{F}$ has maximal elements. Let $I$ be one such maximal element.

        Suppose $R / I$ is not Noetherian. That is, there is some ideal of the form
        $J / I$ which is not finitely generated. Then $J$ is an ideal of $R$
        containing $I$ and it is not finitely generated. But by the maximality
        of $I$, we have $J = R$ which is finitely generated by 1, a
        contradiction. Thus $R / I$ is Noetherian.

        Since $I$ is not finitely generated, it is not prime. Thus, there exist
        elements $a, b \notin I$ with $ab \in I$. Then $J_1 = I + (a)$ and $J_2
        = I + (b)$ both properly contain $I$ (and thus are finitely generated)
        and elements of $J_1 J_2$ are of the form
        \[
            (r_1 + ax) (r_2 + by) = r_1 \cdot r_2 + r_1 \cdot by + r_2 \cdot ax
            + ab \cdot xy \in I,
        \] 
        so $J_1 J_2 \subseteq I$.

        We can give the quotient $I / J_1 J_2$ the structure of an $R / I$
        module by defining
        \[
            (r + I) x = rx
        \] 
        for $r \in R$ and $x \in I / J_1 J_2$. Indeed, since $x = a + J_1 J_2$
        for $a \in I$, we find 
        \[
            r(a + J_1 J_2) = ra + r J_1 J_2 \in \frac{I}{J_1 J_2}
        \] 
        The other module axioms can be checked easily. We can define the same
        structure on $J_1 / J_1 J_2$. 

        Recall that $J_1$ is finitely generated. Then $J_1 / J_1 J_2$ is also
        finitely generated over $R$ and hence over $R / I$. Since $R / I$ is
        Noetherian and $I / J_1 J_2$ is a submodule of $J_1 / J_1 J_2$, we find
        that $I / J_1 J_2$ is finitely generated.

        Finally, observe that $J_1 J_2 \subseteq I$ is finitely generated and $I
        / J_1 J_2$ is finitely generated. Thus, $I$ is finitely generated and we
        arrive at a contradiction. Therefore, a ring is Noetherian if and only
        if its prime ideals are finitely generated.
    \end{solution}
\end{document}
