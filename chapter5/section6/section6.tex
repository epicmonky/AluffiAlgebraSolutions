\documentclass[../../master.tex]{subfiles}

\begin{document}
\section{Further remarks and examples}

% Problem 6.1
\begin{problem}
    Generalize the CRT for two ideals, as follows.
    Let $I, J$ be ideals in a commutative ring $R$; prove that there is an exact sequence of $R$-modules
    \[
        \begin{tikzcd}
                0 & {I \cap J} & R & {\frac{R}{I} \times \frac{R}{J}} & {\frac{R}{I + J}} & 0
                \arrow[from=1-1, to=1-2]
                \arrow[from=1-2, to=1-3]
                \arrow["\varphi", from=1-3, to=1-4]
                \arrow[from=1-4, to=1-5]
                \arrow[from=1-5, to=1-6]
        \end{tikzcd}
    \]
    where $\varphi$ is the natural map. 
    (Also, explain why this implies the first part of Theorem 6.1, for $k = 2$.)
\end{problem}

\begin{solution}
    Let the map for $I \cap J \to R$ be the inclusion.
    Since it is injective, its kernel is 0 and the first part of the sequence is exact.
    Furthermore, its image is merely $I \cap J$.
    Now consider the map $\varphi$ which sends $r \in R$ ot $(r + I, \; r + J)$.
    Certainly the kernel of this map is the set of elements in $R$ which are in both $I$ and $J$;
    that is, the kernel is $I \cap J$.
    The image of this map is merely the set $\{r + I, \; r + J) \mid r \in R\}$.
    Note that this may not be the entirety of $(R / I) \times (R / J)$.
    Define a map from $(R / I) \times (R / J)$ to $R / (I + J)$ which sends $(a + I, b + J)$ to $a - b + (I + J)$.
    One can easily verify that this is indeed a homomorphism of modules.
    Note that the kernel of this image is precisely the image of $\varphi$.
    Furthermore, the homomorphism is surjective;
    and arbitrary $a + (I + J)$ is mapped to by $(a + I, 0 + J)$.
    With these homomorphisms, we have shown the existence of such an exact sequence of $R$-modules.

    In the case where $I + J = (1)$, then the map $\varphi$ is surjective.
    This can be seen by noting that there exist $i \in I$, $j \in J$ such that $i + j = 1$.
    Then for all $(r + I, s + J)$, we have
    \begin{align*}
        \varphi(rj + si) &= (rj + I, si + J) \\
                   &= (rj + ri + I, si + sj + J) \\
                   &= (r(j + i) + I, s(i + j) + J) \\
                   &= (r + I, s + J).
    \end{align*}
    Thus, we have recovered the desired statement.
\end{solution}

% Problem 6.2
\begin{problem}
    Let $R$ be a commutative ring, and let $a \in R$ be an element such that $a^2 = a$.
    Prove that $R \cong R / (a) \times R / (1-a)$.

    Show that the multiplication in $R$ endows the ideal $(a)$ with a \textit{ring} structure, with $a$ as the identity.
    Prove that $(a) \cong R / (1-a)$ as rings.
    Prove that $R \cong (a) \times (1-a)$ as rings.
\end{problem}

\begin{solution}
    Consider the natural homomorphism $\varphi$ from $R$ to $R / (a) \times R / (1 - a)$ which sends $r$ to $(r + (a), r + (1-a))$. 
    The kernel of this homomorphism is the set of elements in $(a) \cap (1-a)$.
    Let $x \in (a) \cap (1-a)$ so $x = ra = s(1-a)$ for some $r, s \in R$.
    Multiplying both sides by $a$ yields $ra^2 = sa - sa^2$. 
    But then we have
    \[
    x = ra = sa - sa = 0.
    \]
    Thus, $(a) \cap (1-a) = 0$ so $\varphi$ is injective.
    To see that it is surjective, note that $(a) + (1-a) = (1)$.
    By Exercise 6.1, the natural homomorphism is surjective.
    Therefore, $\varphi$ is a bijective ring homomorphism and thus an isomorphism.

    The ideal $(a)$ is already an abelian group under addition.
    To see that it is also a ring under multiplication in $R$ with $a$ as an identity, note that for $ax \in (a)$, we have $a \cdot ax = a^2x = ax$.
    Distributivity is inherited from $R$, making $(a)$ a ring.

    Consider the natural map from $(a)$ to $R / (1-a)$ which sends $ax$ to $ax + (1-a)$.
    This map is surjective as any $x + (1-a) = ax + (x - ax) + (1 - a) = ax + (1 - a) = \varphi(ax)$.
    Furthermore, the kernel of this map is the set of elements $ax \in (1-a)$.
    But $ax = (1-a)y \Longrightarrow a(x + y) = y \Longrightarrow a(x + y) = ay \Longrightarrow ax = 0$ so $x = 0$ and the homomorphism is injective.
    Thus, we have a bijective homomorphism from $(a) \to R / (1-a)$ so the rings are isomorphic.
    The third isomorphism is relatively similar to show.
\end{solution}

% Problem 6.3
\begin{problem}
    Recall (Exercise III.3.15) that a ring $R$ is called  \textit{Boolean} if $a^2 = a$ for all $a \in R$.
    Let $R$ be a finite Boolean ring;
    prove that $R \cong \mathbb{Z} / 2\mathbb{Z} \times \cdots \times \mathbb{Z} / 2\mathbb{Z}$.
\end{problem}

\begin{solution}
    Suppose $R$ has only two elements; then $R \cong \mathbb{Z} / 2\mathbb{Z}$.
    If $R$ has more than two elements, then there is some idempotent $e \notin \{0, 1\}$.
    Per Exercise 6.2, we can split $R$ into $(e) \times (1 - e)$, both of which have strictly fewer elements than $R$.
    Repeating this process will eventually yield a direct product in which each component is isomorphic to $\mathbb{Z} / 2\mathbb{Z}$.
\end{solution}

% Problem 6.4
\begin{problem}
    Let $R$ be a finite commutative ring, and let $p$ be the smallest prime dividing $|R|$.
    Let $I_1, \ldots, I_k$ be proper ideals such that $I_i + I_j = (1)$ for $i \neq j$.
    Prove that $k \leq \log_p |R|$.
    (Hint: Prove $|R|^{k-1} \leq |I_1| \cdots |I_k| \leq (|R| / p)^{k}$.)
\end{problem}

\begin{solution}
    To do.
\end{solution}

% Problem 6.5
\begin{problem}
    Show that the map $\mathbb{Z}[x] \to \mathbb{Z}[x] / (2) \times \mathbb{Z}[x] / (x)$ is not surjective.
\end{problem}

\begin{solution}
    Consider the element $(1, 2) \in \mathbb{Z}[x] / (2) \times \mathbb{Z}[x] / (x)$.
    Suppose some polynomial $f \in \mathbb{Z}[x]$ is sent to this element.
    Since $f \equiv 2 (\mod x)$, this forces the constant term of $f$ to be 2.
    However, if this were the case then the constant term of $f \mod 2$ would be 0, a contradiction.
    Thus, there is no polynomial mapped to this element and the mapping is not surjective.
\end{solution}

% Problem 6.6
\begin{problem}
    Let $R$ be a UFD.
    \begin{itemize}
        \item Let $a, b \in R$ such that $\gcd(a, b) = 1$.
            Prove that $(a) \cap (b) = (ab)$.
        \item Under the hypotheses of Corollary 6.4 (but only assuming that $R$ is a UFD) prove that the function $\varphi$ is injective.
    \end{itemize}
\end{problem}

\begin{solution}
    To do.
\end{solution}

% Problem 6.7
\begin{problem}
    Find a polynomial $f \in \mathbb{Q}[x]$ such that $f \equiv 1 \mod (x^2 + 1)$ and $f \equiv x \mod x^{100}$.
\end{problem}

\begin{solution}
    To do.
\end{solution}

% Problem 6.8
\begin{problem}
    Let $n \in \mathbb{Z}$ be a positive integer and $n = p_1^{a_1} \cdots  p_r^{a_r}$ its prime factorization.
    By the classification theorem for finite abelian groups (or, in fact, simplier considerations; cf. Exercise II.4.9)
    \[
        \frac{\mathbb{Z}}{(n)} \cong \frac{\mathbb{Z}}{(p_1^{a_1})} \times \cdots \times \frac{\mathbb{Z}}{(p_r^{a_r})}
    \]
    \textit{as abelian groups}.
    \begin{itemize}
        \item Use the CRT to prove that this is in fact a \textit{ring} isomorphism.
        \item Prove that
            \[
                \left( \frac{\mathbb{Z}}{(n)}\right)^{*} \cong \left( \frac{\mathbb{Z}}{(p_1^{a_1})}\right)^{*} \times \cdots \times \left( \frac{\mathbb{Z}}{(p_r^{a_r})}\right)^{*}
            \]
            (recall that $(\mathbb{Z} / n\mathbb{Z})^{*}$ denotes the group of units of $\mathbb{Z} / n\mathbb{Z}$).
        \item Recall (Exercise II.6.14) that \textit{Euler's $\phi$-function} $\phi(n)$ denotes the number of positive integers $\leq n$ that are relatively prime to $n$. 
            Prove that
            \[
                \phi(n) = p_1^{a_1 - 1}(p_1 - 1) \cdots p_r^{a_r - 1}(p_r - 1).
            \]
    \end{itemize}
\end{problem}

\begin{solution}
    To do.
\end{solution}

% Problem 6.9
\begin{problem}
    Let $I$ be a nonzero ideal of $\mathbb{Z}[i]$. 
    Prove that $\mathbb{Z}[i] / I$ is finite.
\end{problem}

\begin{solution}
    To do.
\end{solution}

% Problem 6.10
\begin{problem}
    Let $z, w \in \mathbb{Z}[i]$.
    Show that if $z$ and $w$ are associates, then $N(z) = N(w)$.
    Show that if $w \in (z)$ and $N(z) = N(w)$, then $z$ and $w$ are associates.
\end{problem}

\begin{solution}
    To do.
\end{solution}

% Problem 6.11
\begin{problem}
    Prove that the irreducible elements in $\mathbb{Z}[i]$ are, up to associates:
    $1 + i$;
    the integer primes congruent to $3 \mod 4$;
    and the elements $a \pm bi$ with $a^2 + b^2$ an integer prime congruent to $1 \mod 4$.
\end{problem}
\end{document}
