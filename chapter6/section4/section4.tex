\documentclass[../../master.tex]{subfiles}

\begin{document}
\section{Presentations and resolutions}

% Problem 4.1
\begin{problem}
    Prove that if $R$ is an integral domain and $M$ is an $R$-module, then $\text{Tor}(M)$ is a submodule of $M$.
    Give an example showing that the hypothesis that $R$ is an integral domain is necessary.
\end{problem}

\begin{solution}
    Clearly $\text{Tor}(M) \neq \emptyset$ since $0 \in \text{Tor}(M)$.
    Now suppose $a, b \in \text{Tor}(M)$.
    Then $\exists r, s \in R$ such that $ra = sb = 0$.
    Therefore, $rs(a + b) = s(ra) + r(sb) = 0$ so $a + b \in \text{Tor}(M)$.
    Similarly, for all $s \in R$, we have $r(sa) = s(ra) = 0$ so $sa \in \text{Tor}(M)$.
    Thus, $\text{Tor}(M)$ is a submodule of $M$.

    To see that $R$ is an integral domain is necessary, consider $R = M = \mathbb{Z}/6\mathbb{Z}$.
    Then $\text{Tor}(M) = \{0, 2, 3, 4\}$.
    But then $2 + 3 = 5 \notin \text{Tor}(M)$ so $\text{Tor}(M)$ is not a submodule of $M$.
\end{solution}

% Problem 4.2
\begin{problem}
    Let $M$ be a module over an integral domain $R$, and let $N$ be a torsion-free module.
    Prove that $\text{Hom}_R(M, N)$ is torsion-free.
    In particular, $\text{Hom}_R(M, R)$ is torsion-free.
    (We will run into this fact again; see Proposition VIII.5.16.)
\end{problem}

\begin{solution}
    Let $f \in \text{Hom}_R(M, N)$ and suppose $r \cdot f = 0$ for some $r \in R$.
    That is, for all $m \in M$, 
    \[
        r \cdot f(m) = 0.
    \]
    But since $f(m) \in N$, $f(m)$ is not a torsion element and $r = 0$.
    Thus, $\text{Hom}_R(M, N)$ is torsion-free.
\end{solution}

% Problem 4.3
\begin{problem}
    Prove that an integral domain $R$ is a PID if and only if every submodule of $R$ itself is free.
\end{problem}

\begin{solution}
    Note that the submodules of $R$ are its ideals.
    If $R$ is a PID, then every submodule of $R$ is generated by a single element.
    That is, every submodule of $R$ has a basis, making it free.
    Now suppose every submodule of $R$ is free.
    Recall that if $M$ is a submodule of $R$, then $\dim(M) \leq \dim(R)$.
    In particular, $\dim(M) \leq 1$.
    Thus, every ideal of $R$ is generated by at most one element so $R$ is a PID.
\end{solution}

% Problem 4.4
\begin{problem}
    Let $R$ be a commutative ring and $M$ an $R$-module.
    \begin{itemize}
        \item Prove that $\text{Ann}(M)$ is an ideal of $R$.
        \item If $R$ is an integral domain and $M$ is finitely generated, prove that $M$ is torsion if and only if $\text{Ann}(M) \neq 0$.
        \item Give an example of a torsion module $M$ over an integral domain, such that $\text{Ann}(M) = 0$.
            (Of course this example cannot be finitely generated!)
    \end{itemize}
\end{problem}

\begin{solution}
    Let $a, b \in \text{Ann}(M)$.
    That is, for all $m \in M$, we have $am = bm = 0$.
    Then $(a + b)m = am + bm = 0$ so $a + b \in \text{Ann}(M)$.
    Similarly, for all $r \in R$, we find $(ra) \cdot m = r \cdot (am) = r \cdot 0 = 0$ so $ra \in \text{Ann}(M)$, proving that it is an ideal.

    If $\text{Ann}(M) \neq 0$, there exists an $r \in R$ such that $rm = 0$ for all $m \in M$.
    Thus, every element of $M$ is torsion.
    Now suppose $M$ is torsion.
    That is, for every element $m_i \in M$, there exists an $r_i \in R, r_i \neq 0$ such that $r_i m_i = 0$.
    In particular, there is such an $r_i$ for each generator of $M$.
    Then we may consider $s$ to be the product of these $r_i$.
    Since $R$ is an integral domain, $s \neq 0$.
    Furthermore, since all $m \in M$ is a linear combination of these generators, we have $sm = 0$ for all $m \in M$.
    Thus, $s \in \text{Ann}(M)$.

    Let $R = \mathbb{Z}$ and consider the $\mathbb{Z}$-module
    \[
        M = \bigoplus_{i=1}^{\infty} \frac{\mathbb{Z}}{2^{i}\mathbb{Z}}.
    \]
    Then each element of $M$ has the form
    \[
        a = (a_1 + \mathbb{Z}/2\mathbb{Z}, a_2 + \mathbb{Z}/2^2\mathbb{Z}, \ldots, a_k + \mathbb{Z}/2^{k}\mathbb{Z}, 0, 0, \ldots)
    \]
    so $2^{k}a = 0$ which makes $M$ a torsion module.
    Now suppose $r \in \text{Ann}(M)$.
    Choose $k \in \mathbb{Z}$ such that $r < 2^{k}$ and consider the element
    \[
        a = (0, 0, \ldots, 1 + \mathbb{Z}/2^{k}\mathbb{Z}, 0, 0, \ldots).
    \]
    Then $ra = 0$, but since $r < 2^{k}$, it must be the case that $r = 0$.
    Thus, $\text{Ann}(M) = 0$.
\end{solution}

% Problem 4.5
\begin{problem}
    Let $M$ be a module over a commutative ring $R$.
    Prove that an ideal $I$ of $R$ is the annihilator of an element of $M$ if and only if $M$ contains an isomorphic copy of $R / I$ (viewed as an $R$-module).

    The \textit{associated primes} of $M$ are the prime ideals among the ideals $\text{Ann}(m)$, for $m \in M$.
    The set of the associated primes of a module $M$ is denoted $\text{Ass}_R(M)$.
    Note that every prime in $\text{Ass}_R(M)$ contains $\text{Ann}_R(M)$.
\end{problem}

\begin{solution}
    Let $I$ be the annihilator of an element $m \in M$.
    That is, for all $r \in I$, $rm = 0$.
    Consider the map $\varphi : R \to M$ which sends $r$ to $rm$.
    The kernel of this map is the set of $r$ such that $rm = 0$.
    That is, $\text{ker}(\varphi) = I$ so, by the isomorphism theorem,
    \[
        \frac{R}{I} \cong \text{im}(\varphi) \subseteq M.
    \]
    Now suppose $M$ contains a submodule $N \cong R / I$ for an ideal $I \subseteq R$ and let $\varphi: R \to M$ be the composition of the natural projection and inclusion.
    We claim that $I$ is the annihilator of $m = \varphi(1)$.
    Indeed, if $r \in I$ then
    \[
        rm = r\varphi(1) = \varphi(r) = i(\pi(r)) = i(0) = 0
    \]
    so $r \in \text{Ann}(m)$ and $I \subseteq \text{Ann}(m)$.
    Similarly, if $r \in \text{Ann}(m)$ then
    \[
        rm = 0 \Longrightarrow \varphi(r) = 0 \Longrightarrow \pi(r) = 0
    \]
    so $r \in I$ and $\text{Ann}(m) = I$.
\end{solution}

% Problem 4.6
\begin{problem}
    Let $M$ be a module over a commutative ring $R$, and consider the family of ideals $\text{Ann}(m)$, as $m$ ranges over the nonzero elements of $M$.
    Prove that the maximal elements in this family are prime ideals of $R$.
    Conclude that if $R$ is Noetherian, then $\text{Ass}_R(M) \neq \emptyset$ (cf. Exercise 4.5).
\end{problem}

\begin{solution}
    Let $\mathfrak{m}$ be a maximal element in this family of ideals, say $\mathfrak{m} = \text{Ann}(m)$.
    Suppose $rs \in \mathfrak{m}$.
    If $r \in \mathfrak{m}$ then there is nothing to prove so suppose otherwise.
    We know $rs \cdot m = 0$ but $rm \neq 0$.
    Thus, $s \in \text{Ann}(rm)$.
    Furthermore, it is clear that $\text{Ann}(m) \subseteq \text{Ann}(rm)$ since if $am = 0$ then $a (rm) = 0$.
    Then, by the maximality of $\text{Ann}(m)$, we have $\text{Ann}(m) = \text{Ann}(rm)$ so $s \in \mathfrak{m}$ and the ideal is prime.

    If $R$ is Noetherian, then every family of ideals has a maximal element.
    In particular, given a module $M$, the family of ideals $\text{Ann}(m)$ as $m$ ranges over the nonzero elements of $M$ has a maximal element which is a prime ideal.
    Such prime ideals are elements of $\text{Ass}_R(M)$, meaning the set is nonempty.
\end{solution}

% Problem 4.7
\begin{problem}
    Let $R$ be a commutative Noetherian ring, and let $M$ be a finitely generated module over $R$.
    Prove that $M$ admits a finite series
    \[
        M = M_0 \supsetneq M_1 \supsetneq \cdots \supsetneq M_m = \langle0\rangle
    \]
    in which all quotients $M_i / M_{i+1}$ are of the form $R / \mathfrak{p}$ for some prime ideal $\mathfrak{p}$ of $R$.
    (Hint: Use Exercises 4.5 and 4.6 to show that $M$ contains an isomorphic copy $M'$ of $R / \mathfrak{p}_1$ for some prime $\mathfrak{p}_1$.
    Then do the same with $M/M'$, producing an $M'' \supseteq M'$ such that $M''/M' \cong R/\mathfrak{p_2}$ for some prime $\mathfrak{p_2}$.
    Why must this process stop after finitely many steps?)
\end{problem}

\begin{solution}
    By Exercise 4.6, $\text{Ass}_R(M) \neq \emptyset$ so let $\mathfrak{p}_1 \in \text{Ass}_R(M)$.
    Then by Exercise 4.5, $M$ contains a submodule $M' \cong R / \mathfrak{p}_1$.
    Now consider $M/M'$, which is also an $R$-module.
    Thus, $\text{Ass}_R(M/M') \neq \emptyset$ and there is a submodule $M'' \supseteq M'$ of $M$ such that $M''/M' \cong R/\mathfrak{p}_2$ for some prime $\mathfrak{p}_2$.
    That is, we have a chain
    \[
        M \supsetneq M'' \supsetneq M' \supsetneq \langle 0 \rangle
    \]
    such that $M'' / M' \cong R / \mathfrak{p}_2$ and $M' / 0 \cong R / \mathfrak{p}_1$ for prime ideals of $R$.
    Since $M$ is finitely generated over a Noetherian ring, it is a Noetherian module and all chains of submodules eventually stabilize.
    Thus, iterating this process yields a finite series whose quotients are isomorphic to $R / \mathfrak{p}$ for prime ideals.
\end{solution}

% Problem 4.8
\begin{problem}
    Let $R$ be a commutative Noetherian ring, and let $M$ be a finitely generated module over $R$.
    Prove that every prime in $\text{Ass}_R(M)$ appears in the list of primes produced by the procedure presented in Exercise 4.7.
    (If $\mathfrak{p}$ is an associated prime, then $M$ contains an isomorphic copy $N$ of $R / \mathfrak{p}$. 
    With notation as in the hint in Exercise 4.7, prove that either $\mathfrak{p}_1 = \mathfrak{p}$ or $N \cap M' = 0$.
    In the latter case, $N$ maps isomorphically to a copy of $R/\mathfrak{p}$ in $M/M'$;
    iterate the reasoning.)

    In particular, if $M$ is a finitely generated module over a Noetherian ring, then $\text{Ass}(M)$ is \textit{finite}.
\end{problem}

\begin{solution}
    Let $\mathfrak{p} \in \text{Ass}_R(M)$ and suppose $R / \mathfrak{p} \cong N \subseteq M$.
    In particular, if $x \in M$ such that $\text{Ann}_R(x) = \mathfrak{p}$, then $N = Rx$.
    If $Rx \cap M' \neq 0$, say $rx = m$ is a nonzero element, then $\text{Ann}_R(m) \subseteq \mathfrak{p}$.
    But by definition, $\text{Ann}_R(m) = \mathfrak{p}_1$ so $\mathfrak{p}_1 \subseteq \mathfrak{p}$.
    The reverse inclusion can be shown similarly.
    Thus, if $M'$ and $N$ have nontrivial intersection, $\mathfrak{p} = \mathfrak{p}_1$.
    Otherwise, $M' \cap N = 0$.
    In the latter case, $N$ is isomorphic to some $R / \mathfrak{p}$ in $M / M' \cong R / \mathfrak{p}_2$.
    Thus, we may repeat the above reasoning which eventually terminates.
\end{solution}

% Problem 4.9
\begin{problem}
    Let $M$ be a module over a commutative Noetherian ring $R$.
    Prove that the union of all annihilators of nonzero elements equals the union of all associated primes of $M$.
    (Use Exercise 4.6)

    Deduce that the \textit{union} of the associated primes of a Noetherian ring $R$ (viewed as a module over itself) equals the set of zero-divisors of $R$.
\end{problem}

\begin{solution}
    Certainly every associated prime is the annihilator of some element $m \in M$, so we only need to show the other direction.
    If $I \in \text{Ann}_R(m)$ for some $m \in M$, then $I \subseteq \mathfrak{p}$ for some maximal element in the family of annihilators of elements of $M$.
    By Exercise 4.6, $\mathfrak{p}$ is prime in $R$ so $I$ is in the union of all associated primes, proving the result.
\end{solution}

% Problem 4.10
\begin{problem}
    Let $R$ be a commutative Noetherian ring.
    One can prove that the minimal primes of $\text{Ann}(M)$ (cf. Exercise V.1.9) are in $\text{Ass}(M)$.
    Assuming this, prove that the \textit{intersection} of the associated primes of a Noetherian ring $R$ (viewed as a module over itself) equals the nilradical of $R$.
\end{problem}

\begin{solution}
    Recall that the nilradical of $R$ is the set of elements $r \in R$ such that $r^{n} = 0$ for some $n > 0$.
    If $x \in \text{nil}(R)$ then $x$ is in the intersection of all prime ideals of $R$, particularly the intersection of associated primes of $R$.
    Now suppose $x$ is in the intersection of the associated primes of $R$.
    Then it is in the minimal primes of $\text{Ann}(R)$.
    Since every prime ideal contains a minimal prime ideal, the intersection of all prime ideals equals the intersection of all minimal prime ideals.
    Thus, $x \in \text{nil}(R)$.
\end{solution}

% Problem 4.11
\begin{problem}
    Review the notion of presentation \textit{of a group}, and relate it to the notion of presentation introduced in \S 4.2.
\end{problem}

\begin{solution}
    Recall that a presentation of a group $G$ is an explicit isomorphism
    \[
        G \cong \frac{F(A)}{R}
    \]
    for a set $A$ and a subgroup $R$ of relations.
    A presentation of an $R$-module $M$ is an exact sequence
    \[
    \begin{tikzcd}
        R^{n} & R^{m} & M & 0
        \arrow[from=1-1, to=1-2]
        \arrow[from=1-2, to=1-3] 
        \arrow[from=1-3, to=1-4] 
    \end{tikzcd}
    \]
    In particular, if $G$ is an abelian group, then we have the exact sequence
    \[
    \begin{tikzcd}
        R & F(A) & G
        \arrow[from=1-1, to=1-2] 
        \arrow[from=1-2, to=1-3] 
    \end{tikzcd}
    \]
    where $R$ is also a free module since it is a submodule of $F(A)$.
\end{solution}

% Problem 4.12
\begin{problem}
    Let $\mathfrak{p}$ be a prime ideal of a polynomial ring $k[x_1, \ldots, x_n]$ over a field $k$, and let $R = k[x_1, \ldots, x_n] / \mathfrak{p}$.
    Prove that every finitely generated module over $R$ has a finite presentation.
\end{problem}

\begin{solution}
    Let $M$ be a finitely generated module over $R$.
    Then there is a surjection $\pi: R^{a} \to M$ for some $a \in \mathbb{Z}$ where $\text{ker}(\pi)$ is a submodule of $R^{a}$.
    Since $k$ is a field, by Hilbert's basis theorem, $k[x_1, \ldots, x_n]$ is also Noetherian.
    But then $R$ is a quotient of a Noetherian ring and is Noetherian itself.
    Thus, $\text{ker}(\pi)$ is finitely generated and there is an exact sequence
    \[
    \begin{tikzcd}
        R^{b} & \text{ker}(\pi) & 0
        \arrow[from=1-1, to=1-2] 
        \arrow[from=1-2, to=1-3] 
    \end{tikzcd}
    \]
    which yields the exact sequence
    \[
    \begin{tikzcd}
        R^{b} & R^{a} & M & 0
        \arrow[from=1-1, to=1-2] 
        \arrow[from=1-2, to=1-3] 
        \arrow[from=1-3, to=1-4] 
    \end{tikzcd}
    \]
    so $M$ is finitely presented.
\end{solution}

% Problem 4.13
\begin{problem}
    Let $R$ be a commutative ring.
    A tuple $(a_1, a_2, \ldots, a_n)$ of elements of $R$ is a \textit{regular sequence} if $a_1$ is a non-zero-divisor in $R$, $a_2$ is a non-zero-divisor modulo $(a_1)$, $a_3$ is a non-zero-divisor modulo $(a_1, a_2)$, and so on.

    For $a, b$ in $R$, consider the following complex of $R$-modules:
    \[
        \tag{*}
        \begin{tikzcd}
            0 & R & R \oplus R & R & \frac{R}{(a, b)} & 0
            \arrow[from=1-1, to=1-2] 
            \arrow[from=1-2, to=1-3, "d_2"] 
            \arrow[from=1-3, to=1-4, "d_1"]
            \arrow[from=1-4, to=1-5, "\pi"]
            \arrow[from=1-5, to=1-6] 
        \end{tikzcd}
    \]
    where $\pi$ is the canonical projection, $d_1(r, s) = ra + sb$, and $d_2(t) = (bt, -at)$.
    Put otherwise, $d_1$ and $d_2$ correspond, respectively, to the matrices
    \[
    \begin{pmatrix}
        a & b
    \end{pmatrix}, \quad
    \begin{pmatrix}
        b \\
        -a
    \end{pmatrix}.
    \]
    \begin{itemize}
        \item Prove that this is indeed a complex, for every $a$ and $b$.
        \item Prove that if $(a, b)$ is a regular sequence, this complex is \textit{exact}.
    \end{itemize}
    The complex (*) is called the \textit{Koszul complex} of $(a, b)$.
    Thus, when $(a, b)$ is a regular sequence, the Koszul complex provides us with a free resolution of the module $R / (a,b)$.
\end{problem}

\begin{solution}
    First we verify that this is a complex for all $a$ and $b$.
    Certainly the image of the zero map is a subset of $\text{ker}(d_2)$.
    Let $(r, s) \in \text{im}(d_2)$.
    Then $(r, s) = (bt, -at)$ for some $t \in R$ and
    \[
        d_1(bt, -at) = bta - bta = 0
    \]
    so $\text{im}(d_2) \subseteq \text{ker}(d_1)$.
    Furthermore, let $ra + sb \in \text{im}(d_1)$.
    Then $\pi(ra + sb) = 0 \in R / (a, b)$ so $\text{im}(d_1) \subseteq \text{ker}(\pi)$.
    Finally, the image of $\pi$ is clearly a subset of the kernel of the zero map.
    Thus, we have verified that this is in fact a complex.

    Now suppose $(a, b)$ is a regular sequence.
    Let $t \in \text{ker}(d_2)$.
    That is, $(bt, -at) = (0, 0)$.
    Since $a \neq 0$, it must be the case that $t = 0$ so $t$ is in the image of the zero map, proving the two are equal.

    Now suppose $(r, s) \in \text{ker}(d_1)$.
    Then $ra + sb = 0$.
    Consider the equation mod $a$: $sb = 0$.
    Since $b$ is not a zero-divisor in $R / (a)$, $s \in (a)$ so $s = at$ for some $t \in R$.
    Then we have $ra + atb = 0$, or $(r + tb)a = 0$.
    Since $a$ is not a zero-divisor in $R$, it must be the case that $r + tb = 0$, or $r = -tb$.
    That is, $(r, s) = (-tb, at) \in \text{im}(d_2)$ so the two sets must be equal.

    Now let $x \in \text{ker}(\pi)$ so $\pi(x) = 0 \Longrightarrow x = ra + sb$ for $r, s \in R$.
    Then $x = d_1(r, s) \in \text{im}(d_1)$ and the two sets are equal.

    Finally, the projection is surjective and the kernel of the zero map is all of its domain so the last map is exact.
\end{solution}

% Problem 4.14
\begin{problem}
    A Koszul complex may be defined for any sequence $a_1, \ldots, a_n$ of elements of a commutative ring $R$.
    The case $n = 2$ seen in Exercise 4.13 and the case $n = 3$ reviewed here will hopefully suffice to get a gist of the general construction;
    the general case will be given in Exercise VIII.4.22.

    Let $a, b, c \in R$. 
    Consider the following complex:
    \[
    \begin{tikzcd}
        0 & R & R \oplus R \oplus R & R \oplus R \oplus R & R & \frac{R}{(a, b, c)} & 0
        \arrow[from=1-1, to=1-2] 
        \arrow[from=1-2, to=1-3, "d_3"] 
        \arrow[from=1-3, to=1-4, "d_2"] 
        \arrow[from=1-4, to=1-5, "d_1"] 
        \arrow[from=1-5, to=1-6, "\pi"] 
        \arrow[from=1-6, to=1-7] 
    \end{tikzcd}
    \]
    where $\pi$ is the canonical projection and the matrices for $d_1, d_2, d_3$ are, respectively,
    \[
    \begin{pmatrix}
        a & b & c
    \end{pmatrix}, \quad
    \begin{pmatrix}
        0 & -c & -b \\
        -c & 0 & a \\
        b & a & 0
    \end{pmatrix}, \quad
    \begin{pmatrix}
        a \\
        -b \\
        c
    \end{pmatrix}.
    \]
    \begin{itemize}
        \item Prove that this is indeed a complex, for every $a, b, c$.
        \item Prove that if $(a, b, c)$ is a regular sequence, this complex is \textit{exact}.
    \end{itemize}
    Koszul complexes are very important in commutative algebra and algebraic geometry.
\end{problem}

\begin{solution}
    Clearly the image of the zero map is in the kernel of $d_3$.
    Let $(ar, -br, cr) \in \text{im}(d_3)$.
    Then
    \[
        \begin{pmatrix}
            0 & -c & -b \\
            -c & 0 & a \\
            b & a & 0
        \end{pmatrix} \cdot
        \begin{pmatrix}
            ar \\
            -br \\
            cr
        \end{pmatrix} =
        \begin{pmatrix}
            bcr - bcr \\
            -acr + acr \\
            abr - abr
        \end{pmatrix} =
        \begin{pmatrix}
            0 \\
            0 \\
            0
        \end{pmatrix}
    \]
    so $(ar, -br, cr) \in \text{ker}(d_2)$.
    Now let $(-cs - bt, -cr + at, br + as) = d_2(r, s, t) \in \text{im}(d_2)$.
    Then
    \[
    \begin{pmatrix}
        a & b & c
    \end{pmatrix} \cdot
    \begin{pmatrix}
        -cs -bt \\
        -cr + at \\
        br + as
    \end{pmatrix} =
    -acs - abt - bcr + abt + bcr + acs = 0
    \]
    so $\text{im}(d_2) \subseteq \text{ker}(d_1)$.
    Now consider $ra + sb + ct = d_1(r, s, t) \in \text{im}(d_1)$.
    We have
    \[
        \pi(ra + sb + ct) = 0
    \]
    by definition of the projection to a quotient so $\text{im}(d_1) \subseteq \text{ker}(\pi)$.
    The image of projection is obviously a subset of the kernel of the zero map.
    Thus, this is indeed a complex.

    Now suppose $(a, b, c)$ is a regular sequence.
    If $r \in \text{ker}(d_3)$ then $d_3(r) = (0, 0, 0)$.
    In particular, $ar = 0$ and since $a$ is not a zero-divisor, we must have $r = 0$ so $r$ is in the image of the zero map, hence it equals the image of $d_3$.

    If $(r_1, r_2, r_3) \in \text{ker}(d_2)$, then
    \[
    \begin{pmatrix}
        0 & -c & -b \\
        -c & 0 & a \\
        b & a & 0
    \end{pmatrix} \cdot
    \begin{pmatrix}
        r_1 \\
        r_2 \\
        r_3
    \end{pmatrix} =
    \begin{pmatrix}
        -cr_2 - br_3 \\
        -cr_1 + ar_3 \\
        br_1 + ar_2
    \end{pmatrix} =
    \begin{pmatrix}
        0 \\
        0 \\
        0
    \end{pmatrix}.
    \]
    The third equation mod $a$ yields $br_1 = 0$ in $R / (a)$.
    Since $b$ is not a zero-divisor in this ring, we must have $r_1 = at$ for some $t \in R$.
    Substituting this back into the third equation, we have $abt + ar_2 = 0$, or $r_2 = -bt$ (since $a$ is not a zero-divisor in $R$).
    Substituting this into the second equation yields $-act + ar_3 = 0$ so $r_3 = ct$ by the same reasoning as above.
    But then
    \[
        (r_1, r_2, r_3) = (at, -bt, ct) = d_3(t)
    \]
    so $\text{im}(d_3) = \text{ker}(d_2)$.

    If $(r_1, r_2, r_3) \in \text{ker}(d_1)$, then
    \[
    \begin{pmatrix}
        a & b & c
    \end{pmatrix} \cdot
    \begin{pmatrix}
        r_1 \\
        r_2 \\
        r_3
    \end{pmatrix} =
    ar_1 + br_2 + cr_3 = 0.
    \]
    Considering this equation mod $(a, b)$ yields $cr_3 = 0$ in $R / (a, b)$ and since $c$ is not a zero-divisor in this ring, we must have $r_3 \in (a, b)$ or $r_3 = ar + bs$ for $r, s \in R$.
    Substituting this into the equation yields
    \[
    ar_1 + br_2 + acr + bcs = 0
    \]
    which we can consider mod $a$ to yield $br_2 + bcs = 0$ in $R / (a)$, or $r_2 + cs = at$ for some $t \in R$.
    That is, $r_2 = at - cs$, which we can again substitute into the equation to obtain
    \[
    ar_1 + abt - bcs + acr + bcs = 0
    \]
    which yields $a(r_1 + bt + cs) = 0$ so $r_1 = -bt - cs$.
    But then
    \[
        (r_1, r_2, r_3) = (-bt -cs, at - cs, ar + bs) = d_2(r, s, t)
    \]
    so $\text{im}(d_2) = \text{ker}(d_1)$.

    Finally, suppose $x \in \text{ker}(\pi)$.
    That is, $x \in (a, b, c)$.
    Then $x = ra + bs + ct = d_1(r, s, t)$ and $\text{im}(d_1) = \text{ker}(\pi)$.
    The last equality is obvious.
    Thus, the complex is exact.
\end{solution}

% Problem 4.15
\begin{problem}
    View $\mathbb{Z}$ as a module over the ring $R = \mathbb{Z}[x, y]$, where $x$ and $y$ act by 0.
    Find a free resolution of $\mathbb{Z}$ over $R$.
\end{problem}

\begin{solution}
    Recall that a free resolution of an $R$-module $M$ is an exact complex
    \[
    \begin{tikzcd}
        \cdots & R^{m_3} & R^{m_2} & R^{m_1} & R^{m_0} & M & 0.
        \arrow[from=1-1, to=1-2] 
        \arrow[from=1-2, to=1-3] 
        \arrow[from=1-3, to=1-4] 
        \arrow[from=1-4, to=1-5] 
        \arrow[from=1-5, to=1-6] 
        \arrow[from=1-6, to=1-7] 
    \end{tikzcd}
    \]
    Consider the complex
    \[
    \begin{tikzcd}
        0 & R & R^2 & R & \mathbb{Z} & 0
        \arrow[from=1-1, to=1-2] 
        \arrow[from=1-2, to=1-3, "d_2"] 
        \arrow[from=1-3, to=1-4, "d_1"] 
        \arrow[from=1-4, to=1-5, "\pi"] 
        \arrow[from=1-5, to=1-6] 
    \end{tikzcd}
    \]
    where $d_1$ and $d_2$ correspond to the matrices
    \[
    \begin{pmatrix}
        x & y
    \end{pmatrix}, \quad
    \begin{pmatrix}
        y \\
        -x
    \end{pmatrix}
    \]
    and $\pi$ is the natural projection to the constant term.
    It is easy to see that this is in fact a complex.
    To see that it is exact, let $f(x, y) \in \ker(\pi)$.
    That is, $f$ has no constant term, so it may be written as
    \[
        f = 
        \begin{pmatrix}
            x & y
        \end{pmatrix} \cdot
        \begin{pmatrix}
            f_1(y) \\
            f_2(x)
        \end{pmatrix}
    \]
    so $f \in \text{im}(d_1)$.
    Similarly, if $(f, g) \in \ker(d_1)$ then $fx + gy = 0$.
    Gathering terms, this is only possible if $f = h y$ and $g = -h x$ for some $h \in R$.
    That is, $(f, g) = d_2(h)$ so $\ker(d_1) = \im(d_2)$ and the sequence is exact.
    Thus, this is a free resolution of $\mathbb{Z}$ over $R$.
\end{solution}

% Problem 4.16
\begin{problem}
    Let $\varphi : R^{n} \to R^{m}$ and $\psi : R^{p} \to R^{q}$ be two $R$-module homomorphisms, and let
    \[
    \varphi \oplus \psi : R^{n} \oplus R^{p} \to R^{m} \oplus R^{q}
    \]
    be the morphism induced on direct sums.
    Prove that
    \[
        \operatorname{coker}(\varphi \oplus \psi) = \operatorname{coker} \varphi \oplus \operatorname{coker} \psi.
    \]
\end{problem}

\begin{solution}
    First note that
    \[
        \im(\varphi \oplus \psi) = \im(\varphi) \oplus \im(\psi).
    \]
    Now consider the map
    \[
        R^{m} \oplus R^{q} \to \frac{R^{m}}{\im \varphi} \oplus \frac{R^{q}}{\im \psi}.
    \]
    The kernel of this map is $\im(\varphi) \oplus \im(\psi)$ so by the first isomorphism theorem, we have
    \[
        \frac{R^{m} \oplus R^{q}}{\im(\varphi \oplus \psi)} \cong \frac{R^{m}}{\im \varphi} \oplus \frac{R^{q}}{\im \psi}
    \]
    and $\coker(\varphi \oplus \psi) = \coker(\varphi) \oplus \coker(\psi)$.
\end{solution}

% Problem 4.17
\begin{problem}
    Determine (as a better known entity) the module represented by the matrix
    \[
    \begin{pmatrix}
        1 + 3x & 2x & 3x \\
        1 + 2x & 1 + 2x - x^2 & 2x \\
        x & x^2 & x
    \end{pmatrix}
    \]
    over the polynomial ring $k[x]$ over a field.
\end{problem}

\begin{solution}
    We perform Gaussian elimination to reduce the matrix to a simpler but equivalent form.
    Subtracting three times the third row from the first yields a unit in the $1, 1$ position so we are reduced to the $2 \times 2$ matrix
    \[
    \begin{pmatrix}
        1 + 2x - x^2 & 2x \\
        x^2 & x
    \end{pmatrix}.
    \]
    Adding the second row to the first and subtracting $\frac{2}{3}$ times the second column from the first yields another unit in the $1, 1$ position so we have reduced the matrix to
    \[
    \begin{pmatrix}
        x
    \end{pmatrix}.
    \]
    The module represented by the original matrix is isomorphic to the cokernel of the homomorphism
    \[
        \varphi : k[x] \to k[x]
    \]
    which maps 1 to $x$.
    That is,
    \[
        M \cong \coker \varphi \cong \frac{k[x]}{(x)} \cong k.
    \]
\end{solution}
\end{document}
