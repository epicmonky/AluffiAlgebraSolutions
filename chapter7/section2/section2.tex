\documentclass[../../master.tex]{subfiles}

\begin{document}
\section{Algebraic closure, Nullstellensatz, and a little algebraic geometry}

% Problem 2.1
\begin{problem}
    Prove Lemma 2.1.
    \begin{proposition}[Lemma 2.1] 
        For a field $K$, the following are equivalent:
        \begin{itemize}
            \item $K$ is algebraically closed.
            \item $K$ has no nontrivial algebraic extensions.
            \item If $K \subseteq L$ is any extension and $\alpha \in L$ is algebraic over $K$, then $\alpha \in K$.
        \end{itemize}
    \end{proposition}
\end{problem}

\begin{solution}
    First we show that the first point implies the third point.
    Suppose $K$ is algebraically closed and let $K \subseteq L$ be an extension and consider $\alpha \in L$ to be algebraic over $K$.
    That is, there is a polynomial $p(x) \in K[x]$ such that $p(\alpha) = 0$.
    But since $K$ is algebraically closed, $p$ factors as a product of linear terms.
    Then $p(\alpha) = 0$ implies that $p$ has a factor of $(x - \alpha)$, which in turn implies that $\alpha \in K$.

    To see that the third point implies the second, let $L$ be an algebraic extension of $K$.
    That is, every $\alpha \in L$ is algebraic over $K$, so $\alpha \in K$.
    But then $L \subseteq K$, and obviously $K \subseteq L$, hence the two are equal and $K$ has no nontrivial algebraic extensions.

    Finally, we show that the second point implies the first by proving the contrapositive.
    Suppose $K$ is not algebraically closed.
    That is, there is an irreducible polynomial which is not linear, say $p(x)$.
    Then $K[x] / (p(x)) = L$ is a nontrivial algebraic extension of $K$ (since the coset of $x$ is a root of $p$ in $L$).
    Therefore, if $K$ has no nontrivial algebraic extensions, it must be algebraically closed.
\end{solution}

% Problem 2.2
\begin{problem}
    Let $k \subseteq \bar{k}$ be an algebraic closure, and let $L$ be an intermediate field.
    Assume that every polynomial $f(x) \in k[x] \subseteq L[x]$ factors as a product of linear terms in $L[x]$.
    Prove that $L = \bar{k}$.
\end{problem}

\begin{solution}
    Certainly $L \subseteq \bar{k}$ so we only need to show the reverse inclusion.
    Let $\alpha \in \bar{k}$.
    Then $\alpha$ is algebraic over $k$, so there is a polynomial $p(x) \in k[x]$ such that $p(\alpha) = 0$.
    Then $p(x)$ factors as a product of linear terms in $L[x]$, so $p(x)$ contains a factor of $(x - \alpha)$ in $L[x]$.
    That is, $\alpha \in L$, proving that $L = \bar{k}$.
\end{solution}

% Problem 2.3
\begin{problem}
    Prove that if $k$ is a countable field, then so is $\bar{k}$.
\end{problem}

\begin{solution}
    Recall that $\bar{k}$ may be considered as the field containing all elements which are algebraic over $k$.
    Then there is a surjection from $F[x] \times \mathbb{N}$ onto $\bar{k}$ which sends a polynomial to its roots (the second component counts the number of roots, where excess points are sent to 0).
    The product of countable sets is countable (simply consider a table formed from the product and count diagonally), so it suffices to show that $F[x]$ is countable.
    A polynomial of degree $n$ in $F[x]$ can be identified with a sequence of length $n$ with elements in $F$.
    Thus, $F[x]$ is equal to the countable union $F \cup F^2 \cup F^3 \cup \cdots$.
    We know each $F^{k}$ is countable since the cartesian product of countable sets is countable.
    Finally, a countable union of countable sets is countable.
    Simply form a table where $a_{ij}$ is the $j$-th element of the set $S_i$.
    Thus, $F[x]$ is countable, so $F[x] \times \mathbb{N}$ is countable, so $\bar{k}$ is countable.
\end{solution}

% Problem 2.4
\begin{problem}
    Let $k$ be a field, let $c_1, \ldots, c_m \in k$ be distinct elements, and let $\lambda_1, \ldots, \lambda_m$ be nonzero elements of $k$.
    Prove that 
    \[
    \frac{\lambda_1}{x - c_1} + \cdots + \frac{\lambda_1}{x - c_m} \neq 0.
    \]
    (This fact is used in the proof of Theorem 2.9.)
\end{problem}

\begin{solution}
    The left side can be rewritten with a common denominator as
    \[
        \frac{\lambda_1 (x-c_2) \cdots (x-c_m) + \cdots + \lambda_m (x-c_1) \cdots (x-c_{m-1})}{(x-c_1) \cdots (x-c_m)}.
    \]
    Let $f(x)$ be the numerator and $g(x)$ be the denominator.
    It suffices to show that $f(x) \neq 0$ (I think?)
    Certainly $f(c_1) = \lambda_1(c_1 - c_2) \cdots (c_1 - c_m)$ and this must be nonzero since all of the $c_i$ are distinct and the $\lambda_i$ are nonzero.
\end{solution}

% Problem 2.5
\begin{problem}
    Let $K$ be a field, let $A$ be a subset of $K[x_1, \ldots, x_n]$ and let $I$ be the ideal generated by $A$.
    Prove that $\mathscr{V}(A) = \mathscr{V}(I)$ in $\mathbb{A}^{n}_K$.
\end{problem}

\begin{solution}
    It is obvious that $\mathscr{V}(I) \subseteq \mathscr{V}(A)$.
    Now let $p \in \mathscr{V}(A)$.
    Let $f, g \in A$.
    Then $(f + g)(p) = f(p) + g(p) = 0$.
    Similarly, let $h \in K[x_1, \ldots, x_n]$.
    Then $(h \cdot f)(p) = h(p) \cdot f(p) = 0$.
    Thus, $p \in \mathscr{V}(I)$, showing the reverse inclusion.
\end{solution}

% Problem 2.6
\begin{problem}
    Let $K$ be your favorite infinite field.
    Find examples of subsets $S \subseteq \mathbb{A}^{n}_K$ which cannot be realized as $V(I)$ for any ideal $I \subseteq K[x_1, \ldots, x_n]$.
    Prove that if $K$ is a finite field, then every subset $S \subseteq \mathbb{A}^{n}_K$ equals $V(I)$ for some ideal $I \subseteq K[x_1, \ldots, x_n]$.
\end{problem}

\begin{solution}
    Consider the set $S = \mathbb{A}^2_K - \{(0, 0\}$.
    Suppose this can be realized as $V(I)$ for some ideal $I \subseteq K[x, y]$.
    That is, for all $f(x) \in I$, we have $f(p) = 0$ for all $p \in S$.
    But since $S$ is infinite, $f(x) = 0$ for infinitely many points but has finite degree.
    Therefore, we must have $f(x) = 0$.
    However, then $f(x) \notin I$ since $f(0, 0) = 0$ and $0 \notin S$.

    Now suppose $K$ is a finite field and let $S \subseteq \mathbb{A}^{n}_K$.
    Note that $S$ must be finite.
    Consider the polynomial
    \[
        f(x_1, \ldots, x_n) = (x - p_1) (x-p_2) \cdots (x-p_r)
    \]
    for all $p_i \in S$.
    Let $I$ be the ideal generated by $f$.
    We claim that $S = V(I)$.
    Certainly if $p \in V(I)$, then $f(p) = 0$, so $p = p_i$ for some $p_i \in S$.
    That is, $V(I) \subseteq S$.
    The reverse inclusion follows from Exercise 2.5.
    Thus, the two are equal.
\end{solution}

% Problem 2.7
\begin{problem}
    Let $K$ be a field and $n$ a nonnegative integer.
    Prove that the set of algebraic subsets of $\mathbb{A}^{n}_K$ is the family of closed sets of a topology on $\mathbb{A}^{n}_K$.
\end{problem}

\begin{solution}
    We simply show that $\mathbb{A}^{n}_K$ is a topological space with the closed sets defined as $V(I)$ for some ideal $I \subseteq K[x_1, \ldots, x_n]$.
    Indeed, $\emptyset = V(1)$ and $\mathbb{A}^{n}_K = V(0)$ so the empty set and the whole set are closed.

    If $V(I)$ and $V(J)$ are closed, then $V(I) \cup V(J) = V(IJ)$.
    Indeed, if $p \in V(I) \cup V(J)$, then $f(p) = 0$ for all $f \in I$ or $g(p) = 0$ for all $g \in J$, so $(f_1 g_1 + \cdots f_k g_k)(p) = 0$.
    For the other other direction, suppose $f(p) g(p) = 0$ for all $f \in I$ and $g \in J$.
    If $f(p) = 0$ for all $f \in I$, then $p \in V(I)$ by definition.
    Suppose now that there exists some $f_0 \in I$ such that $f_0(p) \neq 0$.
    Then for any $g \in J$, since we assumed that $f_0(p) g(p) = 0$, it must be the case that $g(p) = 0$.
    Thus, $p \in V(J)$, so $p \in V(I) \cup V(J)$.
    That is, the union of two closed sets is closed.

    Finally, if $V(I)$ and $V(J)$ are closed, then $V(I) \cap V(J) = V(I + J)$.
    Indeed, if $p \in V(I) \cap V(J)$, then $f(p) = 0$ for all $f \in I$ and $g(p) = 0$ for all $g \in J$.
    Then $f(p) + g(p) = 0$ so $p \in V(I + J)$.
    For the other direction, suppose $f(p) + g(p) = 0$ for all $f \in I, g \in J$.
    In particular, we have $f(p) = 0$ since $0 \in J$.
    Similarly, we have $g(p) = 0$ for all $g \in J$.
    Thus, $p \in V(I) \cap V(J)$ and the two sets are equal.
    That is, the intersection of two closed sets is closed.

    This is the Zariski topology on $\mathbb{A}^{n}_K$.
\end{solution}

% Problem 2.8
\begin{problem}
    With notation as in Definition 2.13:
    \begin{itemize}
        \item Prove that the set $\sqrt{I}$ is an ideal of $R$.
        \item Prove that $\sqrt{I}$ corresponds to the nilradical of $R / I$ via the correspondence between ideals of $R / I$ and ideals of $R$ containing $I$.
        \item Prove that $\sqrt{I}$ is in fact the intersection of all prime ideals of $R$ containing $I$.
        \item Prove that $I$ is radical if and only if $R / I$ is reduced.
    \end{itemize}
\end{problem}

\begin{solution}
    Recall that the set $\sqrt{I} = \{r \in R \mid \exists k \geq 0, r^{k} \in I\}$.
    Certainly $0 \in \sqrt{I}$ as $0^{1} = 0 \in I$.
    Suppose $a, b \in \sqrt{I}$.
    That is, $a^{k} \in I$ and $b^{l} \in I$ for some $k, l \geq 0$.
    Then $(a + b)^{k + l}$ can be expanded via the binomial theorem, and each term in the expansion contains a factor of either $a^{k}$ or $b^{l}$.
    Thus, each of the terms in the expansion is in $I$ so the entire expression is in $I$, showing that $a + b \in \sqrt{I}$.
    Now suppose $a \in \sqrt{I}$.
    That is, $a^{k} \in I$ for some $k \geq 0$.
    Then for all $r \in R$, we have $(ra)^{k} = r^{k} \cdot a^{k} \in I$ since $a^{k} \in I$.
    Thus, $ra \in \sqrt{I}$, proving that $\sqrt{I}$ is an ideal.

    By the third isomorphism theorem, the ideals of $R / I$ are in bijection with the ideals of $R$ containing $I$.
    Since $\sqrt{I}$ is an ideal of $R$ which contains $I$, it corresponds to an ideal of $R / I$.
    In particular, it corresponds to the ideal containing the elements $x + I \in R / I$ such that $(x + I)^{n} = x^{n} + I = I$, which is precisely the nilradical of $R / I$.

    Let $x \in \sqrt{I}$.
    That is, $x^{k} \in I$ for some $k \geq 0$.
    Now let $\mathfrak{p}$ be a prime ideal of $R$ containing $I$.
    Then $x^{k} \in \mathfrak{p}$.
    But then $x \cdot x^{k - 1} \in \mathfrak{p}$, so either $x \in \mathfrak{p}$ or $x^{k-1} \in \mathfrak{p}$.
    In the first case we are done; otherwise we repeat inductively, showing that $x \in \mathfrak{p}$.
    Thus, $\sqrt{I}$ is in the intersection of all prime ideals containing $I$.
    For the other direction, suppose $x \notin \sqrt{I}$.
    That is, $x^{k} \notin I$ for any $k \geq 0$.
    Then the set $S = \{1, x, x^2, x^3, \ldots\}$ is multiplicatively closed and the localization $(R / I)_S$ is nonzero.
    Thus, it contains a prime ideal and the preimage of this prime ideal in $R / I$ is a prime ideal of $R / I$ which does not contain $x$.
    Hence, the preimage of this is a prime ideal of $R$ containing $I$, which does not contain $x$, so $x$ is not in the intersection of prime ideals of $R$ containing $I$.

    Finally, suppose $I$ is radical.
    That is, $I = \sqrt{I}$.
    In particular, since $\sqrt{I}$ is the preimage of the nilradical in $R / I$, the niradical is trivial in $R / I$.
    Then, $R / I$ has no nilpotents.
    The other direction works exactly the same way.
\end{solution}

% Problem 2.9
\begin{problem}
    Prove that every affine algebraic set equals $\mathscr{V}(I)$ for a \textit{radical} ideal $I$.
\end{problem}

\begin{solution}
    Let $S \subseteq \mathbb{A}^{n}_K$ be an affine algebraic set.
    We know $\mathscr{I}(S)$ is radical, so let $I = \mathscr{I}(S)$.
    We claim that $S = \mathscr{V}(I)$.
    Suppose $p \in S$.
    Then $f(p) = 0$ for all $f \in I$, so $p \in \mathscr{V}(I)$, showing that $S \subseteq \mathscr{V}(I)$.
    Now suppose $p \notin S$.
    That is, there exists some $f \in I$ such that $f(p) \neq 0$.
    Then $p \notin \mathscr{V}(I)$, hence $\mathscr{V}(I) \subseteq S$ and the two sets are equal.
    That is, $S = \mathscr{V}(I)$ for a radical ideal $I$ of $K[x_1, \ldots, x_n]$.
\end{solution}

% Problem 2.10
\begin{problem}
    Prove that every ideal in a Noetherian ring contains a power of its radical.
\end{problem}

\begin{solution}
    Let $I$ be an ideal in a Noetherian ring $R$.
    Since $\sqrt{I}$ is an ideal of $R$, it is finitely generated: say $\sqrt{I} = (a_1, \ldots, a_n)$.
    Since $a_i \in \sqrt{I}$, there exists some $k_i \geq 0$ such that $a_i^{k_i} \in I$.
    Let $K = k_1 + k_2 + \cdots + k_n$.
    Note that $\sqrt{I}^{K}$ is generated by elements of the form $a_1^{r_1} \cdots a_n^{r_n}$ where $r_1 + r_2 + \cdots + r_n = K$.
    Then we must have $r_i \geq k_i$ for some $i$, or else we contradict the fact that the sum of the $r_i$ is $K$.
    That is, each element $a_1^{r_1} \cdots a_n^{r_n} \in I$, so $\sqrt{I}^{K} \subseteq I$.
\end{solution}

% Problem 2.11
\begin{problem}
    Assume a field is \textit{not} algebraically closed.
    Find a reduced finite-type $K$-algebra which is not the coordinate ring of any affine algebraic set.
\end{problem}

\begin{solution}
    Since $K$ is not algebraically closed, there exists a nonlinear irreducible polynomial $p(x) \in K[x]$ which has a root not contained in $K$.
    Consider $L = K[x] / (p(x))$.
    Certainly $L$ is a finite-type $K$-algebra.
    To see that it is reduced, suppose note that since $p(x)$ is irreducible, the ideal $(p(x))$ is prime, hence radical, hence $L$ is reduced.
    Finally, $L$ is not the coordinate ring of an algebraic set $S \subseteq \mathbb{A}^{1}_K$.
    Indeed, suppose $(p(x)) = \mathscr{I}(S)$ for some algebraic set $S$.
    In particular, $p(\alpha) = 0$ for all $\alpha \in S$.
    But then $p(x)$ factors into linear terms in $K[x]$, contradicting our choice of $p(x)$.
\end{solution}

% Problem 2.12
\begin{problem}
    Let $K$ be an infinite field.
    A \textit{polynomial function} on an affine algebraic set $S \subseteq \mathbb{A}^{n}_K$ is the restriction to $S$ of (the evaluation function of) a polynomial $f(x_1, \ldots, x_n) \in K[x_1, \ldots, x_n]$.
    Polynomial function on an algebraic $S$ manifestly form a ring and in fact a $K$-algebra.
    Prove that this $K$-algebra is isomorphic to the coordinate ring of $S$.
\end{problem}

\begin{solution}
    Consider the map sending a polynomial $f \in K[x_1, \ldots, x_n]$ to $f |_S$.
    Since the codomain is a ring under pointwise addition and multiplication, this becomes a surjective ring homomorphism.
    The kernel of this map consists of the set of polynomials which vanish when restricted to $S$.
    In particular, these maps are precisely the ideal $\mathscr{I}(S)$.
    Thus, we find
    \[
        K[S] \cong \frac{K[x_1, \ldots, x_n]}{\mathscr{I}(S)}
    \]
    where $K[S]$ denotes the ring of polynomials on $S$.
\end{solution}

% Problem 2.13
\begin{problem}
    Let $K$ be an algebraically closed field.
    Prove that every reduced commutative $K$-algebra of finite type is the coordinate ring of an algebraic set $S$ in some affine space $\mathbb{A}^{n}_K$.
\end{problem}

\begin{solution}
    Let $L$ be a reduced commutative $K$-algebra of finite type.
    That is, there is a surjection $\pi : K[x_1, \ldots, x_n] \to L$ so
    \[
        L \cong \frac{K[x_1, \ldots, x_n]}{\ker(\pi)}.
    \]
    Furthermore, since $L$ is reduced, we find $\ker(\pi)$ is radical.
    Then, since $K$ is algebraically closed, $\mathscr{V}(\ker(\pi))$ is an algebraic subset $S \subseteq \mathbb{A}^{n}_K$.
    That is, $L$ is the coordinate ring of $S$.
\end{solution}

% Problem 2.14
\begin{problem}
    Prove that, over an algebraically closed field $K$, the points of an algebraic set $S$ correspond to the maximal ideals of the coordinate ring $K[S]$ of $S$, in such a way that if $p$ corresponds to the maximal ideal $\mathfrak{m}_p$, then the value of the function $f \in K[S]$ at $p$ equals the coset of $f$ in $K[S] / \mathfrak{m}_p \cong K$.
\end{problem}

\begin{solution}
    By the correspondence theorems, the maximal ideals of $K[S]$ are of the form $\mathfrak{m}_p + \mathscr{I}(S)$ for some maximal ideal $\mathfrak{m}_p \subseteq K[x_1, \ldots, x_n]$.
    Then by the Nullstellensatz, $\mathfrak{m}_p = (x_1 - c_1, \ldots, x_n - c_n)$, so it is natural to consider the map
    \[
        (c_1, \ldots, c_n) \mapsto (\overline{x_1 - c_1}, \ldots, \overline{x_n - c_n})
    \]
    where $\overline{x_i - c_i}$ denotes the image of $x_i - c_i$ in $K[S]$.

    Consider the map which sends $f \in K[S]$ to $f(p) \in K$.
    It is quick to verify that this map is indeed well-defined (since $f$ is technically an equivalence class of polynomials).
    The kernel of this map is the set of polynomials $g$ on $K[S]$ such that $g(p) = 0$.
    But $g(p) = 0$ if and only if $g \in \mathfrak{m}_p$.
    In particular, we have an isomorphism
    \[
        \frac{K[S]}{\mathfrak{m}_p} \cong K
    \]
    via the evaluation map, so $f(p)$ is equal to the coset of $f$ in this quotient.
\end{solution}

% Problem 2.15
\begin{problem}
    Let $K$ be an algebraically closed field.
    An algebraic subset $S$ of $\mathbb{A}^{n}_K$ is \textit{irreducible} if it \textit{cannot} be written as the union of two algebraic subsets properly contained in it.
    Prove that $S$ is irreducible if and only if its ideal $\mathscr{I}(S)$ is prime, if and only if its coordinate ring $K[S]$ is an integral domain.

    An irreducible algebraic set is `all in one piece', like $\mathbb{A}^{n}_K$ itself, and unlike (for example) $\mathscr{V}(xy)$ in the affine plane $\mathbb{A}^{n}_K$ with coordinates $x, y$.
    Irreducible affine algebraic sets are called (affine algebraic) \textit{varieties}.
\end{problem}

\begin{solution}
    The second and third statements are equivalent by the definition of the coordinate ring and a prime ideal.
    Thus, we only need to show that the first is equivalent to these.
    If $\mathscr{I}(S)$ is not prime, then there exist $f, g \notin \mathscr{I}(S)$ such that $fg \in \mathscr{I}(S)$.
    We claim that
    \[
        S = (S \cap \mathscr{V}(f)) \cup (S \cap \mathscr{V}(g)).
    \]
    The direction $\supseteq$ is clear.
    Now suppose $p \in S$.
    Then, since $(fg)(p) = 0$, we must have either $f(p) = 0$ or $g(p) = 0$.
    WLOG, suppose $f(p) = 0$.
    Then $p \in S \cap \mathscr{V}(f)$, proving $\subseteq$.
    Thus, the sets are equal and $S$ is reducible.

    Now suppose $V$ is reducible so $V = V_1 \cup V_2$ with $V_i \subset V$.
    Then $\mathscr{I}(V_i) \supset \mathscr{I}(V)$.
    That is, there exist $f_i \in \mathscr{I}(V_i) \setminus \mathscr{I}(V)$ such that $f_1 f_2 \in \mathscr{I}(V)$, hence $\mathscr{I}(V)$ is not prime.
\end{solution}

% Problem 2.16
\begin{problem}
    Let $K$ be an algebraically closed field.
    The field of rational functions $K(x_1, \ldots, x_n)$ is the field of fractions of $K[\mathbb{A}^{n}_K] = K[x_1, \ldots, x_n]$;
    every rational function $\alpha = \frac{F}{G}$ (with $G \neq 0$ and $F, G$ relatively prime) may be viewed as defining a function on the open set $\mathbb{A}^{n}_K \setminus \mathscr{V}(G)$;
    we say that $\alpha$ is `defined' for all points in the complement of $\mathscr{V}(G)$.

    Let $G \in K[x_1, \ldots, x_n]$ be irreducible.
    The set of rational functions that are defined in the complement of $\mathscr{V}(G)$ is a subring of $K(x_1, \ldots, x_n)$.
    Prove that this subring may be identified with the \textit{localization} of $K[\mathbb{A}^{n}_K]$ at the multiplicative set $\{1, G, G^2, G^3, \ldots\}$.
    (Use the Nullstellensatz.)

    The same considerations may be carried out for any irreducible algebraic set $S$, adopting as field of `rational functions' $K(S)$ the field of fractions of  the integral domain $K[S]$.
\end{problem}

\begin{solution}
    Let $S = \mathscr{V}(G)$ so that the ring of rational functions defined in the complement of $\mathscr{V}(G)$ is the field of fractions of $K[S]$.
    That is, we define $K(S) = \{\frac{f}{g} \mid f, g \in K[S], g \neq 0\} / \sim$ where $\frac{f_1}{g_1} \sim \frac{f_2}{g_2}$ iff $f_1 g_2 - f_2 g_1 = 0 \in K[S]$.
    Since $G$ is irreducible, and $K[\mathbb{A}^{n}_K]$ is a UFD, $(G)$ is prime and $S$ is irreducible.
    Consider now the condition $g \neq 0$ in the ring of rational functions, which implies that $g$ does not vanish anywhere on the complement of $S$.
    We claim that the set of all functions which do not vanish in the complement of $S$ is equal to the set $\{1, G, G^2, \ldots\}$.
    The $\supseteq$ direction follows easily.
    Now suppose $g$ is a function which does not vanish on the complement of $S$.
    If $g$ vanishes everywhere on $S$, then $g = G^{k}$.
    If $g$ vanishes on a subset of $S$, then $g$ must be constant $\neq 0$ because $S$ is irreducible.
    Thus, the two sets are equal and the identification via localization follows naturally.
\end{solution}

% Problem 2.17
\begin{problem}
    Let $K$ be an algebraically closed field, and let $\mathfrak{m}$ be a maximal ideal of $K[x_1, \ldots, x_n]$, corresponding to a point $p$ of $\mathbb{A}^{n}_K$.
    A \textit{germ} of a function at $p$ is determined by an open set containing $p$ and a function defined on that open set;
    in our context (dealing with rational functions and where the open set may be taken to be the complement of a function that does not vanish at $p$) this is the same  information as a rational function defined at $p$, in the sense of Exercise 2.16.

    Show how to identify the ring of germs with the localization $K[\mathbb{A}^{n}_K]_\mathfrak{m}$ (defined in Exercise V.4.11).

    As in Exercise 2.16, the same discussion can be carried out for any algebraic set.
    This is the origin of the name `localization': localizing the coordinate ring of a variety $V$ at the maximal ideal corresponding to a point $p$ amounts to considering only functions defined in a neighborhood of $p$, thus studying $V$ `locally', `near $p$'.
\end{problem}

\begin{solution}
    The complement of $\mathfrak{m}$, that is, the set of polynomials which do not vanish at $p$, is a multiplicatively closed set.
    Let $A$ denote the ring of germs at $p$.
    This is equivalent to the ring of rational functions defined at $p$, or the set of functions $\frac{f}{g}$ such that $g(p) \neq 0$.
    The localization $K[\mathbb{A}^{n}_K]_\mathfrak{m}$ is the ring of fractions whose denominators are functions not in $\mathfrak{m}$, or functions which do not vanish at $p$.
    There is a natural surjection from $A \to K[\mathbb{A}^{n}_K]_\mathfrak{m}$.
    To see that it is an isomorphism, note that a germ $f$ can be identified by its value at $p$.
    In particular, if $\frac{f_1}{g_1} = \frac{f_2}{g_2}$ in the localization, then there exists some $h$ which vanishes at $p$ such that $p(f_1 g_2 - f_2 g_1) = 0$.
    If this is the case, then $f_1 g_2 - f_2 g_1$ vanish in a neighborhood $U$ around $p$, so $\frac{f_1}{g_1} = \frac{g_2}{g_2}$ are equivalent as germs.
\end{solution}

% Problem 2.18
\begin{problem}
    Let $K$ be an algebraically closed field.
    Consider the two `curves' $C_1 : y = x^2, C_2 : y^2 = x^3$ in $\mathbb{A}^2_K$ (pictures of the real points of these algebraic sets are shown in Example 2.12).
    \begin{itemize}
        \item Prove that $K[C_1] \cong K[t] = K[\mathbb{A}^{1}_K]$, while $K[C_2]$ may be identified with the subring $K[t^2, t^3]$ of $K[t]$ consisting of polynomials $a_0 + a_2 t^2 + \cdots + a_d t^{d}$ with zero $t$-coefficient.
            (Note that every polynomial in $K[x, y]$ may be written as $f(x) + g(x)y + h(x, y) (y^2 - x^3)$ for uniquely determined polynomials $f(x), g(x), h(x, y)$.)
        \item Show that $C_1, C_2$ are both irreducible.
        \item Prove that $K[C_1]$ is a UFD, while $K[C_2]$ is not.
        \item Show that the Krull dimension of both $K[C_1]$ and $K[C_2]$ is 1.
            (This is why these sets would be called `curves'.
            You may use the fact that maximal chains of prime ideals in $K[x, y]$ have length 2.)
        \item The origin $(0, 0)$ is in both $C_1, C_2$ and corresponds to the maximal ideals $\mathfrak{m}_1$, resp., $\mathfrak{m}_2$, in $K[C_1]$, resp., $K[C_2]$, generated by the classes of $x$ and $y$.
        \item Prove that the localization $K[C_1]_{\mathfrak{m}_1}$ is a DVR.
            Prove that the localization $K[C_2]_{\mathfrak{m}_2}$ is \textit{not} a DVR.
            (Note that the relation $y^2 = x^3$ still holds in this ring;
            prove that $K[C_2]_{\mathfrak{m}_2}$ is not a UFD.)
    \end{itemize}
    The fact that a DVR admits a local parameter, that is, a single generator for its maximal ideal, is a good algebraic translation of the fact that a curve such as $C_1$ has a single, smooth branch through $(0, 0)$.
    The maximal ideal of $K[C_2]_{\mathfrak{m}_2}$ cannot be generated by just one element, as the reader may verify.
\end{problem}

\begin{solution}
    We find that $\mathscr{V}(C_1) = (y - x^2)$ so
    \[
        K[C_1] = \frac{K[x, y]}{(y - x^2)} \cong K[t] 
    \]
    by mapping $x \mapsto t$ and $y \mapsto t^2$.
    Similarly, we find that $\mathscr{V}(C_2) = (y^2 - x^3)$ so
    \[
        K[C_2] = \frac{K[x, y]}{(y^2 - x^3)} \cong K[t^2, t^3].
    \]
    To see the latter isomorphism, note that every polynomial in $K[x, y]$ may be written as $f(x) + g(x)y + h(x, y)(y^2 - x^3)$ where $\deg(g) < 3$.
    Consider the map which sends $x \mapsto t^2$ and $y \mapsto t^3$.
    Clearly $t$ is not in the image of this map, and the kernel of this map is generated by $(t^3)^2 - (t^2)^3 = y^2 - x^3$.

    Recall that an algebraic set is irreducible if and only if its ideal $\mathsf{I}(S)$ is prime, if and only if $K[S]$ is an integral domain.
    Since both $K[C_1]$ and $K[C_2]$ are quotients by prime ideals, they are integral domains.

    Note that $K[C_1] = K[t]$ is a Euclidean domain (where the valuation is given by the degree of a polynomial), so it is a UFD.
    However, we find that in $K[C_2] = K[t^2, t^3]$ we have $(t^2)^3 = (t^3)^2$ so we have two factorizations of $t^{6}$, hence the ring is not a UFD.

    Since $K[C_1]$ is a PID and is not a field, it must have Krull dimension 1.
    Similarly, since $K[C_2]$ is not a field, it must have Krull dimension at least 1.
    Suppose it has dimension $n \geq 2$.
    Then there is a chain of prime ideals of length $n + 1$ in $K[x, y]$.
    But this implies that the dimension of $K[x, y]$ is at least 3, a contradiction.
    Thus, the Krull dimension of $K[C_2]$ is 1.

    It is easy to see that $(0, 0)$ is in both $C_1$ and $C_2$ since $0 = 0^2$ and $0^2 = 0^3$.
    The corresponding maximal ideals generated by this point are $\mathfrak{m}_1 = (t)$ and $\mathfrak{m}_2 = (t^2, t^3)$.

    The localization $K[C_1]_{\mathfrak{m}_1} = \{\frac{f}{g} \mid g \notin (t)\}$. 
    Since $K[t]$ is a UFD and $t$ is irreducible, we can express $f = t^{k} \cdot h$ for some $k \geq 0$ and $h \in K[t]$ such that $t \nmid h$.
    Then we define a valuation $v(\frac{f}{g}) = k$ and it easy to check that this is indeed a discrete valuation.
    On the other hand, the localization $K[C_2]_{\mathfrak{m}_2} = \{\frac{f}{g} \mid g \notin(t^2, t^3)\}$ is not a DVR.
    Indeed, a DVR must be a UFD, but as noted earlier, $t^{6} = (t^2)^3 = (t^3)^2$, both of which are irreducible in the localization.
    Thus, $K[C_2]_{\mathfrak{m}_2}$ is not a DVR.
\end{solution}

% Problem 2.19
\begin{problem}
    Prove that the fields of rational functions (Exercise 2.16) of the curves $C_1$ and $C_2$ of Exercise 2.18 are isomorphic and both have transcendence degree 1 over $k$ (cf. Exercise 1.27).

    This is another reason why we should think of $C_1$ and $C_2$ as `curves'.
    In fact, it can be proven that the Krull dimension of the coordinate ring of a variety equals the transcendence degree of its field of rational functions.
    This is a consequence of \textit{Noether's normalization theorem}, a cornerstone of commutative algebra.
\end{problem}

\begin{solution}
    Recall that the field of rational functions of the curve $C_1 : y = x^2$ can be identified with the localization of $K[x, y]$ at the set $S = \{1, f, f^2, f^3, \ldots\}$ where $f = y - x^2$.
    Similarly, the ring of rational functions of the curve $C_2$ is identified with the localization of $K[x, y]$ at the set $T = \{1, g, g^2, g^3, \ldots\}$ where $g = y^2 - x^3$.
    We must show that $S^{-1}K[x, y] \cong T^{-1}K[x, y]$.
    The map 
    \[
        \frac{h(x, y)}{f^{k}} \mapsto \frac{h(x, y)}{g^{k}}
    \]
    is a ring homomorphism.
    To do.
\end{solution}

% Problem 2.20
\begin{problem}
    Recall from Exercise VI.2.13 that $\mathbb{P}^{n}_K$ denotes the `projective space' parametrizing lines in the vector space $K^{n+1}$.
    Every such line consists of multiples of a nonzero vector $(c_0, \ldots, c_n) \in K^{n+1}$, so that $\mathbb{P}^{n}_K$ may be identified with the quotient of $K^{n+1} \setminus \{(0, \ldots, 0)\}$ by the equivalence relation $\sim$ defined by
    \[
        (c_0, \ldots, c_n) \sim (c'_0, \ldots, c'_n) \Longleftrightarrow (\exists \lambda \in K^{*}), (c'_0, \ldots, c'_n) = (\lambda c_0, \ldots, \lambda c_n).
    \]
    The `point' in $\mathbb{P}^{n}_K$ determined by the vector $(c_0, \ldots, c_n)$ is denoted $(c_0 : \ldots : c_n)$;
    these are the `projective coordinates' of the point.
    Note that there is no `point' $(0 : \ldots : 0)$.

    Prove that the function $\mathbb{A}^{n}_K \to \mathbb{P}^{n}_K$ defined by
    \[
        (c_1, \ldots, c_n) \mapsto (1 : c_1 : \ldots : c_n)
    \]
    is a bijection.
    This function is used to realize $\mathbb{A}^{n}_K$ as a subset of $\mathbb{P}^{n}_K$.
    By using similar functions, prove that $\mathbb{P}^{n}_K$ can be covered with $n + 1$ copies of $\mathbb{A}^{n}_K$, and relate this fact to the cell decomposition obtained in Exercise VI.2.13.
    (Suggestion: Work out carefully the case $n = 2$.)
\end{problem}

\begin{solution}
    Note that the given function is merely a function between sets.
    It is clearly injective because if $(1 : c_1 : \ldots : c_n) = (1 : c_1' : \ldots : c_n')$ then the corresponding scalar $\lambda \in K*$ is simply 1, hence $c_1 = c_1', \ldots, c_n = c_n'$ so the affine coordinates are equal.
    Similarly, it is easily seen to be surjective since the projective coordinate $(1 : c_1 : \ldots : c_n)$ is mapped to by $(c_1, \ldots, c_n)$.
    Thus, the function is a bijection.
    
    To cover $\mathbb{P}^{n}_K$ with $n + 1$ copies of $\mathbb{A}^{n}_K$, simply consider the images of the bijections $f_i : \mathbb{A}^{n}_K \to \mathbb{P}^{n}_K$ which sends $(c_1, \ldots, c_n) \mapsto (c_1 : \ldots : 1 \ldots : c_n)$ with a $1$ in the $i$-th coordinate for $0 \leq i \leq n$.
    By this convention, the above map is $f_0$.
    Then every projective point is equal to a point in the image of one of the $f_i$ since not all of the projective coordinates are equal to 0.

    Working this out explicitly for $n = 2$, the functions are
    \begin{gather*}
        f_0 : (a, b) \mapsto (1 : a : b) \\
        f_1 : (a, b) \mapsto (a : 1 : b) \\
        f_2 : (a, b) \mapsto (a : b : 1)
    \end{gather*}
    For any projective point $(c_0 : c_1 : c_2)$, it is equivalent to at least one of the following: $(1 : \frac{c_1}{c_0} : \frac{c_2}{c_0})$, $(\frac{c_0}{c_1} : 1 : \frac{c_2}{c_1})$, or $(\frac{c_0}{c_2} : \frac{c_1}{c_2} : 1)$ as at least one of the $c_i \neq 0$.
    In any case, these are all contained in the image of the $f_i$.

    Recall that the cell decomposition in Exercise VI.2.13 states that $\mathbb{P}^{n-1}_K = k^{n-1} \cup k^{n-2} \cup \cdots k^{1} \cup k^{0}$.
    The connection is that the map $f_{n}$ identifies the subset $k^{n-1} \subset \mathbb{P}^{n-1}_K$ since by the realization of projective space as a union, the $n$-th projective coordinate is 1 for all points in the hyperplane lying above the origin.
    Then the remaining $f_{i}$ can be realized as identifying $k^{i-1} \subset \mathbb{P}^{i-1}_K$ inductively.
    Returning to our explicit example, $f_2$ identifies $\mathbb{A}^{2}_K$ with the plane $z = 1$ in $K^3$.
    Then $f_1$ restricted to points of the form $(a, 0)$ maps to projective coordinates in the plane $z = 0$ where lines are identified with a unique point on the line $y = 1$.
    Finally, there is a single point identified by the restriction of $f_0$ to $k$ since all lines in this space pass throught the point $x = 1$.
\end{solution}

% Problem 2.21
\begin{problem}
    Let $F(x_0, \ldots, x_n) \in K[x_0, \ldots, x_n]$ be a \textit{homogeneous} polynomial.
    With notation as in Exercise 2.20, prove that the condition `$F(c_0, \ldots, c_n) = 0$' for a point $(c_0 : \ldots : c_n) \in \mathbb{P}^{n}_K$ is well-defined:
    it does not depend on the representative $(c_0, \ldots, c_n)$ chosen for the points $(c_0 : \ldots : c_n)$.
    We can then define the following subset of $\mathbb{P}^{n}_K$ :
    \[
        \mathscr{V}(F) := \{(c_0 : \ldots : c_n) \in \mathbb{P}^{n}_K \mid F(c_0, \ldots, c_n) = 0\}.
    \]
    Prove that this `projective algebraic set' can be covered with $n + 1$ affine algebraic sets.

    The basic definitions in `projective algebraic geometry' can be developed along essentially the same path taken in this section for affine algebraic geometry, using `homogeneous ideals' (that is, ideals generated by homogeneous polynomials; see \S VIII.4.3) rather than ordinary ideals.
    This problem shows one way to relate projective and affine algebraic sets, in one template example.
\end{problem}

\begin{solution}
    Recall that a homogeneous polynomial is one whose nonzero terms all have the same degree.
    Suppose $F$ is homogeneous and let $(c_0 : \ldots : c_n) \in \mathbb{P}^{n}_K$ be such that $F(c_0, \ldots, c_n) = 0$.
    Choose another representative, say $(\lambda c_0, \ldots, \lambda c_n)$.
    Then since $F$ is homogeneous (say of degree $i$), every nonzero term has a factor $\lambda^{i}$.
    Thus, we may factor this out, yielding
    \[
    F(\lambda c_0, \ldots, \lambda c_n) = \lambda^{i} F(c_0, \ldots, c_n) = 0
    \]
    proving that the choice of representative is irrelevant.
    Thus, the notion of a `projective algebraic set' is well-defined.

    To cover this set with affine algebraic sets, we effectively use the same process as in the previous exercise.
    In particular, consider the $n+1$ polynomials $F_i := F(x_0, \ldots, 1, \ldots, x_n)$ defined by setting the $i$-th argument of $F$ to be 1.
    Letting $f_i$ denote the same injection as in the previous exercise, we find that for a point $(c_0 : \ldots : c_n) \in \mathscr{V}(F)$, 
    \[
        F(c_0, \ldots, c_n) = F\left(\frac{c_0}{c_i}, \ldots, 1, \ldots, \frac{c_n}{c_i}\right) = F(f_i(c_1, \ldots c_n)) = F_i(c_1, \ldots, c_n) = 0.
    \]
    Thus, every point of the projective algebraic set is in the image of one of the $F_i$.
\end{solution}
\end{document}
