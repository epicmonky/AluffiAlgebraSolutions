\documentclass[../../master.tex]{subfiles}

\begin{document}
\section{Field extensions, II}

% Problem 4.1
\begin{problem}
    Let $k$ be a field, $f(x) \in k[x]$, and let $F$ be the splitting field for $f(x)$ over $k$.
    Let $k \subseteq K$ be an extension such that $f(x)$ splits as a product of linear factors over $K$.
    Prove that there is a homomorphism $F \to K$ extending the identity on $k$.
\end{problem}

\begin{solution}
    Since $f$ splits over $F$, we may write
    \[
        f(x) = (x - \alpha_1) \cdots (x - \alpha_r)
    \]
    where $\alpha_i \in F$.
    But then since $f$ splits over $K$, we may also write
    \[
        f(x) = (x - \beta_1) \cdots (x - \beta_r)
    \]
    where $\beta \in K$.
    Then we may consider the homomorphism $\varphi: F \to K$ which extends the identity on $k$ and maps $\alpha_i \mapsto \beta_i$.
    Certainly this is a homomorphism.
\end{solution}

% Problem 4.2
\begin{problem}
    Describe the splitting field of $x^6 + x^3 + 1$ over $\mathbb{Q}$.
    Do the same for  $x^{4} + 4$.
\end{problem}

\begin{solution}
    Note that $(x^{6} + x^3 + 1)(x^3 - 1) = x^{9} - 1$ so the roots of the polynomial are 9th roots of unity.
    Furthermore, they are not the roots of $x^3 - 1 = (x^2 + x + 1) (x - 1)$.
    Thus, the roots of the polynomial are $\zeta, \zeta^2, \zeta^{4}, \zeta^{5}, \zeta^{7}, \zeta^{8}$ where $\zeta = e^{2\pi i / 9}$.
    Since these roots are all generated by $\zeta$, we find that the splitting field of $x^{6} + x^3 + 1$ is $\mathbb{Q}(\zeta)$.

    Note that $x^{4} + 4 = (x^2 - 2x + 2) (x^2 + 2x + 2)$.
    We can easily find the roots of these polynomials to be $1 \pm i, -1 \pm i$ respectively, so the splitting field is merely $\mathbb{Q}(i)$.
\end{solution}

% Problem 4.3
\begin{problem}
    Find the order of the automorphism group of the splitting field of $x^{4} + 2$ over $\mathbb{Q}$.
\end{problem}

\begin{solution}
    The splitting field of $x^{4} + 2$ over $\mathbb{Q}$ is $\mathbb{Q}(i, \sqrt[4]{2})$.
    Let $\sigma$ be an automorphism of this field which fixes $\mathbb{Q}$.
    Then $\sigma(\sqrt[4]{2})^{4} = \sigma(\sqrt[4]{2}^{4}) = \sigma(2) = 2$ so $\sigma( \sqrt[4]{2})$ is a root of $x^{4} - 2$.
    Furthermore, $\sigma(i)^2 = \sigma(i^2) = \sigma(-1) = -1$ so $\sigma(i)$ is a root of $x^2 + 1$.
    Since there are four choices for the first and two choices for the second, the automorphism group has order 8.
\end{solution}

% Problem 4.4
\begin{problem}
    Prove that the field $\mathbb{Q}(\sqrt[4]{2})$ is not the splitting field of any polynomial over $\mathbb{Q}$.
\end{problem}

\begin{solution}
    This is equivalent to showing that $F = \mathbb{Q}(\sqrt[4]{2})$ is not normal; that is, there is an irreducible polynomial $f(x) \in \mathbb{Q}[x]$ with a root in $F$ which does not split as a product of linear factors over $F$.
    Consider $f(x) = x^4 - 2$.
    This is irreducible over $\mathbb{Q}$ by Eisenstein and it factors over $F$ as $(x - \sqrt[4]{2})(x + \sqrt[4]{2})(x^2 + \sqrt{2})$.
    The last factor does not have roots over $F$ since its splitting field is $\mathbb{Q}(\sqrt[4]{2}, i)$.
    Thus, $F$ is not the splitting field of any polynomial over $\mathbb{Q}$.
\end{solution}

% Problem 4.5
\begin{problem}
    Let $F$ be a splitting field for a polynomial $f(x) \in k[x]$, and let $g(x) \in k[x]$ be a factor of $f(x)$.
    Prove that $F$ contains a unique copy of the splitting field of $g(x)$.
\end{problem}

\begin{solution}
    Since $F$ is a splitting field for $f(x)$, we may write
    \[
        f(x) = (x - \alpha_1) \cdots (x - \alpha_n)
    \]
    where $\alpha_i \in F$.
    Furthermore, we have $f(x) = g(x) \cdot h(x)$ where $g, h \in k[x]$.
    Suppose WLOG that $\alpha_1, \ldots, \alpha_m$ are the roots of $g$ in $F$.
    Then we may consider the algebraic extension $k(\alpha_1, \ldots, \alpha_m)$.
    Clearly this is a splitting field of $g(x)$ contained within $F$.
    Uniqueness follows from the minimality of $F$ as a splitting field for $f(x)$.
    Indeed, if there were other roots of $g$ in $F$, and hence other roots of $f$, then $F$ would properly contain the splitting field of $f(x)$.
\end{solution}

% Problem 4.6
\begin{problem}
    Let $k \subseteq F_1, k \subseteq F_2$ be two finite extensions, viewed as embedded in the algebraic closure $\overline{k}$ of $k$.
    Assume that $F_1$ and $F_2$ are splitting fields of polynomials in $k[x]$.
    Prove that the intersection $F_1 \cap F_2$ and the composite $F_1 F_2$ (the smallest subfield of $\overline{k}$ containing both $F_1$ and $F_2$) are both also splitting fields over $k$.
    (Theorem 4.8 is likely going to be helpful.)
\end{problem}

\begin{solution}
    We show that $F_1 \cap F_2$ is finite and normal.
    Certainly if $F_1$ and $F_2$ are finite, then their intersection is finite, generated by the extension elements in both fields.
    Furthermore, suppose $f(x) \in k[x]$ is irreducible and has a root in $F_1 \cap F_2$.
    Then $f(x)$ splits as a product of linear factors over both $F_1$ and $F_2$.
    Thus, all of the roots of $f$ are contained in both $F_1$ and $F_2$, hence in their intersection.
    A similar proof follows for $F_1 F_2$.
\end{solution}

% Problem 4.7
\begin{problem}
    Let $k \subseteq F = k(\alpha)$ be a simple algebraic extension.
    Prove that $F$ is normal over $k$ if and only if for every algebraic extension $F \subseteq K$ and every $\sigma \in \Aut_k(K)$, $\sigma(F) = F$.
\end{problem}

\begin{solution}
    Suppose $F$ is normal.
    In particular, the minimal polynomial $p(x)$ of $\alpha$ splits into linear factors over $F$ since it's irreducible over $k$.
    Let $\sigma \in \Aut_k(K)$.
    Then $\sigma(\alpha)$ is a root of $p$, hence $\sigma(\alpha) \in F$.
    Since $k$ is fixed by $\sigma$, we find that $\sigma(F) = F$.

    For the other direction, let $p(x)$ be irreducible over $k$ and suppose $F$ contains a root $\alpha$ of $p(x)$.
    Let $K$ be the splitting field of $p(x)$ and let $\beta$ be a different root of $p(x)$.
    Then there is an automorphism $\sigma \in \Aut_k(K)$ which sends $\alpha \mapsto \beta$.
    Since $\sigma(F) = F$ and $\alpha \in F$, it must be the case that $\beta \in F$.
    Thus, $p(x)$ splits into linear factors over $F$.
\end{solution}

% Problem 4.8
\begin{problem}
    Let $p$ be a prime, and let $k$ be a field of characteristic $p$.
    For $a, b \in K$, prove that $(a + b)^{p} = a^{p} + b^{p}$.
\end{problem}

\begin{solution}
    By the binomial theorem, we have
    \begin{align*}
        (a + b)^{p} &= \sum_{k=0}^{p} {p \choose k} a^{p-k} b^{k} \\
                    &= a^{p} + b^{p}
    \end{align*}
    since for all $0 < k < p$ we have
    \[
        {p \choose k} = \frac{p!}{k! (p - k)!}
    \]
    has a factor of $p$ since $p$ is prime (so no term in the denominator cancels the $p$ in the numerator).
    Since $k$ has characteristic $p$, each of these coefficients is 0.
    The only remaining terms are $a^{p} + b^{p}$.
\end{solution}

% Problem 4.9
\begin{problem}
    Using the notion of `derivative' given in \S 4.2, prove that $(fg)' = f'g + fg'$ for all polynomials $f, g$.
\end{problem}

\begin{solution}
    Rather than messing with series and coefficients, we can first note the following:
    \begin{gather*}
        (f + g)' = f' + g', \\
        (cf)' = cf', \\
        (xf') = f + f',
        c' = 0
    \end{gather*}
    Let $A \subseteq R[x]$ be the set of polynomials $f$ with the property that $(fg)' = f'g + fg'$ for all polynomials $g$.

    If $f_1, f_2 \in A$, then for every polynomial $g$,
    \begin{align*}
        ((f_1 + f_2)g)' &= (f_1g + f_2g)' \\
                        &= (f_1g)' + (f_2g)' \\
                        &= f_1g' + f_1'g + f_2g' + f_2'g \\
                        &= (f_1 + f_2)g' + (f_1' + f_2')g \\
                        &= (f_1 + f_2)g' + (f_1 + f_2)'g
    \end{align*}
    so $A$ is closed under addition.

    Similarly,
    \begin{align*}
        ((f_1 f_2)g)' &= (f_1 (f_2g))' \\
                      &= f_1' f_2g + f_1 (f_2g)' \\
                      &= f_1' f_2g + f_1 (f_2'g + f_2g') \\
                      &= (f_1' f_2 + f_1f_2') g + f_1 f_2 g'.
    \end{align*}
    That is, $A$ is closed under multiplication.
    Finally, since $A$ contains all constants and $x \in A$, we can conclude that $A = R[x]$.
\end{solution}

% Problem 4.10
\begin{problem}
    Let $k \subseteq F$ be a finite extension in characteristic $p > 0$.
    Assume that $p$ does not divide $[F : k]$.
    Prove that $k \subseteq F$ is separable.
\end{problem}

\begin{solution}
    Since $F$ is finite, we have $k \subseteq F = k(\alpha_1, \ldots, \alpha_n)$.
    Let $f_i$ be the minimal polynomial of $\alpha_i$.
    Since $[F : k]$ is the product of the degrees of the minimal polynomials of each  $\alpha_i$ and $p \nmid [F : k]$, it must be the case that $p \nmid \deg f_i$ for all $i$.
    In particular, $f_i$ is not a polynomial in $x^{p}$, hence it is not inseparable over $k$.
\end{solution}

% Problem 4.11
\begin{problem}
    Let $p$ be a prime integer.
    Prove that the Frobenius homomoprhism on $\mathbb{F}_p$ is the identity.
    (Hint: Fermat.)
\end{problem}

\begin{solution}
    Recall that that the Frobenius homomorphism is the map $x \mapsto x^{p}$.
    For all $x \in \mathbb{F}_p$ we find
    \[
    x^{p} \equiv x \pmod p
    \]
    so the homomorphism is the identity map.
\end{solution}

% Problem 4.12
\begin{problem}
    Let $k$ be a field, and assume that $k$ is not perfect.
    Prove that there are inseparable irreducible polynomials in $k[x]$.
    (If $\text{char } k = p$ and $u \in k$, how many roots does $x^{p} - u$ have in $\overline{k}$?)
\end{problem}

\begin{solution}
    We may assume that $\text{char } k = p > 0$ since all fields of characteristic 0 are perfect.
    Then consider the polynomial $f(x) = x^{p} - u$.
    We find $f'(x) = px^{p-1} = 0$ (since $k$ has characteristic $p$).
    Thus, $\gcd(f, f') = x^{p} - u \neq 0$ and $f$ has a repeated root $\sqrt[p]{u}$ of multiplicity $p$.
\end{solution}

% Problem 4.13
\begin{problem}
    Let $k$ be a field of positive characteristic $p$, and let $f(x)$ be an irreducible polynomial.
    Prove that there exists an integer $d$ and a \textit{separable} irreducible polynomial $f_{\text{sep}}(x)$ such that
    \[
        f(x) = f_{\text{sep}}(x^{p^{d}}).
    \]
    The number $p^{d}$ is called the \textit{inseparable degree} of $f(x)$.
    If $f(x)$ is the minimal polynomial of an algebraic element $\alpha$, the inseparable degree of $\alpha$ is defined to be the inseparable degree of $f(x)$.
    Prove that $\alpha$ is inseparable if and only if its inseparable degree is $\geq p$.

    The picture to keep in mind is as follows:
    the roots of the minimal polynomial $f(x)$ of $\alpha$ are distributed into $\deg f_{\text{sep}}$ `clumps', each collecting a number of coincident roots equal to the inseparable degree of $\alpha$.
    We say that $\alpha$ is `purely inseparable' if there is only one clump, that is, if all roots of $f(x)$ coincide (see Exercise 4.14).
\end{problem}

\begin{solution}
    Let $d$ be the greatest integer such that $f(x)$ is a polynomial in $x^{p^{d}}$, say $f(x) = g(x^{p^{d}})$.
    Then $g$ is irreducible (or else $f$ wouldn't be) and is not a polynomial in $x^{p}$ by the maximality of $d$, hence it is separable.

    Suppose $\alpha$ is inseparable.
    That is, its minimal polynomial $f(x)$ is inseparable, hence a polynomial in $x^{p}$.
    Then the inseparable degree of $\alpha$ is $\geq p$.
    For the other direction, suppose the inseparable degree of $\alpha$ is $\geq p$.
    That is, $f(x)$ can be written as a polynomial in $x^{p}$.
    Thus, $f(x)$ is inseparable, hence $\alpha$ is inseparable.
\end{solution}

% Problem 4.14
\begin{problem}
    Let $k \subseteq F$ be an algebraic extension, in positive characteristic $p$.
    An element $\alpha \in F$ is \textit{purely inseparable} over $k$ if $\alpha^{p^{d}} \in k$ for some $d \geq 0$.
    The extension is defined to be purely inseparable if every $\alpha \in F$ is purely inseparable over $k$.

    Prove that $\alpha$ is purely inseparable if and only if $[k(\alpha) : k]_s = 1$, if and only if its degree equals its \textit{inseparability} degree.
\end{problem}

\begin{solution}
    If $\alpha$ is purely inseparable, then $\alpha^{p^{d}} = a \in k$ for some $d \geq 0$.
    Consider the polynomial $f(x) = x^{p^{d}} - a \in k[x]$ which factors as $(x - \alpha)^{p^{d}} \in k(\alpha)[x]$.
    That is, $f(x)$ has exactly one root in $\overline{k}$, namely $\alpha$.
    Thus, $[k(\alpha) : k]_s = 1$ as there is only one embedding of $k(\alpha)$ into $\overline{k}$.

    Now suppose $[k(\alpha) : k]_s = 1$.
    That is, there is one embedding of $k(\alpha)$ into $\overline{k}$, so $\alpha$ is the unique root of its minimal polynomial $f(x)$.
    Since $f(x)$ is irreducible, it can be written as a separable polynomial $g$ in $x^{p^{d}}$ where $p^{d}$ is the inseparability degree of $\alpha$.
    Furthermore, since $f(x)$ only has one distinct root, $g$ has degree 1.
    Thus, $\deg f = p^{d}$.

    Finally, if the degree of $\alpha$ is equal to its inseparability degree then the minimal polynomial $f$ of $\alpha$ has the form $x^{p^{d}} - a$ where $p^{d}$ is the inseparability degree of $\alpha$.
    In particular, $\alpha^{p^{d}} = a \in k$.
\end{solution}

% Problem 4.15
\begin{problem}
    Let $k \subseteq F$ be an algebraic extension, and let $\alpha \in F$ be separable over $k$.
    For every intermediate field $k \subseteq E \subseteq F$, prove that $\alpha$ is separable over $E$.
\end{problem}

\begin{solution}
    Let $f$ be the minimal polynomial of $\alpha$ over $k$.
    Since $\alpha$ is separable, we have $\gcd(f, f') = 1$.
    Note that for any intermediate extension $E \subseteq F$, the minimal polynomial $g$ of $\alpha$ over $E$ divides $f$.
    In particular, since $f$ splits into linear factors over $F$, so do any factors of $f$.
    Hence, $g$ is separable over $F$.
\end{solution}

% Problem 4.16
\begin{problem}
    Let $k \subseteq E \subseteq F$ be algebraic field extensions, and assume that $k \subseteq E$ is separable.
    Prove that if $\alpha \in F$ is separable over $E$, then $k \subseteq E(\alpha)$ is a separable extensions.
    (Reduce to the case of finite extensions.)

    Deduce that the set of elements of $F$ which are separable over $k$ form an intermediate field $F_{\text{sep}}$, such that every element $\alpha \in F, \alpha \notin F_{\text{sep}}$ is \textit{inseparable} over $k$.

    For $F = \overline{k}$, $\overline{k}_{\text{sep}}$ is called the \textit{separable closure} of $k$.
\end{problem}

\begin{solution}
    Recall that an extension is separable iff $[E : k]_s = [E : k]$.
    That is, the number of embeddings of $E$ into $\overline{k}$ is equal to the degree of the extension.
    Then it is easy to see that
    \[
        [E(\alpha) : k]_s = [E(\alpha) : E]_s \cdot [E : k]_s = [E(\alpha) : E] \cdot [E : k] = [E(\alpha) : k]
    \]
    so the extension $k \subseteq E(\alpha)$ is separable.
\end{solution}

% Problem 4.17
\begin{problem}
    Let $k \subseteq F$ be an algebraic extension, in positive characteristic.
    With notation as in Exercises 4.14 and 4.16, prove that the extension $F_{\text{sep}} \subseteq F$ is purely inseparable.
    Prove that an extension $k \subseteq F$ is purely inseparable if and only if $F_{\text{sep}} = k$.
\end{problem}

\begin{solution}
    We may assume that $\text{char} k = p > 0$ since otherwise $k$ is perfect and $F_{\text{sep}} = F$.
    Let $\alpha \in F$ and let $f(x) \in k[x]$ be the minimal polynomial of $\alpha$.
    We may assume that $f(x)$ is not separable or else $\alpha^{p^{0}} \in F_{\text{sep}}$ and $\alpha$ is purely inseparable.
    Since $f$ is inseparable, we can write $f(x) = g(x^{p})$ for some $g(x) \in k[x]$.
    In particular, $\alpha^{p}$ is algebraic over $F_{\text{sep}}$ of degree less than $\alpha$, so $\alpha^{p^{n+1}} = (\alpha^{p})^{p^{n}}$ is separable over $F$ for some $n \geq 0$.
    It follows that the minimal polynomial of $\alpha$ over $F_{\text{sep}}$ is $(x - \alpha)^{p^{m}}$ for some $m$, which is equivalent to saying that $\alpha^{p^{m}} \in F_{\text{sep}}$ so $\alpha$ is purely inseparable.
\end{solution}

% Problem 4.18
\begin{problem}
    Let $k \subseteq F$ be a finite extension, in positive characteristic.
    Define the \textit{inseparable degree} $[F : k]_i$ to be the quotient $[F : k] / [F : k]_s$.
    \begin{itemize}
        \item Prove that $[k(\alpha) : k]_i$ equals the inseparable degree of $\alpha_i$ as defined in Exercise 4.13.
        \item Prove that the inseparable degree is multiplicative:
            if $k \subseteq E \subseteq F$ are finite extensions, then $[F : k]_i = [F : E]_i [E : k]_i]$.
        \item Prove that a finite extension is purely inseparable if and only if its inseparable degree equals its degree.
        \item With notation as in Exercise 4.16, prove that $[F : k]_s = [F_{\text{sep}} : k]$ and $[F : k]_i = [F : F_{\text{sep}}]$.
            (Use Exercise 4.17.)
    \end{itemize}
\end{problem}

\begin{solution}
    To do.
\end{solution}
\end{document}
