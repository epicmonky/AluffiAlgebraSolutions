\documentclass[../../master.tex]{subfiles}

\begin{document}
\section{Preliminaries, reprise}

% Problem 1.1
\begin{problem}
    Let $\mathscr{F} : \mathsf{C} \to \mathsf{D}$ be a covariant functor, and assume that both $\mathsf{C}$ and $\mathsf{D}$ have products.
    Prove that for all objects $A, B$ of $\mathsf{C}$, there is a unique morphism $\mathscr{F}(A \times B) \to \mathscr{F}(A) \times \mathscr{F}(B)$ such that the relevant diagram involving natural projections commutes.

    If $\mathsf{D}$ has \textit{co}products (denoted $\coprod$ ) and $\mathscr{G} : \mathsf{C} \to \mathsf{D}$ is contravariant, prove that there is a unique morphism $\mathscr{G}(A) \coprod \mathscr{G}(B) \to \mathscr{G}(A \times B)$ (again, such that an appropriate diagram commutes).
\end{problem}

\begin{solution}
    Recall that the product $A \times B$ in $\mathsf{C}$ comes equipped with natural projections $\pi_A$ and $\pi_B$ to $A$ and $B$ respectively.
    Then, by the universal property of products, we have the following diagram in $\mathsf{D}$.
    \[
    \begin{tikzcd}
    && {\mathscr{F}(A)} \\
        {\mathscr{F}(A \times B)} & {\mathscr{F}(A) \times \mathscr{F}(B)} \\
            && {\mathscr{F}(B)}
                \arrow["{\pi'_A}", from=2-2, to=1-3]
                    \arrow["{\pi'_B}"', from=2-2, to=3-3]
                        \arrow["{\mathscr{F}(\pi_A)}"{description}, bend left=12, from=2-1, to=1-3]
                            \arrow["{\mathscr{F}(\pi_B)}"{description}, bend right=12, from=2-1, to=3-3]
                                \arrow["{\exists!f}", from=2-1, to=2-2]
    \end{tikzcd}
    \]
    where the morphism from $\mathscr{F}(A \times B) \to \mathscr{F}(A) \times \mathscr{F}(B)$ is unique.

    If $\mathscr{G}$ is contravariant, then there are instead morphisms $\mathscr{G}(\pi_A) : \mathscr{G}(A) \to \mathscr{G}(A \times B)$ and similarly for  $\mathscr{G}(\pi_B)$.
    Then the universal property for coproducts induces a unique morphism from $\mathscr{G}(A) \coprod \mathscr{G}(B) \to \mathscr{G}(A \times B)$.
\end{solution}

% Problem 1.2
\begin{problem}
    Let $\mathscr{F} : \mathsf{C} \to \mathsf{D}$ be a fully faithful functor.
    If $A, B$ are objects in $\mathsf{C}$, prove that $A \cong B$ in $\mathsf{C}$ if and only if $\mathscr{F}(A) \cong \mathscr{F}(B)$ in $\mathsf{D}$.
\end{problem}

\begin{solution}
    Recall that $A \cong B$ means there exist morphisms $f \in \Hom_{\mathsf{C}}(A, B)$ and $g \in \Hom_{\mathsf{C}}(B, A)$ such that $g \circ f = 1_A$ and $f \circ g = 1_B$.
    Furthermore, since functors preserve composition and identity, the forward direction is trivial.
    Now suppose $\mathscr{F}(A) \cong \mathscr{F}(B)$ in $\mathsf{D}$.
    Then there exist isomorphisms $f \in \Hom_{\mathsf{D}}(\mathscr{F}(A), \mathscr{F}(B))$ and $g \in \Hom_{\mathsf{D}}(\mathscr{F}(B), \mathscr{F}(A))$.
    The two Hom sets are in bijection with the sets $\Hom_{\mathsf{C}}(A, B)$ and $\Hom_{\mathsf{C}}(B, A)$ so we have corresponding morphisms $f'$ and $g'$.
    In particular, we find
    \[
        f \circ g = \mathscr{F}(f') \circ \mathscr{F}(g') = \mathscr{F}(f' \circ g') = \mathscr{F}(1_B) = 1_{\mathscr{F}(B)}
    \]
    and the bijectivity on morphisms implies that $f' \circ g' = 1_B$.
    A similar argument holds to show that $g' \circ f' = 1_A$ so these are isomorphisms and $A \cong B$ in $\mathsf{C}$.
\end{solution}

% Problem 1.3
\begin{problem}
    Recall that a group $G$ may be thought of as a groupoid $\mathsf{G}$ with a single object.
    Prove that defining the action of $G$ on an object of a category $\mathsf{C}$ is equivalent to defining a functor $\mathsf{G} \to \mathsf{C}$.
\end{problem}

\begin{solution}
    Indeed, recall that a group $G$ can be considered as a category $\mathsf{G}$ with one object, $X$, where $\Hom_{\mathsf{G}}(X, X) = \{g \cdot \mid g \in G\}$.
    Since every morphism is an isomorphism, this Hom set contains inverses, there is an identity, and composition guarantees associativity.
    To define a group action of $G$ on an object $A$ of $\mathsf{C}$, let $\mathscr{F} : \mathsf{G} \to \mathsf{C}$ be a functor sending $X \mapsto A$.
    Similarly, we send each element of $\Hom_{\mathsf{G}}(X, X)$ to an element of $\Hom_{\mathsf{C}}(A, A)$.
    Since functors preserve identities, $\mathscr{F}(1_X) = 1_A$ which corresponds to $e \cdot a = a$ for all $a \in A$ (if it has some set structure).
    Similarly, since functors preserve composition, we find $\mathscr{F}(g \circ h) = \mathscr{F}(g) \circ \mathscr{F}(h)$, or $(gh)(a) = g(h(a))$.
    Thus, we have defined an action.
    An action can be converted into a functor in a similar manner.
\end{solution}

% Problem 1.4
\begin{problem}
    Let $R$ be a commutative ring, and let $S \subseteq R$ be a \textit{multiplicative subset} in the sense of Exercise V.4.7.
    Prove that `localization is a functor': 
    associating with every $R$-module $M$ the localization $S^{-1}M$ (Exercise V.4.8) and with every $R$-module homomorphism $\varphi : M \to N$ the naturally induced homomorphism $S^{-1}M \to S^{-1}N$ defines a covariant functor from the category of $R$-modules to the category of $S^{-1}R$-modules.
\end{problem}

\begin{solution}
    The map assigns every object of $R\textsf{-Mod}$ to an object of $S^{-1}R\textsf{-Mod}$.
    Furthermore, given a module homomorphism $\varphi: M \to N$, we have an induced homomorphism which maps $\frac{m}{s} \mapsto \frac{\varphi(m)}{s}$.
    We show that it preserves identities and composition.
    Let $1_M : M \to M$ be the identity.
    Then $\mathscr{F}(1_M) : S^{-1}M \to S^{-1}M$ is defined as $\frac{m}{s} \mapsto \frac{m}{s}$ which is equivalent to the identity on $S^{-1}M$.
    Now let $\alpha: M \to N$ and $\beta: N \to P$ be module homomorphisms.
    Then $\mathscr{F}(\alpha)$ sends $\frac{m}{s} \mapsto \frac{\alpha(m)}{s}$.
    Similarly, $\mathscr{F}(\beta)$ sends $\frac{n}{s} \mapsto \frac{\beta(n)}{s}$.
    Then we find that
    \[
        \mathscr{F}(\beta) \circ \mathscr{F}(\alpha) \left(\frac{m}{s}\right) = \mathscr{F}(\beta) \left(\frac{\alpha(m)}{s}\right) = \frac{\beta(\alpha(m))}{s} = \mathscr{F}(\beta \circ \alpha) \left(\frac{m}{s}\right)
    \]
    so this map preserves composition, hence it is a functor.
\end{solution}

% Problem 1.5
\begin{problem}
    For $F$ a field, denote by $F^{*}$ the group of nonzero elements of $F$, with multiplication.
    The assignment $\mathsf{Fld} \to \mathsf{Grp}$ mapping $F$ to $F^{*}$ and a homomorphism of fields $\varphi: k \to F$ to the restriction $\varphi|_{k^{*}} :k^{*} \to F^{*}$ is clearly a covariant functor.

    On the other hand, a homomorphism of fields $k \to F$ is nothing but a field extension $k \subseteq F$.
    Prove that the assignment $F \mapsto F^{*}$ on objects, together with the prescription associating with every $k \subseteq F$ the \textit{norm} $N_{k \subseteq F} : F^{*} \to k^{*}$ (cf. Exercise VII.1.12), gives a \textit{contravariant} functor $\mathsf{Fld} \to \mathsf{Grp}$.
    State and prove an analogous statement for the \textit{trace} (cf. Exercise VII.1.13).
\end{problem}

\begin{solution}
    To do.
\end{solution}

% Problem 1.6
\begin{problem}
    Formalize the notion of presheaf of abelian groups on a topological space $T$.
    If $\mathscr{F}$ is a presheaf on $T$, elements of $\mathscr{F}(U)$ are called \textit{sections} of $\mathscr{F}$ on $U$.
    The homomorphism $\rho_{UV} : \mathscr{F}(U) \to \mathscr{F}(V)$ induced by an inclusion $V \subseteq U$ is called the \textit{restriction map}.

    Show that an example of a presheaf is obtained by letting $\mathscr{C}(U)$ be the additive abelian group of continuous complex-valued functions on $U$, with restriction of sections defined by ordinary restrictions of functions.

    For this presheaf, prove that one can uniquely glue sections agreeing on overlapping open sets.
    That is, if $U$ and $V$ are open sets and $s_U \in \mathscr{C}(U), s_V \in \mathscr{C}(V)$ agree after restriction to $U \cap V$, prove that there exists a unique $s \in \mathscr{C}(U \cup V)$ such that $s$ restricts to $s_U$ on $U$ and to $s_V$ on $V$.

    This is essentially the condition making $\mathscr{C}$ a \textit{sheaf}.
\end{problem}

\begin{solution}
    A presheaf of abelian groups on a topological space $T$ is a map $\mathscr{F}$ which assigns each open set $U$ of $T$ to an abelian group.
    By assumption, the restriction maps $\rho_{UV}$ are homomorphisms of abelian groups.
    Indeed, we can define the presheaf of continuous complex-valued functions on a topological space.
    To any open set $U$ of $T$, let $\mathscr{C}(U)$ be the abelian group of continuous functions on $U$ (under pointwise addition).
    For an inclusion $U \subseteq V$, we can define the restriction map $\rho_{UV}: \mathscr{C}(V) \to \mathscr{C}(U)$ by sending $f \mapsto f|_U$.
    Certainly if $f$ is continuous on $V$, it is continuous on a subset of $V$.
    To prove that this is in fact a group homomorphism, we see that
    \[
        \rho_{UV} (f + g) = (f + g)|_{U} = f|_{U} + g|_{U} = \rho_{UV}(f) + \rho_{UV}(g)
    \]
    where the second equality follows from addition being defined as point-wise.
    Thus, this is in fact a presheaf of abelian groups.

    To see that it is also a sheaf, let $s_U \in \mathscr{C}(U)$ and $s_V \in \mathscr{C}(V)$ such that $\rho_{U, U \cap V}(s_U) = \rho_{V, U \cap V}(s_V)$.
    Define
    \[
    s := 
    \begin{cases}
        s_U(x) & \text{if $x \in U$} \\
        s_V(x) & \text{if $x \in V$}
    \end{cases}
    \]
    Certainly $s$ is continuous on $U \cup V$ since it is continuous on both $U$ and $V$, as well as on $U \cap V$.
    It is also unique by construction, as for any function $f$ which agrees with $s$ on $U \cup V$, we find $s - f = 0$ so they are equivalent.
\end{solution}

% Problem 1.7
\begin{problem}
    Define a topology on $\Spec R$ by declaring the closed sets to be the sets $V(I)$, where $I \subseteq R$ is an ideal and $V(I)$ denotes the set of prime ideals \textit{containing} $I$.
    \begin{itemize}
        \item Verify that this indeed defines a topology on $\Spec R$.
            (This is the \textit{Zariski topology} on $\Spec R$.)
        \item Relate this topology to the Zariski topology defined in \S VII.2.3.
        \item Prove that $\Spec$ is then a contravariant functor from the category of commutative rings to the category of topological spaces (where morphisms are continuous functions).
    \end{itemize}
\end{problem}

\begin{solution}
    We verify that this is a topology on $\Spec R$.
    Certainly $\emptyset$ is closed since $R \subseteq R$ is an ideal and no prime ideals contain $R$.
    Similarly, $\Spec R$ is closed as $\{0\} \subseteq R$ is an ideal and every ideal contains $\{0\}$.
    Now let $V(I)$ and $V(J)$ be closed sets.
    Recall that $IJ$ is the ideal generated by elements of the form $ab$ where $a \in I, b \in J$.
    We claim that $V(IJ) = V(I) \cup V(J)$.
    Suppose $\mathfrak{p} \in V(I) \cup V(J)$ and WLOG, assume $\mathfrak{p} \in V(I)$.
    Then, since $IJ \subseteq I$, we have $IJ \subseteq I \subseteq \mathfrak{p}$ so $\mathfrak{p} \in V(IJ)$.
    For the other direction, suppose $\mathfrak{p} \in V(IJ)$.
    Then $\mathfrak{p} \in V(I)$ or $\mathfrak{p} \in V(J)$.
    Indeed, otherwise we could find an element $ab \in IJ \subseteq \mathfrak{p}$ such that $a, b \notin \mathfrak{p}$, contradicting the assumption that $\mathfrak{p}$ is prime.
    Thus, $V(IJ) = V(I) \cup V(J)$ and the topology is closed under finite unions.
    Finally, we claim that $V(I + J) = V(I) \cap V(J)$.
    Suppose $\mathfrak{p} \in V(I+J)$.
    That is, $I + J \subseteq \mathfrak{p}$.
    Since $I \subseteq I + J$ and $J \subseteq I + J$, we find that $\mathfrak{p} \in V(I) \cap V(J)$.
    Now suppose $\mathfrak{p} \in V(I) \cap V(J)$.
    That is, $I \subseteq \mathfrak{p}$ and $J \subseteq \mathfrak{p}$.
    Now let $x \in I + J$.
    That is, $x = a + b$ for some $a \in I, b \in J$.
    Then since $a,b \in \mathfrak{p}$, we have $x \in \mathfrak{p}$, thus $I + J \subseteq \mathfrak{p}$ and  $\mathfrak{p} \in V(I + J)$.
    Thus, the intersection of two closed sets is closed, proving that this is in fact a topology on $\Spec R$.

    Recall that the Zariski topology defined in \S VII.2.3 is defined on $\mathbb{A}^{n}_K$ by setting algebraic subsets to be the closed sets.
    Given a set $S \subseteq K[x_1, \ldots, x_n]$, the points of $V(S)$ correspond to the maximal ideals of $K[x_1, \ldots, x_n]$ which contain $S$.
    Thus, this is a natural generalization where instead of only using maximal ideals, one extends to prime ideals.

    To see that this is indeed a contravariant functor from $\mathsf{CRing} \to \mathsf{Top}$, first note that $\Spec$ maps every commutative ring to a topological space (as shown above).
    Now let $\varphi: R \to S$ be a homomorphism of rings.
    Then $\Spec R$ induces a morphism $\Spec S \to \Spec R$ which sends $\mathfrak{p} \mapsto \varphi^{-1}(\mathfrak{p})$ (it is easy to verify that the preimage of a prime ideal is prime, which verifies that this is a continuous map).
    Certainly $\Spec$ takes the identity to the identity, and it can quickly be seen that it preserves composition.
\end{solution}

% Problem 1.8
\begin{problem}
    Let $K$ be an algebraically closed field, and consider the category $K\textsf{-Aff}$ defined in Example 1.9.
    \begin{itemize}
        \item Denote by $h_S$ the functor $\Hom_{K\mathsf{-Aff}}(\_, S)$ (as in \S 1.2), and let $p = \mathbb{A}^{0}_K$ be a point.
            Show that there is a natural bijection between $S$ and $h_S(p)$.
            (Use Exercise VII.2.14.)
        \item Show how every $\varphi \in \Hom_{K\mathsf{-Aff}}(S, T)$ determines a function \textit{of sets} $S \to T$.
        \item If $S \subseteq \mathbb{A}^{m}_K, T \subseteq \mathbb{A}^{n}_K$, show that the function $S \to T$ determined by a morphism $\varphi \in \Hom_{K\mathsf{-Aff}}(S, T)$ is the restriction of a `polynomial function' $\mathbb{A}^{m}_K \to \mathbb{A}^{n}_K$.
            (Part of this exercise is to make sense of what this means!)
    \end{itemize}
\end{problem}

\begin{solution}
    Note that $h_S(p) = \Hom_{K\mathsf{-Aff}}(p, S)$. 
    Each map in this set is uniquely determined by the point in $q$ where $p$ is sent to.
    To formalize this notion, recall that we define $\Hom_{K\mathsf{-Aff}}(p, S) = \Hom_{K\mathsf{-Alg}}(K[S], K)$.
    By Exercise VII.2.14, there is a natural bijection between the points of $S$ and the maximal ideals of $K[S]$ such that if $q$ corresponds to the ideal $\mathfrak{m}_q$, then the evaluation map from $K[S]$ sending $f \mapsto f(q)$ has kernel $\mathfrak{m}_q$.
    Thus, each point of $S$ corresponds to a map in the $\Hom$ set.

    This can be extended to see that every $\varphi \in \Hom_{K\mathsf{-Aff}}(S, T)$ determines a set function $S \to T$.
    Intuitively, this reflects nothing more than the fact that $\varphi$ maps points of $S$ to points of $T$.
    Formally, we have that $\varphi : K[T] \to K[S]$ is determined by sending $y_i \mapsto f_i(x_1, \ldots, x_m)$.
    Thus, given a point $p = (x_1, \ldots, x_m) \in S$, we find that $\varphi$ induces a set function sending $p \mapsto (f_1(p), \ldots, f_n(p))$.

    To do.
\end{solution}

% Problem 1.9
\begin{problem}
    Let $\mathsf{C}, \mathsf{D}$ be categories, and assume $\mathsf{C}$ to be small.
    Define a \textit{functor category} $\mathsf{D^{C}}$, whose objects are covariant functors $\mathsf{C} \to \mathsf{D}$ and whose morphisms are natural transformations.

    Prove that the assignment $X \mapsto h_X := \Hom_{\mathsf{C}}(\_, X)$ defines a covariant functor $\mathsf{C} \to \mathsf{Set}^{\mathsf{C^{\mathsf{op}}}}$.
    (Define the action on morphisms in the natural way.)
\end{problem}

\begin{solution}
    Note that $\mathsf{Set^{C^{op}}}$ is the category whose objects are covariant functors $\mathsf{C^{op}} \to \mathsf{Set}$.
    Indeed, since $\mathsf{C}$ is small, we find $\Hom_{\mathsf{C}}(A, X)$ is a set for all objects $A$ of $\mathsf{C}$.
    Given a morphism $f: X \to Y$ in $\mathsf{C}$, we set $\mathscr{F}(f):h_X \to h_Y$ to be the natural transformation $v_A: \Hom_{\mathsf{C}}(A, X) \to \Hom_{\mathsf{C}}(A, Y)$ which maps $\alpha: A \to X$ to $\beta: A \to Y$ where $\beta = f \circ \alpha$.
    Verifying that that this is in fact a natural transformation is a brief diagram chase.
    We check that this functor $\mathscr{F}$ preserves identities.
    Indeed, consider $\mathscr{F}(1_X) : h_X \to h_X$ to be the natural transformation $v_A: \Hom_C(A, X) \to \Hom_C(A, X)$ which sends $\alpha: A \to X$ to $\alpha = 1_X \circ \alpha$.
    Then clearly $v$ is the identity on all Hom sets.
    Similarly, since natural transformations can be composed, it is quick to check that $\mathscr{F}$ preserves compositions.
\end{solution}

% Problem 1.10
\begin{problem}
    Let $\mathsf{C}$ be a category, $X$ and object of $\mathsf{C}$, and consider the contravariant functor $h_X := \Hom_{\mathsf{C}}(_, X)$.
    For every contravariant functor $\mathscr{F} : \mathsf{C} \to \mathsf{Set}$, prove that there is a bijection between the set of natural transformations $h_x \rightsquigarrow \mathscr{F}$ and $\mathscr{F}(X)$ as follows.
    The datum of a natural transformation $h_X \rightsquigarrow \mathscr{F}$ consists of a morphism from $h_X(A) = \Hom_{\mathsf{C}}(A, X)$ to $\mathscr{F}(A)$ for every object $A$ of $\mathsf{C}$.
    Map $h_X$ to the image of $\text{id}_X \in h_X(X)$ in $\mathscr{F}(X)$.
    (Hint: Produce an inverse of the specified map.
    For every $f \in \mathscr{F}(X)$ and every $\varphi \in \Hom_{\mathsf{C}}(A, X)$, how do you construct an element of $\mathscr{F}(A)$?)

    This result is called the \textit{Yoneda lemma}.
\end{problem}

\begin{solution}
    Given an element $f \in F(x)$, we construct a natural transform $\alpha_f : h_X \to F$.
    The associated morphism is
    \[
        \alpha_f(A) : \Hom(A, X) \to F(A), \quad \varphi \mapsto F(\varphi)(f) \in F(A)
    \]
    since $F$ is contravariant.
    To verify that this is indeed a natural transformation, let $f \in F(x)$, $g \in \Hom_{\mathsf{C}}(A, B)$, and $\varphi \in \Hom(B, X)$.
    Then, 
    \begin{gather*}
        (\alpha_f(A) \circ h_x(g))(\varphi) = F(\varphi \circ g)(f) \\
        (F(g) \circ \alpha_f(B))(\varphi) = F(\varphi)(f) \circ F(g) = F(\varphi \circ g)(f).
    \end{gather*}
    The two are equal, hence $\alpha_f$ is indeed a natural transformation.
    Finally, we show that the two constructed functions are inverses.
    Let $f \in F(X)$.
    Then we can construct the natural transformation $\alpha_f$.
    But then we obtain an element of $F(X)$ by sending this transformation to  $\alpha_f(X)(\text{id}_X) = F(\text{id}_X)(f) = \text{id}_X(f) = f$ since functors preserve identity morphisms.
    Thus, the two are inverses and there is a bijection between the set of natural transformations from $h_x \to F$ and the set $F(X)$.
\end{solution}

% Problem 1.11
\begin{problem}
    Let $\mathsf{C}$ be a small category.
    A contravariant functor $\mathsf{C} \to \mathsf{Set}$ is \textit{representable} if it is naturally isomorphic to a functor $h_X$.
    In this case, $X$ `represents' the functor.
    Prove that $\mathsf{C}$ is equivalent to the subcategory of representable functors in $\mathsf{Set}^{\mathsf{C}^{\mathsf{op}}}$.

    Thus, every (small) category is equivalent to a subcategory of a functor category.
\end{problem}

\begin{solution}
    Consider the functor $F : C \to \mathsf{Set}^{\mathsf{C}^{\mathsf{op}}}$ sending $X \mapsto h_X$ and which sends morphisms $f: X \to Y$ to the natural transformation $\alpha : h_X \to h_Y$ such that $\alpha_A(\varphi) = f \circ \varphi$.
    We prove that $F$ is an equivalence of categories.
    By the Yoneda lemma, there is a bijection between the set of natural transformations $h_X \to h_Y$ and the elements of $h_Y(X)$.
    In particular, there is a bijection between $\Hom_{\mathsf{C}}(X, Y)$ and $\Hom_{\mathsf{C}^{\mathsf{Set}^{\mathsf{op}}}}(h_X, h_Y)$.
    Thus, $F$ is fully faithful.
    To show that $F$ is essentially surjective, let $G$ be a representable functor.
    That is, $G$ is naturally isomorphic to a functor $h_X$ for some object $X$ in $\mathsf{C}$.
    Then $G \cong h_X = F(X)$ in $\mathsf{C}^{\mathsf{Set}^{\mathsf{op}}}$.
    Thus, $F$ is an equivalence of categories.
\end{solution}

% Problem 1.12
\begin{problem}
    Let $\mathsf{C}, \mathsf{D}$ be categories, and let $\mathscr{F} : \mathsf{C} \to \mathsf{D}$, $\mathscr{G} : \mathsf{D} \to \mathsf{C}$ be functors.
    Prove that $\mathscr{F}$ is left-adjoint to $\mathscr{G}$ if and only if, for every object $Y$ in $\mathsf{D}$, the object $\mathscr{G}(Y)$ represents the functor $h_Y \circ \mathscr{F}$.
\end{problem}

\begin{solution}
    Recall that $\mathscr{F}$ is left-adjoint to $\mathscr{G}$ iff there is a natural isomorphism such that for all objects $X$ of $\mathsf{C}$ and $Y$ of $\mathsf{D}$, $\Hom_{\mathsf{C}}(X, \mathscr{G}(Y)) \cong \Hom_{\mathsf{D}}(\mathscr{F}(X), Y)$.
    In particular, if we fix $Y$, then there is a natural isomorphism between $h_{\mathscr{G}(Y)}$ and $h_Y \circ \mathscr{F}$ (since $h_Y \circ \mathscr{F}(X) = \Hom_{\mathsf{D}}(\mathscr{F}(X), Y)$).
    That is, $\mathscr{G}(Y)$ represents $h_Y \circ \mathscr{F}$.
    The other direction is effectively the same.
\end{solution}

% Problem 1.13
\begin{problem}
    Let $\mathsf{Z}$ be the `Zen' category consisting of no objects and no morphisms.
    One can contemplate a functor $\mathscr{L}$ from $\mathsf{Z}$ to any category $\mathsf{C}$:
    no datum whatsoever need be specified.
    What is $\varprojlim \mathscr{L}$ (when such an object exists)?
\end{problem}

\begin{solution}
    Since no datum is specified and by the definition of a limit, the object is final with respect to the property defined by a cone up to isomorphism, $\varprojlim \mathscr{L}$ is a final object of $\mathsf{C}$.
\end{solution}

% Problem 1.14
\begin{problem}
    Verify that the construction described in Example 1.11 indeed recovers the kernel of a homomorphism of $R$-modules, as claimed.
\end{problem}

\begin{solution}
    The construction describes taking the limit of a functor from a two-object category with parallel morphisms in which one is sent to the zero morphism.
    In particular, such a limit is determined by the choice of two modules, $A_1$ and $A_2$, along with morphisms $\varphi: A_2 \to A_1$ and $0: A_2 \to A_1$.
    The limit of this functor, say $\varprojlim \mathscr{K}$, is a module $K$ equipped with morphisms $\lambda_i : K \to A_i$ such that $\lambda_1 = \varphi \circ \lambda_2$ and $\lambda_1 = 0 \circ \lambda_2$.
    That is, any morphism from $K \to A_1$ is the zero map and uniquely factors through $A_2$.
    Since $K$ is final with respect to this property, we recover the definition of the kernel of a module homomorphism.
\end{solution}

% Problem 1.15
\begin{problem}
    Verify that the construction given in the proof of Claim 1.13 is an inverse limit, as claimed.
\end{problem}

\begin{solution}
    Claim 1.13 constructs the limit $\varprojlim A_i$ in $R\mathsf{-Mod}$.
    Consider the product $\Pi A_i$ which consists of arbitrary sequences $(a_i)_{i > 0}$ where $a_i \in A_i$.
    A sequence $(A_i)_{i > 0}$ is \textit{coherent} if for all $i > 0$, we have $a_i = \varphi_{i, i+1}(a_{i+1})$.
    Coherent sequences form an $R$-submodule $A$ of $\Pi A_i$ where the canonical projections restrict to homomorphisms $\varphi_i : A \to A_i$.
    Indeed, we have $\varphi_i(a) = a_i = \varphi_{i, i+1} \circ \varphi_{i+1}(a)$.
    Furthermore, suppose $M$ is another module endowed with morphisms satisfying the requirement.
    Then, since there are morphisms $\lambda_i : M \to A_i$, there is a unique morphism $\lambda: M \to A$ sending $m \mapsto (\lambda_i(m))_{i > 0}$.
    This morphism makes all relevant diagrams to commute and is entirley determined by $M$, hence $A$ is final with respect to this property, making it a limit.
\end{solution}

% Problem 1.16
\begin{problem}
    Flesh out the sketch of the constructions of colimits in $\mathsf{Set}$ and $R\mathsf{-Mod}$ given in \S 1.4, for an indexing poset.
    In $\mathsf{Set}$, observe that the construction of the colimit is simpler if the poset $\mathsf{I}$ is \textit{directed};
    that is, if $\forall i, j \in \mathsf{I}$, there exists a $k \in \mathsf{I}$ such that $i \leq k$, $j \leq k$.
\end{problem}

\begin{solution}
    No, I don't think I will. / To do.
\end{solution}

% Problem 1.23
\begin{problem}
    Let $R, S$ be rings.
    Prove that an additive covariant functor $\mathscr{F} : R\mathsf{-Mod} \to S\mathsf{-Mod}$ is exact if and only if $\mathscr{F}(A) \longrightarrow \mathscr{F}(B) \longrightarrow \mathscr{F}(C)$ is exact in $S\mathsf{-Mod}$ whenever $A \longrightarrow B \longrightarrow C$ is exact in $R\mathsf{-Mod}$.
    Deduce that an exact functor sends exact complexes to exact complexes.
\end{problem}

\begin{proof}
    One direction is trivial since if
    \[
    \begin{tikzcd}
        0 & A & B & C & 0
        \arrow[from=1-1, to=1-2]
        \arrow[from=1-2, to=1-3]
        \arrow[from=1-3, to=1-4]
        \arrow[from=1-4, to=1-5]
    \end{tikzcd}
    \]
    is exact, then so is
    \[
    \begin{tikzcd}
        A & B & C.
        \arrow[from=1-1, to=1-2]
        \arrow[from=1-2, to=1-3]
    \end{tikzcd}
    \]
    For the other direction, suppose $\mathscr{F}$ is an additive covariant functor satisfying the specified property.
    Let
    \[
    \begin{tikzcd}
        0 & A & B & C & 0
        \arrow[from=1-1, to=1-2]
        \arrow[from=1-2, to=1-3]
        \arrow[from=1-3, to=1-4]
        \arrow[from=1-4, to=1-5]
    \end{tikzcd}
    \]
    be an exact sequence of $R$-modules.
    In particular, each `sub-sequence' is exact.
    Then applying $\mathscr{F}$ to each `sub-sequence' preserves exactness, and since $\mathscr{F}(d)^2 = 0$, we may concatenate the `sub-sequences` to obtain the necessary exact sequence of $S$-modules.
    Thus, $\mathscr{F}$ is an exact functor.
    The same logic applies to arbitrary exact sequences.
\end{proof}

% Problem 1.24
\begin{problem}
    Let $R, S$ be rings.
    An additive covariant functor $\mathscr{F} : R\mathsf{-Mod} \to S\mathsf{-Mod}$ is \textit{faithfully} exact if `$\mathscr{F}(A) \longrightarrow \mathscr{F}(B) \longrightarrow \mathscr{F}(C)$ is exact in $S\mathsf{-Mod}$ if and only if $A \longrightarrow B \longrightarrow C$ is exact in $R\mathsf{-Mod}$'.
    Prove that an exact functor $\mathscr{F} : R\mathsf{-Mod} \to S\mathsf{-Mod}$ is faithfully exact if and only if $\mathscr{F}(M) \neq 0$ for every nonzero $R$-module $M$, if and only if $\mathscr{F}(\varphi) \neq 0$ for every nonzero morphism $\varphi$ in $R\mathsf{-Mod}$.
\end{problem}

\begin{solution}
    Suppose $\mathscr{F}$ is faithfully exact.
    Suppose $\mathscr{F}(M) = 0$ for some $R$-module $M$.
    Then there is an exact sequence $0 \longrightarrow \mathscr{F}(M) \longrightarrow 0$.
    But then $0 \longrightarrow M \longrightarrow 0$ is an exact sequence of $R$-modules, implying that $M = 0$.

    Now suppose $\mathscr{F}(M) \neq 0$ for every nonzero $R$-module $M$.
    Let $\varphi : M \to N$ be a homomorphism of $R$-modules.
    We obtain a commutative diagram with exact rows and columns
    \[
    \begin{tikzcd}
	    & 0 \\
	    {\mathscr{F}(M)} & {\mathscr{F}(\im \varphi)} & 0 \\
	    & {\mathscr{F}(N)}
            \arrow["\mathscr{F}(\varphi')", from=2-1, to=2-2]
	    \arrow[from=2-2, to=2-3]
	    \arrow[from=1-2, to=2-2]
            \arrow["\mathscr{F}(i)", from=2-2, to=3-2]
            \arrow["\mathscr{F}(\varphi)"', from=2-1, to=3-2]
    \end{tikzcd}
    \]
    Suppose $\mathscr{F}(\varphi) = 0$.
    Then $\mathscr{F}(i) \circ \mathscr{F}(\varphi') = 0$, and by the injectivity of $i$, we have $\mathscr{F}(\varphi') = 0$.
    This implies that $\mathscr{F}(\im \varphi) = 0$, hence $\im \varphi = 0$, hence $\varphi = 0$.

    Finally, suppose $\mathscr{F}(\varphi) \neq 0$ for every nonzero $R$-module homomorphism $\varphi$.
    Let
    $
    \begin{tikzcd}
        {\mathscr{F}(A)} & {\mathscr{F}(B)} & {\mathscr{F}(C)}
        \arrow["\mathscr{F}(f)", from=1-1, to=1-2]
        \arrow["\mathscr{F}(g)", from=1-2, to=1-3] 
    \end{tikzcd}
    $
    be exact.
    Since $\mathscr{F}(g) \circ \mathscr{F}(f) = \mathscr{F}(g \circ f) = 0$, we find that $g \circ f = 0$ and $\im f \subseteq \ker g$.
    Then we have the following commutative diagram with exact rows and columns.
    \[
    \begin{tikzcd}
	& 0 \\
	0 & {\mathscr{F} (\im f)} & {\mathscr{F}(A)} \\
	0 & {\mathscr{F}(\ker g)} & {\mathscr{F}(B)} & {\mathscr{F}(C)} \\
	& {\mathscr{F}(\ker g / \im f)} \\
	& 0
	\arrow[from=1-2, to=2-2]
	\arrow["{\mathscr{F}(i)}"', from=2-2, to=3-2]
	\arrow["{\mathscr{F}(p)}"', from=3-2, to=4-2]
	\arrow[from=4-2, to=5-2]
	\arrow[from=3-1, to=3-2]
	\arrow["{\mathscr{F}(j)}"', from=3-2, to=3-3]
	\arrow["{\mathscr{F}(g)}"', from=3-3, to=3-4]
	\arrow["{\mathscr{F}(f)}", from=2-3, to=3-3]
	\arrow[from=2-2, to=2-1]
	\arrow["{\mathscr{F}(f')}"', from=2-3, to=2-2]
    \end{tikzcd}
   \]
   We show that $\mathscr{F}(i)$ is surjective.
   Indeed, let $x \in \mathscr{F}(\ker g)$.
   Since $\mathscr{F}(g) \circ \mathscr{F}(j) = 0$,  $\mathscr{F}(j)(x) \in \ker \mathscr{F}(g) = \im \mathscr{F}(f)$.
   That is, there exists some $y \in \mathscr{F}(A)$ such that $\mathscr{F}(f)(y) = \mathscr{F}(j)(x)$.
   Thus, $\mathscr{F}(j)(x) = \mathscr{F}(f)(y) = \mathscr{F}(j) \circ \mathscr{F}(i) \circ \mathscr{F}(f')(y)$.
   By the injectivity of $\mathscr{F}(j)$, we find that $x = \mathscr{F}(i) \circ \mathscr{F}(f')(y)$.
   Thus, $\mathscr{F}(i)$ is surjective, as well as injective.
   Therefore, $\mathscr{F}(p) = 0$*, hence $p = 0$, hence $\ker g = \im f$, so the corresponding sequence of $R$-modules is exact. 
\end{solution}

% Problem 1.25
\begin{problem}
    Prove that localization is an \textit{exact} functor.

    In fact, prove that localization `preserves homology': if
    \[
        M_{\bullet}: 
        \begin{tikzcd}
            {\cdots} & M_{i + 1} & M_{i} & M_{i-1} & {\cdots}
            \arrow[from=1-1, to=1-2]
            \arrow["d_{i+1}", from=1-2, to=1-3]
            \arrow["d_i", from=1-3, to=1-4]
            \arrow[from=1-4, to=1-5] 
        \end{tikzcd}
    \]
    is a complex of $R$-modules and $S$ is a multiplicative subset of $R$, then the localization $S^{-1}H_i(M_{\bullet})$ of the $i$-th homology of $M_{\bullet}$ is the $i$-th homology $H_i(S^{-1}M_{\bullet})$ of the localized complex
    \[
        S^{-1}M_{\bullet}:
        \begin{tikzcd}
            {\cdots} & S^{-1}M_{i+1} & S^{-1}M_i & S^{-1}M_{i-1} & {\cdots}
            \arrow[from=1-1, to=1-2]
            \arrow["S^{-1}d_{i+1}", from=1-2, to=1-3]
            \arrow["S^{-1}d_i", from=1-3, to=1-4]
            \arrow[from=1-4, to=1-5] 
        \end{tikzcd}
    \]
\end{problem}

\begin{solution}
    We first prove that localization is an exact functor.
    Let
    $
    \begin{tikzcd}
        A & B & C
        \arrow["f", from=1-1, to=1-2]
        \arrow["g", from=1-2, to=1-3] 
    \end{tikzcd}
    $
    be an exact sequence of $R$-modules.
    That is, $\ker g = \im f$.
    Localizing yields a complex
    $
    \begin{tikzcd}
        S^{-1}A & S^{-1}B & S^{-1}C
        \arrow["S^{-1}f", from=1-1, to=1-2]
        \arrow["S^{-1}g", from=1-2, to=1-3] 
    \end{tikzcd}
    $
    where $\im S^{-1}f \subseteq \ker S^{-1}g$.
    To see the other inclusion, let $m / s \in \ker S^{-1}g$.
    That is, $S^{-1}g(m / s) = 0$, so $g(m) / s = 0$, hence there exists $t \in S$ such that $t g(m) = 0$.
    But $t g(m) = g(tm)$, hence $tm \in \ker g = \im f$.
    Therefore, there exists $a \in A$ such that $f(a) = tm$.
    Thus, $m / s = mt / st = f(a) / st \in \im S^{-1}f$.
    Hence, localization is an exact functor.

    The $i$-th homology of $M_{\bullet}$ is given by $\frac{\ker d_i}{\im d_{i+1}}$, which inherits the structure of an $R$-module.
    Thus, we may localize it to obtain the $S^{-1}R$-module $S^{-1}H_i(M_{\bullet})$.
    On the other hand, localizing the complex yields induced morphisms $S^{-1}d_i$ such that the homology $H_i(S^{-1}M_{\bullet}) = \frac{\ker S^{-1}d_i}{\im S^{-1}d_{i+1}}$.

    We have the following exact sequences:
    \[
    \begin{tikzcd}
        0 & \ker d_i & M_i & M_{i-1}
        \arrow[from=1-1, to=1-2]
        \arrow["i", from=1-2, to=1-3]
        \arrow["d_i", from=1-3, to=1-4] 
    \end{tikzcd}
    \] 
    \[
    \begin{tikzcd}
        M_{i+1} & \im d_{i+1} & 0
        \arrow["d_{i+1}", from=1-1, to=1-2]
        \arrow[from=1-2, to=1-3]
    \end{tikzcd}
    \]
    Localizing both of these yields exact sequences which show that $S^{-1} \ker d_i \cong \ker S^{-1} d_i$ and that $\im S^{-1}d_{i+1} \cong S^{-1} \im d_{i+1}$.
    Finally, we have the exact sequence
    \[
    \begin{tikzcd}
        0 & \im d_{i+1} & \ker d_i & \frac{\ker d_i}{\im d_{i+1}} & 0
        \arrow[from=1-1, to=1-2]
        \arrow[from=1-2, to=1-3] 
        \arrow[from=1-3, to=1-4]
        \arrow[from=1-4, to=1-5] 
    \end{tikzcd}
    \]
    Localizing yields the exact sequence
    \[
    \begin{tikzcd}
        0 & S^{-1} \im d_{i+1} & S^{-1} \ker d_i & S^{-1} \frac{\ker d_i}{\im d_{i+1}} & 0
        \arrow[from=1-1, to=1-2]
        \arrow[from=1-2, to=1-3] 
        \arrow[from=1-3, to=1-4]
        \arrow[from=1-4, to=1-5] 
    \end{tikzcd}
    \]
    Finally, combining all of the above yields
    \[
        S^{-1}H_i(M_{\bullet}) \cong S^{-1} \frac{\ker d_i}{\im d_{i+1}} \cong \frac{S^{-1}\ker d_i}{S^{-1} \im d_{i+1}} \cong \frac{\ker S^{-1} d_i}{\im S^{-1} d_{i+1}} \cong H_i(S^{-1} M_{\bullet}).
    \]
\end{solution}

% Problem 1.26
\begin{problem}
    Prove that localization is faithfully exact in the following sense:
    let $R$ be a commutative ring and let
    \[
        \begin{tikzcd}
            0 & A & B & C & 0
            \arrow[from=1-1, to=1-2]
            \arrow[from=1-2, to=1-3]
            \arrow[from=1-3, to=1-4]
            \arrow[from=1-4, to=1-5] 
        \end{tikzcd}\tag{*}
    \]
    be a sequence of $R$-modules.
    Then (*) is exact if and only if the induced sequence of $R_{\mathfrak{p}}$-modules
    \[
        \begin{tikzcd}
            0 & A_{\mathfrak{p}} & B_{\mathfrak{p}} & C_{\mathfrak{p}} & 0
            \arrow[from=1-1, to=1-2]
            \arrow[from=1-2, to=1-3]
            \arrow[from=1-3, to=1-4]
            \arrow[from=1-4, to=1-5] 
        \end{tikzcd}
    \]
    is exact for evey prime ideal $\mathfrak{p}$ of $R$, if and only if it is exact for every maximal ideal $\mathfrak{p}$.
\end{problem}

\begin{solution}
    Suppose (*) is exact and let $\mathfrak{p}$ be a prime ideal of $R$.
    Then $S = R \setminus \mathfrak{p}$ so we may consider the localization $R_{\mathfrak{p}}$.
    Since $(*)$ is exact, each homology group vanishes.
    Furthermore, since localization preserves homology, the homology of the induced sequence of $R_{\mathfrak{p}}$-modules also vanishes, hence it is exact.

    Now suppose the induced sequence is exact for every prime ideal of $R$.
    In particular, since maximal ideals are prime, the sequence is exact for every maximal ideal of  $R$.

    Finally, suppose the induced sequence of $R_{\mathfrak{p}}$-modules is exact for every maximal ideal $\mathfrak{p}$ of $R$.
    Consider a homology group $H$ of (*) with the inherited $R$-module structure and suppose that $m \neq 0$ is in $H$.
    Then the ideal $\{r \in R | rm = 0\}$ is a proper ideal of $R$ (since $1 \cdot m \neq 0$).
    In particular, it is contained in some maximal ideal $\mathfrak{m}$.
    But then we may consider the localized homology group $H_{\mathfrak{m}}$, which is nonempty in this case.
    This implies that the sequence of $R_{\mathfrak{m}}$-modules is not exact.
    Thus, the contrapositive yields the desired result.
\end{solution}

% Problem 1.27
\begin{problem}
    Let $R, S$ be rings.
    Prove that right-adjoint functors $R\mathsf{-Mod} \to S\mathsf{-Mod}$ are left-exact and left-adjoint functors are right-exact.
\end{problem}

\begin{solution}
    Let $\mathscr{F} : R\mathsf{-Mod} \to S\mathsf{-Mod}$ be a right-adjoint functor, say it is right-adjoint to $\mathscr{G}$.
    Consider an exact sequence of $R$-modules
    \[
    \begin{tikzcd}
        0 & A & B & C & 0
        \arrow[from=1-1, to=1-2]
        \arrow["f", from=1-2, to=1-3]
        \arrow["g", from=1-3, to=1-4]
        \arrow[from=1-4, to=1-5] 
    \end{tikzcd}
    \]
    We want to show that the sequence
    \[
        \begin{tikzcd}
            \mathscr{F}(0) & \mathscr{F}(A) & \mathscr{F}(B) & \mathscr{F}(C)
            \arrow[from=1-1, to=1-2] 
            \arrow["\mathscr{F}(f)", from=1-2, to=1-3]
            \arrow["\mathscr{F}(g)", from=1-3, to=1-4]
        \end{tikzcd}
    \]
    is exact.
    First, recall that $0$ as a zero object, hence a final object, is a limit.
    Furthermore, right-adjoints commute with limits.
    Therefore, $\mathscr{F}$ preserves this object; that is, $\mathscr{F}(0) = 0$.
    Thus, it suffices to show that $\ker\mathscr{F}(f) = 0$ and $\ker \mathscr{F}(g) = \im \mathscr{F}(f)$.

    Recall that the kernel is a categorical limit, so $\mathscr{F}$ preserves kernels.
    Since $\ker f = 0$, we find that
    \[
        \ker \mathscr{F}(f) = \mathscr{F}(\ker f) = \mathscr{F}(0) = 0.
    \]
    Furthermore, we find that $\ker \mathscr{F}(g) = \mathscr{F}(\ker g) = \mathscr{F}(\im f)$.
    Since $\ker \mathscr{F}(f) = 0$, $\mathscr{F}(f)$ is injective, hence $\mathscr{F}(A) \cong \im \mathscr{F}(f)$.
    But from this, we find that
    \[
        \ker \mathscr{F}(g) = \mathscr{F}(\ker g) = \mathscr{F}(\im f) \cong \mathscr{F}(A) \cong \im \mathscr{F}(f).
    \]
    To tell the truth, this probably doesn't verify what's necessary but I'm stuck at this point so idk.
    I figure it's a similar strategy for showing that left-adjoints are right-exact.
    Note to come back and finish missing problems / To do.
\end{solution}
\end{document}
