\documentclass[../../master.tex]{subfiles}

\begin{document}
\section{Tensor products and the Tor functors}

% Problem 2.1
\begin{problem}
    Let $M, N$ be $R$-modules, and assume that $N$ is \textit{cyclic}.
    Prove that every element of $M \otimes_R N$ may be written as a pure tensor.
\end{problem}

\begin{solution}
    Recall that a cyclic module is generated by one element, say $N = \langle a \rangle$.
    That is, for all $n \in N$, we can write $n = s \cdot a$ for some $s \in R$.
    Let
    \[
        \sum_{i} r_i(m_i \otimes n_i) \in M \otimes_R N.
    \]
    Then by the $R$-bilinearity of $\otimes$, we find
    \[
        \sum_{i} r_i (m_i \otimes n_i) = \sum_{i} (r_i m_i) \otimes s_i a = \left( \sum_{i} r_i s_i m_i \right) \otimes a
    \]
    which is a pure tensor since $\sum_{i} r_i s_i m_i \in M$ because $M$ is an $R$-module.
    Thus, every element can be written as a pure tensor.
\end{solution}

% Problem 2.2
\begin{problem}
    Prove `by hand' (that is, without appealing to the right-exactness of tensor) that $\mathbb{Z} / n\mathbb{Z} \otimes_{\mathbb{Z}} \mathbb{Z}/m\mathbb{Z} \cong 0$ if $m, n$ are relatively prime integers.
\end{problem}

\begin{solution}
    Since $\gcd(m, n) = 1$, there exist integers $x, y$ such that $mx + ny = 1$.
    Consider an element $a \otimes b$.
    Then we have
    \[
        a \otimes b = 1 \cdot a \otimes b = (mx + ny) \cdot a \otimes b = mx a \otimes b + a \otimes ny b = 0
    \]
    via the bilinearity of $\otimes$.
\end{solution}

% Problem 2.3
\begin{problem}
    Prove that $R[x_1, \ldots, x_n] \otimes_R R[y_1, \ldots, y_m] \cong R[x_1, \ldots, x_n, y_1, \ldots, y_m]$.
\end{problem}

\begin{solution}
    The polynomial rings in question are free over $R$ with bases consisting of the monomials.
    Then the pure tensors which are tensor products of monomials generate $R[x_1, \ldots, x_n] \otimes R[y_1, \ldots, y_m]$.
    In particular, letting $a \otimes b = a \cdot b$ shows that the pure tensors are mapped to all monomials, hence generating $R[x_1, \ldots, x_n, y_1, \ldots, y_m]$.
    The homomorphism in the other direction is merely separating monomials into the $x_i$ and $y_j$.
\end{solution}

% Problem 2.4
\begin{problem}
    Let $S, T$ be commutative $R$-algebras.
    Verify the following:
    \begin{itemize}
        \item The tensor product $S \otimes_R T$ has an operation of multiplication, defined on pure tensors by $(s_1 \otimes t_1) \cdot (s_2 \otimes t_2) := s_1 s_2 \otimes t_1 t_2$ and making it into a commutative $R$-algebra.
        \item With respect to this structure, there are $R$-algebra homomorphisms $i_S : S \to S \otimes T$, resp., $i_T : T \to S \otimes T$, defined by $i_S(s) := s \otimes 1$, $i_T(t) := 1 \otimes t$.
        \item The $R$-algebra $S \otimes_R T$, with these two structure homomorphisms, is a coproduct of $S$ and $T$ in the category of commutative $R$-algebras:
            if $U$ is a commutative $R$-algebra and $f_S : S \to U$, $f_T : T \to U$ are $R$-algebra homomorphisms, then there exists a unique $R$-algebra homomorphism $f_S \otimes f_T$ making the following diagram commute:
            \[
            \begin{tikzcd}
                S \\
                && {S \otimes_R T} && U \\
                T
                \arrow["{f_S \otimes f_T}", from=2-3, to=2-5]
                \arrow["{i_T}"', from=3-1, to=2-3]
                \arrow["{i_S}", from=1-1, to=2-3]
                \arrow["{f_S}", bend left=12, from=1-1, to=2-5]
                \arrow["{f_T}"', bend right=12, from=3-1, to=2-5]
            \end{tikzcd}
            \]
            In particular, if $S$ and $T$ are simply commutative rings, then $S \otimes_\mathbb{Z} T$ is a coproduct of $S$ and $T$ in the category of commutative rings.
            This settles an issue left open at the end of \S III.2.4.
    \end{itemize}
\end{problem}

\begin{solution}
    To verify that this defines a commutative $R$-algebra structure, we check that the multiplication defines a ring structure on $S \otimes_R T$.
    It suffices to show that this holds on pure tensors.
    Indeed,
    \begin{align*}
        (s \otimes t) \cdot (a \otimes b + a \otimes c) &= (s \otimes t) \cdot (a \otimes (b + c)) \\
                                                        &= sa \otimes t(b + c) \\
                                                        &= sa \otimes (tb + tc) \\
                                                        &= (sa \otimes tb) + (sa \otimes tc) \\
                                                        &= (s \otimes t) \cdot (a \otimes b) + (s \otimes t) \cdot (a \otimes c).
    \end{align*}
    It is just as straightforward to show distributivity of the other part of addition.
    Furthermore, the multiplication respects scalar multiplication:
    \[
        r \cdot ((s_1 \otimes t_1) \cdot (s_2 \otimes t_2)) = (r \cdot s_1 s_2) \otimes t_1 t_2 = s_1 s_2 \otimes (r t_1 t_2)
    \]
    by the bilinearity of $\otimes$.
    Commutativity is inherited from the fact that $S$ and $T$ are commutative.

    It is clear that the two maps $i_S$ and $i_T$ are indeed homomorphisms; we verify it for the former:
    \[
        i_S(s_1 + s_2) = (s_1 + s_2) \otimes 1 = s_1 \otimes 1 + s_2 \otimes 1 = i_S(s_1) + i_S(s_2),
    \]
    \[
        i_S(s_1 \cdot s_2) = (s_1 \cdot s_2) \otimes 1 = (s_1 \otimes 1) \cdot (s_2 \otimes 1) = i_S(s_1) \cdot i_S(s_2),
    \]
    and
    \[
        i_S(r \cdot s) = (r \cdot s) \otimes 1 = r (s \otimes 1) = r \cdot i_S(s).
    \]
    
    Finally, consider a commutative $R$-algebra $U$ equipped with the described homomorphisms $f_S$ and $f_T$.
    Then there exists an $R$-algebra homomorphism $f_S \otimes f_T : S \otimes_R T \to U$ defined on pure tensors by $f_S \otimes f_T (s \otimes t) = f_S(s) \cdot f_T(t)$.
    It is clear that this makes the necessary diagram commute as given $s \in S$, we have $f_S(s) = f_S \otimes f_T (s \otimes 1) = f_S \otimes f_T (i_S(s))$ (and similarly for $t \in T$).
    For uniqueness, let $h: S \otimes_R T \to U$ be defined such that $h \circ i_S = f_S$ and $h \circ i_T = f_T$.
    Then these properties imply that
    \[
        h(s \otimes t) = h((s \otimes 1) \cdot (1 \otimes t)) = h(i_S(s)) \cdot h(i_T(t)) = f_S(s) \cdot f_T(t).
    \]
    Therefore, $h = f_S \otimes f_T$.
\end{solution}

% Problem 2.5
\begin{problem}
    Let $S$ be a multiplicative subset of $R$, and let $M$ be an $R$-module.
    Prove that $S^{-1}M \cong M \otimes_R S^{-1}R$ as $R$-modules.
    (Use the universal property of the tensor product.)

    Through this isomorphism, $M \otimes_R S^{-1}$ inherits an $S^{-1}R$-module structure.
\end{problem}

\begin{solution}
    Consider the homomorphism $f : M \otimes_R S^{-1}R$ defined by $f(m \otimes \frac{r}{s}) = \frac{r \cdot m}{s}$.
    The homomorphism has an inverse given by $\frac{m}{s} \mapsto m \otimes \frac{1}{s}$.
    Indeed, we have
    \[
        f^{-1}\left(f\left(m \otimes \frac{r}{s}\right)\right) = f^{-1}\left(\frac{r \cdot m}{s}\right) = r \cdot f^{-1}\left(\frac{m}{s}\right) = r\left(m \otimes \frac{1}{s}\right) = m \otimes \frac{r}{s}.
    \]
    The other direction is verified just as easily.
\end{solution}

% Problem 2.6
\begin{problem}
    Let $S$ be a multiplicative subset of $R$, and let $M$ be an $R$-module.
    \begin{itemize}
        \item Let $N$ be an $S^{-1}R$-module.
            Prove that $(S^{-1}M) \otimes_{S^{-1}R} N \cong M \otimes_R N$.
        \item Let $A$ be an $R$-module.
            Prove that $(S^{-1}A) \otimes_R M \cong S^{-1}(A \otimes_R M)$.
    \end{itemize}
    (Both can be done `by hand' by analyzing the construction in Lemma 2.3.
    For example, there is a homomorphism $M \otimes_R N \to (S^{-1}M) \otimes_{S^{-1}R} N$ which is surjective because, with evident notation, $\frac{m}{s} \otimes n = m \otimes \frac{n}{s}$ in $(S^{-1}M) \otimes_{S^{-1}R} N$;
    checking that it is injective amounts to easy manipulation of the relations defining the two tensor products.

    Both isomorphisms will be easy consequences of the associativity of tensor products.)
\end{problem}

\begin{solution}
    For the first isomorphism, consider the map $f$ which sends $m \otimes n \mapsto \frac{m}{1} \otimes n$.
    To see that this map is surjective, let $\frac{m}{s} \otimes n \in (S^{-1}M) \otimes_{S^{-1}R} N$.
    Then, by the bilinearity of $\otimes$, $\frac{m}{s} \otimes n = m \otimes \frac{n}{s} = f(m \otimes \frac{n}{s})$ where the latter element is in $M \otimes_R N$ because $N$ is an $S^{-1}R$-module.
    To see that this map is injective, suppose $f(m_1 \otimes n_1) = f(m_2 \otimes n_2)$; that is, $\frac{m_1}{1} \otimes n_1 = \frac{m_2}{1} \otimes n_2$.
    Again by bilinearity, this implies that $m_1 \otimes n_1 = m_2 \otimes n_2$ (where we move the denominator of 1 to the other term and ``cancel'' it out).
    This equivalence holds over the base ring $R$, hence the map is injective.

    For the second isomorphism, consider the map $f$ which sends $\frac{a}{s} \otimes m \mapsto \frac{a \otimes m}{s}$.
    Clearly this map is surjective.
    To see that it is injective, suppose $f(\frac{a_1}{s_1} \otimes m_1) = f(\frac{a_2}{s_2} \otimes m_2)$; that is, $\frac{a_1 \otimes m_1}{s_1} = \frac{a_2 \otimes m_2}{s_2}$.
    Then, by the definition of equivalence in localization, we find that there exists some $u \in R$ such that
    \[
        u(s_2 a_1 \otimes m_1 - s_1 a_2 \otimes m_2) = 0.
    \]
    Using the properties of the tensor product, we reduce this to
    \[
    u a_1 s_2 \otimes m_1 - u a_2 s_1 \otimes m_2 = 0
    \]
    which implies that $m_1 = m_2$ and $u(a_1 s_2 - a_2 s_1) = 0$, hence $\frac{a_1}{s_1} = \frac{a_2}{s_2}$ as elements of $S^{-1}A$.
    Thus, $\frac{a_1}{s_1} \otimes m_1 = \frac{a_2}{s_2} \otimes m_2$ and $f$ is injective.
\end{solution}

% Problem 2.7
\begin{problem}
    Changing the base ring in a tensor may or may not make a difference:
    \begin{itemize}
        \item Prove that $\mathbb{Q} \otimes_\mathbb{Z} \mathbb{Q} \cong \mathbb{Q} \otimes_\mathbb{Q} \mathbb{Q}$.
        \item Prove that $\mathbb{C} \otimes_\mathbb{R} \mathbb{C} \not\cong \mathbb{C} \otimes_\mathbb{C} \mathbb{C}$.
    \end{itemize}
\end{problem}

\begin{solution}
    For the first isomorphism, note that $\mathbb{Q} \otimes_\mathbb{Q} \mathbb{Q} \cong \mathbb{Q}$.
    Now consider the map $f : \mathbb{Q} \otimes_\mathbb{Z} \mathbb{Q} \to \mathbb{Q}$ which sends $\frac{a}{b} \otimes \frac{c}{d} \mapsto \frac{ac}{bd}$.
    It is clear that this map is surjective, and to see that it is injective, suppose $\frac{ac}{bd} = \frac{eg}{fh}$.
    Then
    \[
        \frac{a}{b} \otimes \frac{c}{d} = \frac{a}{b} \otimes \left( \frac{c}{d} \cdot \frac{b}{b} \right) = \frac{a}{b} \otimes \left(b \frac{c}{d} \cdot \frac{1}{b} \right) = a \otimes \frac{c}{bd} = 1 \otimes \frac{ac}{bd} = 1 \otimes \frac{eg}{fh} = \frac{e}{f} \otimes \frac{g}{h}.
    \] 
    hence the map is an isomorphism.

    To see that $\mathbb{C} \otimes_\mathbb{R} \mathbb{C} \not\cong \mathbb{C} \otimes_{\mathbb{C}} \mathbb{C}$, first note that the latter is isomorphic to $\mathbb{C}$.
    Suppose that there were some isomorphism between the two; clearly it must map $1 \otimes 1 \mapsto 1$.
    Then, by the linearity of the map, we find that $i \otimes 1 \mapsto i$ and $1 \otimes i \mapsto i$, hence $i \otimes 1 = 1 \otimes i$.
    However, this contradicts the relations induced on $\mathbb{C} \otimes_\mathbb{R} \mathbb{C}$ as $i \notin \mathbb{R}$.
    Another way to see this is that the elements $1 \otimes i, i \otimes 1$ are linearly independent in this module.
\end{solution}

% Problem 2.8
\begin{problem}
    Let $R$ be an integral domain, with field of fractions $K$, and let $M$ be a finitely generated $R$-module.
    The tensor product $V := M \otimes_R K$ is a $K$-vector space (Exercise 2.5).
    Prove that $\dim_K V$ equals the rank of $M$ as an $R$-module, in the sense of Definition VI.5.5.
\end{problem}

\begin{solution}
    Recall that the rank of $M$ as an $R$-module is the cardinality of a maximal linearly independent set.
    Let $S$ be such a set for $M$ and consider the set $\{s_i \otimes 1\}$ which has the same cardinality as $S$ and forms a basis for $V$.
    Indeed, let $m \otimes \frac{r}{s} \in V$.
    Then since $S$ is a basis for $M$, $m$ can be written as a linear combination of the $s_i$ so we find
    \[
        m \otimes \frac{r}{s} = \frac{r}{s} \sum_{i} r_i \left( s_i \otimes 1 \right),
    \]
    hence $\dim_K V$ agrees with the rank of $M$.
\end{solution}

% Problem 2.9
\begin{problem}
    Let $G$ be a finitely generated abelian group of rank $r$.
    Prove that $G \otimes_\mathbb{Z} \mathbb{Q} \cong \mathbb{Q}^{r}$.
    Prove that for infinitely many primes $p$, $G \otimes_\mathbb{Z} (\mathbb{Z} / p\mathbb{Z}) \cong (\mathbb{Z} / p\mathbb{Z})^{r}$.
\end{problem}

\begin{solution}
    By the classification of f.g. abelian groups, $G \cong \mathbb{Z}^{r} \oplus \mathbb{Z}_{q_1} \oplus \cdots \oplus \mathbb{Z}_{q_n}$.
    Since the tensor product distributes over direct sums, we find that
    \[
        G \otimes_\mathbb{Z} \mathbb{Q} \cong (\mathbb{Z}^{r} \otimes_\mathbb{Z} \mathbb{Q}) \oplus (\mathbb{Z}_{q_1} \otimes_\mathbb{Z} \mathbb{Q}) \oplus \cdots \oplus (\mathbb{Z}_{q_n} \otimes_\mathbb{Z} \mathbb{Q}).
    \]
    Since $\mathbb{Q}$ is a $\mathbb{Z}$-module, the first term is isomorphic to $\mathbb{Q}^{r}$.
    For the remaining terms, consider $[a] \otimes \frac{m}{n} \in \mathbb{Q}_{q_i} \otimes_\mathbb{Z} \mathbb{Q}$.
    We find that
    \[
        [a] \otimes \frac{m}{n} \cdot \frac{q_i}{q_i} = q_i \cdot [a] \otimes \frac{m}{n \cdot q_i} = 0 \otimes \frac{m}{n \cdot q_i} = 0
    \]
    so the remaining summands are all zero.
    Thus, we find $G \otimes_\mathbb{Z} \mathbb{Q} \cong \mathbb{Q}^{r}$.

    For the second part, simply let $G = \mathbb{Z}^{r}$.
    Then the tensor product distributes over the direct sum, and since $\mathbb{Z} / p\mathbb{Z}$ is a $\mathbb{Z}$-module, we find that
    \[
        G \otimes_\mathbb{Z} (\mathbb{Z}/p\mathbb{Z}) \cong (\mathbb{Z}/p\mathbb{Z})^{r}.
    \]
    as desired.
\end{solution}

% Problem 2.10
\begin{problem}
    Let $k \subseteq k(\alpha) = F$ be a finite simple field extension.
    Note that $F \otimes_k F$ has a natural ring structure: cf. Exercise 2.4.
    \begin{itemize}
        \item Prove that $\alpha$ is separable over $k$ if and only if $F \otimes_k F$ is \textit{reduced} as a ring.
        \item Prove that $k \subseteq F$ is Galois if and only if $F \otimes_k F$ is isomorphic to $F^{[F : k]}$ as a ring.
    \end{itemize}
    (Use Corollary 2.8 to `compute' the tensor.
    The CRT from \S V.6.1 will likely be helpful.)
\end{problem}

\begin{solution}
    To do.
\end{solution}

% Problem 2.11
\begin{problem}
    Complete the proof of Lemma 2.4.
    \begin{proposition}[Lemma 2.4] 
        For all $R$-modules $M, N, P$, there is an isomorphism of $R$-modules
        \[
            \Hom_R(M, \Hom_R(N, P)) \cong \Hom_R(M \otimes_R N, P).
        \]
    \end{proposition}
\end{problem}

\begin{solution}
    Every $\alpha \in \Hom_R(M, \Hom_R(N, P))$ determines an $R$-bilinear map $\varphi : M \times N \to P$, by $(m, n) \mapsto \alpha(m)(n)$.
    The map $\varphi$ factors uniquely through an $R$-linear map $\overline{\varphi} : M \otimes_R N \to P$, hence $\alpha$ determines an element $\overline{\varphi} \in \Hom_R(M \otimes_R N, P)$.
    We verify that this map $\alpha \mapsto \overline{\varphi}$ is $R$-linear and has an inverse.

    Let $\alpha, \beta \in \Hom_R(M, \Hom_R(N, P))$ and note that these correspond to maps $\varphi_\alpha, \varphi_\beta \in \Hom_R(M \otimes_R N, P)$.
    Now consider $\alpha + \beta$.
    We wish to show that $\varphi_{\alpha + \beta} = \varphi_\alpha + \varphi_\beta$.
    Indeed, on elements, we find that
    \[
        \varphi_{\alpha + \beta}(m \otimes n) = (\alpha + \beta)(m)(n) = \alpha(m)(n) + \beta(m)(n) = \varphi_\alpha(m \otimes n) + \varphi_\beta(m \otimes n)
    \]
    by the linearity of the maps.
    Furthermore, given $\alpha \in \Hom_R(M, \Hom_R(N, P))$, we have that
    \[
        \varphi_{r \cdot \alpha}(m \otimes n) (r \cdot \alpha)(m)(n) = r \cdot \alpha(m)(n) = r \cdot \varphi_\alpha (m \otimes n)
    \]
    so the correspondence is $R$-linear.
    
    To conceive of an inverse, consider an element $\varphi \in \Hom_R(M \otimes_R N, P)$.
    This determines a unique bilinear map $\overline{\varphi} : M \times N \to P$ by $\overline{\varphi}(m, n) = \varphi(m \otimes n)$.
    Then $\overline{\varphi}$ determines an element $\alpha \in \Hom_R(M, \Hom_R(N, P))$ by $\alpha(m) = \overline{\varphi}(m, \_)$.

    To see that these maps are inverses, let $\alpha \in \Hom_R(M, \Hom_R(N, P))$.
    Then the corresponding $\varphi_\alpha$ is defined by $m \otimes n \mapsto \alpha(m)(n)$.
    To go back, we send $\varphi_{\alpha}$ to $\overline{\varphi_\alpha}$ which maps $(m, n) \mapsto \varphi_\alpha(m \otimes n) = \alpha(m)(n)$.
    Finally, we return to the original function $\alpha$ by mapping $\overline{\varphi_\alpha}$ to $\alpha(m) = \overline{\varphi_\alpha}(m, n) = \varphi_\alpha(m \otimes n) = \alpha(m)(n)$ as desired.
\end{solution}

% Problem 2.12
\begin{problem}
    Let $S$ be a multiplicative subset of $R$.
    Prove that $S^{-1}R$ is flat over $R$.
    (Hint: Exercise 2.5 and 1.25.)
\end{problem}

\begin{solution}
    Let
    \[
    \begin{tikzcd}
        0 & A & B & C & 0
        \arrow[from=1-1, to=1-2]
        \arrow[from=1-2, to=1-3]
        \arrow[from=1-3, to=1-4]
        \arrow[from=1-4, to=1-5] 
    \end{tikzcd}
    \]
    be an exact sequence of $R$-modules.
    Tensoring by $S^{-1}R$ yields the chain complex
    \[
    \begin{tikzcd}
        0 & A \otimes_R S^{-1}R & B \otimes_R S^{-1}R & C \otimes_R S^{-1}R & 0
        \arrow[from=1-1, to=1-2]
        \arrow[from=1-2, to=1-3]
        \arrow[from=1-3, to=1-4]
        \arrow[from=1-4, to=1-5] 
    \end{tikzcd}
    \]
    but by Exercise 2.5, for all $R$-modules $M$, we have the isomorphism $M \otimes_R S^{-1}R \cong S^{-1}M$.
    Thus, we are looking at the chain complex
    \[
    \begin{tikzcd}
        0 & S^{-1}A & S^{-1}B & S^{-1}C & 0
        \arrow[from=1-1, to=1-2]
        \arrow[from=1-2, to=1-3]
        \arrow[from=1-3, to=1-4]
        \arrow[from=1-4, to=1-5] 
    \end{tikzcd}
    \]
    and since localization is an exact functor, we find that this is also an exact sequence.
    Thus, the functor $\underline{\hspace{1em}} \otimes_R S^{-1}R$ is exact, hence $S^{-1}R$ is a flat $R$-modules.
\end{solution}

% Problem 2.13
\begin{problem}
    Prove that direct sums of flat modules are flat.
\end{problem}

\begin{solution}
    Suppose $M, N$ are flat $R$-modules and let
    \[
    \begin{tikzcd}
        0 & A & B & C & 0
        \arrow[from=1-1, to=1-2]
        \arrow[from=1-2, to=1-3]
        \arrow[from=1-3, to=1-4]
        \arrow[from=1-4, to=1-5] 
    \end{tikzcd}
    \]
    be an exact sequence of $R$-modules.
    Tensoring by $M \oplus N$ yields the complex
    \[
    \begin{tikzcd}
        0 & A \otimes_R (M \oplus N) & B \otimes_R (M \oplus N) & C \otimes_R (M \oplus N) & 0
        \arrow[from=1-1, to=1-2]
        \arrow[from=1-2, to=1-3]
        \arrow[from=1-3, to=1-4]
        \arrow[from=1-4, to=1-5] 
    \end{tikzcd}
    \]
    Since the tensor functor is left-adjoint, it commutes with colimits.
    In particular, we have $A \otimes_R (M \oplus N) \cong (A \otimes_R M) \oplus (A \otimes_R N)$.
    Since the two sequences induced by $\underline{\hspace{1em}} \otimes_R M$ and $\underline{\hspace{1em}} \otimes_R N$ are exact (as $M$ and $N$ are flat), the direct sum of these sequences is also exact (where the differential maps are defined componentwise).
    Thus, $M \oplus N$ is also a flat $R$-module.
\end{solution}

% Problem 2.14
\begin{problem}
    Prove that, according to the definition given in \S 2.4, $\Tor_0^{R}(M, N)$ is isomorphic to $M \otimes_R N$.
\end{problem}

\begin{solution}
    Given an $R$-module $M$, we begin computing $\Tor_0^{R}(M, N)$ by finding a free resolution of $M$:
    \[
    \begin{tikzcd}
        \cdots & R^{\oplus S_1} & R^{\oplus S_0} & M & 0
        \arrow[from=1-1, to=1-2]
        \arrow[from=1-2, to=1-3]
        \arrow[from=1-3, to=1-4]
        \arrow[from=1-4, to=1-5] 
    \end{tikzcd}
    \]
    Then we tensor this resolution by $N$ to obtain the right-exact sequence
    \[
    \begin{tikzcd}
        \cdots & R^{\oplus S_1} \otimes_R N & R^{\oplus S_0} \otimes_R N & M \otimes_R N & 0
        \arrow["d_2", from=1-1, to=1-2]
        \arrow["d_1", from=1-2, to=1-3]
        \arrow["d_0", from=1-3, to=1-4]
        \arrow[from=1-4, to=1-5] 
    \end{tikzcd}
    \]
    Finally, we throw away the term $M \otimes_R N$ to obtain a chain complex, and we compute the $\Tor$ functors by taking the homology of this complex.
    In particular, we have
    \[
        \Tor_0^{R}(M, N) = \frac{R^{\oplus S_0} \otimes_R N}{\im d_1} \cong \frac{R^{\oplus S_0} \otimes_R N}{\ker d_0} \cong M \otimes_R N
    \]
    where the first isomorphism follows from the right-exactness of the free resolution after tensoring and the second isomorphism follows from the surjectivity of $d_0$ and the isomorphism theorem.
\end{solution}

% Problem 2.15
\begin{problem}
    Prove that for $r \in R$, a non-zero-divisor and $N$ an $R$-module, the module $\Tor_1^{R}(R / (r), N)$ is isomorphic to the $r$-torsion of $N$, that is, the submodule of elements $n \in N$ such that $rn = 0$ (cf. \S VI.4.1).
    (This is the reason why $\Tor$ is called $\Tor$.)
\end{problem}

\begin{solution}
    We begin by finding a free resolution for $R / (r)$:
    \[
    \begin{tikzcd}
        0 & R & R & R/(r) & 0
        \arrow["i", from=1-1, to=1-2]
        \arrow["\cdot r", from=1-2, to=1-3]
        \arrow["\pi", from=1-3, to=1-4]
        \arrow[from=1-4, to=1-5] 
    \end{tikzcd}
    \]
    and then tensoring by $N$ to obtain the right-exact sequence
    \[
    \begin{tikzcd}
        0 & N & N & R/(r) \otimes_R N & 0
        \arrow["d_2", from=1-1, to=1-2] 
        \arrow["d_1", from=1-2, to=1-3]
        \arrow["d_0", from=1-3, to=1-4]
        \arrow[from=1-4, to=1-5] 
    \end{tikzcd}
    \]
    Then, by our standard procedure, we have
    \[
        \Tor_1^{R}(R/(r), N) = \frac{\ker d_1}{\im d_2} = \{n \in N \mid rn = 0\}
    \]
    which is precisely the $r$-torsion of $N$.
\end{solution}

% Problem 2.16
\begin{problem}
    Let $I, J$ be ideals of $R$.
    Prove that $\Tor_1^{R}(R/I, R/J) \cong (I \cap J)/IJ$.
    (For example, this $\Tor_1^{R}$ vanishes if $I + J = R$, by Lemma V.6.2.)
    Prove that $\Tor_i^{R}(R/I, R/J)$ is isomorphic to $\Tor_{i-1}^{R}(I, R/J)$ for $i > 1$.
\end{problem}

\begin{solution}
    We have the exact sequence of $R$-modules
    \[
    \begin{tikzcd}
        0 & I & R & \frac{R}{I} & 0
        \arrow[from=1-1, to=1-2]
        \arrow[from=1-2, to=1-3]
        \arrow[from=1-3, to=1-4]
        \arrow[from=1-4, to=1-5] 
    \end{tikzcd}
    \]
    and by the definition of $\Tor$, tensoring by $R/J$ yields the long exact sequence
    \[
    \begin{tikzcd}
        \cdots & \Tor_1^{R}(R, \frac{R}{J}) & \Tor_1^{R}(\frac{R}{I}, \frac{R}{J}) & \frac{I}{IJ} & \frac{R}{J} & \frac{R}{I + J} & 0.
        \arrow[from=1-1, to=1-2]
        \arrow[from=1-2, to=1-3]
        \arrow[from=1-3, to=1-4]
        \arrow[from=1-4, to=1-5] 
        \arrow[from=1-5, to=1-6] 
        \arrow[from=1-6, to=1-7] 
    \end{tikzcd}
    \]
    Note that $\Tor_1^{R}(R, R/J) = 0$ because $R$ is free as an $R$-module, hence flat.
    Thus, $\Tor_1^{R}(R/I, R/J)$ is the kernel of the map $I/IJ \to R/J$ which sends $i + IJ \mapsto i + J$.
    An element $i \in I$ is sent to $J$ if and only if $i \in I \cap J$.
    Thus, we find
    \[
        \Tor_1^{R}(R/I, R/J) \cong \frac{I \cap J}{IJ}
    \]
    as desired.

    For the higher $\Tor$ groups, we look at more terms of the induced long exact sequence.
    \[
        \begin{tikzcd}
        \cdots & \Tor_2^{R}(I, R/J) & \Tor_2^{R}(R, R/J) & \Tor_2^{R}(R/I, R/J) \\
               & \Tor_1^{R}(I, R/J) & \Tor_1^{R}(R, R/J) & \Tor_1^{R}(R/I, R/J) \\
               & \frac{I}{IJ} & \frac{R}{J} & \frac{R}{I+J} & 0.
               \arrow[from=1-1, to=1-2]
               \arrow[from=1-2, to=1-3]
               \arrow[from=1-3, to=1-4] 
               % Arrow from 1-4 to 2-2
               \arrow[from=1-4, to=2-2, out=355, in=175, overlay] 
               \arrow[from=2-2, to=2-3] 
               \arrow[from=2-3, to=2-4] 
               % Arrow from 2-4 to 3-2
               \arrow[from=2-4, to=3-2, out=355, in=175, overlay] 
               \arrow[from=3-2, to=3-3]
               \arrow[from=3-3, to=3-4]
               \arrow[from=3-4, to=3-5] 
    \end{tikzcd}
    \]
    For all $i > 0$, we have $\Tor_i^{R}(R, R/J) = 0$ since $R$ is flat.
    Thus, at each step we have the short exact sequence
    \[
    \begin{tikzcd}
        0 & \Tor_i^{R}(R/I, R/J) & \Tor_{i-1}^{R}(I, R/J) & 0
        \arrow[from=1-1, to=1-2]
        \arrow[from=1-2, to=1-3]
        \arrow[from=1-3, to=1-4] 
    \end{tikzcd}
    \]
    which implies that $\Tor_i^{R}(R/I, R/J) \cong \Tor_{i-1}^{R}(I, R/J)$.
\end{solution}

% Problem 2.17
\begin{problem}
    Let $M, N$ be modules over a PID $R$.
    Prove that $\Tor_i^{R}(M, N) = 0$ for $i \geq 2$.
    (Assume $M, N$ are finitely generated, for simplicity.)
\end{problem}

\begin{solution}
    Recall that every module over a PID admits a free resolution of length 1:
    \[
    \begin{tikzcd}
        0 & R^{\oplus m_1} & R^{\oplus m_0} & M & 0.
        \arrow[from=1-1, to=1-2]
        \arrow[from=1-2, to=1-3]
        \arrow[from=1-3, to=1-4] 
        \arrow[from=1-4, to=1-5] 
    \end{tikzcd}
    \]
    Applying the procedure to compute $\Tor$, we tensor by $N$ and then take the homology.
    However, since the induced complex is zero in degrees $i \geq 2$, we have $\Tor_i^{R}(M, N) = 0$ for $i \geq 2$.
\end{solution}

% Problem 2.18
\begin{problem}
    Let $R$ be an integral domain.
    Prove that a cyclic $R$-module is flat if and only if it is free.
\end{problem}

\begin{solution}
    Since every free module is flat, this direction is trivial.
    For the other direction, let $M = \langle m \rangle$ be a cyclic $R$-module and suppose $M$ is flat.
    That is $\Tor_1^{R}(N, M) = 0$ for all $R$-modules $N$;
    equivalently, $\underline{\hspace{1em}} \otimes_R M$ is an exact functor.
    In particular, since $R$ is an integral domain it has no nontrivial zero-divisors.
    Thus, for all $r \in R$, $\Tor_1^{R}(R/(r), M) = 0$.
    But by Exercise 2.15, $\Tor_1^{R}(R/(r), M) = \{m \in M \mid rm = 0\}$, so for all $r \in R$, $r \neq 0$, $rm = 0$ if and only if $m = 0$.

    Consider the surjective map $\varphi : R \to M$ given by $\varphi(r) = rm$.
    The kernel of this map is the set of $r \in R$ such that $rm = 0$ for all $m \in M$, but this is precisely 0.
    Thus, the map is injective, hence an isomorphism.
    In particular, $R \cong M$ and $M$ is free.
\end{solution}

% Problem 2.19
\begin{problem}
    The following criterion is quite useful.
    \begin{itemize}
        \item Prove that an $R$-module $M$ is flat if and only if every monomorphism of $R$-modules $A \hookrightarrow B$ induces a monomorphism of $R$-modules $A \otimes_R M \hookrightarrow B \otimes_R M$.
        \item Prove that it suffices to verify this condition for all \textit{finitely generated} modules $B$.
            (Hint: For once, refer back to the construction of tensor products given in Lemma 2.3.
            An element $\sum_i a_i \otimes m_i \in A \otimes_R M$ goes to zero in $B \otimes_R M$ if the corresponding element $\sum_i (a_i, m_i)$ equals a combination of the relations defining $B \otimes_R M$ in the free $R$-module $F^{R}(B \times M)$.
            This will be an identity involving only finitely many elements of $B$;
            hence\ldots.)
        \item Prove that it suffices to verify this condition when $B = R$ and $A = I$ is an ideal of $R$.
            (Hint: We may now assume that $B$ is finitely generated.
            Find submodules $B_j$ such that $A = B_0 \subseteq B_1 \subseteq \cdots \subseteq B_r = B$, with each $B_j / B_{j-1}$ cyclic.
            Reduce to verify that $A \otimes_R M$ injects in $B \otimes_R M$ when $B / A$ is cyclic, hence $\cong R/I$ for some ideal $I$.
            Conclude by a $\Tor_1^{R}$ argument or - but this requires a little more stamina - by judicious use of the snake lemma.)
        \item Deduce that an $R$-module is flat if and only if the natural homomorphism $I \otimes_R M \to IM$ is an isomorphism for every ideal $I$ of $R$.
    \end{itemize}
    If you believe in $\Tor$'s, now you can also show that an $R$-module $M$ is flat if and only if $\Tor_1^{R}(R/I, M) = 0$ for all ideals $I$ of $R$.
\end{problem}

\begin{solution}
    The first part effectively follows from the definition of flat modules.
    Monomorphisms of $R$-modules are the same as injective maps.
    Therefore, if tensoring preserves monomorphisms, then the functor $\underline{\hspace{1em}} \otimes_R M$ is left-exact and right-exact, hence it is exact.
    This is equivalent to $M$ being flat.

    For the second point, suppose an element $\sum_i a_i \otimes m_i \in A \otimes_R M$ goes to zero in $B \otimes_R M$.
    This occurs only if the corresponding element $\sum_i (a_i, m_i)$ is a combination of the relations defining $B \otimes_R M$ as a quotient of the free module on $B \times M$.
    This will only involve finitely many elements of $B$, hence we may restrict our attention to modules generated by such elements.
    Thus, it suffices to show that the condition holds for finitely generated modules.

    For the third point, we may now assume that $B$ is finitely generated.
    Let $S$ be a generating set for $B$ and construct a chain of submodules $A = B_0 \subseteq B_1 \subseteq \cdots \subseteq B_r = B$ where each module includes one more element from $S$.
    Then each quotient $B_j / B_{j-1}$ is cyclic.
    Inducting on the chain, we may restrict our attention to the case where $B/A$ is cyclic, hence isomorphic to $R/I$.
    Then we have the exact sequence
    \[
    \begin{tikzcd}
        0 & I & R & \frac{R}{I} & 0
        \arrow[from=1-1, to=1-2] 
        \arrow[from=1-2, to=1-3]
        \arrow[from=1-3, to=1-4]
        \arrow[from=1-4, to=1-5] 
    \end{tikzcd}
    \]
    and our condition is reduced to verifying that tensoring by $M$ preserves the injection $I \hookrightarrow R$.

    Writing out the tensor product, we have the map $I \otimes_R M \to R \otimes_R M = M$ given by $i \otimes m \mapsto im$.
    Clearly, the image is sent to the submodule $IM \subseteq M$, where the map is surjective.
    Thus, we simply need the map to be injective, in which case it would be an isomorphism.
\end{solution}

% Problem 2.20
\begin{problem}
    Let $R$ be a PID.
    Prove that an $R$-module $M$ is flat if and only if it is torsion-free.
    (If $M$ is finitely generated, the classification theorem of \S VI.5.3 makes this particularly easy.
    Otherwise, use Exercise 2.19.)

    Geometrically, this says roughly that an algebraic set fails to be `flat' over a nonsingular curve if and only if some component of the set is contracted to a point.
    This phenomenon is displayed in the picture in Example 2.14.
\end{problem}

\begin{solution}
    Every flat module is torsion-free.
    Indeed, a module $M$ is flat if and only if the map $I \otimes_R M \to IM$ is an isomorphism for every ideal $I$ of $R$.
    In particular, let $I = R$.
    Then the map $R \otimes_R M \to M$ given by $r \otimes m \mapsto rm$ is injective, which implies that $M$ is torsion-free.

    For the other direction, suppose $M$ is a torsion-free module.
    Since $R$ is a PID, we may write $I = (a)$.
    Then we have the exact sequence
    \[
    \begin{tikzcd}
        0 & (a) & R & R/(a) & 0
        \arrow[from=1-1, to=1-2]
        \arrow[from=1-2, to=1-3]
        \arrow[from=1-3, to=1-4]
        \arrow[from=1-4, to=1-5] 
    \end{tikzcd}
    \]
    and computing $\Tor_1^{R}(R/(a), M)$ gives the $a$-torsion of $M$, which is 0 since $M$ is torsion-free.
    Since this holds for all ideals of $R$, $M$ is a flat module.
\end{solution}

% Problem 2.21
\begin{problem}
    Prove that \textit{flatness is a local property}:
    an $R$-module $M$ is flat if and only if $M_{\mathfrak{p}}$ is a flat $R_{\mathfrak{p}}$-module for all prime ideals $\mathfrak{p}$, if and only if $M_{\mathfrak{m}}$ is a flat $R_{\mathfrak{m}}$-module for all maximal ideals $\mathfrak{m}$.
    (Hint: Use Exercises 1.25 and 2.6.
    The $\Longrightarrow$ direction will be straightforward.
    For the converse, let $A \subseteq B$ be $R$-modules, and let $K$ be the kernel of the induced homomorphism $A \otimes_R M \to B \otimes_R M$.
    Prove that the kernel of the localized homomorphism $A_{\mathfrak{m}} \otimes_{R_{\mathfrak{m}}} M_{\mathfrak{m}} \to B_{\mathfrak{m}} \otimes_{R_{\mathfrak{m}}} M_{\mathfrak{m}}$ is isomorphic to $K_{\mathfrak{m}}$, and use Exercise V.4.12.)
\end{problem}

\begin{solution}
    Suppose $M$ is a flat $R$-module.
    Let
    \[
    \begin{tikzcd}
        0 & A_{\mathfrak{p}} & B_{\mathfrak{p}} & C_{\mathfrak{p}} & 0
        \arrow[from=1-1, to=1-2]
        \arrow[from=1-2, to=1-3]
        \arrow[from=1-3, to=1-4]
        \arrow[from=1-4, to=1-5] 
    \end{tikzcd}
    \]
    be an exact sequence of $R_{\mathfrak{p}}$-modules.
    Tensoring by $M_{\mathfrak{p}}$ yields the complex
    \[
    \begin{tikzcd}
        0 & A_{\mathfrak{p}} \otimes_{R_\mathfrak{p}} M_{\mathfrak{p}} & B_{\mathfrak{p}} \otimes_{R_\mathfrak{p}} M_{\mathfrak{p}} & C_{\mathfrak{p}} \otimes_{R_\mathfrak{p}} M_{\mathfrak{p}} & 0
        \arrow[from=1-1, to=1-2]
        \arrow[from=1-2, to=1-3]
        \arrow[from=1-3, to=1-4]
        \arrow[from=1-4, to=1-5] 
    \end{tikzcd}
    \]
    By Exercise 2.6, we have isomorphisms $A_{\mathfrak{p}} \otimes_{R_{\mathfrak{p}}} M_{\mathfrak{p}} \cong A \otimes_R M_{\mathfrak{p}} \cong (A \otimes_R M)_{\mathfrak{p}}$.
    Thus, this complex is isomorphic to
    \[
    \begin{tikzcd}
        0 & (A \otimes M)_{\mathfrak{p}} & (B \otimes_R M)_{\mathfrak{p}} & (C \otimes_R M)_{\mathfrak{p}} & 0
        \arrow[from=1-1, to=1-2]
        \arrow[from=1-2, to=1-3]
        \arrow[from=1-3, to=1-4]
        \arrow[from=1-4, to=1-5] 
    \end{tikzcd}
    \]
    By the flatness of $M$ and the exactness of localization, this complex is exact.
    Therefore, $M_{\mathfrak{p}}$ is a flat $R_{\mathfrak{p}}$-module.

    Suppose $M_{\mathfrak{p}}$ is a flat $R_{\mathfrak{p}}$-module for all prime ideals $\mathfrak{p}$.
    Since maximal ideals are prime, $M_{\mathfrak{m}}$ is a flat $R_{\mathfrak{m}}$-module for all maximal ideals $\mathfrak{m}$.

    Finally, suppose $M_{\mathfrak{m}}$ is a flat $R_{\mathfrak{m}}$-module for all maximal ideals $\mathfrak{m}$.
    Consider an injection of $R$-modules $A \hookrightarrow B$.
    Let $K$ be the kernel of the induced homomorphism $A \otimes_R M \mapsto B \otimes_R M$.
    We have the exact sequence
    \[
    \begin{tikzcd}
        0 & K & A \otimes_R M & B \otimes_R M
        \arrow[from=1-1, to=1-2]
        \arrow[from=1-2, to=1-3]
        \arrow[from=1-3, to=1-4] 
    \end{tikzcd}
    \]
    Localizing by $\mathfrak{m}$ and using the exactness of localization yields the exact sequence
    \[
    \begin{tikzcd}
        0 & K_{\mathfrak{m}} & A_{\mathfrak{m}} \otimes_{R_{\mathfrak{m}}} M_{\mathfrak{m}} & B_{\mathfrak{m}} \otimes_{R_{\mathfrak{m}}} M_{\mathfrak{m}}
        \arrow[from=1-1, to=1-2]
        \arrow[from=1-2, to=1-3]
        \arrow[from=1-3, to=1-4] 
    \end{tikzcd}
    \]
    Therefore, $K_{\mathfrak{m}}$ is the kernel of the induced map, but by the exactness of localization and the flatness of $M_{\mathfrak{m}}$, $K_{\mathfrak{m}} = 0$.
    Then by Exercise V.4.12, $K = 0$ so the induced map is injective, hence tensoring by $M$ is exact.
\end{solution}

% Problem 2.22
\begin{problem}
    Let $M, N$ be $R$-modules, and let $S$ be a multiplicative subset of $R$.
    Use the definition of $\Tor$ given in \S 2.4 to show $S^{-1}\Tor_i^{R}(M, N) \cong \Tor_i^{S^{-1}R}(S^{-1}M, S^{-1}N)$.
    (Use Exercise 1.25.)
    Use this fact to give a leaner proof that flatness is a local property (Exercise 2.21).
\end{problem}

\begin{solution}
    Consider the exact sequence
    \[
    \begin{tikzcd}
        \Tor_1^{R}(M, N) & A \otimes_R N & B \otimes_R N & M \otimes_R N & 0
        \arrow[from=1-1, to=1-2]
        \arrow[from=1-2, to=1-3]
        \arrow[from=1-3, to=1-4]
        \arrow[from=1-4, to=1-5] 
    \end{tikzcd}
    \]
    Localizing gives us an exact sequence whose first term is $S^{-1}\Tor_1^{R}(M, N)$.
    Recall that we have the isomorphisms $S^{-1}(M \otimes_R N) \cong (S^{-1}M) \otimes_R N \cong (S^{-1}M) \otimes_{S^{-1}R} (S^{-1}N)$.
    Applying this to the induced exact sequence and using the property of the first $\Tor$ group of said sequence implies the isomorphism $S^{-1}\Tor_1^{R}(M, N) \cong \Tor_1^{S^{-1}R}(S^{-1}M, S^{-1}N)$.
    Inducting on the index of the $\Tor$ groups and using the property of the induced long exact sequence yields the isomorphism for higher $\Tor$ groups.

    To see that this implies that flatness is a local property, let $M$ be an $R$-module such that $M_{\mathfrak{m}}$ is a flat $R_{\mathfrak{m}}$-module for every maximal ideal $\mathfrak{m}$ of $R$.
    In particular, we have the isomorphism
    \[
        \Tor_1^{R}(M, N)_\mathfrak{m} \cong \Tor_1^{R_{\mathfrak{m}}}(M_{\mathfrak{m}}, N_{\mathfrak{m}}) \cong 0
    \]
    for all maximal ideals of $R$.
    By Exercise V.4.12, this implies that $\Tor_1^{R}(M, N) \cong 0$, hence $M$ is flat.
\end{solution}

% Problem 2.23
\begin{problem}
    Let
    \[
    \begin{tikzcd}
        0 & M & N & P & 0
        \arrow[from=1-1, to=1-2]
        \arrow[from=1-2, to=1-3]
        \arrow[from=1-3, to=1-4]
        \arrow[from=1-4, to=1-5] 
    \end{tikzcd}
    \]
    be an exact sequence of $R$-modules, and assume that $P$ is flat.
    \begin{itemize}
        \item Prove that $M$ is flat if and only if $N$ is flat.
        \item Prove that for all $R$-modules $Q$, the induced sequence
            \[
            \begin{tikzcd}
                0 & M \otimes_R Q & N \otimes_R Q & P \otimes_R Q & 0
                \arrow[from=1-1, to=1-2]
                \arrow[from=1-2, to=1-3]
                \arrow[from=1-3, to=1-4]
                \arrow[from=1-4, to=1-5] 
            \end{tikzcd}
            \]
            is exact.
    \end{itemize}
\end{problem}

\begin{solution}
    Suppose $M$ is flat and let $A$ be an $R$-module.
    By the definition of $\Tor$, we have an exact sequence
    \[
    \begin{tikzcd}
        \Tor_1^{R}(A, M) & \Tor_1^{R}(A, N) & \Tor_1^{R}(A, P)
        \arrow[from=1-1, to=1-2]
        \arrow[from=1-2, to=1-3]
    \end{tikzcd}
    \]
    where the first and last terms are 0 by the flatness of $M$ and $P$.
    Thus, $\Tor_1^{R}(A, N) = 0$ for all $R$-modules $A$, hence $N$ is flat.

    For the other direction, suppose $N$ is flat and let $A$ be an $R$-module.
    We have an exact sequence
    \[
    \begin{tikzcd}
        \Tor_2^{R}(A, P) & \Tor_1^{R}(A, M) & \Tor_1^{R}(A, N)
        \arrow[from=1-1, to=1-2]
        \arrow[from=1-2, to=1-3] 
    \end{tikzcd}
    \]
    where the first and last terms are 0 by the flatness of $N$ and $P$.
    Thus, $\Tor_1^{R}(A, M) = 0$ for all $R$-modules $A$, hence $M$ is flat.

    The exactness of the induced sequence is equivalent to the vanishing of $\Tor_1^{R}(P, Q)$.
    However, this holds due to the flatness of $P$.
    Thus, the induced sequence is exact.
\end{solution}

% Problem 2.24
\begin{problem}
    Let $R$ be a commutative Noetherian local ring with (single) maximal ideal $\mathfrak{m}$, and let $M$ be a finitely generated flat $R$-module.
    \begin{itemize}
        \item Choose elements $m_1, \ldots, m_r \in M$ whose cosets mod $\mathfrak{m}M$ are a basis of $M/\mathfrak{m}M$ as a vector space over the field $R/\mathfrak{m}$.
            By Nakayama's lemma, $M = \langle m_1, \ldots, m_r \rangle$ (Exercise VI.3.10).
        \item Obtain an exact sequence
            \[
            \begin{tikzcd}
                0 & N & R^{\oplus r} & M & 0
                \arrow[from=1-1, to=1-2]
                \arrow[from=1-2, to=1-3]
                \arrow[from=1-3, to=1-4]
                \arrow[from=1-4, to=1-5] 
            \end{tikzcd}
            \]
            where $N$ is finitely generated.
        \item Prove that this sequence induces an exact sequence
            \[
            \begin{tikzcd}
                0 & N/\mathfrak{m}N & (R/\mathfrak{m})^{\oplus r} & M/\mathfrak{m}M & 0
                \arrow[from=1-1, to=1-2]
                \arrow[from=1-2, to=1-3]
                \arrow[from=1-3, to=1-4]
                \arrow[from=1-4, to=1-5] 
            \end{tikzcd}
            \]
            (Use Exercise 2.23.)
        \item Deduce that $N = 0$. (Nakayama.)
        \item Conclude that $M$ is free.
    \end{itemize}
    Thus, a finitely generated module over a (Noetherian) local ring is flat if and only if it is free.
    Compare with Exercise VI.5.5.
\end{problem}

\begin{solution}
    Since $M$ is finitely generated, $M/\mathfrak{m}M$ is a finite dimensional vector space over $R/\mathfrak{m}$, hence we can choose elements $m_1, \ldots, m_r \in M$ whose cosets mod $\mathfrak{m}M$ are a basis of this vector space.
    Then these elements can be lifted to a generating set for $M$ by Nakayama's lemma.

    As a result, we have an exact sequence
    \[
    \begin{tikzcd}
        0 & N & R^{\oplus r} & M & 0
        \arrow[from=1-1, to=1-2]
        \arrow[from=1-2, to=1-3]
        \arrow[from=1-3, to=1-4]
        \arrow[from=1-4, to=1-5] 
    \end{tikzcd}
    \]
    where the map from the free module sends basis elements to the generators of $M$ given above and $N$ is the kernel of this map.
    Since $R$ is Noetherian and finite direct sums of Noetherian rings are Noetherian, $R^{\oplus r}$ is Noetherian.
    $N$ is a submodule of $R^{\oplus r}$, hence $N$ is finitely generated.

    Now we may tensor by $R/\mathfrak{m}$ and since $M$ is flat, the induced sequence is exact by Exercise 2.23.
    We now have the exact sequence
    \[
    \begin{tikzcd}
        0 & N/\mathfrak{m}N & (R/\mathfrak{m})^{\oplus r} & M/\mathfrak{m}M & 0
        \arrow[from=1-1, to=1-2]
        \arrow[from=1-2, to=1-3]
        \arrow[from=1-3, to=1-4]
        \arrow[from=1-4, to=1-5] 
    \end{tikzcd}
    \]
    Since $M/\mathfrak{m}M$ is a vector space, it is free over $R/\mathfrak{m}$.
    In particular, the induced map is an isomorphism, hence the kernel is 0.
    Therefore, $N/\mathfrak{m}N = 0$, and by Nakayama's lemma, $N = 0$.

    Finally, we may conclude that the map $R^{\oplus r} \to M$ is an isomorphism, hence $M$ is free.
    That is, a finitely generated module over a Noetherian local ring is flat if and only if it is free.
    We have a similar result in Exercise VI.5.5, where we show that direct summands of finitely generated free modules over local rings are free.
    I suppose the conclusion is that things tend to be quite nice over local rings.
\end{solution}

% Problem 2.25
\begin{problem}
    Let $R$ be a commutative Noetherian ring, and let $M$ be a finitely generated $R$-module.
    Prove that
    \[
        M \text{ is flat} \Longleftrightarrow \Tor_1^{R}(M, R/\mathfrak{m}) = 0 \text{ for every maximal ideal $\mathfrak{m}$ of $R$.}
    \]
    (Use Exercise 2.21, and refine the argument you used in Exercise 2.24;
    remember that $\Tor$ localizes, by Exercise 2.22.
    The Noetherian hypothesis is actually unnecessary, but the proofs are harder without it.)
\end{problem}

\begin{solution}
    One direction is trivial since $M$ is flat if and only if $\Tor_1^{R}(M, R/I) = 0$ for all ideals $I \subseteq R$.

    Suppose that $\Tor_1^{R}(M, R/\mathfrak{m}) = 0$ for every maximal ideal $\mathfrak{m}$ of $R$.
    Since flatness is a local property, it suffices to show that $M_\mathfrak{m}$ is a flat $R_\mathfrak{m}$-module for every maximal ideal.
    Recall that the localization $R_\mathfrak{m}$ is a local ring with unique maximal ideal $\mathfrak{m}_\mathfrak{m}$.
    Furthermore, standard properties of localization imply that $M_\mathfrak{m}$ is a finitely generated $R_\mathfrak{m}$-module and that $R_\mathfrak{m}$ is a Noetherian ring.

    Following the argument of Exercise 2.24, we can consider $M_\mathfrak{m}/\mathfrak{m}M_\mathfrak{m}$ as a finite-dimensional vector space over $R_\mathfrak{m}/\mathfrak{m}_\mathfrak{m} \cong (R/\mathfrak{m})_\mathfrak{m}$.
    Taking a basis, we may lift this to a generating set for $M_\mathfrak{m}$.

    Thus, we may obtain an exact sequence
    \[
    \begin{tikzcd}
        0 & N_\mathfrak{m} & R_\mathfrak{m}^{\oplus r} & M_\mathfrak{m} & 0
        \arrow[from=1-1, to=1-2]
        \arrow[from=1-2, to=1-3]
        \arrow[from=1-3, to=1-4]
        \arrow[from=1-4, to=1-5] 
    \end{tikzcd}
    \]
    where $N_\mathfrak{m}$ is finitely generated as a submodule of a Noetherian module.
    Note that since $\Tor_1^{R}(M, R/\mathfrak{m}) = 0$, localizing the $\Tor$ module and using the isomorphism in Exercise 2.22 implies that $\Tor_1^{R_\mathfrak{m}}(M_\mathfrak{m}, (R/\mathfrak{m})_\mathfrak{m}) = 0$.
    Thus, tensoring our exact sequence by $(R/\mathfrak{m})_\mathfrak{m}$ yields the exact sequence
    \[
    \begin{tikzcd}
        0 & N_\mathfrak{m}/\mathfrak{m}N_\mathfrak{m} & (R/\mathfrak{m})_\mathfrak{m}^{\oplus r} & M_\mathfrak{m}/\mathfrak{m}M_\mathfrak{m} & 0
        \arrow[from=1-1, to=1-2]
        \arrow[from=1-2, to=1-3]
        \arrow[from=1-3, to=1-4]
        \arrow[from=1-4, to=1-5] 
    \end{tikzcd}
    \]
    The induced map from the surjection becomes an isomorphism as we chose the map to send basis elements to basis elements, hence $N_\mathfrak{m}/\mathfrak{m}N_\mathfrak{m} = 0$.
    Thus, we may apply Nakayama's lemma to deduce that $N_\mathfrak{m} = 0$, hence $M_\mathfrak{m}$ is free, hence it is flat.
    Since this holds for every maximal ideal $\mathfrak{m}$ of $R$, the localness of flatness implies that $M$ is a flat module.
\end{solution}
\end{document}
