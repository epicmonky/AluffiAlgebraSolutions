\documentclass[../../master.tex]{subfiles}

\begin{document}
\section{Base change}

% Problem 3.1
\begin{problem}
    Verify that a combination of pure tensors $\sum_i (m_i \otimes n_i)$ is zero in the tensor product $M \otimes_R N$ if and only if $\sum_i (m_i, n_i) \in \mathbb{Z}^{\oplus(M \times N)}$ is a combination of elements of the form
    \begin{gather*}
        (m, n_1 + n_2) - (m, n_1) - (m, n_2), \\
        (m_1 + m_2, n) - (m_1, n) - (m_2, n), \\
        (rm, n) - (m, rn),
    \end{gather*}
    with $m, m_1, m_2 \in M$, $n, n_1, n_2 \in N$, $r \in R$.
\end{problem}

\begin{solution}
    Certainly a combination of such elements is sent to zero in the tensor product as the kernel of the surjection from the free $\mathbb{Z}$-module is generated by elements of this form.
    On the other hand, suppose $\sum_i (m_i, n_i)$ contains an element which is not of the above form.
    Then it is not contained in the kernel of the surjection, hence it is not sent to zero in $M \otimes_R N$.
\end{solution}

% Problem 3.2
\begin{problem}
    If $f: R \to S$ is a ring homomorphism and $M, N$ are $S$-modules (hence $R$-modules by restriction of scalars), prove that there is a canonical homomorphism of $R$-modules $M \otimes_R N \to M \otimes_S N$.
\end{problem}

\begin{solution}
    Consider the map $\varphi : M \otimes_R N \to M \otimes_S N$ which sends $m \otimes_R n \mapsto m \otimes_S n$.
    The map is clearly a homomorphism of abelian groups.
    Furthermore, this map is a homomorphism of $R$-modules as
    \[
        \varphi(r \cdot m \otimes_R n) = \varphi(f(r) m \otimes_R n) = f(r) m \otimes_S n = r \cdot m \otimes_S n = r \cdot \varphi(m \otimes_R n).
    \]
    Since the map only depends on $f: R \to S$, it is a canonical homomorphism of $R$-modules.
\end{solution}

% Problem 3.3
\begin{problem}
    Let $R, S$ be commutative rings, and let $M$ be an $R$-module, $N$ an $(R, S)$-bimodule, and $P$ an $S$-module.
    Prove that there is an isomorphism of $R$-modules
    \[
        M \otimes_R (N \otimes_S P) \cong (M \otimes_R N) \otimes_S P.
    \]
    In this sense, $\otimes$ is `associative'.
\end{problem}

\begin{solution}
    Consider the map $\varphi : M \otimes_R (N \otimes_S P) \to (M \otimes_R N) \otimes_S P$ which sends $m \otimes_R (n \otimes_S p) \mapsto (m \otimes_R n) \otimes_S p$.
    Again, it is clear that the map is a homomorphism of abelian groups.
    To see that it preserves the action of $R$, we find that
    \begin{align*}
        \varphi(rm \otimes_R (n \otimes_S p)) &= (rm \otimes_R n) \otimes_S p \\
                                              &= r(m \otimes_R n) \otimes_S p \\
                                              &= r\varphi(m \otimes_R (n \otimes_S p)).
    \end{align*}
    Furthermore, the map has an obvious inverse, hence it is an isomorphism.
    Thus, the tensor product is associative in the sense of isomorphism (perhaps the structure of a monoidal structure is lurking somewhere in the background).
\end{solution}

% Problem 3.4
\begin{problem}
    Use the associativity of the tensor product (Exercise 3.3) to prove again the formulas given in Exercise 2.6.
    (Use Exercise 2.5.)

    Let $S$ be a multiplicative subset of $R$ and let $M$ be an $R$-module.
    \begin{itemize}
    \item Let $N$ be an $S^{-1}R$-module.
        Prove that $(S^{-1}M) \otimes_{S^{-1}R} N \cong M \otimes_R N$.
    \item Let $A$ be an $R$-module.
        Prove that $(S^{-1}A) \otimes_R M \cong S^{-1}(A \otimes_R M)$.
    \end{itemize}
\end{problem}

\begin{solution}
    By Exercise 2.5, we have $S^{-1}M \cong M \otimes_R S^{-1}R$ as $R$-modules.
    Then the first isomorphism follows from
    \[
        (S^{-1}M) \otimes_{S^{-1}R} N \cong (M \otimes_R S^{-1}R) \otimes_{S^{-1}R} N \cong M \otimes_R (S^{-1}R \otimes_{S^{-1}R} N) \cong M \otimes_R N.
    \]
    Similarly, the second isomorphism is given by
    \begin{align*}
        (S^{-1}A) \otimes_R M &\cong (A \otimes_R S^{-1}R) \otimes_R M \\
                              &\cong A \otimes_R (S^{-1}R \otimes_R M) \\
                              &\cong A \otimes_R (M \otimes_R S^{-1}R) \\
                              &\cong (A \otimes_R M) \otimes_R S^{-1}R \\
                              &\cong S^{-1}(A \otimes_R M).
    \end{align*}
\end{solution}

% Problem 3.5
\begin{problem}
    Let $f: R \to S$ be a ring homomorphism.
    Prove that $f^{!}$ commutes with limits, $f^{*}$ commutes with colimits, and $f_{*}$ commutes with both.
    In particular, deduce that these three functors all preserve finite direct sums.
\end{problem}

\begin{solution}
    Since $f^{!}$ is right-adjoint to $f_{*}$, $f^{!}$ commutes with limits.
    Similarly, since $f^{*}$ is left-adjoint to $f_{*}$, $f^{*}$ commutes with colimits.
    Finally, since $f_{*}$ is right-adjoint to $f^{*}$ and left-adjoint to $f^{!}$, it commutes with both limits and colimits.

    In particular, since finite direct sums are both products and coproducts in module categories (hence both limits and colimits), all three functors preserve finite direct sums.
\end{solution}

% Problem 3.6
\begin{problem}
    Let $f: R \to S$ be a ring homomorphism, and let $\varphi : N_1 \to N_2$ be a homomorphism of $S$-modules.
    Prove that $\varphi$ is an isomorphism if and only if $f_{*}(\varphi)$ is an isomorphism.
    (Functors with this property are said to be conservative.)
    In fact, prove that $f_{*}$ is \textit{faithfully} exact:
    a sequence of $S$-modules
    \[
    \begin{tikzcd}
        0 & L & M & N & 0
        \arrow[from=1-1, to=1-2] 
        \arrow[from=1-2, to=1-3] 
        \arrow[from=1-3, to=1-4] 
        \arrow[from=1-4, to=1-5] 
    \end{tikzcd}
    \]
    is exact \textit{if and only if} the sequence of $R$-modules
    \[
    \begin{tikzcd}
        0 & f_{*}(L) & f_{*}(M) & f_{*}(N) & 0
        \arrow[from=1-1, to=1-2] 
        \arrow[from=1-2, to=1-3] 
        \arrow[from=1-3, to=1-4] 
        \arrow[from=1-4, to=1-5] 
    \end{tikzcd}
    \]
    is exact.
    In particular, a sequence of $R$-modules is exact if and only if it is exact as a sequence of abelian groups.
    (This is completely trivial but useful nonetheless.)
\end{problem}

\begin{solution}
    Note that restriction of scalars does not change the underlying sets of the modules or the homomorphisms between said modules.
    Thus, faithful exactness is trivial since the sequence of modules doesn't actually change on the level of sets.
\end{solution}

% Problem 3.7
\begin{problem}
    Let $i : k \subseteq F$ be a finite field extension, and let $W$ be an $F$-vector space of finite dimension $n$.
    Compute the dimension of $i_{*}(W)$ as a $k$-vector space (where $i_{*}$ is a restriction of scalars).
\end{problem}

\begin{solution}
    We claim that $\dim_k(i_{*}(W)) = [F : k] \cdot n$.
    Indeed, $F$ is a $k$-vector space of dimension $[F : k]$, hence we may consider $W \cong F^{n} \cong k^{[F : k] \cdot n}$.
    More explicitly, writing $F = k(\alpha_1, \ldots, \alpha_m)$ where $m = [F : k]$, a basis $\{v_1, \ldots, v_n\}$ for $W$ over $F$ is extended to a basis $\{v_1, \ldots, v_n, \alpha_1 v_1, \ldots, \alpha_1 v_1, \ldots, \alpha_m v_n\}$ which has order $mn$.
\end{solution}

% Problem 3.8
\begin{problem}
    Let $i : k \subseteq F$ be a finite field extension, and let $V$ be a $k$-vector space of dimension $n$.
    Compute the dimension of $i^{*}(V)$ and $i^{!}(V)$ as $F$-vector spaces.
\end{problem}

\begin{solution}
    Recall that $i^{*}(V) = F \otimes_{k} V$.
    But $F \otimes_k V \cong F \otimes_k k^{n} \cong (F \otimes_k k)^{n} \cong F^{n}$, hence $i^{*}(V)$ has dimension $n$ as an $F$-vector space.
    More explicitly, we have that $F \otimes_k V$ is generated by $\{1 \otimes v_1, \ldots, 1 \otimes v_n\}$ where $1 \in F$ and $v_i$ form a basis for $V$.

    On the other hand, $i^{!}(V) = \Hom_k(F, V)$.
    Viewing $F$ as a $k$-vector space, it has dimension $[F : k]$.
    We have isomorphisms $\Hom_R(R^{n}, R^{m}) \cong \mathcal{M}_{m \times n} \cong R^{mn}$, hence $\dim_F i^{!}(V) = [F : k] \cdot n$.
\end{solution}

% Problem 3.9
\begin{problem}
    Let $f : R \to S$ be a ring homomorphism, and let $M$ be an $R$-module.
    Prove that the extension $f^{*}(M)$ satisfies the following universal property:
    if $N$ is an $S$-module and $\varphi : M \to N$ is an $R$-linear map, then there exists a unique $S$-linear map $\bar{\varphi} : f^{*}(M) \to N$ making the diagram
    \[
    \begin{tikzcd}
        M & & N \\
        \\
        f^{*}(M)
        \arrow[from=1-1, to=1-3, "\varphi"]
        \arrow[from=1-1, to=3-1, "i"'] 
        \arrow[from=3-1, to=1-3, "\exists! \bar{\varphi}"'] 
    \end{tikzcd}
    \]
    commute, where $i : M \to f^{*}(M) = M \otimes_R S$ is defined by $m \mapsto m \otimes 1$.
    (Thus, $f^{*}(M)$ is the `best approximation' to the $R$-module $M$ in the category of $S$-modules.)
\end{problem}

\begin{solution}
    We start by showing uniqueness.
    Suppose $\bar{\varphi}'$ is another map making the diagram commute.
    Then for all $m \otimes s \in f^{*}(M)$, we have $(\bar{\varphi} - \bar{\varphi}')(m \otimes s) = s\bar{\varphi}(m \otimes 1) - s\bar{\varphi}'(m \otimes 1) = s\varphi(m) - s\varphi(m) = 0$, hence the two maps are equal.
    For existence, we claim that $\bar{\varphi} : f^{*}(M) \to N$ which sends $m \otimes s \mapsto \varphi(m) \cdot s$ satisfies the universal property.
    Indeed, it is clear that the map is $S$-linear and for all $m \in M$, we have $\bar{\varphi} \circ i(m) = \bar{\varphi}(m \otimes 1) = \varphi(m)$ as required.
\end{solution}

% Problem 3.10
\begin{problem}
    Prove the following \textit{projection formula:}
    if $f : R \to S$ is a ring homomorphism, $M$ is an $R$-module, and $N$ is an $S$-module, then $f_{*}(f^{*}(M) \otimes_S N) \cong M \otimes_R f_{*}(N)$ as $R$-modules.
\end{problem}

\begin{solution}
    Consider the map $\varphi$ which sends $m \otimes s \otimes n \mapsto m \otimes sn$.
    Certainly this is a map of abelian groups, and to see that it is a morphims of $R$-modules, note that $r \cdot (m \otimes s \otimes n) = rm \otimes s \otimes n \mapsto rm \otimes sn = r \cdot (m \otimes sn)$.
    Finally, the map has an inverse given by $m \otimes n \mapsto m \otimes 1 \otimes n$.
    Indeed, we have $\varphi(m \otimes s \otimes n) = m \otimes sn \mapsto m \otimes 1 \otimes sn = m \otimes s \otimes n$.
    Thus, the two are isomorphic as $R$-modules.
\end{solution}

% Problem 3.11
\begin{problem}
    Let $f: R \to S$ be a ring homomorphism, and let $M$ be a \textit{flat} $R$-module.
    Prove that $f^{*}(M)$ is a flat $S$-module.
\end{problem}

\begin{solution}
    Recall that a $R$-module $M$ is flat if tensoring by $M$ is an exact functor.
    Let
    \[
        \begin{tikzcd}
            0 & A & B & C & 0
            \arrow[from=1-1, to=1-2] 
            \arrow[from=1-2, to=1-3] 
            \arrow[from=1-3, to=1-4] 
            \arrow[from=1-4, to=1-5] 
        \end{tikzcd}
    \]
    be an exact sequence of $S$-modules.
    By the faithful exactness of $f_{*}$, the sequence of $R$-modules
    \[
        \begin{tikzcd}
            0 & f_{*}(A) & f_{*}(B) & f_{*}(C) & 0
            \arrow[from=1-1, to=1-2] 
            \arrow[from=1-2, to=1-3] 
            \arrow[from=1-3, to=1-4] 
            \arrow[from=1-4, to=1-5] 
        \end{tikzcd}
    \]
    is exact.
    Tensoring by $M$ yields the exact sequence
    \[
        \begin{tikzcd}
            0 & M \otimes_R f_{*}(A) & M \otimes_R f_{*}(B) & M \otimes_R f_{*}(C) & 0
            \arrow[from=1-1, to=1-2] 
            \arrow[from=1-2, to=1-3] 
            \arrow[from=1-3, to=1-4] 
            \arrow[from=1-4, to=1-5] 
        \end{tikzcd}
    \]
    by the flatness of $M$.
    But by Exercise 10, $M \otimes_R f_{*}(N) \cong f_{*}(f^{*}(M) \otimes_S N)$ as $R$-modules for all $S$-modules $N$.
    Thus, we have the exact sequence
    \[
        \begin{tikzcd}
            0 & f_{*}(f^{*}(M) \otimes_S A) & f_{*}(f^{*}(M) \otimes_S B) & f_{*}(f^{*}(M) \otimes_S C) & 0
            \arrow[from=1-1, to=1-2] 
            \arrow[from=1-2, to=1-3] 
            \arrow[from=1-3, to=1-4] 
            \arrow[from=1-4, to=1-5] 
        \end{tikzcd}
    \]
    Finally, since $f_{*}$ is a faithfully exact functor we conclude that the sequence
    \[
        \begin{tikzcd}
            0 & f^{*}(M) \otimes_S A & f^{*}(M) \otimes_S B & f^{*}(M) \otimes_S C & 0
            \arrow[from=1-1, to=1-2] 
            \arrow[from=1-2, to=1-3] 
            \arrow[from=1-3, to=1-4] 
            \arrow[from=1-4, to=1-5] 
        \end{tikzcd}
    \]
    is exact, hence $f^{*}(M)$ is a flat $S$-module.
\end{solution}

% Problem 3.12
\begin{problem}
    In `geometric' context (such as the one hinted at in Remark 3.7), one would actually work with categories which are \textit{opposite} to the category of commutative rings; cf. Example 1.9.
    A ring homomorphism $f: R \to S$ corresponds to a morphism $f^{\circ}: S^{\circ} \to R^{\circ}$ in the opposite category, and we can simply define $f^{\circ}_{*}$, etc., to be $f_*$, etc.

    For morphisms $f^{\circ}: S^{\circ} \to R^{\circ}$ and $g^{\circ}: T^{\circ} \to S^{\circ}$ in the opposite category, prove that
    \begin{itemize}
        \item $(f^{\circ} \circ g^{\circ})_{*} \cong f^{\circ}_* \circ g^{\circ}_*$,
        \item $(f^{\circ} \circ g^{\circ})^{*} \cong g^{\circ *} \circ f^{\circ *}$,
        \item $(f^{\circ} \circ g^{\circ})^{!} \cong g^{\circ !} \circ f^{\circ !}$,
    \end{itemize}
    where $\cong$ stands for `naturally isomorphic'.
    (These are the formulas suggested by the notation:
    a $*$ in the subscript invariably suggests a basic `covariance' property of the notation, while modifiers in the subscript usually suggest contravariance.
    The switch to the opposite category is natural in the algebro-geometric context.)
\end{problem}

\begin{solution}
    Per Example 1.9, the category of affine $K$-algebraic sets is opposite to the subcategory of reduced, commutative, finite-type $K$-algebras.
    More generally, the category of affine schemes is opposite to the category of commutative rings.
    For the remainder of the solution, let $F$ denote the functor of the composition and let $G$ denote the composition of the functors.
    
    By definition, $(f^{\circ} \circ g^{\circ})_{*}$ is a (covariant) functor from $ T\mathsf{-Mod} \to R\mathsf{-Mod}$.
    Given a $T$-module $M$, define the morphism $(f^{\circ} \circ g^{\circ})_{*}(M) \to f^{\circ}_* \circ g^{\circ}_*(M)$ to be the identity map.
    Indeed, since restriction of scalars is the identity on module homomorphisms, commutativity of the relevant diagram is reduced to $id_Y \circ F(h) = G(h) \circ id_X$ where $h : X \to Y$ is any homomorphism of $T$-modules.

    Similarly, $(f^{\circ} \circ g^{\circ})^{*}$ is a (covariant) functor from $R\mathsf{-Mod} \to T\mathsf{-Mod}$.
    In particular, $g \circ f: R \to T$ induces an $R$-module structure on $T$ via $r \cdot t = g(f(r))t$, hence the functor sends an $R$-module $M$ to $M \otimes_R T$.
    On the other hand, the composition of functors sends $M \mapsto M \otimes_R S \mapsto M \otimes_R S \otimes_S T$.
    Thus, given an $R$-module $M$ consider the map $M \otimes_R S \otimes_S T \to M \otimes_R T$ which sends $m \otimes s \otimes t \mapsto m \otimes g(s) t$.
    This forces the relevant diagram to commute since we are again reduced to checking that morphisms commute with identity maps.

    I'll skip the last one for now, but the natural isomorphism is effectively the same as above with composition of morphisms in the Hom sets.
\end{solution}

% Problem 3.13
\begin{problem}
    Let $p > 0$ be a prime integer, and let $\pi : \mathbb{Z} \to \mathbb{Z}/p\mathbb{Z}$ be the natural projection.
    Compute $\pi^{*}(A)$ and $\pi^{!}(A)$ for all finitely generated abelian groups $A$, as a vector space over $\mathbb{Z}/p\mathbb{Z}$.
    Compute $i^{*}(A)$ and $i^{!}(A)$ for all finitely generated abelian groups $A$, where $i : \mathbb{Z} \hookrightarrow \mathbb{Q}$ is the natural inclusion.
\end{problem}

\begin{solution}
    Recall that by the classification of f.g. modules over a PID, every finitely generated abelian group (or $\mathbb{Z}$-module) is isomorphic to $\mathbb{Z}^{r} \oplus (\mathbb{Z}/p_1\mathbb{Z})^{a_1} \oplus \cdots \oplus (\mathbb{Z}/p_n\mathbb{Z})^{a_n}$.
    Then given a f.g. abelian group $A$, we have $\pi^{*}(A) = A \otimes_\mathbb{Z} \mathbb{Z}/p\mathbb{Z}$.
    Using the above expansion, and recalling that tensor products commute with finite direct sums, we have
    \[
    A \otimes_\mathbb{Z} \cong (\mathbb{Z}/p\mathbb{Z})^{r} \oplus (\mathbb{Z}/p\mathbb{Z})^{a_i} \cong (\mathbb{Z}/p\mathbb{Z})^{r + a_i}
    \]
    where $a_i$ is the multiplicity of $\mathbb{Z}/p\mathbb{Z}$ in $A$.
    All the other factors vanish by Exercise 2.2.
    On the other hand, we have that $\pi^{!}(A) = \Hom_\mathbb{Z}(\mathbb{Z}/p\mathbb{Z}, A)$.
    Since $\mathbb{Z}/p\mathbb{Z}$ is cyclic, any map $\varphi \in \Hom_\mathbb{Z}(\mathbb{Z}/p\mathbb{Z}, A)$ is determined by $\varphi(1)$.
    Further, since the domain has order $p$, the image of $\varphi(1)$ must have order $p$.
    Thus, we find $\Hom_\mathbb{Z}(\mathbb{Z}/p\mathbb{Z}, A) \cong (\mathbb{Z}/p\mathbb{Z})^{a_i}$ where $a_i$ is the multiplicity of $\mathbb{Z}/p\mathbb{Z}$ in $A$.

    For the inclusion into $\mathbb{Q}$, we have $i^{*}(A) = A \otimes_\mathbb{Z} \mathbb{Q}$.
    Again expanding, we find that
    \[
    A \otimes_\mathbb{Z} \mathbb{Q} \cong \mathbb{Q}^{r}
    \]
    where $r$ is the rank of $\mathbb{Z}$ in $A$.
    Indeed, for all cylic groups we have $\mathbb{Q} \otimes_\mathbb{Z} \mathbb{Z}/n\mathbb{Z} = 0$ since we have $\frac{a}{b} \otimes m = \frac{a}{bn} \otimes mn = 0$.
    For the other functor, we have $i^{!}(A) = \Hom_\mathbb{Z}(\mathbb{Q}, A) = 0$.
    Indeed, suppose $\varphi : \mathbb{Q} \to \mathbb{Z}$ is nontrivial.
    Let $n$ be the smallest positive integer in $\im \varphi$ and let $\frac{a}{b} \in \varphi^{-1}(n)$.
    Then
    \[
    \varphi\left( \frac{a}{2b} + \frac{a}{2b}\right) = \varphi\left( \frac{a}{2b}\right) + \varphi\left( \frac{a}{2b} \right) = n
    \]
    so $\varphi \left( \frac{a}{2b}\right)$ is a positive integer less than $n$, a contradiction.
    A similar argument holds for $\mathbb{Z}/m\mathbb{Z}$.
\end{solution}

% Problem 3.14
\begin{problem}
    Let $f : R \to S$ be an \emph{onto} ring homomorphism;
    thus, $S \cong R/I$ for some ideal $I$ of $R$.
    \begin{itemize}
        \item Prove that, for all $R$-modules $M$, $f^{!}(M) \cong \{m \in M | \forall a \in I, am = 0\}$, while $f^{*}(M) \cong M/IM$.
            (Exercise III.7.7 may help.)
        \item Prove that, for all $S$-modules $N$, $f^{!}f_*(N) \cong N$ and $f^{*}f_*(N) \cong N$.
        \item Prove that $f_*$ is fully faithful (Definition 1.6).
        \item Deduce that if there is an onto homomorphism $R \to S$, then $S\mathsf{-Mod}$ is equivalent to a full subcategory of $R\mathsf{-Mod}$.
    \end{itemize}
\end{problem}

\begin{solution}
    We have $f^{!}(M) = \Hom_R(R/I, M)$ for some ideal $I$ of $R$.
    By Exercise III.7.7, we have an exact sequence
    \[
    \begin{tikzcd}
        0 & \Hom_R(R/I, M) & \Hom_R(R, M) & \Hom_R(I, M)
        \arrow[from=1-1, to=1-2] 
        \arrow[from=1-2, to=1-3] 
        \arrow[from=1-3, to=1-4] 
    \end{tikzcd}
    \]
    where the latter map $\varphi$ sends a morphism $f: R \to M$ to $\bar{f} = f \circ i: I \to M$ where $i$ is the inclusion.
    Since the sequence is exact, $\Hom_R(R/I, M) \cong \ker \varphi$.
    But $\ker \varphi = \{f \in \Hom_R(R, M) \mid \forall a \in I, f(a) = 0\}$.
    Note that a module homomorphism whose domain is $R$ is entirely determined by the image of 1.
    In particular, $\Hom_R(R, M) \cong M$.
    Thus, $f(a) = 0$ iff $a \cdot f(1) = 0$.
    Thus, we conclude that $\Hom_R(R/I, M) \cong \{m \in M \mid \forall a \in I, am = 0\}$.
    The much simpler case is that $f^{*}(M) = M \otimes_R R/I \cong M/IM$ per Section 2.
    More specifically, we have an induced right-exact sequence after tensoring by $M$ which identifies $R/I \otimes_R M \cong (R \otimes_R M) / (I \otimes_R M) \cong M/IM$.

    Let $N$ be an $S$-module.
    Then $f^{!}f_*(N) \cong \Hom_R(R/I, N)$.
    A module homomorphism from $R/I$ is entirely determined by the image of $1$, and since there are no restrictions on where 1 can map, we have $\Hom_R(R/I, N) \cong N$.
    As shown above, $f^{*}f_*(N) \cong N/IN$ so it suffices to show that $N/IN \cong N$ as $R/I$-modules.
    Indeed, consider the map $\varphi: n \mapsto n + IN$.
    Certainly it is a homomorphism of abelian groups and for all $r + I \in R/I$, we have
     \[
         \varphi((r + I) \cdot n) = \varphi(f(r)n) = f(r)n + IN = (r + I) \cdot \varphi(n)
    \]
    with a well-defined inverse since the choice of representative does not matter under the action of $R/I$.
    Thus, $f^{*}f_*(N) \cong N$.

    Recall that a functor is fully faithful if it is bijective on Hom sets.
    I think $f_*$ is always fully faithful since it is the identity map on morphisms.
    We can conclude that $S\mathsf{-Mod}$ is equivalent to a full subcategory of $R\mathsf{-Mod}$ since $f_*$ is a fully faithful functor.
\end{solution}

% Problem 3.15
\begin{problem}
    Let $f: R \to S$ be a ring homomorphism, and assume that the functor $f_* : S\mathsf{-Mod} \to R\mathsf{-Mod}$ is an equivalence of categories.
    \begin{itemize}
        \item Prove that there is a homomorphism of rings $\bar{g} : S \to \End_{\mathsf{Ab}}(R)$ such that the composition $R \to S \to \End_{\mathsf{Ab}}(R)$ is the homomorphism realizing $R$ as a module over itself (that is, the homomorphism studied in Proposition III.2.7).
        \item Use the facts that $S$ is commutative and $f_*$ is fully faithful to deduce that $\bar{g}(S)$ is isomorphic to $R$.
            (Refine the result of Exercise III.2.17.)
            Deduce that $f$ has a left-inverse $g: S \to R$.
        \item Therefore, $f_* \circ g_*$ is naturally isomorphic to the identity;
            and it follows that $g_* \circ f_*(S) \cong S$ as an $S$-module.
            Prove that this implies that $g$ is injective.
            (If $a \in \ker g$, prove that $a$ is in the annihilator of $g_* \circ f_*(S)$.)
        \item Conclude that $f$ is an isomorphism.
    \end{itemize}

    Two rings are \emph{Morita equivalent} if their categories of left-modules are equivalent.
    The result of this exercise is a (very) particular case of the fact that two \emph{commutative} rings are Morita equivalent if and only if they are isomorphic.
    In fact, this more general statement is perhaps easier (!) to prove than the particular case worked out in this exercise.
    The reader can verify that if $R$ is a commutative ring, then it is isomorphic to the endomorphism ring of the \textit{identity functor} on $R\mathsf{-Mod}$.
    It follows that if $R\mathsf{-Mod}$ is equivalent to $S\mathsf{-Mod}$, then $R$ and $S$ must be isomorphic. 
    The commutativity is crucial in this statement:
    for example, it can be shown that any ring $R$ is Morita equivalent to the ring of matrices $\mathcal{M}_{n, n}(R)$, for all $n > 0$.
\end{problem}

\begin{solution}
    Let $M$ be an $S$-module such that $f_*(M) \cong R$ (such a module exists by the equivalence of categories, call the isomorphism $\varphi$).
    Then $\bar{g}: S \to \End(M) \cong \End(R)$ maps $s \mapsto \lambda_s$ where $\lambda_s(m) = sm$.
    Certainly for $r \in R$, we have $\bar{g}(f(r)) = \lambda_{f(r)}$ but by our isomorphism of $R$-modules $\varphi(\lambda_{f(r)}(m)) = r \varphi(m)$, hence $\bar{g}$ corresponds to normal multiplication in $R$.

    It is clear that $\bar{g}$ identifies $S$ with a subring of $\End(R)$.
    Since $S$ is commutative, so is the image $\bar{g}(S)$; in particular, it is contained in the center of $\End(R)$.
    Furthermore, $f_*$ is fully faithful, so $\bar{g}(S)$ is actually in bijection with the set of commuting endomorphisms of $R$.
    But $R$ is commutative, hence $Z(\End(R)) \cong R$.
    Thus, $\bar{g}(S) \cong R$ as rings, and $S$ can be identified with a subring of $\bar{g}(S)$.
    Therefore, $f$ has a left-inverse $g : S \to R$.
    
    Suppose $a \in \ker g$.
    That is, $g(a) \in R = 0$, so for $s \in S$ we have
    \[
    a \cdot g_* \circ f_*(S) = g(a) \cdot f_*(S) = f \circ g(a) \cdot S = 0 \cdot S = 0,
    \]
    hence $a \in \Ann(g_* \circ f_*(S)) \cong \Ann(S)$, but the only annihilator of $S$ as a module over itself is $0$.
    Thus, $\ker g = 0$, hence $g$ is injective.

    Since $g$ has a right-inverse, it is surjective, hence an isomorphism.
    Finally, since it is the inverse to $f$, we conclude that $f$ is an isomorphism.
\end{solution}
\end{document}
