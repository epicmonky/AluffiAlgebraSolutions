\documentclass[../../master.tex]{subfiles}

\begin{document}
\section{Projective and injective modules and the Ext functors}

% Exercise 6.1
\begin{problem}
    Prove that an $R$-module $P$ is projective if and only if every epimorphism $M \to P$ splits and $Q$ is injective if and only if every monomorphism $Q \to M$ splits.
\end{problem}

\begin{solution}
    First suppose $P$ is projective.
    Let $f: M \to P$ be an epimorphism; 
    that is, we there is an exact sequence
    \[
    \begin{tikzcd}
        M & P & 0
        \arrow[from=1-1, to=1-2, "f"]
        \arrow[from=1-2, to=1-3] 
    \end{tikzcd}
    \]
    Then $P$ being projective is equivalent to $\Hom_R(P, -)$ being right-exact.
    That is, the induced map $f^{*} : \Hom_R(P, M) \to \Hom_R(P, P)$ is surjective.
    Equivalently, $P$ satisfies the projective lifting property, specifically for $1_P : P \to P$.
    That is, there exists a morphism $g : P \to M$ such that the following diagram commutes:
    \[
    \begin{tikzcd}
        & P & \\
        M & P & 0
        \arrow[from=1-2, to=2-1, "g"', dashed]
        \arrow[from=1-2, to=2-2, "1_P"] 
        \arrow[from=2-1, to=2-2, "f"]
        \arrow[from=2-2, to=2-3] 
    \end{tikzcd}
    \]
    That is, $f \circ g = 1_P$, hence $f$ has a right-inverse, hence it is a split epimorphism.

    Now suppose every epimorphism $M \to P$ admits a right-inverse.
    That is, given a surjection $f: M \to P$, there exists a map $g : P \to M$ such that $f \circ g = 1_P$.
    But this is precisely the characterization of projective modules in terms of the projective lifting property.

    The case for $Q$ an injective module is similar.
    Let $f : Q \to M$ be a monomorphism, hence there is an exact sequence
    \[
    \begin{tikzcd}
        0 & Q & M
        \arrow[from=1-1, to=1-2]
        \arrow[from=1-2, to=1-3, "f"]
    \end{tikzcd}
    \]
    Then $Q$ being injective is equivalent to the induced map $f^{*} : \Hom_R(M, Q) \to \Hom_R(Q, Q)$ being surjective.
    In other words, there for all morphisms $Q \to Q$, there exists a map $g : M \to Q$ such that the following diagram commutes:
    \[
    \begin{tikzcd}
        0 & Q & M \\
          & Q & 
        \arrow[from=1-1, to=1-2]
        \arrow[from=1-2, to=1-3, "f"]
        \arrow[from=1-2, to=2-2]
        \arrow[from=1-3, to=2-2, "g"] 
    \end{tikzcd}
    \]
    In particular, there exists $g: M \to Q$ such that $g \circ f = 1_Q$, but this is precisely what it means for $f$ to be a split monomorphism.
    The other direction is entirely analogous to the case for projective modules.
\end{solution}

% Exercise 6.2
\begin{problem}
    Prove that the result of Proposition 5.13 holds more generally whenever $P$ is a \textit{projective} module.
    \begin{proposition}[Proposition 5.13] 
        Let
        \[
        \begin{tikzcd}
            0 & M & N & P & 0
            \arrow[from=1-1, to=1-2]
            \arrow[from=1-2, to=1-3, "\mu"]
            \arrow[from=1-3, to=1-4, "\nu"] 
            \arrow[from=1-4, to=1-5] 
        \end{tikzcd}
        \]
        be an exact sequence of $R$-modules, with $P$ \text{projective}.
        Then the induced sequence
        \[
        \begin{tikzcd}
            0 & P^{\vee} & N^{\vee} & M^{\vee} & 0
            \arrow[from=1-1, to=1-2]
            \arrow[from=1-2, to=1-3, "\nu^{\vee}"]
            \arrow[from=1-3, to=1-4, "\mu^{\vee}"] 
            \arrow[from=1-4, to=1-5] 
        \end{tikzcd}
        \]
        is exact.
    \end{proposition}
\end{problem}

\begin{solution}
    Recall that $M^{\vee} = \Hom_R(M, R)$.
    It suffices to show that the map $\mu^{\vee}: N^{\vee} \to M^{\vee}$ is surjective.
    Let $f: M \to R$ be an $R$-linear map.
    Since $P$ is projective, the epimorphism $\nu : N \to P$ splits;
    that is, there exists a map $\rho : P \to N$ such that $\nu \circ \rho = 1_P$.

    Now let $n \in N$.
    Then $\nu(\rho(\nu(n))) = \nu(n)$, hence $\nu(\rho(\nu(n)) - n) = 0$, hence $\rho(\nu(n)) - n \in \ker \nu = \im \mu$.
    That is, there exists a unique (by injectivity) $m \in M$ such that $\mu(m) = \rho(\nu(n)) - n$.
    Define $g : N \to R$ to be $g(n) := f(m)$.
    Then it is clear that $g \in \Hom_R(N, R) = N^{\vee}$ and
    \[
    \mu^{\vee}(g)(m) = (g \circ \mu)(m) = g(\rho(\nu(n)) - n) = f(m)
    \]
    by definition, hence $\mu^{\vee}$ is surjective.
\end{solution}

% Exercise 6.3
\begin{problem}
    Prove that an $R$-linear map $A \to B$ is injective if and only if the induced map $\Hom_R(B, Q) \to \Hom_R(A, Q)$ is surjective for all \textit{injective} modules $Q$.
    (Hint: $R\mathsf{-Mod}$ has enough injectives.)
    This strengthens the result of Exercise 5.2.
\end{problem}

\begin{solution}
    If $\varphi: A \to B$ is injective and $C = \coker \varphi$, then we have the exact sequence
    \[
    \begin{tikzcd}
        0 & A & B & C & 0
        \arrow[from=1-1, to=1-2]
        \arrow[from=1-2, to=1-3, "\varphi"] 
        \arrow[from=1-3, to=1-4]
        \arrow[from=1-4, to=1-5] 
    \end{tikzcd}
    \]
    Applying $\Hom_R(-, Q)$ to both sides, where $Q$ is an injective module yields the exact sequence
    \[
    \begin{tikzcd}
        0 & \Hom_R(C, Q) & \Hom_R(B, Q) & \Hom_R(A, Q) & 0
        \arrow[from=1-1, to=1-2]
        \arrow[from=1-2, to=1-3] 
        \arrow[from=1-3, to=1-4, "\varphi^{*}"]
        \arrow[from=1-4, to=1-5] 
    \end{tikzcd}
    \]
    hence $\varphi^{*}$ is surjective for all injective modules $Q$.

    Now suppose $\varphi^{*} : \Hom_R(B, Q) \to \Hom_R(A, Q)$ is surjective for all injective modules $Q$.
    Let
    \[
    \begin{tikzcd}
        0 & A & Q_0 & Q_1 & Q_2 & \cdots
        \arrow[from=1-1, to=1-2] 
        \arrow[from=1-2, to=1-3, "f"]
        \arrow[from=1-3, to=1-4] 
        \arrow[from=1-4, to=1-5] 
        \arrow[from=1-5, to=1-6] 
    \end{tikzcd}
    \]
    be an injective resolution of $A$.
    Then by the surjectivity of $\varphi^{*}$, there exists a map $g : B \to Q_0$ such that $g \circ \varphi = f$.
    But $\ker(f) = 0$, hence $\ker(\varphi) \subseteq \ker(g \circ \varphi) = 0$.
    We conclude that $\varphi$ is injective.
\end{solution}

% Exercise 6.4
\begin{problem}
    Let $\{P_i\}_{i \in I}$ be a family of $R$-modules.
    Prove that $\bigoplus_i P_i$ is projective if and only if each $P_i$ is projective.
\end{problem}

\begin{solution}
    Suppose each $P_i$ is projective.
    That is, for each $i \in I$ there exists a $K_i$ such that $P_i \oplus K_i \cong F_i$ where $f_i$ is a projective module.
    Then
    \[
    \bigoplus_i P_i \oplus \bigoplus_i K_i \cong \bigoplus_i (P_i \oplus K_i) \cong \bigoplus_i F_i
    \]
    so $\bigoplus_i P_i$ is a summand of a free module, hence it is projective.

    Now suppose $\bigoplus_i P_i$ is projective.
    Let $\mu : M \to N$ be an epimorphism and let $p_i : P_i \to N$ be a morphism.
    Then by the universal property of coproducts, there exists a morphism $p : \bigoplus_i P_i \to N$ such that $p \circ i = p_i$ where $i$ is the natural inclusion $P_i \hookrightarrow \bigoplus_i P_i$.
    By the projective lifting property, there exists a map $\hat{p} : \bigoplus_i P_i \to M$ such that $\mu \circ \hat{p} = p$.
    Then the restriction $\hat{p}|_{P_i}$ satisfies the projective lift property;
    indeed
    \[
        \mu \circ \hat{p}|_{P_i} = p|_{P_i} = p_i.
    \]
    Thus, $P_i$ is a projective module for all $i \in I$.

    For a more categorical proof using adjunction, recall that $\Hom_R(P, -)$ is a left-adjoint functor, hence it preserves colimits.
    In particular,
    \[
    \Hom_R(\bigoplus_i P_i, -) \cong \bigoplus_i \Hom_R(P_i, -)
    \]
    where the two functors are naturally isomorphic.
    Then this is reduced to checking when either side consists of exact functors.
\end{solution}

% Exercise 6.5
\begin{problem}
    Prove that the dual of a finitely generated projective module is projective.
    Prove that the finitely generated projective modules are reflexive (that is, isomorphic to their biduals).
\end{problem}

\begin{solution}
    If $P$ is a finitely generated projective module, then it is a direct summand of a free module of finite rank, say $F$.
    Then
    \[
        P^{\vee} \oplus K^{\vee} \cong (P \oplus K)^{\vee} \cong F^{\vee} \cong F,
    \]
    so $P^{\vee}$ is a direct summand of a free module, hence it is projective.

    Since we have $P^{\vee} \oplus K^{\vee} \cong F^{\vee}$, applying the dual functor again yields
    \[
    P^{\vee \vee} \oplus K^{\vee \vee} \cong F^{\vee \vee} \cong F.
    \]
    Then since $F$ splits, the restriction of the isomorphism $F \to F^{\vee \vee}$ induces an isomorphism $P \to P^{\vee \vee}$.
\end{solution}

% Exercise 6.6
\begin{problem}
    If $f: S \to R$ is a ring homomorphism and $P$ is a projective $S$-module, then $f^{*}(P)$ is a projective $R$-module.
\end{problem}

\begin{solution}
    Recall that $f^{*}(P) = P \otimes_S R$.
    Since $P$ is a projective $S$-module, there exists an $S$-module $K$ such that $P \oplus K \cong S^{n}$.
    Then since $f^{*}$ is left-adjoint, it preserves colimits, hence
    \[
    f^{*}(P) \oplus f^{*}(K) \cong f^{*}(P \oplus K) \cong f^{*}(S^{n}) = R^{n},
    \]
    hence $f^{*}(P)$ is a projective $R$-module.
\end{solution}

% Exercise 6.7
\begin{problem}
    Give a proof of the fact that projective modules are flat, using adjunction.
    (Hint: $R\mathsf{-Mod}$ has enough injectives, and Exercise 6.3.)
\end{problem}

\begin{solution}
    Let $P$ be a projective module.
    We want to show that the functor $- \otimes_R P$ is (left-)exact;
    that is, given an injective map $\varphi : M \to N$, the induced map $M \otimes_R P \to N \otimes_R P$ is also injective.

    Let
    \[
    \begin{tikzcd}
        0 & A & B
        \arrow[from=1-1, to=1-2] 
        \arrow[from=1-2, to=1-3] 
    \end{tikzcd}
    \]
    is exact.
    Then by Exercise 6.3, the sequence
    \[
    \begin{tikzcd}
        \Hom_R(B, Q) & \Hom_R(A, Q) & 0
        \arrow[from=1-1, to=1-2]
        \arrow[from=1-2, to=1-3] 
    \end{tikzcd}
    \]
    is exact for all injective modules $Q$.
    Applying $\Hom_R(P, -)$ to the above sequence induces an exact sequence
    \[
    \begin{tikzcd}
        \Hom_R(P, \Hom_R(B, Q)) & \Hom_R(P, \Hom_R(A, Q)) & 0
        \arrow[from=1-1, to=1-2]
        \arrow[from=1-2, to=1-3]
    \end{tikzcd}
    \]
    Applying tensor-hom adjunction (and the commutativity of $\otimes_R$) yields the exact sequence
    \[
    \begin{tikzcd}
        \Hom_R(B \otimes_R P, Q) & \Hom_R(A \otimes_R P, Q) & 0
        \arrow[from=1-1, to=1-2]
        \arrow[from=1-2, to=1-3]
    \end{tikzcd}
    \]
    Since this holds for all injective modules $Q$, Exercise 6.3 implies that the map $A \otimes_R P \hookrightarrow B \otimes_R P$ is injective, hence $P$ is a flat $R$-module.
\end{solution}

% Exercise 6.8
\begin{problem}
    Prove that Exercises 2.24 and 6.7 together imply the result of Exercise VI.5.5.
\end{problem}

\begin{solution}
    Recall that Exercise VI.5.5 asks to show that projective modules over local rings are free.
    In Exercise 2.24, we show that a finitely generated module over a local ring is flat if and only if it is free.
    Combining this with the results of Exercise 6.7, we find that a projective module over a local ring is flat, hence it is free.
\end{solution}

% Exercise 6.9
\begin{problem}
    Prove that vector spaces are flat, injective, and projective.
    (Try to do this directly, without invoking Baer's criterion.)
\end{problem}

\begin{solution}
    Let $f: A \hookrightarrow B$ be an injective map of vector spaces over a field $k$ and let $V$ be a vector space of dimension $n$.
    We claim that the induced map $f \otimes 1 : A \otimes_k V \to B \otimes_k V$ is injective.
    It suffices to check that the map is injective on a basis.
    Let $a \otimes v$ be a basis vector and suppose $f \otimes 1(a \otimes v) = f(a) \otimes v = 0$.
    Then either $f(a) = 0$, in which case $a = 0$, or $v = 0$.
    Thus, $f \otimes 1$ is injective, hence $V$ is flat.

    Let $f: A \hookrightarrow B$ be an injective map of vector spaces.
    We claim that the induced map $f^{*}: \Hom_k(B, V) \to \Hom_k(A, V)$ is surjective.
    Indeed, let $\{x_i\}$ be a basis for $A$.
    Then by the injectivity of $f$, $\{f(x_i)\}$ is a linearly independent set of vectors in $B$, hence it may be extended to a basis for $B$.
    Now let $g: A \to V$ be a map and define a map $\bar{g} : B \to V$ on a basis by $\bar{g}(f(x_i)) = g(x_i)$ and $\bar{g}(x) = 0$ for all basis vectors not in the image of $f$.
    Then it is clear that $\bar{g} \circ f(x) = g(x)$ for all $x \in A$, hence $V$ is injective.

    It is clear that vector spaces are projective because they are free.
\end{solution}

% Exercise 6.10
\begin{problem}
    Prove that a finitely generated module over a PID is free if and only if it is projective if and only if it is flat.
    (Do this `by hand', without invoking Exercise 6.12.)
\end{problem}

\begin{solution}
    Suppose $M$ is a finitely generated module over a PID.
    All free modules are projective, so if $M$ is free then it is projective.
    Similarly, all projective modules are flat, so if $M$ is projective, then it is flat.

    The only nontrivial part is showing that $M$ flat implies it is free.
    Recall that a flat module is torsion free since the map $R \otimes_R M \to RM$ is an isomorphism, hence injective, hence $rm = 0$ for all $m \in M$ implies $r = 0$.
    But then by the classification of finitely generated modules over a PID, $M$ is free.
\end{solution}

% Exercise 6.11
\begin{problem}
    Prove that a finitely generated module over a local ring is projective if and only if it is free.
    (You have already done this, in Exercise VI.5.5.)

    Besides PIDs and local rings, there are other classes of rings for which projective and free modules coincide.
    Topological considerations suggested to Serre that projective modules over a polynomial ring over a field should necessarily by free, but it took two decades to prove that this is indeed the case.
\end{problem}

\begin{solution}
    One direction is trivial since free modules are projective.
    Let $P$ be a finitely generated projective module over a local ring $R$.
    Then $P/\mathfrak{m}P$ is a finite-dimensional vector space over $R/\mathfrak{m}$.
    Choose a basis for this vector space and apply Nakayama's lemma to lift it to a minimal generating set for $P$ over $R$.
    Then we have an exact sequence
    \[
    \begin{tikzcd}
        0 & K & R^{n} & P & 0
        \arrow[from=1-1, to=1-2]
        \arrow[from=1-2, to=1-3] 
        \arrow[from=1-3, to=1-4]
        \arrow[from=1-4, to=1-5] 
    \end{tikzcd}
    \]
    where $K$ is the kernel of the surjection onto $P$.
    Since $P$ is projective, this exact sequence splits, hence we have $R^{n} \cong P \oplus K$.
    Quotienting by $\mathfrak{m}$, we have an isomorphism
    \[
        \left(\frac{R}{\mathfrak{m}}\right)^{n} \cong \frac{P}{\mathfrak{m}P} \oplus \frac{K}{\mathfrak{m}K}.
    \]
    In particular, $K/\mathfrak{m}K$ is finite-dimensional, hence $K$ is finitely generated.
    But since $P/\mathfrak{m}P$ is an $n$-dimensional vector space, we must have $K/\mathfrak{m}K = 0$, hence $K = \mathfrak{m}K$, so by Nakayama's lemma, $K = 0$.
    That is, $P \cong R^{n}$, hence $P$ is free.
\end{solution}

% Exercise 6.12
\begin{problem}
    An $R$-module $M$ is `locally free' if $M_\mathfrak{p}$ is free as an $R_\mathfrak{p}$-module for every prime ideal $\mathfrak{p}$ of $R$.

    Prove that a finitely generated module over a Noetherian ring is locally free if and only if it is projective if and only if it is flat.
    (Use the results of Exercise 2.24 and 6.11.)

    The hypothesis of finite generation is necessary (cf. Exercise 6.13).
\end{problem}

\begin{solution}
    Suppose $M$ is a finitely generated locally free $R$-module.
    That is, $M_\mathfrak{p}$ is a finitely-generated free $R_\mathfrak{p}$-module for every prime ideal $\mathfrak{p}$ of $R$.
    In particular, $M_\mathfrak{p}$ is a flat $R_\mathfrak{p}$-module, since free modules are flat.
    But flatness is a local property (cf. Exercise 2.21), hence $M$ is a flat $R$-module.

    Conversely, if $M$ is a flat $R$-module, then $M_\mathfrak{p}$ is a flat $R_\mathfrak{p}$-module for every prime ideal $\mathfrak{p}$.
    But $R_\mathfrak{p}$ is a local Noetherian ring, hence flat modules are free.
    That is, $M_\mathfrak{p}$ is a free $R_\mathfrak{p}$-module for every prime ideal, hence $M$ is a locally free $R$-module.

    Similarly, if $M$ is a projective $R$-module, then $M_\mathfrak{p}$ is a projective $R_\mathfrak{p}$-module.
    Indeed, if $M \oplus K \cong R^{n}$, then $M_\mathfrak{p} \oplus K_\mathfrak{p} \cong R_\mathfrak{p}^{n}$ since localization is an exact functor.
    But $R_\mathfrak{p}$ is a local ring, hence finitely generated projective modules are free.
    Thus, $M$ is a locally free module.

    It remains to show that if $M$ is a finitely generated locally free module, then it is projective.
    Since $M$ is finitely generated, there is a surjection $R^{n} \to M$ with kernel $K$.
    Furthermore, $K$ is a submodule of $R^{n}$, which is Noetherian.
    Therefore, $K$ is finitely generated, hence there is a surjection $R^{m} \to K$, which induces an exact sequence
    \[
    \begin{tikzcd}
        0 & R^{m} & R^{n} & M & 0
        \arrow[from=1-1, to=1-2]
        \arrow[from=1-2, to=1-3]
        \arrow[from=1-3, to=1-4]
        \arrow[from=1-4, to=1-5] 
    \end{tikzcd}
    \]
    Let $N$ be an $R$-module.
    Then applying $\Hom_R(-, N)$ and localizing yields the exact sequence
    \[
    \begin{tikzcd}
        0 & \Hom_R(M, N)_\mathfrak{p} & N^{n}_\mathfrak{p} & N^{m}_\mathfrak{p}
        \arrow[from=1-1, to=1-2]
        \arrow[from=1-2, to=1-3]
        \arrow[from=1-3, to=1-4]
    \end{tikzcd}
    \]
    On the other hand, localizing and then applying $\Hom_{R_\mathfrak{p}}(-, N_\mathfrak{p})$ yields the exact sequence
    \[
    \begin{tikzcd}
        0 & \Hom_{R_\mathfrak{p}}(M_\mathfrak{p}, N_\mathfrak{p}) & N_\mathfrak{p}^{n} & N_\mathfrak{p}^{m}
        \arrow[from=1-1, to=1-2]
        \arrow[from=1-2, to=1-3]
        \arrow[from=1-3, to=1-4]
    \end{tikzcd}
    \]
    It follows that $\Hom_R(M, N)_\mathfrak{p} \cong \Hom_{R_\mathfrak{p}}(M_\mathfrak{p}, N_\mathfrak{p})$ for finitely presented modules $M$.
    The result follows trivially from this;
    indeed, if $M$ is locally free, then for all $R$-modules $N$ we have $\Ext^{1}_{R_\mathfrak{p}}(M_\mathfrak{p}, N_\mathfrak{p}) = 0$ because $M_\mathfrak{p}$ is free, hence projective.
    But since $M$ is finitely presented, $\Ext_{R_\mathfrak{p}}^{1}(M_\mathfrak{p}, N_\mathfrak{p}) \cong \Ext_R^{1}(M, N)_\mathfrak{p}$.
    Since this module is 0 for all prime ideals $\mathfrak{p} \subseteq R$, $\Ext_R^{1}(M, N) = 0$.
    This holds for all $R$-modules $N$, hence $M$ is projective.
\end{solution}

% Exercise 6.13
\begin{problem}
    Prove that $\mathbb{Q}$ is flat, but not projective, as a $\mathbb{Z}$-module.
\end{problem}

\begin{solution}
    By Exercise 2.20, a module over a PID is flat if and only if it is torsion-free.
    Therefore, $\mathbb{Q}$ is flat over $\mathbb{Z}$.
    On the other hand, $\mathbb{Q}$ is not projective over $\mathbb{Z}$.
    Indeed, if $\mathbb{Q}$ were projective then it would be a direct summand of a free module, hence free itself (submodules of free modules over a PID are free).
    We claim $\mathbb{Q}$ is not free.
    Suppose $\mathbb{Q} = \mathbb{Z}^{S}$ for some nonempty set $S$.
    Then given a nonzero set map $S \to \mathbb{Z}$, by the universal property of free modules, this induces a nonzero module morphism $\mathbb{Q} \to \mathbb{Z}$.
    But $\Hom_\mathbb{Z}(\mathbb{Q}, \mathbb{Z}) = 0$, a contradiction.
    Thus, $\mathbb{Q}$ is not a projective $\mathbb{Z}$-module.
\end{solution}

% Exercise 6.14
\begin{problem}
    Prove that a module over a PID is injective if and only if it is divisible.
    (Use Baer's criterion.)
\end{problem}

\begin{solution}
    Recall that an $R$-module $M$ is divisible if $rM = M$ for every non-zero-divisor $r \in R$.
    Baer's criterion states that a module $M$ is injective if and only if every $R$-linear map $f: I \to M$ where $I$ is an ideal of $R$ extends to a map $\hat{f}:R \to M$ such that $f = \hat{f} \circ i$ where $i$ is the inclusion of the ideal.

    First suppose $M$ is injective, fix $a \in R$, $a \neq 0$, and let $m \in M$. 
    Consider the map $f : R \to M$ defined by $f(r) = rm$.
    Since $M$ is injective, there exists a map $\hat{f} : R \to M$ which makes the diagram
    \[
    \begin{tikzcd}
        0 & R & R \\
          & M
          \arrow[from=1-1, to=1-2]
          \arrow[hook, from=1-2, to=1-3, "\cdot a"]
          \arrow[from=1-2, to=2-2, "f"']
          \arrow[from=1-3, to=2-2, "\exists \hat{f}", dashed] 
    \end{tikzcd}
    \]
    commute.
    In particular, we use the fact that since $M$ is injective, the map $(a) \to M$ lifts to a map $R \to M$.
    Now consider $m = f(1)$.
    By commutativity of the diagram, $m = \hat{f}(s) \cdot a$ for some $s \in R$, hence $M$ is divisible.

    Now suppose $M$ is a divisible module.
    By Baer's criterion, it suffices to check that every map $f: (a) \to M$ lifts to a map $\hat{f} : R \to M$.
    If $a = 0$, simply take $\hat{f}(r) = 0$.
    If $a \neq 0$, since $M$ is divisible, there exists $n \in M$ such that $an = f(a)$.
    Then define $\hat{f}(1) = n$ and extend by linearity.
    We verify that this map satisfies the relevant commutative diagram:
    \[
        \hat{f} \circ i(a) = \hat{f}(a) = an = f(a).
    \]
    Thus, $M$ is injective.
\end{solution}

% Exercise 6.15
\begin{problem}
    Prove that $Q_1 \oplus Q_2$ is injective if and only if $Q_1, Q_2$ are both injective.
    More generally, prove that if $\{Q_i\}_{i \in I}$ is a family of $R$-modules, then $\prod_i Q_i$ is injective if and only if each $Q_i$ is injective.
\end{problem}

\begin{solution}
    For a categorical proof using adjunction, note that $\Hom_R(-, Q)$ is a right-adjoint functor (right-adjoint to its corresponding functor in the opposite category).
    Thus, it preserves limits;
    that is, for all $R$-modules $M$, 
    \[
    \Hom_R(M, \prod_i Q_i) \cong \prod_i \Hom_R(M, Q_i).
    \]
    Now if $M \to N$ is an injective map, then $\prod_i \Hom_R(N, Q_i) \to \prod_i \Hom_R(M, Q_i)$ is surjective if and only if $\Hom_R(N, \prod_i Q_i) \to \Hom_R(M, \prod_i Q_i)$ is surjective.
    Thus, $\prod_i Q_i$ is injective if and only if each $Q_i$ is injective.
\end{solution}

% Exercise 6.16
\begin{problem}
    Prove that $\mathbb{Z} \oplus \mathbb{Q}$ is flat as a $\mathbb{Z}$-module but neither projective nor injective.
\end{problem}

\begin{solution}
    First note that $\mathbb{Z}$ is a free, hence flat, $\mathbb{Z}$-module.
    Similarly, $\mathbb{Q}$ is a localization of $\mathbb{Z}$, hence flat.
    Furthermore, $\otimes$ commutes with colimits, hence a direct sum of flat modules is flat, hence $\mathbb{Z} \oplus \mathbb{Q}$ is a flat $\mathbb{Z}$-module.

    On the other hand, a direct sum of modules is projective if and only if each summand is projective.
    $\mathbb{Q}$ is not a projective module, hence $\mathbb{Z} \oplus \mathbb{Q}$ is not projective.
    Similarly, a direct sum of modules is injective if and only if each summand is injective.
    $\mathbb{Z}$ is not an injective module (one cannot extend the injection $n \mapsto 2n$ to the identity map), hence $\mathbb{Z} \oplus \mathbb{Q}$ is not injective.
\end{solution}

% Exercise 6.17
\begin{problem}
    Prove that $\Hom_R(M, N) \cong \Ext_R^{0}(M, N)$, using any of the definitions provided for $\Ext_R^{i}(M, N)$.
\end{problem}

\begin{solution}
    Take a projective resolution of $M$:
    \[
    \begin{tikzcd}
        \cdots & P_2 & P_1 & P_0 & M & 0,
        \arrow[from=1-1, to=1-2]
        \arrow[from=1-2, to=1-3]
        \arrow[from=1-3, to=1-4]
        \arrow[from=1-4, to=1-5]
        \arrow[from=1-5, to=1-6] 
    \end{tikzcd}
    \]
    apply the contravariant functor $\Hom_R(-, N)$ to the projective part to obtain the new complex
    \[
    \begin{tikzcd}
        0 & \Hom_R(P_0, N) & \Hom_R(P_1, N) & \Hom_R(P_2, N) & \cdots
        \arrow[from=1-1, to=1-2]
        \arrow[from=1-2, to=1-3]
        \arrow[from=1-3, to=1-4]
        \arrow[from=1-4, to=1-5]
    \end{tikzcd}
    \]
    and define $\Ext_R^{i}(M, N)$ to be the $i$-th cohomology group of this complex.
    Note that $\Hom_R(-, N)$ is left-exact, hence the kernel of the map $\Hom_R(P_0, N) \to \Hom_R(P_1, N)$ is precisely the image of the map $\Hom_R(M, N) \to \Hom_R(P_0, N)$, but this map is injective, hence its image is isomorphic to $\Hom_R(M, N)$.
    Thus, $\Ext_R^{0}(M, N) \cong \Hom_R(M, N)$.
\end{solution}

% Exercise 6.18
\begin{problem}
    Prove that if $R$ is a PID, then $\Ext_R^{i}(M, N) = 0$ for all $i \geq 2$ and all $R$-modules $M, N$.
\end{problem}

\begin{solution}
    Recall that every module over a PID admits a free resolution of length 2.
    Thus, start by taking a free resolution of $M$:
    \[
    \begin{tikzcd}
        0 & R^{m} & R^{n} & M & 0
        \arrow[from=1-1, to=1-2]
        \arrow[from=1-2, to=1-3]
        \arrow[from=1-3, to=1-4]
        \arrow[from=1-4, to=1-5]
    \end{tikzcd}
    \]
    Then we apply the contravariant $\Hom_R(-, N)$ to the free part, obtaining the complex
    \[
    \begin{tikzcd}
        0 & \Hom_R(R^{n}, N) & \Hom_R(R^{m}, N) & 0 & \cdots
        \arrow[from=1-1, to=1-2]
        \arrow[from=1-2, to=1-3]
        \arrow[from=1-3, to=1-4]
        \arrow[from=1-4, to=1-5]
    \end{tikzcd}
    \]
    Taking cohomology, it is clear that $\Ext_R^{i}(M, N) = 0$ for $i \geq 2$.
\end{solution}

% Exercise 6.19
\begin{problem}
    Let $r$ be a non-zero-divisor in $R$, and let $M$ be an $R$-module.
    Compute all $\Ext^{i}(R/(r), M)$.
\end{problem}

\begin{solution}
    Start with the free resolution
    \[
    \begin{tikzcd}
        0 & R & R & R/(r) & 0
        \arrow[from=1-1, to=1-2]
        \arrow[from=1-2, to=1-3, "\cdot r"]
        \arrow[from=1-3, to=1-4, "\pi"]
        \arrow[from=1-4, to=1-5]
    \end{tikzcd}
    \]
    and apply the contravariant functor $\Hom_R(-, M)$ to obtain the complex
    \[
    \begin{tikzcd}
        0 & \Hom_R(R, M) & \Hom_R(R, M) & 0 & \cdots
        \arrow[from=1-1, to=1-2]
        \arrow[from=1-2, to=1-3, "\cdot r"]
        \arrow[from=1-3, to=1-4]
        \arrow[from=1-4, to=1-5]
    \end{tikzcd}
    \]
    which is isomorphic to the complex
    \[
    \begin{tikzcd}
        0 & M & M & 0 & \cdots
        \arrow[from=1-1, to=1-2]
        \arrow[from=1-2, to=1-3, "\cdot r"]
        \arrow[from=1-3, to=1-4]
        \arrow[from=1-4, to=1-5]
    \end{tikzcd}
    \]
    Then we compute $\Ext^{0}(R/(r), M) = \{m \in M : rm = 0\}$ is the $r$-torsion of $M$.
    We also compute
    \[
        \Ext^{1}(R/(r), M) = \frac{M}{rM}
    \]
    and $\Ext^{i}(R/(r), M) = 0$ for all $i > 1$.
\end{solution}

% Exercise 6.20
\begin{problem}
    Let $I$ be an ideal of $R$.
    Prove that $\Ext_R^{1}(R/I, R/I) \cong \Hom_R(I/I^2, R/I)$.
    (Cf. Exercise 5.4.
    This says that this Ext module essentially computes the normal bundle of an embedding.)
\end{problem}

\begin{solution}
    Consider the exact sequence
    \[
    \begin{tikzcd}
        0 & I & R & R/I & 0.
        \arrow[from=1-1, to=1-2]
        \arrow[from=1-2, to=1-3]
        \arrow[from=1-3, to=1-4]
        \arrow[from=1-4, to=1-5] 
    \end{tikzcd}
    \]
    Applying contravariant $\Hom_R(-, R/I)$, there is an induced long exact sequence
    \[
    \begin{tikzcd}
        \cdots & \Hom_R(R, \frac{R}{I}) & \Hom_R(I, \frac{R}{I}) & \Ext_R^{1}(\frac{R}{I}, \frac{R}{I}) & 0
        \arrow[from=1-1, to=1-2]
        \arrow[from=1-2, to=1-3]
        \arrow[from=1-3, to=1-4]
        \arrow[from=1-4, to=1-5] 
    \end{tikzcd}
    \]
    where $\Ext_R^{1}(R, R/I) = 0$ because $R$ is a projective $R$-module.
    Furthermore, $\Hom_R(R, R/I) \cong R/I$.

    Now consider the image of the map $\Hom_R(R, R/I) \to \Hom_R(I, R/I)$ which sends a map $f \mapsto f \circ i$.
    If $a \in I$, then $f \circ i(a) = 0 \in R/I$, hence the image of the map is 0.
    This implies that the map $\Hom_R(I, R/I) \to \Ext_R^{1}(R/I, R/I)$ is injective.
    It is surjective by the exactness of the sequence.
    Thus, it is an isomorphism.
    By Exercise 5.4, $\Hom_R(I, R/I) \cong \Hom_R(I^2, R/I)$.
    We conclude that $\Ext_R^{1}(R/I, R/I) \cong \Hom_R(I^2, R/I)$.
\end{solution}

% Exercise 6.21
\begin{problem}
    In Exercise 2.15, we have seen why $\Tor$ is called $\Tor$.
    Why is $\Ext$ called $\Ext$?

    Let $M, N, E$ be $R$-modules.
    An \textit{extension} of $M$ by $N$ is an exact sequence
    \[
        \tag{*}
        \begin{tikzcd}
            0 & N & E & M & 0.
            \arrow[from=1-1, to=1-2]
            \arrow[from=1-2, to=1-3]
            \arrow[from=1-3, to=1-4]
            \arrow[from=1-4, to=1-5] 
        \end{tikzcd}
    \]
    Two extensions are `isomorphic' if there is a commutative diagram
    \[
        \begin{tikzcd}
            0 & N & E & M & 0 \\
            0 & N & E' & M & 0
            \arrow[from=1-1, to=1-2]
            \arrow[from=1-2, to=1-3]
            \arrow[from=1-3, to=1-4]
            \arrow[from=1-4, to=1-5] 
            \arrow[from=2-1, to=2-2]
            \arrow[from=2-2, to=2-3]
            \arrow[from=2-3, to=2-4]
            \arrow[from=2-4, to=2-5] 
            \arrow[from=1-2, to=2-2, equal] 
            \arrow[from=1-3, to=2-3]
            \arrow[from=1-4, to=2-4, equal] 
        \end{tikzcd}
    \]
    linking them.
    (Note that, by the snake lemma, the middle arrow must then be an isomorphism:
    cf. Exercise III.7.10.)
    Extensions that are isomorphic to the standard sequence 
    $
    \begin{tikzcd}
        0 & N & N \oplus M & M & 0
        \arrow[from=1-1, to=1-2]
        \arrow[from=1-2, to=1-3]
        \arrow[from=1-3, to=1-4]
        \arrow[from=1-4, to=1-5] 
    \end{tikzcd}$
    are `trivial'.

    Every extension $\mathscr{E}$ as above determines an element $\epsilon = \epsilon(\mathscr{E}) \in \Ext^{1}(M, N)$ as follows.
    The sequence $(^{*})$ induces a long exact sequence of $\Ext$, i.e.,
    \[
    \begin{tikzcd}
        \cdots & \Hom_R(E, N) & \Hom_R(N, N) & \Ext_R^{1}(M, N) & \cdots,
        \arrow[from=1-1, to=1-2]
        \arrow[from=1-2, to=1-3]
        \arrow[from=1-3, to=1-4]
        \arrow[from=1-4, to=1-5] 
    \end{tikzcd}
    \]
    and we let $\epsilon(\mathscr{E})$ be the image in $\Ext_R^{1}(M, N)$ of the distinguished element $\id_N \in \Hom_R(N, N)$.

    Prove that if $\mathscr{E}$ and $\mathscr{E}'$ are isomorphic extensions, then $\epsilon(\mathscr{E}) = \epsilon(\mathscr{E}')$.
    Prove that if $\mathscr{E}$ is a trivial extension, then $\epsilon(\mathscr{E}) = 0$.
\end{problem}

\begin{solution}
    If $\mathscr{E}$ and $\mathscr{E}'$ are isomorphic extensions, then in particular, $E \cong E'$.
    Then the induced maps in the long exact sequence are the same, hence $\epsilon(\mathscr{E}) = \epsilon(\mathscr{E}')$.
    
    Now suppose $\mathscr{E}$ is the trivial extension;
    that is, $E \cong M \oplus N$.
    Then the induced long exact sequence is
    \[
    \begin{tikzcd}
        \cdots & \Hom_R(M \oplus N, N) & \Hom_R(N, N) & \Ext_R^{1}(M, N) & \cdots
        \arrow[from=1-1, to=1-2]
        \arrow[from=1-2, to=1-3]
        \arrow[from=1-3, to=1-4]
        \arrow[from=1-4, to=1-5] 
    \end{tikzcd}
    \]
    Recall that the induced map $\Hom_R(M \oplus N, N) \to \Hom_R(N, N)$ sends $f \mapsto f \circ i$ where $i : N \hookrightarrow M \oplus N$ is the inclusion.
    In particular, consider $f(m, n) = n$, which is the projection onto $N$.
    Then $f \circ i(n) = f(0, n) = n$, hence $\id_N$ is in the image of the induced map.
    But the image of this is precisely the kernel of $\Hom_R(N, N) \to \Ext_R^{1}(M, N)$.
    Thus, $\id_N$ is sent to 0, which is equivalent to saying that $\epsilon(\mathscr{E}) = 0$.
\end{solution}

% Exercise 6.22
\begin{problem}
    Exercise 6.21 teaches us that extensions determine elements of $\Ext_R^{1}$.
    We get an even sharper statement by constructing an \textit{inverse} to the map $\epsilon$:
    for every element $\kappa \in \Ext_R^{1}(M, N)$, we will construct an extension $e(\kappa)$ such that $\epsilon(e(\kappa)) = \kappa$ and such that $e(\epsilon(\mathscr{E}))$ is isomorphic to $\mathscr{E}$.
    \begin{itemize}[leftmargin=*] 
        \item Let $F$ be any free module surjecting onto $M$, and let $i : K \hookrightarrow F$ be the kernel of $\pi : F \twoheadrightarrow M$. 
            Since $F$ is free (hence projective), a piece of the long exact sequence of $\Ext$, i.e.,
            \[
            \begin{tikzcd}
                \cdots & \Hom_R(K, N) & \Ext_R^{1}(M, N) & \Ext_R^{1}(F, N) = 0 & \cdots,
        \arrow[from=1-1, to=1-2]
        \arrow[from=1-2, to=1-3]
        \arrow[from=1-3, to=1-4]
        \arrow[from=1-4, to=1-5] 
            \end{tikzcd}
            \]
            tells us that there exists a homomorphism $k : K \to N$ mapping to the element $\kappa \in \Ext_R^{1}(M, N)$.
        \item We now have a monomorphism $(i, k) : K \to F \oplus N$.
            Let $E$ be the cokernel $(F \oplus N)/K$.
            Prove that the epimorphism $(\pi, 0) : F \oplus N \to M$ factors through this cokernel, defining an epimorphism $E \twoheadrightarrow M$.
        \item Prove that the natural monomorphism $N \cong 0 \oplus N \hookrightarrow F \oplus N$ defines a monomorphism $N \hookrightarrow E$, identifying $N$ with the kernel of $E \to M$.
        \item We let $e(\kappa)$ be the extension
            \[
            \begin{tikzcd}
                0 & N & E & M & 0
                \arrow[from=1-1, to=1-2]
                \arrow[from=1-2, to=1-3]
                \arrow[from=1-3, to=1-4]
                \arrow[from=1-4, to=1-5] 
            \end{tikzcd}
            \]
            that we have obtained.
            Prove that different choices in the procedure lead to isomorphic extensions.
        \item Prove that $e(\epsilon(\mathscr{E}))$ is isomorphic to $\mathscr{E}$ and that $\epsilon(e(\kappa)) = \kappa$.
    \end{itemize}
    The upshot is that there is a natural \textit{bijection} between $\Ext_R^{1}(M, N)$ and the set of isomorphism classes of extensions of $M$ by $N$.
    Hence the name.
    `Higher' $\Ext_R^{i}$ $(i > 1)$ may also be treated similarly:
    for example, $\Ext_R^2(M, N)$ can be analyzed in terms of two-step extensions consisting of exact sequences
    \[
    \begin{tikzcd}
        0 & N & E_1 & E_2 & M & 0,
        \arrow[from=1-1, to=1-2]
        \arrow[from=1-2, to=1-3]
        \arrow[from=1-3, to=1-4]
        \arrow[from=1-4, to=1-5] 
        \arrow[from=1-5, to=1-6] 
    \end{tikzcd}
    \]
    where two such extensions are `isomorphic' if there is a diagram
    \[
    \begin{tikzcd}
        0 & N & E_1 & E_2 & M & 0 \\
        0 & N & E_1' & E_2' & M & 0
        \arrow[from=1-1, to=1-2]
        \arrow[from=1-2, to=1-3]
        \arrow[from=1-3, to=1-4]
        \arrow[from=1-4, to=1-5] 
        \arrow[from=1-5, to=1-6] 
        \arrow[from=2-1, to=2-2]
        \arrow[from=2-2, to=2-3]
        \arrow[from=2-3, to=2-4]
        \arrow[from=2-4, to=2-5] 
        \arrow[from=2-5, to=2-6] 
        \arrow[from=1-2, to=2-2, equals]
        \arrow[from=1-3, to=2-3]
        \arrow[from=1-4, to=2-4]
        \arrow[from=1-5, to=2-5, equals] 
    \end{tikzcd}
    \]
    This approach to $\Ext$ is attributed to Nobuo Yoneda.
\end{problem}

\begin{solution}
    The epimorphism $(\pi, 0): F \oplus N \to M$ sends $(f, n) \mapsto \pi(f)$.
    Since $F$ surjects onto $M$, there is an isomorphism $\varphi: F/K \to M$.
    Consider the map $(\varphi, 0): (F \oplus N) / K$ which sends $(f, n) + K \mapsto \varphi(f + K) \in M$.
    Then $(\varphi, 0) \circ \pi_K(f, n) = (\varphi, 0)((f, n) + K) = \varphi(f + K) = (\pi, 0)(f, n)$, where $\pi_K$ is the projection $F \twoheadrightarrow F/K$, hence the map factors through $E$.
    Furthermore, this map is a surjection because for all $m \in M$, we may identify $m = \varphi(f + K)$.
    Then we find $(\varphi, 0)((f + K, n)) = \varphi(f + K) = m$, as desired.

    In particular, the kernel of the epimorphism $E \twoheadrightarrow M$ is precisely $N$.
    To verify this, there is a natural embedding $N \hookrightarrow F \oplus N$, which extends to a map $N \to E$ which sends $n \mapsto (0, n) \mapsto (0, n) + K$.
    We claim that this map is injective;
    indeed, $K$ is a submodule of $F$, hence $(m, n) + K = 0$ if and only if $m \in K$.
    In particular, the choice of $n$ does not matter, hence the kernel of the epimorphism $E \twoheadrightarrow M$ is $0 \oplus N \cong N$.

    This yields the exact sequence
    \[
    \begin{tikzcd}
        0 & N & E & M & 0
        \arrow[from=1-1, to=1-2]
        \arrow[from=1-2, to=1-3]
        \arrow[from=1-3, to=1-4]
        \arrow[from=1-4, to=1-5] 
    \end{tikzcd}
    \]
    If $F'$ is a different free module surjecting onto $M$ with kernel $K'$, then we find that $F' \oplus N \to M$ induces a surjection $E' = (F' \oplus N)/K' \to M$ by the same reasoning as above.

    We find that $E \cong E'$ since we can identify an element $(f', n) + K' \in E'$ with its image $f' + K' \in F'/K' \cong M$.
    Then there is exactly one corresponding element $f + K \in F/K \cong M$ mapped to by $(f, n) + K$.
    This essentially yields the needed isomorphism for the diagrams to commute, hence the two extensions are isomorphic.

    The two maps are probably inverses, but I think you have to trace through with the snake lemma to verify this which I can't be bothered to do right now.
    Instead, I'll illustrate an explicit example.
    We will construct all extensions
    \[
    \begin{tikzcd}
        0 & \frac{\mathbb{Z}}{2\mathbb{Z}} & M & \frac{\mathbb{Z}}{2\mathbb{Z}} & 0
        \arrow[from=1-1, to=1-2]
        \arrow[from=1-2, to=1-3]
        \arrow[from=1-3, to=1-4]
        \arrow[from=1-4, to=1-5] 
    \end{tikzcd}
    \]
    First we compute $\Ext^{1}(\mathbb{Z}/2\mathbb{Z}, \mathbb{Z}/2\mathbb{Z})$.
    There is a free resolution
    \[
    \begin{tikzcd}
        0 & \mathbb{Z} & \mathbb{Z} & \frac{\mathbb{Z}}{2\mathbb{Z}} & 0
        \arrow[from=1-1, to=1-2]
        \arrow[from=1-2, to=1-3, "\cdot 2"]
        \arrow[from=1-3, to=1-4]
        \arrow[from=1-4, to=1-5] 
    \end{tikzcd}
    \]
    and applying $\Hom(-, \mathbb{Z}/2\mathbb{Z})$ yields
    \[
    \begin{tikzcd}
        0 & \Hom\left(\mathbb{Z}, \frac{\mathbb{Z}}{2\mathbb{Z}}\right) \cong \frac{\mathbb{Z}}{2\mathbb{Z}} & \Hom\left(\mathbb{Z}, \frac{\mathbb{Z}}{2\mathbb{Z}}\right) \cong \frac{\mathbb{Z}}{2\mathbb{Z}} & 0
        \arrow[from=1-1, to=1-2]
        \arrow[from=1-2, to=1-3, "\cdot 2"]
        \arrow[from=1-3, to=1-4]
    \end{tikzcd}
    \]
    Taking cohomology shows that $\Ext^{1}(\mathbb{Z}/2\mathbb{Z}, \mathbb{Z}/2\mathbb{Z}) = \mathbb{Z}/2\mathbb{Z}$.
    That is, there are two isomorphism classes of extensions of the above form.
    We first construct the trivial one:
    take $\kappa = 0 \in \mathbb{Z}/2\mathbb{Z}$.
    Let $\mathbb{Z} \to \mathbb{Z}/2\mathbb{Z}$ be the projection with kernel $2\mathbb{Z}$.
    That is, we have the exact sequence
    \[
    \begin{tikzcd}
        0 & 2\mathbb{Z} & \mathbb{Z} & \frac{\mathbb{Z}}{2\mathbb{Z}} & 0
        \arrow[from=1-1, to=1-2]
        \arrow[from=1-2, to=1-3]
        \arrow[from=1-3, to=1-4]
        \arrow[from=1-4, to=1-5] 
    \end{tikzcd}
    \]
    Then applying $\Hom(-, \mathbb{Z}/2\mathbb{Z})$, there is an induced exact sequence
    \[
    \begin{tikzcd}
        \cdots & \Hom\left(\mathbb{Z}, \frac{\mathbb{Z}}{2\mathbb{Z}}\right) & \Hom\left(2\mathbb{Z}, \frac{\mathbb{Z}}{2\mathbb{Z}}\right) & \Ext^{1}\left(\frac{\mathbb{Z}}{2\mathbb{Z}}, \frac{\mathbb{Z}}{2\mathbb{Z}}\right) & 0
        \arrow[from=1-1, to=1-2]
        \arrow[from=1-2, to=1-3]
        \arrow[from=1-3, to=1-4]
        \arrow[from=1-4, to=1-5] 
    \end{tikzcd}
    \]
    Since the differentials are all module homomorphisms, the map sent to $\kappa = 0$ is the zero map $k: 2\mathbb{Z} \to \mathbb{Z}/2\mathbb{Z}$, $k(n) = 0$.
    The image of $2\mathbb{Z}$ in $\mathbb{Z}/2\mathbb{Z}$ under this map is simply $0$.
    Then letting $E := (\mathbb{Z} \oplus \mathbb{Z}/2\mathbb{Z}) / 2\mathbb{Z}$, there is a natural epimorphism $\pi: E \to \mathbb{Z}/2\mathbb{Z}$ whose kernel is $\mathbb{Z}/2\mathbb{Z}$.
    Then it becomes clear that $E \cong \mathbb{Z}/2\mathbb{Z} \oplus \mathbb{Z}/2\mathbb{Z}$, reconstructing the trivial sequence.

    On the other hand, we can take $\kappa = 1 \in \Ext^{1}(\mathbb{Z}/2\mathbb{Z}, \mathbb{Z}/2\mathbb{Z})$.
    Then we can take the same exact sequence
    \[
    \begin{tikzcd}
        \cdots & \Hom\left(\mathbb{Z}, \frac{\mathbb{Z}}{2\mathbb{Z}}\right) & \Hom\left(\mathbb{Z}, \frac{\mathbb{Z}}{2\mathbb{Z}}\right) & \Ext^{1}\left(\frac{\mathbb{Z}}{2\mathbb{Z}}, \frac{\mathbb{Z}}{2\mathbb{Z}}\right) & 0
        \arrow[from=1-1, to=1-2]
        \arrow[from=1-2, to=1-3]
        \arrow[from=1-3, to=1-4]
        \arrow[from=1-4, to=1-5] 
    \end{tikzcd}
    \]
    which shows that the non-trivial map $k : \mathbb{Z} \to \mathbb{Z}/2\mathbb{Z}$, $k(n) = n \pmod 2$ maps to $\kappa$.
    The image $\mathbb{Z} \hookrightarrow \mathbb{Z} \oplus \mathbb{Z}/2\mathbb{Z}$ is the set $K = \{(2n, n \pmod 2) : n \in \mathbb{Z}\}$.
    Then we set
    \[
    E := \frac{\mathbb{Z} \oplus \mathbb{Z}/2\mathbb{Z}}{K}.
    \]
    Finally, writing out our addition table makes it clear that $E \cong \mathbb{Z}/4\mathbb{Z}$.
\end{solution}
\end{document}
